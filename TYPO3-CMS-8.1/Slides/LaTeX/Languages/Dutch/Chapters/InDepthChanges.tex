% ------------------------------------------------------------------------------
% TYPO3 CMS 8.1 - What's New - Chapter "In-Depth Changes" (English Version)
%
% @author	Patrick Lobacher <patrick@lobacher.de> and Michael Schams <schams.net>
% @license	Creative Commons BY-NC-SA 3.0
% @link		http://typo3.org/download/release-notes/whats-new/
% @language	English
% ------------------------------------------------------------------------------
% LTXE-CHAPTER-UID:		e2881b46-7caaff58-912a10b9-7d517f6e
% LTXE-CHAPTER-NAME:	In-Depth Changes
% ------------------------------------------------------------------------------

\section{Systeemwijzigingen}
\begin{frame}[fragile]
	\frametitle{Systeemwijzigingen}

	\begin{center}\huge{Hoofdstuk 3:}\end{center}
	\begin{center}\huge{\color{typo3darkgrey}\textbf{Systeemwijzigingen}}\end{center}

\end{frame}


% ------------------------------------------------------------------------------
% LTXE-SLIDE-START
% LTXE-SLIDE-UID:		f2fcf0d4-2478188a-eb21ad15-c912742c
% LTXE-SLIDE-ORIGIN:	861205b5-d3f4a55e-69279ab1-e18dcb15 English
% LTXE-SLIDE-TITLE:		Feature: #75454 - PHP Library Doctrine DBAL (1)
% LTXE-SLIDE-REFERENCE:	Feature-75454-DoctrineDBALForDatabaseConnections.rst
% ------------------------------------------------------------------------------
\begin{frame}[fragile]
	\frametitle{Systeemwijzigingen}
	\framesubtitle{PHP Bibliotheek "Doctrine DBAL" (1)}

	\begin{itemize}

		\item De PHP-bibliotheek
			"\href{http://www.doctrine-project.org}{Doctrine DBAL}"
			is toegevoegd als composer-afhankelijkheid
			om als een krachtige database-abstractielaag te werken met vele opties
			voor database-abstractie, schema-analyse en schemabeheer binnen
			TYPO3 CMS

		\item Een TYPO3-specifieke PHP-klasse
			\texttt{TYPO3\textbackslash
				CMS\textbackslash
				Core\textbackslash
				Database\textbackslash
				ConnectionPool}\newline
			is toegevoegd voor het beheer van databaseconnecties

		\item Alle connecties geconfigureerd onder
			\texttt{\$GLOBALS['TYPO3\_CONF\_VARS']['DB']['Connections']}\newline
			zijn toegankelijk via deze beheerder waardoor parallel gebruik van
			verschillende databasesystemen mogelijk is

	\end{itemize}

\end{frame}

% ------------------------------------------------------------------------------
% LTXE-SLIDE-START
% LTXE-SLIDE-UID:		bbd6271c-788b001c-5f82cbe6-fc6393a3
% LTXE-SLIDE-ORIGIN:	988bc43f-175df2eb-d7158156-f8f36703 English
% LTXE-SLIDE-TITLE:		Feature: #75454 - PHP Library Doctrine DBAL (2)
% LTXE-SLIDE-REFERENCE:	Feature-75454-DoctrineDBALForDatabaseConnections.rst
% ------------------------------------------------------------------------------
\begin{frame}[fragile]
	\frametitle{Systeemwijzigingen}
	\framesubtitle{PHP Bibliotheek "Doctrine DBAL" (2)}

	\begin{itemize}

		\item Met het gebruik van de database-abstractiemogelijkheden en de QueryBuilder
		 	worden de opgebouwde SQL-opdrachten correct gecodeerd en zijn ze automatisch
		 	zoveel mogelijk compatibel met de verschillende DBMS'en

		\item Bestaande opties in \texttt{\$GLOBALS['TYPO3\_CONF\_VARS']['DB']} zijn
			verwijderd en/of gemigreerd naar de nieuwe Doctrine-opties

		\item De \texttt{Connection} klasse biedt handige functies voor
			\texttt{insert}, \texttt{select}, \texttt{update}, \texttt{delete} en
			\texttt{truncate} opdrachten

		\item Voor \texttt{select}, \texttt{update} en \texttt{delete} worden alleen simpele
			vergelijkingen (zoals \texttt{WHERE "aField" = 'aValue'}) ondersteund.
			Voor complexe opdrachten is de \texttt{QueryBuilder} nodig.

	\end{itemize}

\end{frame}

% ------------------------------------------------------------------------------
% LTXE-SLIDE-START
% LTXE-SLIDE-UID:		49923bf4-521dc8d0-c446bd9d-bb47528c
% LTXE-SLIDE-ORIGIN:	4e6e640e-d37fcca8-bd0ca330-a70baaab English
% LTXE-SLIDE-TITLE:		Feature: #75454 - PHP Library Doctrine DBAL (3)
% LTXE-SLIDE-REFERENCE:	!Feature-75454-DoctrineDBALForDatabaseConnections.rst
% ------------------------------------------------------------------------------
\begin{frame}[fragile]
	\frametitle{Systeemwijzigingen}
	\framesubtitle{PHP Bibliotheek "Doctrine DBAL" (3)}

	% decrease font size for code listing
	\lstset{basicstyle=\tiny\ttfamily}

	\begin{itemize}

		\item De \texttt{ConnectionPool} klasse kan zo gebruikt worden:
			\begin{lstlisting}
				// Haal verbinding voor meerdere bewerkingen
				/** @var \TYPO3\CMS\Core\Database\Connecction $conn */
				$conn = GeneralUtility::makeInstance(ConnectionPool::class)->getConnectionForTable('aTable');
				$affectedRows = $conn->insert(
				  'aTable',
				  $fields, // Array met kolom/waarde paren, automatisch gecodeerd
				);

				// Haal QueryBuilder (voor eenmalig gebruik)
				$query = GeneralUtility::makeInstance(ConnectionPool::class)->getQueryBuilderForTable('aTable);
				$query->select('*')
				  ->from('aTable)
				  ->where($query->expr()->eq('aField', $query->createNamedParameter($aValue)))
				  ->andWhere(
					$query->expr()->lte(
					  'anotherField',
					  $query->createNamedParameter($anotherValue)
					)
				  )
				$rows = $query->execute()->fetchAll();
			\end{lstlisting}
	\end{itemize}

\end{frame}

% ------------------------------------------------------------------------------
% LTXE-SLIDE-START
% LTXE-SLIDE-UID:		8c2671a7-837fd891-3d94a0e1-fb12ea56
% LTXE-SLIDE-ORIGIN:	443c416c-63c0f980-719e38f1-1d117e50 English
% LTXE-SLIDE-TITLE:		Feature: #69439 - Enhance SQL query reduction in page tree in workspaces
% LTXE-SLIDE-REFERENCE:	!Feature-69439-EnhanceSQLQueryReductionInPageTreeInWorkspaces.rst
% ------------------------------------------------------------------------------
\begin{frame}[fragile]
	\frametitle{Systeemwijzigingen}
	\framesubtitle{Verbeterde SQL-query reductie in paginaboom in werkruimtes}

	% decrease font size for code listing
	\lstset{basicstyle=\tiny\ttfamily}

	\begin{itemize}

		\item Het vaststellen of een pagina werkruimteversies heeft kan uitgebreid
		 	worden door maatwerkcode via hooks

		\item Hierdoor kan de betekenis van het hebben van versies verder aangepast
		 	worden via hooks

		\item Standaard maakt TYPO3 bijvoorbeeld een werkruimte versie van een record bij
		 	het opslaan van van hetzelfde record in de backend, zonder dat er daadwerkelijk
		 	data in het model is veranderd

			\begin{lstlisting}
				$GLOBALS['TYPO3_CONF_VARS']['SC_OPTIONS']...
				  ...['TYPO3\\CMS\\Workspaces\\Service\\WorkspaceService']['hasPageRecordVersions'];

				$GLOBALS['TYPO3_CONF_VARS']['SC_OPTIONS']...
				  ...['TYPO3\\CMS\\Workspaces\\Service\\WorkspaceService']['fetchPagesWithVersionsInTable']
			\end{lstlisting}

	\end{itemize}

\end{frame}


% ------------------------------------------------------------------------------
% LTXE-SLIDE-START
% LTXE-SLIDE-UID:		b45899ee-c585e72c-1eed5b90-29026676
% LTXE-SLIDE-ORIGIN:	630de169-c2699a8a-ee44dde5-cf39c2ca English
% LTXE-SLIDE-TITLE:		Feature: #70056 - PHP Library "Guzzle" (1)
% LTXE-SLIDE-REFERENCE:	!Feature-70056-GuzzleForHttpRequests.rst
% ------------------------------------------------------------------------------
\begin{frame}[fragile]
	\frametitle{Systeemwijzigingen}
	\framesubtitle{PHP Bibliotheek "Guzzle" (1)}

	\begin{itemize}

		\item De PHP-bibliotheek
			"\href{http://docs.guzzlephp.org}{Guzzle}"
			is toegevoegd via een composerafhankelijkheid als oplossing voor het
			maken van HTTP aanvragen gebaseerd op de PSR-7 interfaces die al
			door TYPO3 gebruikt worden

		\item Guzzle detecteert vanzelf welke onderliggende adapters -- zoals
		 	cURL or stream wrappers -- beschikbaar zijn en kiest dan de beste
		 	oplossing voor het systeem

		\item Een TYPO3-specifiek PHP-klasse
			\texttt{TYPO3\textbackslash
				CMS\textbackslash
				Core\textbackslash
				Http\textbackslash
				RequestFactory}\newline
			is toegevoegd als simpele wrapper voor toegang tot Guzzle clienten

	\end{itemize}

\end{frame}


% ------------------------------------------------------------------------------
% LTXE-SLIDE-START
% LTXE-SLIDE-UID:		117f8848-7445053f-0e97fd4d-78eb43cc
% LTXE-SLIDE-ORIGIN:	db2b31c1-47321ffe-3ed51ebc-5dd524a4 English
% LTXE-SLIDE-TITLE:		Feature: #70056 - PHP Library "Guzzle" (2)
% LTXE-SLIDE-REFERENCE:	!Feature-70056-GuzzleForHttpRequests.rst
% ------------------------------------------------------------------------------
\begin{frame}[fragile]
	\frametitle{Systeemwijzigingen}
	\framesubtitle{PHP Bibliotheek "Guzzle" (2)}

	% decrease font size for code listing
	\lstset{basicstyle=\tiny\ttfamily}

	\begin{itemize}

		\item De \texttt{RequestFactory} klasse kan zo gebruikt worden:

			\begin{lstlisting}
				// Initialiseer RequestFactory

				/** @var \TYPO3\CMS\Core\Http\RequestFactory $requestFactory */
				$requestFactory = GeneralUtility::makeInstance(
				  \TYPO3\CMS\Core\Http\RequestFactory\RequestFactory::class);

				$uri = $additionalOptions = [
				  // extra headers voor deze opdracht
				  'headers' => ['Cache-Control' => 'no-cache'],
				  'allow_redirects' => false,
				  'cookies' => true
				];

				// geef een PSR-7 response object terug
				$response = $requestFactory->request($url, 'GET', $additionalOptions);

				// haal de inhoud op als string bij een succesvolle opdracht
				if ($response->getStatusCode() === 200) {
				  if ($response->getHeader('Content-Type') === 'text/html') {
				    $content = $response->getBody()->getContents();
				  }
				}
			\end{lstlisting}

	\end{itemize}

\end{frame}

% ------------------------------------------------------------------------------
