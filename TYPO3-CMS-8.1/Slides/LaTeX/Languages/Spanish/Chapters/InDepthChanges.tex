% ------------------------------------------------------------------------------
% TYPO3 CMS 8.1 - What's New - Chapter "In-Depth Changes" (English Version)
%
% @author	Patrick Lobacher <patrick@lobacher.de> and Michael Schams <schams.net>
% @license	Creative Commons BY-NC-SA 3.0
% @link		http://typo3.org/download/release-notes/whats-new/
% @language	English
% ------------------------------------------------------------------------------
% LTXE-CHAPTER-UID:		e2881b46-7caaff58-912a10b9-7d517f6e
% LTXE-CHAPTER-NAME:	In-Depth Changes
% ------------------------------------------------------------------------------

\section{Cambios en Profundidad}
\begin{frame}[fragile]
	\frametitle{Cambios en Profundidad}

	\begin{center}\huge{Capítulo 3:}\end{center}
	\begin{center}\huge{\color{typo3darkgrey}\textbf{Cambios en Profundidad}}\end{center}

\end{frame}


% ------------------------------------------------------------------------------
% LTXE-SLIDE-START
% LTXE-SLIDE-UID:		1371c579-9fca1447-ad89f611-872d3b9f
% LTXE-SLIDE-ORIGIN:	861205b5-d3f4a55e-69279ab1-e18dcb15 English
% LTXE-SLIDE-TITLE:		Feature: #75454 - PHP Library Doctrine DBAL (1)
% LTXE-SLIDE-REFERENCE:	Feature-75454-DoctrineDBALForDatabaseConnections.rst
% ------------------------------------------------------------------------------
\begin{frame}[fragile]
	\frametitle{Cambios en Profundidad}
	\framesubtitle{Librería PHP "Doctrine DBAL" (1)}

	\begin{itemize}

		\item La librería PHP
			"\href{http://www.doctrine-project.org}{Doctrine DBAL}"
			ha sido añadida a través de la dependencia de composer
			para trabajar como una capa de abstración de base de datos potente con muchas características para
			la abstración de base de datos, introspección de esquema y manejo de esquema dentro de
			TYPO3 CMS

		\item Una clase PHP específica de TYPO3 llamada
			\texttt{TYPO3\textbackslash
				CMS\textbackslash
				Core\textbackslash
				Database\textbackslash
				ConnectionPool}\newline
			ha sido añadida como manager para conexiones de bases de datos

		\item Todas las conexiones configuradas bajo
			\texttt{\$GLOBALS['TYPO3\_CONF\_VARS']['DB']['Connections']}\newline
			son accesibles usando este configurador, habilitando el uso paralelo de
			múltiples sistemas de bases de datos

	\end{itemize}

\end{frame}

% ------------------------------------------------------------------------------
% LTXE-SLIDE-START
% LTXE-SLIDE-UID:		a62de8bb-cb6c8d94-e111c6d2-fa341c4e
% LTXE-SLIDE-ORIGIN:	988bc43f-175df2eb-d7158156-f8f36703 English
% LTXE-SLIDE-TITLE:		Feature: #75454 - PHP Library Doctrine DBAL (2)
% LTXE-SLIDE-REFERENCE:	Feature-75454-DoctrineDBALForDatabaseConnections.rst
% ------------------------------------------------------------------------------
\begin{frame}[fragile]
	\frametitle{Cambios en Profundidad}
	\framesubtitle{Librería PHP "Doctrine DBAL" (2)}

	\begin{itemize}

		\item Usando las opciones de abstracción de bases de datos y el QueryBuilder
			sentencias SQL proporcionadas siendo construidas serán entrecomilladas correctamente y serán compatibles
			con diferentes DBMS al momento tanto como sea posible

		\item Opciones existentes \texttt{\$GLOBALS['TYPO3\_CONF\_VARS']['DB']} han sido
			eliminadas y/o migradas a las nuevas opciones compatibles con Doctrine

		\item La clase \texttt{Connection} proporciona métodos adecuados para
			sentencias \texttt{insert}, \texttt{select}, \texttt{update}, \texttt{delete} y
			\texttt{truncate}

		\item Para el \texttt{select}, \texttt{update} y \texttt{delete} sólo se soportan comparaciones de igualdad
			(como \texttt{WHERE "aField" = 'aValue'}).
			Para sentencias complejas es necesario usar el \texttt{QueryBuilder}.

	\end{itemize}

\end{frame}

% ------------------------------------------------------------------------------
% LTXE-SLIDE-START
% LTXE-SLIDE-UID:		f8167f29-44c02f7c-b7c272f8-87bd2548
% LTXE-SLIDE-ORIGIN:	4e6e640e-d37fcca8-bd0ca330-a70baaab English
% LTXE-SLIDE-TITLE:		Feature: #75454 - PHP Library Doctrine DBAL (3)
% LTXE-SLIDE-REFERENCE:	!Feature-75454-DoctrineDBALForDatabaseConnections.rst
% ------------------------------------------------------------------------------
\begin{frame}[fragile]
	\frametitle{Cambios en Profundidad}
	\framesubtitle{Librería PHP "Doctrine DBAL" (3)}

	% decrease font size for code listing
	\lstset{basicstyle=\tiny\ttfamily}

	\begin{itemize}

		\item La clase \texttt{ConnectionPool} puede ser usada así:
			\begin{lstlisting}
				// Get a connection which can be used for muliple operations
				/** @var \TYPO3\CMS\Core\Database\Connecction $conn */
				$conn = GeneralUtility::makeInstance(ConnectionPool::class)->getConnectionForTable('aTable');
				$affectedRows = $conn->insert(
				  'aTable',
				  $fields, // Associative array of column/value pairs, automatically quoted & escaped
				);

				// Get a QueryBuilder, which should only be used a single time
				$query = GeneralUtility::makeInstance(ConnectionPool::class)->getQueryBuilderForTable('aTable);
				$query->select('*')
				  ->from('aTable)
				  ->where($query->expr()->eq('aField', $query->createNamedParameter($aValue)))
				  ->andWhere(
					$query->expr()->lte(
					  'anotherField',
					  $query->createNamedParameter($anotherValue)
					)
				  )
				$rows = $query->execute()->fetchAll();
			\end{lstlisting}
	\end{itemize}

\end{frame}

% ------------------------------------------------------------------------------
% LTXE-SLIDE-START
% LTXE-SLIDE-UID:		3e0be25c-66e4f428-bc5ebb74-eb848b51
% LTXE-SLIDE-ORIGIN:	443c416c-63c0f980-719e38f1-1d117e50 English
% LTXE-SLIDE-TITLE:		Feature: #69439 - Enhance SQL query reduction in page tree in workspaces
% LTXE-SLIDE-REFERENCE:	!Feature-69439-EnhanceSQLQueryReductionInPageTreeInWorkspaces.rst
% ------------------------------------------------------------------------------
\begin{frame}[fragile]
	\frametitle{Cambios en Profundidad}
	\framesubtitle{Habilitar reducción query SQL en el árbol de páginas en áreas de trabajo}

	% decrease font size for code listing
	\lstset{basicstyle=\tiny\ttfamily}

	\begin{itemize}

		\item El proceso de determinar si una página tiene versiones de áreas de trabajo
			puede ser extendido por el código personalizado de una aplicación utilizando hooks

		\item De esta manera, el significado de tener versiones puede ser modificado más allá por
		hooks

		\item Por ejemplo el comportamiento por defecto del núcleo de TYPO3 es crear
			un registro de versión de área de trabajo persistiendo el mismo registro en el
			backend - sin cambios reales en el modelo de datos

			\begin{lstlisting}
				$GLOBALS['TYPO3_CONF_VARS']['SC_OPTIONS']...
				  ...['TYPO3\\CMS\\Workspaces\\Service\\WorkspaceService']['hasPageRecordVersions'];

				$GLOBALS['TYPO3_CONF_VARS']['SC_OPTIONS']...
				  ...['TYPO3\\CMS\\Workspaces\\Service\\WorkspaceService']['fetchPagesWithVersionsInTable']
			\end{lstlisting}

	\end{itemize}

\end{frame}


% ------------------------------------------------------------------------------
% LTXE-SLIDE-START
% LTXE-SLIDE-UID:		81a5f1ad-33ea83ad-959b6c3b-74246af4
% LTXE-SLIDE-ORIGIN:	630de169-c2699a8a-ee44dde5-cf39c2ca English
% LTXE-SLIDE-TITLE:		Feature: #70056 - PHP Library "Guzzle" (1)
% LTXE-SLIDE-REFERENCE:	!Feature-70056-GuzzleForHttpRequests.rst
% ------------------------------------------------------------------------------
\begin{frame}[fragile]
	\frametitle{Cambios en Profundidad}
	\framesubtitle{Librería PHP "Guzzle" (1)}

	\begin{itemize}

		\item La librería PHP
			"\href{http://docs.guzzlephp.org}{Guzzle}"
			ha sido añadida a través de una dependencia de composer para trabajar como una solución rica añadida
			para crear peticiones HTTP basadas en los interfaces PSR-7
			ya usados dentro de TYPO3

		\item Guzzle auto-detecta adaptadores subyacentes disponibles en
			el sistema, como cURL o envolturas de stream y escoge la mejor
			solución para el sistema

		\item Una clase PHP específica de TYPO3 llamada
			\texttt{TYPO3\textbackslash
				CMS\textbackslash
				Core\textbackslash
				Http\textbackslash
				RequestFactory}\newline
			ha sido añadida como una envoltura simplificada para acceder a  clientes Guzzle

	\end{itemize}

\end{frame}


% ------------------------------------------------------------------------------
% LTXE-SLIDE-START
% LTXE-SLIDE-UID:		59af77fb-8b696471-40ee3061-9f2c5c4e
% LTXE-SLIDE-ORIGIN:	db2b31c1-47321ffe-3ed51ebc-5dd524a4 English
% LTXE-SLIDE-TITLE:		Feature: #70056 - PHP Library "Guzzle" (2)
% LTXE-SLIDE-REFERENCE:	!Feature-70056-GuzzleForHttpRequests.rst
% ------------------------------------------------------------------------------
\begin{frame}[fragile]
	\frametitle{Cambios en Profundidad}
	\framesubtitle{Librería PHP "Guzzle" (2)}

	% decrease font size for code listing
	\lstset{basicstyle=\tiny\ttfamily}

	\begin{itemize}

		\item La clase \texttt{RequestFactory} puede ser usada así:

			\begin{lstlisting}
				// Initiate RequestFactory

				/** @var \TYPO3\CMS\Core\Http\RequestFactory $requestFactory */
				$requestFactory = GeneralUtility::makeInstance(
				  \TYPO3\CMS\Core\Http\RequestFactory\RequestFactory::class);

				$uri = $additionalOptions = [
				  // additional headers for this specific request
				  'headers' => ['Cache-Control' => 'no-cache'],
				  'allow_redirects' => false,
				  'cookies' => true
				];

				// return a PSR-7 compliant response object
				$response = $requestFactory->request($url, 'GET', $additionalOptions);

				// get the content as a string on a successful request
				if ($response->getStatusCode() === 200) {
				  if ($response->getHeader('Content-Type') === 'text/html') {
				    $content = $response->getBody()->getContents();
				  }
				}
			\end{lstlisting}

	\end{itemize}

\end{frame}

% ------------------------------------------------------------------------------
