% ------------------------------------------------------------------------------
% TYPO3 CMS 8.1 - What's New - Chapter "In-Depth Changes" (French Version)
%
% @author	Patrick Lobacher <patrick@lobacher.de> and Michael Schams <schams.net>
% @license	Creative Commons BY-NC-SA 3.0
% @link		http://typo3.org/download/release-notes/whats-new/
% @language	French
% ------------------------------------------------------------------------------
% LTXE-CHAPTER-UID:		e2881b46-7caaff58-912a10b9-7d517f6e
% LTXE-CHAPTER-NAME:	In-Depth Changes
% ------------------------------------------------------------------------------

\section{Changements en profondeur}
\begin{frame}[fragile]
	\frametitle{Changements en profondeur}

	\begin{center}\huge{Chapitre 3~:}\end{center}
	\begin{center}\huge{\color{typo3darkgrey}\textbf{Changements en profondeur}}\end{center}

\end{frame}


% ------------------------------------------------------------------------------
% LTXE-SLIDE-START
% LTXE-SLIDE-UID:		39b5ffb1-48cadb48-3ca0c5d0-0093b872
% LTXE-SLIDE-ORIGIN:	861205b5-d3f4a55e-69279ab1-e18dcb15 English
% LTXE-SLIDE-TITLE:		Feature: #75454 - PHP Library Doctrine DBAL (1)
% LTXE-SLIDE-REFERENCE:	Feature-75454-DoctrineDBALForDatabaseConnections.rst
% ------------------------------------------------------------------------------
\begin{frame}[fragile]
	\frametitle{Changements en profondeur}
	\framesubtitle{Bibliothèque PHP «~Doctrine DBAL~» (1)}

	\begin{itemize}

		\item La bibliothèque PHP
			"\href{http://www.doctrine-project.org}{Doctrine DBAL}"
			est ajoutée comme dépendance composer pour fournir une
			puissante couche d'abstraction de la base de données avec des
			fonctionnalités d'abstraction, d'introspection et de gestion du
			schéma dans TYPO3 CMS

		\item Une classe spécifique à TYPO3
			\texttt{TYPO3\textbackslash
				CMS\textbackslash
				Core\textbackslash
				Database\textbackslash
				ConnectionPool}\newline
			est ajoutée comme gestionnaire de connexion aux bases de données

		\item Toutes les connexions configurées sous
			\texttt{\$GLOBALS['TYPO3\_CONF\_VARS']['DB']['Connections']}\newline
			sont accessibles sous ce gestionnaire, permettant l'usage en parallèle
			de multiples systèmes de base de données

	\end{itemize}

\end{frame}

% ------------------------------------------------------------------------------
% LTXE-SLIDE-START
% LTXE-SLIDE-UID:		f9a63dc9-e9907ee1-3935f2cf-869f2916
% LTXE-SLIDE-ORIGIN:	988bc43f-175df2eb-d7158156-f8f36703 English
% LTXE-SLIDE-TITLE:		Feature: #75454 - PHP Library Doctrine DBAL (2)
% LTXE-SLIDE-REFERENCE:	Feature-75454-DoctrineDBALForDatabaseConnections.rst
% ------------------------------------------------------------------------------
\begin{frame}[fragile]
	\frametitle{Changements en profondeur}
	\framesubtitle{Bibliothèque PHP «~Doctrine DBAL~» (2)}

	\begin{itemize}

		\item À l'aide des options d'abstraction, les instructions SQL construites
			par le QueryBuilder possèdent le bon échappement des caractères et sont
			directement compatibles avec les différents systèmes de base de données

		\item Les options \texttt{\$GLOBALS['TYPO3\_CONF\_VARS']['DB']} existantes sont
			retirées ou migrées vers les nouvelles options Doctrine

		\item La classe \texttt{Connection} fournit des méthodes pratiques pour les
			instructions \texttt{insert}, \texttt{select}, \texttt{update}, \texttt{delete}
			et \texttt{truncate}

		\item Pour \texttt{select}, \texttt{update} et \texttt{delete} seules les
		 	comparaisons d'égalité simples (like \texttt{WHERE "aField" = 'aValue'}) sont supportées.
			Pour des instructions complexes, le \texttt{QueryBuilder} est requis.

	\end{itemize}

\end{frame}

% ------------------------------------------------------------------------------
% LTXE-SLIDE-START
% LTXE-SLIDE-UID:		2a2db61e-0f76443c-35fbf4a9-83b82940
% LTXE-SLIDE-ORIGIN:	4e6e640e-d37fcca8-bd0ca330-a70baaab English
% LTXE-SLIDE-TITLE:		Feature: #75454 - PHP Library Doctrine DBAL (3)
% LTXE-SLIDE-REFERENCE:	!Feature-75454-DoctrineDBALForDatabaseConnections.rst
% ------------------------------------------------------------------------------
\begin{frame}[fragile]
	\frametitle{Changements en profondeur}
	\framesubtitle{Bibliothèque PHP «~Doctrine DBAL~» (3)}

	% decrease font size for code listing
	\lstset{basicstyle=\tiny\ttfamily}

	\begin{itemize}

		\item La classe \texttt{ConnectionPool} s'utilise comme suit~:
			\begin{lstlisting}
				// Get a connection which can be used for muliple operations
				/** @var \TYPO3\CMS\Core\Database\Connecction $conn */
				$conn = GeneralUtility::makeInstance(ConnectionPool::class)->getConnectionForTable('aTable');
				$affectedRows = $conn->insert(
				  'aTable',
				  $fields, // Associative array of column/value pairs, automatically quoted & escaped
				);

				// Get a QueryBuilder, which should only be used a single time
				$query = GeneralUtility::makeInstance(ConnectionPool::class)->getQueryBuilderForTable('aTable');
				$query->select('*')
				  ->from('aTable)
				  ->where($query->expr()->eq('aField', $query->createNamedParameter($aValue)))
				  ->andWhere(
					$query->expr()->lte(
					  'anotherField',
					  $query->createNamedParameter($anotherValue)
					)
				  )
				$rows = $query->execute()->fetchAll();
			\end{lstlisting}
	\end{itemize}

\end{frame}

% ------------------------------------------------------------------------------
% LTXE-SLIDE-START
% LTXE-SLIDE-UID:		9c13cfb2-7a054d39-56aa3507-f6c2cafd
% LTXE-SLIDE-ORIGIN:	443c416c-63c0f980-719e38f1-1d117e50 English
% LTXE-SLIDE-TITLE:		Feature: #69439 - Enhance SQL query reduction in page tree in workspaces
% LTXE-SLIDE-REFERENCE:	!Feature-69439-EnhanceSQLQueryReductionInPageTreeInWorkspaces.rst
% ------------------------------------------------------------------------------
\begin{frame}[fragile]
	\frametitle{Changements en profondeur}
	\framesubtitle{Amélioration de la réduction des requêtes SQL dans l'arborescence de pages des espaces de travail}

	% decrease font size for code listing
	\lstset{basicstyle=\tiny\ttfamily}

	\begin{itemize}

		\item La détermination de la présence d'une version dans l'espace de travail
			pour une page peut être étendue en utilisant les hooks

		\item Ainsi, la signification d'avoir une version peut être modifiée par
			les hooks

		\item Par exemple, le comportement par défaut de TYPO3 est de créer une
			version dans l'espace de travail lors de la sauvegarde d'un enregistrement
			dans le backend - sans changer le modèle de données

			\begin{lstlisting}
				$GLOBALS['TYPO3_CONF_VARS']['SC_OPTIONS']...
				  ...['TYPO3\\CMS\\Workspaces\\Service\\WorkspaceService']['hasPageRecordVersions'];

				$GLOBALS['TYPO3_CONF_VARS']['SC_OPTIONS']...
				  ...['TYPO3\\CMS\\Workspaces\\Service\\WorkspaceService']['fetchPagesWithVersionsInTable']
			\end{lstlisting}

	\end{itemize}

\end{frame}


% ------------------------------------------------------------------------------
% LTXE-SLIDE-START
% LTXE-SLIDE-UID:		5ec313aa-908de0d7-be34fa0a-bed276a7
% LTXE-SLIDE-ORIGIN:	630de169-c2699a8a-ee44dde5-cf39c2ca English
% LTXE-SLIDE-TITLE:		Feature: #70056 - PHP Library "Guzzle" (1)
% LTXE-SLIDE-REFERENCE:	!Feature-70056-GuzzleForHttpRequests.rst
% ------------------------------------------------------------------------------
\begin{frame}[fragile]
	\frametitle{Changements en profondeur}
	\framesubtitle{Bibliothèque PHP «~Guzzle~» (1)}

	\begin{itemize}

		\item La bibliothèque PHP
			"\href{http://docs.guzzlephp.org}{Guzzle}"
			est ajoutée en dépendance composer pour fournir une solution de
			création de requêtes HTTP compatibles avec l'interface PSR-7 déjà
			utilisée dans TYPO3

		\item Guzzle détecte automatiquement les adapteurs disponibles
			sur le système, comme cURL ou les stream wrappers et choisie le
			meilleur pour le système

		\item Une classe spécifique TYPO3 appelée
			\texttt{TYPO3\textbackslash
				CMS\textbackslash
				Core\textbackslash
				Http\textbackslash
				RequestFactory}\newline
			est ajoutée comme point d'entrée simplifié des clients Guzzle

	\end{itemize}

\end{frame}


% ------------------------------------------------------------------------------
% LTXE-SLIDE-START
% LTXE-SLIDE-UID:		4191685c-7037256d-92613374-d0aecb00
% LTXE-SLIDE-ORIGIN:	db2b31c1-47321ffe-3ed51ebc-5dd524a4 English
% LTXE-SLIDE-TITLE:		Feature: #70056 - PHP Library "Guzzle" (2)
% LTXE-SLIDE-REFERENCE:	!Feature-70056-GuzzleForHttpRequests.rst
% ------------------------------------------------------------------------------
\begin{frame}[fragile]
	\frametitle{Changements en profondeur}
	\framesubtitle{Bibliothèque PHP «~Guzzle~» (2)}

	% decrease font size for code listing
	\lstset{basicstyle=\tiny\ttfamily}

	\begin{itemize}

		\item La classe \texttt{RequestFactory} s'utilise comme ceci~:

			\begin{lstlisting}
				// Initiate RequestFactory

				/** @var \TYPO3\CMS\Core\Http\RequestFactory $requestFactory */
				$requestFactory = GeneralUtility::makeInstance(
				  \TYPO3\CMS\Core\Http\RequestFactory\RequestFactory::class);

				$url = 'http://typo3.org/';
				$additionalOptions = [
				  // additional headers for this specific request
				  'headers' => ['Cache-Control' => 'no-cache'],
				  'allow_redirects' => false,
				  'cookies' => true
				];

				// return a PSR-7 compliant response object
				$response = $requestFactory->request($url, 'GET', $additionalOptions);

				// get the content as a string on a successful request
				if ($response->getStatusCode() === 200) {
				  if ($response->getHeader('Content-Type') === 'text/html') {
				    $content = $response->getBody()->getContents();
				  }
				}
			\end{lstlisting}

	\end{itemize}

\end{frame}

% ------------------------------------------------------------------------------
