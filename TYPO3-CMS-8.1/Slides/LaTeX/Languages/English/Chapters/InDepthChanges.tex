% ------------------------------------------------------------------------------
% TYPO3 CMS 8.1 - What's New - Chapter "In-Depth Changes" (English Version)
%
% @author	Patrick Lobacher <patrick@lobacher.de> and Michael Schams <schams.net>
% @license	Creative Commons BY-NC-SA 3.0
% @link		http://typo3.org/download/release-notes/whats-new/
% @language	English
% ------------------------------------------------------------------------------
% LTXE-CHAPTER-UID:		e2881b46-7caaff58-912a10b9-7d517f6e
% LTXE-CHAPTER-NAME:	In-Depth Changes
% ------------------------------------------------------------------------------

\section{In-Depth Changes}
\begin{frame}[fragile]
	\frametitle{In-Depth Changes}

	\begin{center}\huge{Chapter 3:}\end{center}
	\begin{center}\huge{\color{typo3darkgrey}\textbf{In-Depth Changes}}\end{center}

\end{frame}


% ------------------------------------------------------------------------------
% LTXE-SLIDE-START
% LTXE-SLIDE-UID:		861205b5-d3f4a55e-69279ab1-e18dcb15
% LTXE-SLIDE-TITLE:		Feature: #75454 - PHP Library Doctrine DBAL (1)
% LTXE-SLIDE-REFERENCE:	Feature-75454-DoctrineDBALForDatabaseConnections.rst
% ------------------------------------------------------------------------------
\begin{frame}[fragile]
	\frametitle{In-Depth Changes}
	\framesubtitle{PHP Library "Doctrine DBAL" (1)}

	\begin{itemize}

		\item The PHP library
			"\href{http://www.doctrine-project.org}{Doctrine DBAL}"
			has been added via composer dependency
			to work as a powerful database abstraction layer with many features for
			database abstraction, schema introspection and schema management within
			TYPO3 CMS

		\item A TYPO3-specific PHP class called
			\texttt{TYPO3\textbackslash
				CMS\textbackslash
				Core\textbackslash
				Database\textbackslash
				ConnectionPool}\newline
			has been added as a manager for database connections

		\item All connections configured under
			\texttt{\$GLOBALS['TYPO3\_CONF\_VARS']['DB']['Connections']}\newline
			are accessible using this manager, enabling the parallel usage of
			multiple database systems

	\end{itemize}

\end{frame}

% ------------------------------------------------------------------------------
% LTXE-SLIDE-START
% LTXE-SLIDE-UID:		988bc43f-175df2eb-d7158156-f8f36703
% LTXE-SLIDE-TITLE:		Feature: #75454 - PHP Library Doctrine DBAL (2)
% LTXE-SLIDE-REFERENCE:	Feature-75454-DoctrineDBALForDatabaseConnections.rst
% ------------------------------------------------------------------------------
\begin{frame}[fragile]
	\frametitle{In-Depth Changes}
	\framesubtitle{PHP Library "Doctrine DBAL" (2)}

	\begin{itemize}

		\item By using the database abstraction options and the QueryBuilder
			provided SQL statements being built will be properly quoted and compatible
			with different DBMS out of the box as far as possible

		\item Existing \texttt{\$GLOBALS['TYPO3\_CONF\_VARS']['DB']} options have been
			removed and/or migrated to the new Doctrine-compliant options

		\item The \texttt{Connection} class provides convenience methods for
			\texttt{insert}, \texttt{select}, \texttt{update}, \texttt{delete} and
			\texttt{truncate} statements

		\item For \texttt{select}, \texttt{update} an \texttt{delete} only simple
			equality comparisons (like \texttt{WHERE "aField" = 'aValue'}) are supported.
			For complex statements it is required to use the \texttt{QueryBuilder}.

	\end{itemize}

\end{frame}

% ------------------------------------------------------------------------------
% LTXE-SLIDE-START
% LTXE-SLIDE-UID:		4e6e640e-d37fcca8-bd0ca330-a70baaab
% LTXE-SLIDE-TITLE:		Feature: #75454 - PHP Library Doctrine DBAL (3)
% LTXE-SLIDE-REFERENCE:	!Feature-75454-DoctrineDBALForDatabaseConnections.rst
% ------------------------------------------------------------------------------
\begin{frame}[fragile]
	\frametitle{In-Depth Changes}
	\framesubtitle{PHP Library "Doctrine DBAL" (3)}

	% decrease font size for code listing
	\lstset{basicstyle=\tiny\ttfamily}

	\begin{itemize}

		\item The \texttt{ConnectionPool} class can be used like this:
			\begin{lstlisting}
				// Get a connection which can be used for muliple operations
				/** @var \TYPO3\CMS\Core\Database\Connecction $conn */
				$conn = GeneralUtility::makeInstance(ConnectionPool::class)->getConnectionForTable('aTable');
				$affectedRows = $conn->insert(
				  'aTable',
				  $fields, // Associative array of column/value pairs, automatically quoted & escaped
				);

				// Get a QueryBuilder, which should only be used a single time
				$query = GeneralUtility::makeInstance(ConnectionPool::class)->getQueryBuilderForTable('aTable);
				$query->select('*')
				  ->from('aTable)
				  ->where($query->expr()->eq('aField', $query->createNamedParameter($aValue)))
				  ->andWhere(
					$query->expr()->lte(
					  'anotherField',
					  $query->createNamedParameter($anotherValue)
					)
				  )
				$rows = $query->execute()->fetchAll();
			\end{lstlisting}
	\end{itemize}

\end{frame}

% ------------------------------------------------------------------------------
% LTXE-SLIDE-START
% LTXE-SLIDE-UID:		443c416c-63c0f980-719e38f1-1d117e50
% LTXE-SLIDE-TITLE:		Feature: #69439 - Enhance SQL query reduction in page tree in workspaces
% LTXE-SLIDE-REFERENCE:	!Feature-69439-EnhanceSQLQueryReductionInPageTreeInWorkspaces.rst
% ------------------------------------------------------------------------------
\begin{frame}[fragile]
	\frametitle{In-Depth Changes}
	\framesubtitle{Enhance SQL query reduction in page tree in workspaces}

	% decrease font size for code listing
	\lstset{basicstyle=\tiny\ttfamily}

	\begin{itemize}

		\item The process of determining whether a page has workspace versions
			can be extended by custom application code utilizing hooks

		\item This way, the meaning of having versions can be further modified
			by hooks

		\item For instance the default behavior of the TYPO3 core is to create
			a workspace version record on persisting the same record in the
			backend - without any actual changes to the data model

			\begin{lstlisting}
				$GLOBALS['TYPO3_CONF_VARS']['SC_OPTIONS']...
				  ...['TYPO3\\CMS\\Workspaces\\Service\\WorkspaceService']['hasPageRecordVersions'];

				$GLOBALS['TYPO3_CONF_VARS']['SC_OPTIONS']...
				  ...['TYPO3\\CMS\\Workspaces\\Service\\WorkspaceService']['fetchPagesWithVersionsInTable']
			\end{lstlisting}

	\end{itemize}

\end{frame}


% ------------------------------------------------------------------------------
% LTXE-SLIDE-START
% LTXE-SLIDE-UID:		630de169-c2699a8a-ee44dde5-cf39c2ca
% LTXE-SLIDE-TITLE:		Feature: #70056 - PHP Library "Guzzle" (1)
% LTXE-SLIDE-REFERENCE:	!Feature-70056-GuzzleForHttpRequests.rst
% ------------------------------------------------------------------------------
\begin{frame}[fragile]
	\frametitle{In-Depth Changes}
	\framesubtitle{PHP Library "Guzzle" (1)}

	\begin{itemize}

		\item The PHP library
			"\href{http://docs.guzzlephp.org}{Guzzle}"
			has been added via composer dependency to work as a feature rich
			solution for creating HTTP requests based on the PSR-7 interfaces
			already used within TYPO3

		\item Guzzle auto-detects available underlying adapters available on
			the system, like cURL or stream wrappers and chooses the best
			solution for the system

		\item A TYPO3-specific PHP class called
			\texttt{TYPO3\textbackslash
				CMS\textbackslash
				Core\textbackslash
				Http\textbackslash
				RequestFactory}\newline
			has been added as a simplified wrapper to access Guzzle clients

	\end{itemize}

\end{frame}


% ------------------------------------------------------------------------------
% LTXE-SLIDE-START
% LTXE-SLIDE-UID:		db2b31c1-47321ffe-3ed51ebc-5dd524a4
% LTXE-SLIDE-TITLE:		Feature: #70056 - PHP Library "Guzzle" (2)
% LTXE-SLIDE-REFERENCE:	!Feature-70056-GuzzleForHttpRequests.rst
% ------------------------------------------------------------------------------
\begin{frame}[fragile]
	\frametitle{In-Depth Changes}
	\framesubtitle{PHP Library "Guzzle" (2)}

	% decrease font size for code listing
	\lstset{basicstyle=\tiny\ttfamily}

	\begin{itemize}

		\item The \texttt{RequestFactory} class can be used like this:

			\begin{lstlisting}
				// Initiate RequestFactory

				/** @var \TYPO3\CMS\Core\Http\RequestFactory $requestFactory */
				$requestFactory = GeneralUtility::makeInstance(
				  \TYPO3\CMS\Core\Http\RequestFactory\RequestFactory::class);

				$uri = $additionalOptions = [
				  // additional headers for this specific request
				  'headers' => ['Cache-Control' => 'no-cache'],
				  'allow_redirects' => false,
				  'cookies' => true
				];

				// return a PSR-7 compliant response object
				$response = $requestFactory->request($url, 'GET', $additionalOptions);

				// get the content as a string on a successful request
				if ($response->getStatusCode() === 200) {
				  if ($response->getHeader('Content-Type') === 'text/html') {
				    $content = $response->getBody()->getContents();
				  }
				}
			\end{lstlisting}

	\end{itemize}

\end{frame}

% ------------------------------------------------------------------------------
