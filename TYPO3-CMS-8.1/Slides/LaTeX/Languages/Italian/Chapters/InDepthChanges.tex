% ------------------------------------------------------------------------------
% TYPO3 CMS 8.1 - What's New - Chapter "In-Depth Changes" (English Version)
%
% @author	Patrick Lobacher <patrick@lobacher.de> and Michael Schams <schams.net>
% @license	Creative Commons BY-NC-SA 3.0
% @link		http://typo3.org/download/release-notes/whats-new/
% @language	English
% ------------------------------------------------------------------------------
% LTXE-CHAPTER-UID:		e2881b46-7caaff58-912a10b9-7d517f6e
% LTXE-CHAPTER-NAME:	In-Depth Changes
% ------------------------------------------------------------------------------

\section{Modifiche rilevanti}
\begin{frame}[fragile]
	\frametitle{Modifiche rilevanti}

	\begin{center}\huge{Capitolo 3:}\end{center}
	\begin{center}\huge{\color{typo3darkgrey}\textbf{Modifiche rilevanti}}\end{center}

\end{frame}


% ------------------------------------------------------------------------------
% LTXE-SLIDE-START
% LTXE-SLIDE-UID:		9fab170d-0b5dd923-502cebc0-6912a809
% LTXE-SLIDE-ORIGIN:	861205b5-d3f4a55e-69279ab1-e18dcb15 English
% LTXE-SLIDE-TITLE:		Feature: #75454 - PHP Library Doctrine DBAL (1)
% LTXE-SLIDE-REFERENCE:	Feature-75454-DoctrineDBALForDatabaseConnections.rst
% ------------------------------------------------------------------------------
\begin{frame}[fragile]
	\frametitle{Modifiche rilevanti}
	\framesubtitle{Libreria PHP "Doctrine DBAL" (1)}

	\begin{itemize}

		\item La libreria PHP
			"\href{http://www.doctrine-project.org}{Doctrine DBAL}"
			è stata aggiunta come dipendenza composer per essere
			un potente strato di astrazione del database, con molte caratteristiche di 
			astrazione del database, l'introspezione dello schema e la gestione di esso 
			in TYPO3 CMS

		\item Una specifica classe TYPO3 denominata
			\texttt{TYPO3\textbackslash
				CMS\textbackslash
				Core\textbackslash
				Database\textbackslash
				ConnectionPool}\newline
			è stata aggiunta come manager per le connessioni al database

		\item Tutte le connessioni configurate in
			\texttt{\$GLOBALS['TYPO3\_CONF\_VARS']['DB']['Connections']}\newline
			sono accessibili attraverso questo manager, consentendo l'utilizzo in
			parallelo di più sistemi di database.

	\end{itemize}

\end{frame}

% ------------------------------------------------------------------------------
% LTXE-SLIDE-START
% LTXE-SLIDE-UID:		ef36a016-557557b6-0beefb6d-6133cd3b
% LTXE-SLIDE-ORIGIN:	988bc43f-175df2eb-d7158156-f8f36703 English
% LTXE-SLIDE-TITLE:		Feature: #75454 - PHP Library Doctrine DBAL (2)
% LTXE-SLIDE-REFERENCE:	Feature-75454-DoctrineDBALForDatabaseConnections.rst
% ------------------------------------------------------------------------------
\begin{frame}[fragile]
	\frametitle{Modifiche rilevanti}
	\framesubtitle{Libreria PHP "Doctrine DBAL" (2)}

	\begin{itemize}

		\item L'utilizzo delle opzioni di astrazione del database e delle istruzioni SQL del
		QueryBuilder, fornite in fase di costruzione, permetteranno di essere compatibili con
		diversi DBMS, per quanto possibile.

		\item La vecchia opzione \texttt{\$GLOBALS['TYPO3\_CONF\_VARS']['DB']} è stata rimossa
			e/o migrata alla nuova opzione Doctrine-compliant

		\item La classe \texttt{Connection} fornisce metodi per le operazioni di 
			\texttt{insert}, \texttt{select}, \texttt{update}, \texttt{delete} e
			\texttt{truncate}

		\item Per \texttt{select}, \texttt{update} e \texttt{delete} sono supportate solo
			semplici operazioni di comparazione (come \texttt{WHERE "aField" = 'aValue'}).
			Per istruzioni più complesse è necessario utilizzare il \texttt{QueryBuilder}.

	\end{itemize}

\end{frame}

% ------------------------------------------------------------------------------
% LTXE-SLIDE-START
% LTXE-SLIDE-UID:		9eee15d8-a6b0d850-dd4ec287-44944a3c
% LTXE-SLIDE-ORIGIN:	4e6e640e-d37fcca8-bd0ca330-a70baaab English
% LTXE-SLIDE-TITLE:		Feature: #75454 - PHP Library Doctrine DBAL (3)
% LTXE-SLIDE-REFERENCE:	!Feature-75454-DoctrineDBALForDatabaseConnections.rst
% ------------------------------------------------------------------------------
\begin{frame}[fragile]
	\frametitle{Modifiche rilevanti}
	\framesubtitle{Libreria PHP "Doctrine DBAL" (3)}

	% decrease font size for code listing
	\lstset{basicstyle=\tiny\ttfamily}

	\begin{itemize}

		\item La classe \texttt{ConnectionPool} può essere utilizzata come segue:
			\begin{lstlisting}
				// Ottiene una connessione utilizzabile per varie operazioni
				/** @var \TYPO3\CMS\Core\Database\Connecction $conn */
				$conn = GeneralUtility::makeInstance(ConnectionPool::class)->getConnectionForTable('aTable');
				$affectedRows = $conn->insert(
				  'aTable',
				  $fields, // array associativo di coppie colonna/valore, automaticamente quoted & escaped
				);

				// Ottiene un QueryBuilder che dovrebbe essere usato una sola volta
				$query = GeneralUtility::makeInstance(ConnectionPool::class)->getQueryBuilderForTable('aTable);
				$query->select('*')
				  ->from('aTable)
				  ->where($query->expr()->eq('aField', $query->createNamedParameter($aValue)))
				  ->andWhere(
					$query->expr()->lte(
					  'anotherField',
					  $query->createNamedParameter($anotherValue)
					)
				  )
				$rows = $query->execute()->fetchAll();
			\end{lstlisting}
	\end{itemize}

\end{frame}

% ------------------------------------------------------------------------------
% LTXE-SLIDE-START
% LTXE-SLIDE-UID:		ef747304-6f7b38c5-fe6108c4-45b3762f
% LTXE-SLIDE-ORIGIN:	443c416c-63c0f980-719e38f1-1d117e50 English
% LTXE-SLIDE-TITLE:		Feature: #69439 - Enhance SQL query reduction in page tree in workspaces
% LTXE-SLIDE-REFERENCE:	!Feature-69439-EnhanceSQLQueryReductionInPageTreeInWorkspaces.rst
% ------------------------------------------------------------------------------
\begin{frame}[fragile]
	\frametitle{Modifiche rilevanti}
	\framesubtitle{Migliorare la riduzione delle query SQL nei workspace dell'albero delle pagine}

	% decrease font size for code listing
	\lstset{basicstyle=\tiny\ttfamily}

	\begin{itemize}

		\item Il processo per determinare se una pagina ha versioni in workspace può essere ampliato
			da codice custom tramite degli hook.

		\item In questo modo, la segnalazione di più versioni può essere ulteriormente modificata
			via hook.

		\item Per esempio il comportamento di base del core di TYPO3 è quello di creare
			un record di versione workspace sulla persistenza dello stesso record nel backend
			- senza nessun cambiamento al modello dei dati.

			\begin{lstlisting}
				$GLOBALS['TYPO3_CONF_VARS']['SC_OPTIONS']...
				  ...['TYPO3\\CMS\\Workspaces\\Service\\WorkspaceService']['hasPageRecordVersions'];

				$GLOBALS['TYPO3_CONF_VARS']['SC_OPTIONS']...
				  ...['TYPO3\\CMS\\Workspaces\\Service\\WorkspaceService']['fetchPagesWithVersionsInTable']
			\end{lstlisting}

	\end{itemize}

\end{frame}


% ------------------------------------------------------------------------------
% LTXE-SLIDE-START
% LTXE-SLIDE-UID:		3152eb18-4814dc82-01a4126b-b95615fc
% LTXE-SLIDE-ORIGIN:	630de169-c2699a8a-ee44dde5-cf39c2ca English
% LTXE-SLIDE-TITLE:		Feature: #70056 - PHP Library "Guzzle" (1)
% LTXE-SLIDE-REFERENCE:	!Feature-70056-GuzzleForHttpRequests.rst
% ------------------------------------------------------------------------------
\begin{frame}[fragile]
	\frametitle{Modifiche rilevanti}
	\framesubtitle{Libreria PHP "Guzzle" (1)}

	\begin{itemize}

		\item La libreria PHP
			"\href{http://docs.guzzlephp.org}{Guzzle}"
			è stata aggiunta come dipendenza di composer per rispondere con una soluzione
			ricca di funzionalità per la creazione di richieste HTTP sulla base di interfacce
			PSR-7 già utilizzate all'interno di TYPO3.

		\item Guzzle rileva automaticamente la presenza di adattatori disponibili nel sistema,
			come cURL o stream wrappers e sceglie la soluzione più adatta per il sistema.

		\item Una specifica classe di TYPO3 chiamata
			\texttt{TYPO3\textbackslash
				CMS\textbackslash
				Core\textbackslash
				Http\textbackslash
				RequestFactory}\newline
			è stata aggiunta per semplificare l'accesso ai client Guzzle

	\end{itemize}

\end{frame}


% ------------------------------------------------------------------------------
% LTXE-SLIDE-START
% LTXE-SLIDE-UID:		bee658a0-5be02467-17520607-e5c46778
% LTXE-SLIDE-ORIGIN:	db2b31c1-47321ffe-3ed51ebc-5dd524a4 English
% LTXE-SLIDE-TITLE:		Feature: #70056 - PHP Library "Guzzle" (2)
% LTXE-SLIDE-REFERENCE:	!Feature-70056-GuzzleForHttpRequests.rst
% ------------------------------------------------------------------------------
\begin{frame}[fragile]
	\frametitle{Modifiche rilevanti}
	\framesubtitle{Libreria PHP "Guzzle" (2)}

	% decrease font size for code listing
	\lstset{basicstyle=\tiny\ttfamily}

	\begin{itemize}

		\item La classe \texttt{RequestFactory} può essere utilizzata come segue:

			\begin{lstlisting}
				// Inizializza RequestFactory

				/** @var \TYPO3\CMS\Core\Http\RequestFactory $requestFactory */
				$requestFactory = GeneralUtility::makeInstance(
				  \TYPO3\CMS\Core\Http\RequestFactory\RequestFactory::class);

				$uri = $additionalOptions = [
				  // additional headers for this specific request
				  'headers' => ['Cache-Control' => 'no-cache'],
				  'allow_redirects' => false,
				  'cookies' => true
				];

				// restituisce un oggetto response compatibile PSR-7
				$response = $requestFactory->request($url, 'GET', $additionalOptions);

				// ottiene il contenuto come stringa ad una richiesta andata a buon fine
				if ($response->getStatusCode() === 200) {
				  if ($response->getHeader('Content-Type') === 'text/html') {
				    $content = $response->getBody()->getContents();
				  }
				}
			\end{lstlisting}

	\end{itemize}

\end{frame}

% ------------------------------------------------------------------------------
