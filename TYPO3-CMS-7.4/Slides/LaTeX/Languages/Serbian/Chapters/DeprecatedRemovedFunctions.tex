% ------------------------------------------------------------------------------
% TYPO3 CMS 7.4 - What's New - Chapter "Deprecated Functions" (English Version)
%
% @author	Michael Schams <schams.net>
% @license	Creative Commons BY-NC-SA 3.0
% @link		http://typo3.org/download/release-notes/whats-new/
% @language	English
% ------------------------------------------------------------------------------
% LTXE-CHAPTER-UID:		052872e9-9f10b97b-dcf0f34b-0bb90d27
% LTXE-CHAPTER-NAME:	Deprecated Functions
% ------------------------------------------------------------------------------

\section{Zastarele/izbacene funkcije}
\begin{frame}[fragile]
	\frametitle{Zastarele/izbacene funkcije}

	\begin{center}\huge{Chapter 6:}\end{center}
	\begin{center}\huge{\color{typo3darkgrey}\textbf{Zastarele/izbacene funkcije}}\end{center}

\end{frame}

% ------------------------------------------------------------------------------
% LTXE-SLIDE-START
% LTXE-SLIDE-UID:		56ef0f1a-1fb49e07-0e3c29eb-af85a2ce
% LTXE-SLIDE-ORIGIN:	685b0aeb-6654ab27-f8128f62-3454735b English
% LTXE-SLIDE-ORIGIN:	ebe40c2e-32da4e43-67f18e6c-866b0496 German
% LTXE-SLIDE-TITLE:		Deprecation: #67991 - Removed EXT:cms (1)
% LTXE-SLIDE-REFERENCE:	Deprecation-67991-RemovedExtCms.rst
% ------------------------------------------------------------------------------

\begin{frame}[fragile]
	\frametitle{Zastarele/izbacene funkcije}
	\framesubtitle{Uklonjeno sistemsko prosirenje \texttt{cms} (1)}

	% decrease font size for code listing
	\lstset{basicstyle=\tiny\ttfamily}

	\begin{itemize}

		\item Sistemsko prosirenje \texttt{cms} je uklonjeno

		\item Developeri treba da provere u fajlu \texttt{ext\_emconf.php} sta je neophodno da bi sistem radio

			\begin{lstlisting}
				[...]
				'constraints' => array(
				  'depends' => array(
				    // 'cms' => ' ... ',           <= WRONG!
				    'typo3' => '7.0.0-7.99.99',
				  ),
				),
				[...]
			\end{lstlisting}

		\item Vecina funkcionalnosti je migrirana u sistemsku ekstenziju \texttt{frontend}
			(ovo moze zahtevati da se osveze jezicki fajlovi, pogledati sledeci slajd)

	\end{itemize}

\end{frame}

% ------------------------------------------------------------------------------
% LTXE-SLIDE-START
% LTXE-SLIDE-UID:		548be795-81ffeadc-83a006ef-266a2b51
% LTXE-SLIDE-ORIGIN:	ff99fb62-81ad4d10-2367bc34-4985628d English
% LTXE-SLIDE-ORIGIN:	4e4332da-8e6c67f1-0496866b-ebe40c2e German
% LTXE-SLIDE-TITLE:		Deprecation: #67991 - Removed EXT:cms (2)
% LTXE-SLIDE-REFERENCE:	Deprecation-67991-RemovedExtCms.rst
% ------------------------------------------------------------------------------

\begin{frame}[fragile]
	\frametitle{Zastarele/izbacene funkcije}
	\framesubtitle{Uklonjeno sistemsko prosirenje \texttt{cms} (2)}

	% decrease font size for code listing
	\lstset{basicstyle=\tiny\ttfamily}

	\begin{itemize}

		\item Neophodno unapredjenje referenci ka jezickim fajlovima:

			% hint for translators:
			% please translate words "old" and "new" in listing below,
			% but DO NOT USE special characters, like German umlauts, etc.
			% (special characters inside "lstlisting" break LaTeX).

			\begin{lstlisting}
				Staro: typo3/sysext/cms/web_info/locallang.xlf
				Novo: typo3/sysext/frontend/Resources/Private/Language/locallang_webinfo.xlf
			\end{lstlisting}
			\vspace{-0.3cm}
			\begin{lstlisting}
				Staro: typo3/sysext/cms/locallang_ttc.xlf
				Novo: typo3/sysext/frontend/Resources/Private/Language/locallang_ttc.xlf
			\end{lstlisting}
			\vspace{-0.3cm}
			\begin{lstlisting}
				Staro: typo3/sysext/cms/locallang_tca.xlf
				Novo: typo3/sysext/frontend/Resources/Private/Language/locallang_tca.xlf
			\end{lstlisting}
			\vspace{-0.3cm}
			\begin{lstlisting}
				Staro: typo3/sysext/cms/layout/locallang_db_new_content_el.xlf
				Novo: typo3/sysext/backend/Resources/Private/Language/locallang_db_new_content_el.xlf
			\end{lstlisting}
			\vspace{-0.3cm}
			\begin{lstlisting}
				Staro: typo3/sysext/cms/layout/locallang.xlf
				Novo: typo3/sysext/backend/Resources/Private/Language/locallang_layout.xlf
			\end{lstlisting}
			\vspace{-0.3cm}
			\begin{lstlisting}
				Staro: typo3/sysext/cms/layout/locallang_mod.xlf
				Novo: typo3/sysext/backend/Resources/Private/Language/locallang_mod.xlf
			\end{lstlisting}
			\vspace{-0.3cm}
			\begin{lstlisting}
				Staro: typo3/sysext/cms/locallang_csh_webinfo.xlf
				Novo: typo3/sysext/frontend/Resources/Private/Language/locallang_csh_webinfo.xlf
			\end{lstlisting}
			\vspace{-0.3cm}
			\begin{lstlisting}
				Staro: typo3/sysext/cms/locallang_csh_weblayout.xlf
				Novo: typo3/sysext/frontend/Resources/Private/Language/locallang_csh_weblayout.xlf
			\end{lstlisting}

	\end{itemize}

\end{frame}

% ------------------------------------------------------------------------------
% LTXE-SLIDE-START
% LTXE-SLIDE-UID:		8279bbe1-d315aa61-30123ee8-e432777f
% LTXE-SLIDE-ORIGIN:	e1a896f5-538a7c07-e462c6f0-db80381c English
% LTXE-SLIDE-ORIGIN:	ae6fc887-9aeff493-158130de-3cdbf052 German
% LTXE-SLIDE-TITLE:		Deprecation: #68074 - Deprecate getPageRenderer() methods
% LTXE-SLIDE-REFERENCE:	Deprecation-68074-DeprecateGetPageRenderer.rst
% ------------------------------------------------------------------------------

\begin{frame}[fragile]
	\frametitle{Zastarele/izbacene funkcije}
	\framesubtitle{Zastarele PageRenderer metode}

	% decrease font size for code listing
	\lstset{basicstyle=\tiny\ttfamily}

	\begin{itemize}
		\item Sledece \texttt{PageRenderer} metode su oznacene kao \textbf{zastarele}:

			\begin{lstlisting}
				TYPO3\CMS\Backend\Controller\BackendController::getPageRenderer()
				TYPO3\CMS\Backend\Template\DocumentTemplate::getPageRenderer()
				TYPO3\CMS\Backend\Template\FrontendDocumentTemplate::getPageRenderer()
				TYPO3\CMS\Frontend\Controller\TypoScriptFrontendController::getPageRenderer()
			\end{lstlisting}

		\item Umesto njih treba koristiti sledeci kod kako bi se doslo do instance PageRenderer-a:

			\begin{lstlisting}
				\TYPO3\CMS\Core\Utility\GeneralUtility::makeInstance(\TYPO3\CMS\Core\Page\PageRenderer::class)
			\end{lstlisting}

	\end{itemize}

\end{frame}

% ------------------------------------------------------------------------------
% LTXE-SLIDE-START
% LTXE-SLIDE-UID:		30114181-f63fc7a6-d0a9fa94-840c75c3
% LTXE-SLIDE-ORIGIN:	440429fa-30e8d4b1-4784ea6f-86bc343e English
% LTXE-SLIDE-ORIGIN:	a58c2f4f-3eb806e5-d256078a-70ed73c0 German
% LTXE-SLIDE-TITLE:		Deprecation #68098 and #68122
% LTXE-SLIDE-REFERENCE:	Deprecation-68098-GeneralUtilityMethods.rst
% LTXE-SLIDE-REFERENCE:	Deprecation-68122-GeneralUtilityReadLLfile.rst
% ------------------------------------------------------------------------------
% Deprecation: #68098 - Deprecate GeneralUtility methods
% Deprecation: #68122 - Deprecate GeneralUtility::readLLfile

\begin{frame}[fragile]
	\frametitle{Zastarele/izbacene funkcije}
	\framesubtitle{Zastarele \texttt{GeneralUtility} metode}

	% decrease font size for code listing
	\lstset{basicstyle=\tiny\ttfamily}

	\begin{itemize}
		\item Sledece \texttt{GeneralUtility} metode su oznacene kao \textbf{zastarele}
			i bice uklonjene u TYPO3 CMS verziji 8:

			\begin{lstlisting}
				GeneralUtility::modifyHTMLColor()
				GeneralUtility::modifyHTMLColorAll()
				GeneralUtility::isBrokenEmailEnvironment()
				GeneralUtility::normalizeMailAddress()
				GeneralUtility::formatForTextarea()
				GeneralUtility::getThisUrl()
				GeneralUtility::cleanOutputBuffers()
				GeneralUtility::readLLfile()
			\end{lstlisting}

		\item Metoda \texttt{readLLfile()} moze se zameniti sledecim kodom:

			\begin{lstlisting}
				/** @var $languageFactory \TYPO3\CMS\Core\Localization\LocalizationFactory */
				$languageFactory = GeneralUtility::makeInstance(
				  \TYPO3\CMS\Core\Localization\LocalizationFactory::class
				);
				$languageFactory->getParsedData($fileToParse, $language, $renderCharset, $errorMode);
			\end{lstlisting}

	\end{itemize}

\end{frame}

% ------------------------------------------------------------------------------
% LTXE-SLIDE-START
% LTXE-SLIDE-UID:		fe4aa37a-883dd6ad-71db99a8-13e4479e
% LTXE-SLIDE-ORIGIN:	0d77c7a7-fd5da718-210555bf-80e688f0 English
% LTXE-SLIDE-ORIGIN:	b0c4d95f-d9fd698f-2d9dfdec-42aaf1af German
% LTXE-SLIDE-TITLE:		Breaking: #39721 - Prototype.js and Scriptaculous removed
% LTXE-SLIDE-REFERENCE:	Breaking-39721-PrototypejsAndScriptaculousRemoved.rst
% ------------------------------------------------------------------------------

\begin{frame}[fragile]
	\frametitle{Zastarele/izbacene funkcije}
	\framesubtitle{Uklonjene JavaScript biblioteke}

	\begin{itemize}

		\item JavaScript biblioteke \texttt{prototype.js} i \texttt{scriptaculous} su uklonjene.
			Kao posledica, sledece TypoScript osobine vise nemaju nikakvu funkcionalnost:

			\begin{itemize}
				\item \texttt{page.javascriptLibs.Prototype}
				\item \texttt{page.javascriptLibs.Scriptaculous.*}
			\end{itemize}

		\item Koriscenje sledecih metoda u ViewHelper \texttt{be.container} rezultirace greskom:
			\begin{lstlisting}
				<f:be.container loadPrototype="false" loadScriptaculous="false" scriptaculousModule="someModule,someOtherModule">
			\end{lstlisting}

		\item Kao zamena trebalo bi koristiti \texttt{jQuery} i \texttt{RequireJS}\newline
			(koji su vec ucitani u administratorski interfejs kao podrazumevano podesavanje)

	\end{itemize}

\end{frame}

% ------------------------------------------------------------------------------
% LTXE-SLIDE-START
% LTXE-SLIDE-UID:		ba1ff6bd-215eedcd-4b3bf3d4-2455ea18
% LTXE-SLIDE-ORIGIN:	4bb27612-517a802f-3b7a7dae-900630fc English
% LTXE-SLIDE-ORIGIN:	1375cbd0-05287ba0-3235df98-19473bf0 German
% LTXE-SLIDE-TITLE:		Deprecation: #68183 - typo3/mod.php
% LTXE-SLIDE-REFERENCE:	Deprecation-68183-Typo3modphp.rst
% ------------------------------------------------------------------------------

\begin{frame}[fragile]
	\frametitle{Zastarele/izbacene funkcije}
	\framesubtitle{Zastareli: \texttt{init.php}, \texttt{mod.php} i \texttt{ajax.php}}

	% decrease font size for code listing
	%\lstset{basicstyle=\tiny\ttfamily}

	\begin{itemize}

		\item Kako bi se ocistio sadrzaj foldera \texttt{typo3}, sledeci fajlovi su oznaceni kao \textbf{zastareli}: \texttt{init.php}, \texttt{mod.php} i \texttt{ajax.php}

		\item Sledeci kod se moze iskoristiti za inicijalizovanje ulaznih tacaka:

			\begin{lstlisting}
				call_user_func(function() {
				  $classLoader = require __DIR__ . '/vendor/autoload.php';
				  (new \TYPO3\CMS\Backend\Http\Application($classLoader))->run();
				});
			\end{lstlisting}

		\item Sledeci poziv metode moze se iskoristiti kako bi se pristupilo \texttt{mod.php}:

			\begin{lstlisting}
				BackendUtility::getModuleUrl()
			\end{lstlisting}

	\end{itemize}

\end{frame}

% ------------------------------------------------------------------------------
% LTXE-SLIDE-START
% LTXE-SLIDE-UID:		6dd9aad3-af97893b-670c170e-88998d47
% LTXE-SLIDE-ORIGIN:	a09aedae-f4adde91-f6b0d3a8-ada31684 English
% LTXE-SLIDE-ORIGIN:	749d4c59-1ec6087a-45020fbd-2f7b8646 German
% LTXE-SLIDE-TITLE:		Deprecation: #67737 - TCA: Drop additional palette
% LTXE-SLIDE-REFERENCE:	Deprecation-67737-TcaDropAdditionalPalette.rst
% ------------------------------------------------------------------------------

\begin{frame}[fragile]
	\frametitle{Zastarele/izbacene funkcije}
	\framesubtitle{TCA: Uklonjena dodatna paleta}

	% decrease font size for code listing
	\lstset{basicstyle=\tiny\ttfamily}

	\begin{itemize}

		\item String \texttt{showitem} TCA kljuca \texttt{types} dozvoljavao je developerima da definisu dodatnu paletu

		\item Ovo je sada uklonjeno i migrirano u normalnu paletu

		\item Pre:

			\begin{lstlisting}
				'types' => array(
				  'aType' => array(
				    'showitem' => 'aField;aLabel;anAdditionalPaletteName',
				  ),
				),
			\end{lstlisting}

		\item Sada:

			\begin{lstlisting}
				'types' => array(
				  'aType' => array(
				    'showitem' => 'aField;aLabel, --palette--;;anAdditionalPaletteName',
				  ),
				),
			\end{lstlisting}

	\end{itemize}

\end{frame}

% ------------------------------------------------------------------------------
% LTXE-SLIDE-START
% LTXE-SLIDE-UID:		12b79593-b3685a24-8d09fa16-b302b225
% LTXE-SLIDE-ORIGIN:	e27efa5c-c3999b32-6715dd2f-da8acf1f English
% LTXE-SLIDE-ORIGIN:	6398e32e-7f467ac7-463d9b47-bc0fd47c German
% LTXE-SLIDE-TITLE:		Breaking: #67577, #67646 and #67824
% LTXE-SLIDE-REFERENCE: Breaking-67577-RteEnabledFlagHandling.rst
% LTXE-SLIDE-REFERENCE: Breaking-67646-LibraryInclusionInFrontend.rst
% LTXE-SLIDE-REFERENCE: Breaking-67824-Typo3ExtFolderRemoved.rst
% ------------------------------------------------------------------------------
% Breaking: #67577 - rte_enabled and flag handling
% Breaking: #67646 - PHP library inclusion in frontend removed
% Breaking: #67824 - typo3/ext folder removed

\begin{frame}[fragile]
	\frametitle{Zastarele/izbacene funkcije}
	\framesubtitle{Razno (1)}

	\begin{itemize}

		\item cObjects "Text" i "Text with Images" u proslosti je imao je cekboks "RTE enabled".
			Ovo je uklonjeno, ukljucujuci i TCA \texttt{opciju}.

		\item Uklonjene su sledece TypoScript opcije za ukljucivanje PHP fajlova:

			\begin{itemize}
				\item \texttt{config.includeLibrary}
				\item \texttt{config.includeLibs}
			\end{itemize}

		\item Uklonjen je direktorijum \texttt{typo3/ext}\newline
			\small
				(ali ne i opcija za koriscenje globalnih prosirenja: direktorijum se moze kreirati rucno)
			\normalsize

	\end{itemize}

\end{frame}

% ------------------------------------------------------------------------------
% LTXE-SLIDE-START
% LTXE-SLIDE-UID:		d588852b-a1fae0a0-e3760df5-270759ff
% LTXE-SLIDE-ORIGIN:	7b47e35b-9ae0fd9a-13d2db50-ebe93a84 English
% LTXE-SLIDE-ORIGIN:	463d9b47-bc0fd47c-6398e32e-7f467ac7 German
% LTXE-SLIDE-TITLE:		Breaking: #68001
% LTXE-SLIDE-REFERENCE: Breaking-68001-RemovedExtJSCoreAndExtJSAdapters.rst
% ------------------------------------------------------------------------------
% Breaking: #68001 - Removed ExtJS Core and ExtJS Adapters
% Breaking: #68020 - Dropped DisableBigButtons

\begin{frame}[fragile]
	\frametitle{Zastarele/izbacene funkcije}
	\framesubtitle{Razno (2)}

	\begin{itemize}

		\item ExtCore (ExtJS adapter) je uklonjen, ukljucujuci i sledece TypoScript opcije:

			\begin{itemize}
				\item \texttt{page.javascriptLibs.ExtCore.*}
				\item \texttt{page.javascriptLibs.ExtJs.*}
			\end{itemize}

			Ovo takodje ukljucuje i opcije u \texttt{<f:be.container>} ViewHelper

		\item Uklonjeni su takozvani "BigButtons" ("Edit Page Properties", "Move Page",...)
			ukljucujuci i njihovo TSconfig podesavanje \texttt{mod.we\_layout.disableBigButtons}

	\end{itemize}

\end{frame}

% ------------------------------------------------------------------------------
% LTXE-SLIDE-START
% LTXE-SLIDE-UID:		4f691d74-d272f481-910f9ad2-58ff98dd
% LTXE-SLIDE-ORIGIN:	8052d34e-ccc2def0-c43053a0-c3c8eba3 English
% LTXE-SLIDE-ORIGIN:	d3a774ba-c1fd32c6-099aa85f-96b6a2a2 German
% LTXE-SLIDE-TITLE:		Breaking: #68020, #68131, #65790 and #67506
% LTXE-SLIDE-REFERENCE: Breaking-68020-DroppedDisableBigButtons.rst
% LTXE-SLIDE-REFERENCE: Breaking-68131-StreamlineErrorAndExceptionHandling.rst
% LTXE-SLIDE-REFERENCE: Deprecation-65790-PagesStoragePidDeprecated.rst
% LTXE-SLIDE-REFERENCE: Deprecation-67506-DeprecateIconUtilitygetIcon.rst
% ------------------------------------------------------------------------------
% Breaking: #68131 - Streamline error and exception handling
% Deprecation: #65790 - Remove pages.storage_pid and logic
% Deprecation: #67506 - Deprecate IconUtility::getIcon

\begin{frame}[fragile]
	\frametitle{Zastarele/izbacene funkcije}
	\framesubtitle{Razno (3)}

	\begin{itemize}

		\item Upravljanje greskama i izuzetcima se vise ne moze konfigurisati u prosirenjima (na primer pregaziti u
			\texttt{ext\_localconf.php}), vec samo u fajlovima \texttt{LocalConfiguration.php} ili
			\texttt{AdditionalConfiguration.php}

		\item Uklonjeno je polje "General Record Storage Page", koje sadrzi storage PID strane.
			Storage PID se sada mora konfigurisati koriscenjem TypoScript-a ili FlexForm-i.

		\item Funkcija \texttt{IconUtility::getIcon()} oznacena je kao \textbf{zastarela} (umesto nje koristiti metodu \texttt{IconUtility::getSpriteIconForRecord()})

	\end{itemize}

\end{frame}

% ------------------------------------------------------------------------------
