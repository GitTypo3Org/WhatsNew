% ------------------------------------------------------------------------------
% TYPO3 CMS 7.4 - What's New - Chapter "TypoScript" (English Version)
%
% @author	Michael Schams <schams.net>
% @license	Creative Commons BY-NC-SA 3.0
% @link		http://typo3.org/download/release-notes/whats-new/
% @language	English
% ------------------------------------------------------------------------------
% LTXE-CHAPTER-UID:		693ffeb9-5ce46f4e-8c0b7d7c-be9f3ee3
% LTXE-CHAPTER-NAME:	TypoScript
% ------------------------------------------------------------------------------

\section{TSconfig i TypoScript}
\begin{frame}[fragile]
	\frametitle{TSconfig i TypoScript}

	\begin{center}\huge{Poglavlje 2:}\end{center}
	\begin{center}\huge{\color{typo3darkgrey}\textbf{TSconfig i TypoScript}}\end{center}

\end{frame}

% ------------------------------------------------------------------------------
% LTXE-SLIDE-START
% LTXE-SLIDE-UID:		4972a68b-20d164d9-57422317-c75b5283
% LTXE-SLIDE-ORIGIN:	69805fcd-5bf6738b-3d31552e-d0add3ee English
% LTXE-SLIDE-ORIGIN:	78394814-b88d7d32-54dd1c41-022129d0 German
% LTXE-SLIDE-TITLE:		Feature: #61903 - PageTS dataprovider for backend layouts (1)
% LTXE-SLIDE-REFERENCE:	Feature-61903-PageTSDataproviderForBackendLayouts.rst
% ------------------------------------------------------------------------------
\begin{frame}[fragile]
	\frametitle{TSconfig i TypoScript}
	\framesubtitle{Data-Provider za sablone administratorskog interfejsa (1)}

	% decrease font size for code listing
	\lstset{basicstyle=\tiny\ttfamily}

	\begin{itemize}
		\item Sada je moguce definisati sablone administratorskog interfejsa preko TSconfig strane i takodje ih sacuvati u fajlovima. Na primer:

		\begin{lstlisting}
			mod {
			  web_layout {
			    BackendLayouts {
			      exampleKey {
			        title = Example
			        config {
			          backend_layout {
			            colCount = 1
			            rowCount = 2
			            rows {
			              1 {
			                columns {
			                  1 {
			                    name = LLL:EXT:frontend/ ... /locallang_ttc.xlf:colPos.I.3
			                    colPos = 3
			                    colspan = 1
			                  }
			                }
			              }
			[...]
		\end{lstlisting}

	\end{itemize}

\end{frame}

% ------------------------------------------------------------------------------
% LTXE-SLIDE-START
% LTXE-SLIDE-UID:		b338b33b-911d1af9-ab7dde2d-ae463dd5
% LTXE-SLIDE-ORIGIN:	0691efc7-c5a339e2-1753851e-9a62d8ed English
% LTXE-SLIDE-ORIGIN:	b88d4814-7d327839-10221c41-54dd29d0 German
% LTXE-SLIDE-TITLE:		Feature: #61903 - PageTS dataprovider for backend layouts (2)
% LTXE-SLIDE-REFERENCE:	Feature-61903-PageTSDataproviderForBackendLayouts.rst
% ------------------------------------------------------------------------------
\begin{frame}[fragile]
	\frametitle{TSconfig i TypoScript}
	\framesubtitle{Data-Provider za sablone administratorskog interfejsa (2)}

	% decrease font size for code listing
	\lstset{basicstyle=\tiny\ttfamily}

	\begin{itemize}
		\item \smaller(nastavak)\normalsize
		\begin{lstlisting}
			[...]
			              2 {
			                columns {
			                  1 {
			                    name = Main
			                    colPos = 0
			                    colspan = 1
			                  }
			                }
			              }
			            }
			          }
			        }
			        icon = EXT:example_extension/Resources/Public/Images/BackendLayouts/default.gif
			      }
			    }
			  }
			}
		\end{lstlisting}
	\end{itemize}

\end{frame}

% ------------------------------------------------------------------------------
% LTXE-SLIDE-START
% LTXE-SLIDE-UID:		c77c9b68-f2906458-54b06ae2-c1ebf270
% LTXE-SLIDE-ORIGIN:	53593b48-60766071-136bf56b-f63caf39 English
% LTXE-SLIDE-ORIGIN:	9fc7749d-aa1d6042-b1b7699d-e9aee9ed German
% LTXE-SLIDE-TITLE:		Feature: #67360 - Custom attribute name and multiple values for meta tags
% LTXE-SLIDE-REFERENCE:	Feature-67360-CustomAttributeNameAndMultipleValuesForMetaTags.rst
% ------------------------------------------------------------------------------

\begin{frame}[fragile]
	\frametitle{TSconfig i TypoScript}
	\framesubtitle{Prosireni meta tagovi}

	% decrease font size for code listing
	\lstset{basicstyle=\tiny\ttfamily}

	\begin{itemize}

		% Note for translators: make sure, the following bullet point is only ONE line long.
		% If it wraps and uses two lines, the code example below does not fit on the page.
		% If not avoidable, split the content and create a second slide for the code.

		\item Opcija \texttt{page.meta} sada podrzava \href{http://ogp.me}{Open Graph} atribute

			\begin{lstlisting}
				page {
				  meta {
				    X-UA-Compatible = IE=edge,chrome=1
				    X-UA-Compatible.attribute = http-equiv
				    keywords = TYPO3
				    # <meta property="og:site_name" content="TYPO3" />
				    og:site_name = TYPO3
				    og:site_name.attribute = property
				    description = Inspiring people to share
				    og:description = Inspiring people to share
				    og:description.attribute = property
				    og:locale = en_GB
				    og:locale.attribute = property
				    og:locale:alternate {
				      attribute = property
				      value.1 = fr_FR
				      value.2 = de_DE
				    }
				    refresh = 5; url=http://example.com/
				    refresh.attribute = http-equiv
				  }
				}
			\end{lstlisting}

	\end{itemize}

\end{frame}

% ------------------------------------------------------------------------------
% LTXE-SLIDE-START
% LTXE-SLIDE-UID:		2034743c-26f9bf12-fcaaaec5-293b53c4
% LTXE-SLIDE-ORIGIN:	5df1ea93-74ddf5d4-c8acd647-7f47e896 English
% LTXE-SLIDE-ORIGIN:	c38a626f-40c7bda9-523650fb-6829234e German
% LTXE-SLIDE-TITLE:		Feature: #68191 - TypoScript .select option languageField is active by default
% LTXE-SLIDE-REFERENCE:	Feature-68191-TypoScriptSelectOptionLanguageFieldIsActiveByDefault.rst
% ------------------------------------------------------------------------------

\begin{frame}[fragile]
	\frametitle{TSconfig i TypoScript}
	\framesubtitle{\texttt{languageField} kao podrazumevano podesavanje}

	% decrease font size for code listing
	\lstset{basicstyle=\tiny\ttfamily}

	\begin{itemize}

		\item TypoScript opcija \texttt{select} (koristi se na primer kod cObject CONTENT) bila je potrebna da bi se \texttt{languageField} eksplicitno setovao

		\item Ovo vise nije obavezno posto se podesavanje sada automatski dohvata iz strukture informacija TCA

			% reminder for translators: special characters such as German umlauts are not
			% allowed inside "lstlisting". They break the LaTeX code. Be careful when you
			% want to translate the sentence "the following line is not required anymore"
			% below (feel free to leave it in English).

			\begin{lstlisting}
				config.sys_language_uid = 2
				page.10 = CONTENT
				page.10 {
				  table = tt_content
				  select.where = colPos=0

				  # the following line is not required anymore:
				  #select.languageField = sys_language_uid

				  renderObj = TEXT
				  renderObj.field = header
				  renderObj.htmlSpecialChars = 1
				}
			\end{lstlisting}

	\end{itemize}

\end{frame}

% ------------------------------------------------------------------------------
% LTXE-SLIDE-START
% LTXE-SLIDE-UID:		7158ff57-6891e803-bc577f27-c31b68c4
% LTXE-SLIDE-ORIGIN:	48fffdc3-1a0b74a9-f2415121-1e9f032a English
% LTXE-SLIDE-ORIGIN:	945ae684-9d826d2a-1c2e0d7b-c6c69465 German
% LTXE-SLIDE-TITLE:		Feature: #64200 - Allow individual content caching
% LTXE-SLIDE-REFERENCE:	Feature-64200-AllowIndividualContentCaching.rst
% ------------------------------------------------------------------------------
\begin{frame}[fragile]
	\frametitle{TSconfig i TypoScript}
	\framesubtitle{Individualno kesiranje sadrzaja}

	% decrease font size for code listing
	\lstset{basicstyle=\tiny\ttfamily}

	\begin{itemize}

		\item Od TYPO3 CMS 7.4 postoji individualno kesiranje sadrzaja koje - uporedjeno sa \texttt{stdWrap.cache} -
			takodje radi sa COA objektima \newline(slicno kao "Magento Block Caching")

			\begin{columns}[T]
				\begin{column}{.07\textwidth}
                \end{column}
				\begin{column}{.465\textwidth}
					\begin{lstlisting}
						page = PAGE
						page.10 = COA
						page.10 {
						  cache.key = coaout
						  cache.lifetime = 60
						  #stdWrap.cache.key = coastdWrap
						  #stdWrap.cache.lifetime = 60
						  10 = TEXT
						  10 {
						    cache.key = mycurrenttimestamp
						    cache.lifetime = 60
						    data = date : U
						    strftime = %H:%M:%S
						    noTrimWrap = |10: | |
						  }
					[...]
					\end{lstlisting}
				\end{column}

				\begin{column}{.465\textwidth}
					\begin{lstlisting}
					[...]
						  20 = TEXT
						  20 {
						    data = date : U
						    strftime = %H:%M:%S
						    noTrimWrap = |20: | |
						  }
						}
					\end{lstlisting}

				\end{column}
			\end{columns}

	\end{itemize}

\end{frame}

% ------------------------------------------------------------------------------
% LTXE-SLIDE-START
% LTXE-SLIDE-UID:		d17365de-02f81dd4-3a9fac0f-8fa52a5f
% LTXE-SLIDE-ORIGIN:	a9f58068-74a67b75-cd3b44ef-4c5422b9 English
% LTXE-SLIDE-ORIGIN:	d4fd55ab-f9f1cd6c-4eb2ca00-d50beeb2 German
% LTXE-SLIDE-TITLE:		Feature: #67880 - Added count to listNum
% LTXE-SLIDE-REFERENCE:	Feature-67880-AddedCountToListNum.rst
% ------------------------------------------------------------------------------

\begin{frame}[fragile]
	\frametitle{TSconfig i TypoScript}
	\framesubtitle{Broj elemenata u listi}

	% decrease font size for code listing
	\lstset{basicstyle=\tiny\ttfamily}

	\begin{itemize}

		\item Nova osobina \texttt{returnCount} dodata je u stdWrap osobinu \texttt{split}

		\item Ovo omogucava da se izbroji broj elemenata u listi ciji su elementi odvojeni zarezom

		\item Sledeci kod, na primer vraca \texttt{9}:

			\begin{lstlisting}
				1 = TEXT
				1 {
				  value = x,y,z,1,2,3,a,b,c
				  split.token = ,
				  split.returnCount = 1
				}
			\end{lstlisting}

	\end{itemize}

\end{frame}

% ------------------------------------------------------------------------------
% LTXE-SLIDE-START
% LTXE-SLIDE-UID:		3edf37a6-e3129c40-709a48e7-0a4486ac
% LTXE-SLIDE-ORIGIN:	525607c3-d2f26693-2b44076d-e90599ca English
% LTXE-SLIDE-ORIGIN:	a31ad0e4-857ff98a-239b23fd-c2e48958 German
% LTXE-SLIDE-TITLE:		Feature: #65550 - Make table display order configurable in List module
% LTXE-SLIDE-REFERENCE:	Feature-65550-MakeTableDisplayOrderConfigurableInListModule.rst
% ------------------------------------------------------------------------------

\begin{frame}[fragile]
	\frametitle{TSconfig i TypoScript}
	\framesubtitle{Sortiranje tabela u List pregledu}

	% decrease font size for code listing
	\lstset{basicstyle=\tiny\ttfamily}

	\begin{itemize}

		\item Dodata je nova TSconfig opcija \texttt{mod.web\_list.tableDisplayOrder} u "List" modul

		\item Sa ovom opcijom moguce je konfigurisati redosled u kome ce se tabele prikazivati

		\item Knjucne reci \texttt{pre} i \texttt{posle} mogu se iskoristiti za postavljanje redosleda u odnosu na druga imena tabela

		\begin{columns}[T]
			\begin{column}{.07\textwidth}
			\end{column}
			\begin{column}{.465\textwidth}

				\small Sintaksa:\normalsize

				\begin{lstlisting}
					mod.web_list.tableDisplayOrder {
					  <tableName> {
					    before = <tableA>, <tableB>, ...
					    after = <tableA>, <tableB>, ...
					  }
					}
				\end{lstlisting}
			\end{column}
			\begin{column}{.465\textwidth}

				\small Na primer:\normalsize

				\begin{lstlisting}
					mod.web_list.tableDisplayOrder {
					  be_users.after = be_groups
					  sys_filemounts.after = be_users
					  pages_language_overlay.before = pages
					  fe_users.after = fe_groups
					  fe_users.before = pages
					}
				\end{lstlisting}

			\end{column}
		\end{columns}

	\end{itemize}

\end{frame}

% ------------------------------------------------------------------------------
% LTXE-SLIDE-START
% LTXE-SLIDE-UID:		808e6ff6-df516e8a-83135864-e00cb943
% LTXE-SLIDE-ORIGIN:	9d764efe-2cd7e74e-f53f456b-f4a88331 English
% LTXE-SLIDE-ORIGIN:	35e3a45d-888c3fe2-26d320d1-ce6c65c3 German
% LTXE-SLIDE-TITLE:		Feature: #33071 - Add the http header "Content-Language" when rendering a page
% LTXE-SLIDE-REFERENCE:	Feature-33071-AddTheHttpHeaderContent-LanguageWhenRenderingAPage.rst
% ------------------------------------------------------------------------------

\begin{frame}[fragile]
	\frametitle{TSconfig i TypoScript}
	\framesubtitle{Content-Language u HTTP hederu}

	\begin{itemize}

		\item HTTP heder \texttt{Content-language: XX} se salje kao podrazumevana vrednost, gde je "XX" ISO kod \texttt{sys\_language\_content} konfiguracije

		\item Ova opcija se moze onemoguciti  setovanjem \texttt{config.disableLanguageHeader = 1}
			(uopste ne salje \texttt{Content-language} heder)

	\end{itemize}

\end{frame}

% ------------------------------------------------------------------------------
% LTXE-SLIDE-START
% LTXE-SLIDE-UID:		d624b2c3-0b583c0d-0d74e334-ec468746
% LTXE-SLIDE-ORIGIN:	17e7f151-0ebb8899-6209180f-bec0fc56 English
% LTXE-SLIDE-ORIGIN:	35a84207-7b86ee45-c39d815d-859811e1 German
% LTXE-SLIDE-TITLE:		Feature: #45725 - Added recursive option to folder based file collections
% LTXE-SLIDE-REFERENCE:	Feature-45725-AddedRecursiveOptionToFolderBasedFileCollections.rst
% ------------------------------------------------------------------------------

\begin{frame}[fragile]
	\frametitle{TSconfig i TypoScript}
	\framesubtitle{Opcija rekurzije za kolekcije fajlova}

	\begin{itemize}

		\item Kolekcije fajlova bazirane na grupisanje po folderima sada imaju opciju da rekurzivno dohvate sve fajlove u datom folderu

		\item Opcija je takodje dostupna za TypoScript objekat \texttt{FILES}

			\begin{lstlisting}
				filecollection = FILES
				filecollection {
				  folders = 1:images/
				  folders.recursive = 1
				  renderObj = IMAGE
				  renderObj {
				    file.import.data = file:current:uid
				  }
				}
			\end{lstlisting}

	\end{itemize}

\end{frame}

% ------------------------------------------------------------------------------
% LTXE-SLIDE-START
% LTXE-SLIDE-UID:		3051f393-2c556be5-1c602910-c5ef6d02
% LTXE-SLIDE-ORIGIN:	b7337338-ab366edd-bd94e9c0-6d7e86de English
% LTXE-SLIDE-ORIGIN:	ece6c712-df09d880-9378940c-e658d6a5 German
% LTXE-SLIDE-TITLE:		Feature: #34922 - Allow .ts file extension for static TypoScript templates
% LTXE-SLIDE-REFERENCE:	Feature-34922-AllowTsFileExtensionForStaticTyposcriptTemplates.rst
% ------------------------------------------------------------------------------

\begin{frame}[fragile]
	\frametitle{TSconfig i TypoScript}
	\framesubtitle{Ekstenzija \texttt{.ts} za Static Templates}

	\begin{itemize}

		\item U TYPO3 CMS < 7.4, samo su sledeca imena fajlova bila dozvoljena kao staticki TypoScript sabloni:

			\begin{itemize}
				\item \texttt{constants.txt}
				\item \texttt{setup.txt}
				\item \texttt{include\_static.txt}
				\item \texttt{include\_static\_files.txt}
			\end{itemize}

		\item Fajl ekstenzija \texttt{.ts} sada je takodje dozvoljena i za \texttt{constants} i \texttt{setup}.

		\item U ovom kontekstu, \texttt{.ts} ima prioritet nad \texttt{.txt}

	\end{itemize}

\end{frame}

% ------------------------------------------------------------------------------
% LTXE-SLIDE-START
% LTXE-SLIDE-UID:		8c08b3dc-4e9c09f3-807eba4d-67bcd6b4
% LTXE-SLIDE-ORIGIN:	6e127c7e-38fe07f8-e6b9545b-9142c306 English
% LTXE-SLIDE-ORIGIN:	168d7ca5-b9a3c55b-d6300076-7abdfd09 German
% LTXE-SLIDE-TITLE:		Feature: #20194 - Configuration for displaying the "save & view" button
% LTXE-SLIDE-REFERENCE:	Feature-20194-ConfigurationForDisplayingTheSaveViewButton.rst
% ------------------------------------------------------------------------------

\begin{frame}[fragile]
	\frametitle{TSconfig i TypoScript}
	\framesubtitle{Save \& view dugme}

	\begin{itemize}

		\item Dugme "save \& view" ada se moze konfigurisati preko TSconfig-a

		\item TSconfig \texttt{TCEMAIN.preview.disableButtonForDokType} prihvata "doktypes" liste razdvojene zarezom

		\item Podrazumevane vrednosti su "254, 255, 199" (za: Storage Folder, Recycler i Menu Separator)

		\item Kao posledica, dugme "save \& view" se vise ne prikazuje u folderima i recycler pages kao podrazumevano

	\end{itemize}

\end{frame}

% ------------------------------------------------------------------------------
% LTXE-SLIDE-START
% LTXE-SLIDE-UID:		48686ed4-0eb403ad-61f20c43-eead58f5
% LTXE-SLIDE-ORIGIN:	eb340ca0-00687bcb-51cda09f-82cb51ce English
% LTXE-SLIDE-ORIGIN:	80efec02-bec87365-a8bc2ab6-6741053f German
% LTXE-SLIDE-TITLE:		Feature: #43984 - Add stdWrap functionality to TreatIdAsReference TypoScript
% LTXE-SLIDE-REFERENCE:	Feature-43984-AddStdWrapFunctionalityToTreatIdAsReferenceTypoScript.rst
% ------------------------------------------------------------------------------
\begin{frame}[fragile]
	\frametitle{TSconfig i TypoScript}
	\framesubtitle{stdWrap za \texttt{treatIdAsReference}}

	\begin{itemize}

		\item Za objekat \texttt{getImgResource} postoji opcija \texttt{treatIdAsReference},
			koja se moze iskoristiti za definisanje da se UID tretitaju kao UID od \texttt{sys\_file\_reference}
			a ne od \texttt{sys\_file}.

		\item Opcija \texttt{treatIdAsReference} sada je dobila stdWrap funkcionalnost


	\end{itemize}

\end{frame}

% ------------------------------------------------------------------------------
