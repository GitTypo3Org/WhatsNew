% ------------------------------------------------------------------------------
% TYPO3 CMS 7.4 - What's New - Chapter "Extbase & Fluid" (German Version)
%
% @author	Patrick Lobacher <patrick@lobacher.de> and Michael Schams <schams.net>
% @license	Creative Commons BY-NC-SA 3.0
% @link		http://typo3.org/download/release-notes/whats-new/
% @language	German
% ------------------------------------------------------------------------------
% LTXE-CHAPTER-UID:		d1a9ab65-989c4bc7-f84807fd-d96febe7
% LTXE-CHAPTER-NAME:	Extbase & Fluid
% ------------------------------------------------------------------------------

\section{Extbase \& Fluid}
\begin{frame}[fragile]
	\frametitle{Extbase \& Fluid}

	\begin{center}\huge{Kapitel 5:}\end{center}
	\begin{center}\huge{\color{typo3darkgrey}\textbf{Extbase \& Fluid}}\end{center}

\end{frame}

% ------------------------------------------------------------------------------
% LTXE-SLIDE-START
% LTXE-SLIDE-UID:		7b7e03da-55cc30d6-8f375c39-c82ffc4f
% LTXE-SLIDE-TITLE:		Feature #66070: Configure anchor for pagination widget
% LTXE-SLIDE-REFERENCE:	Feature-66070-ConfigureSectionForPaginationWidget.rst
% ------------------------------------------------------------------------------

\begin{frame}[fragile]
	\frametitle{Extbase \& Fluid}
	\framesubtitle{Section-Anker für Pagination Widget}

	% decrease font size for code listing
	\lstset{basicstyle=\tiny\ttfamily}

	\begin{itemize}

		\item Es ist nun möglich einen Section-Anker im Pagination Widget zu verwenden
		\item Dazu gibt es den Schlüssel \texttt{section} im Attribut \texttt{configuration}

		\item Im folgenden Beispiel wird der Anker \texttt{\#archive} an jeden Widget-Link angehängt:

			\begin{lstlisting}
				<f:widget.paginate objects="{plantpestWarnings}" as="paginatedWarnings"
				  configuration="{section: 'archive', itemsPerPage: 10, insertAbove: 0, insertBelow: 1,
				  maximumNumberOfLinks: 10}">

				   [...]

				</f:widget.paginate>
			\end{lstlisting}

	\end{itemize}

\end{frame}

% ------------------------------------------------------------------------------
% LTXE-SLIDE-START
% LTXE-SLIDE-UID:		3356cd43-6aa02ac1-54fd276d-0c509286
% LTXE-SLIDE-TITLE:		Feature #68022: Added base date attribute to DateViewHelper
% LTXE-SLIDE-REFERENCE:	Feature-68022-AddedBaseDateAttributeToDateViewHelper.rst
% ------------------------------------------------------------------------------

\begin{frame}[fragile]
	\frametitle{Extbase \& Fluid}
	\framesubtitle{Attribut \texttt{base} für Date-ViewHelper}

	% decrease font size for code listing
	%\lstset{basicstyle=\tiny\ttfamily}

	\begin{itemize}

		\item Der Date-ViewHelper wurde um das optionale Attribut \texttt{base} ergänzt
		\item Damit kann man relative Berechnungen durchführen
		\item Wird das Datum als DateTime angegeben, wird \texttt{base} ignoriert
		\item Erlaubte Werte: siehe \href{http://www.php.net/manual/en/datetime.formats.relative.php}{PHP Dokumentation}
		\item Das folgende Beispiel gibt "2016" zurück, wenn das Objekt "dateObject" ein beliebiges Datum in 2017 enthält:

			\begin{lstlisting}
				<f:format.date format="Y" base="{dateObject}">-1 year</f:format.date>
			\end{lstlisting}

	\end{itemize}

\end{frame}

% ------------------------------------------------------------------------------
% LTXE-SLIDE-START
% LTXE-SLIDE-UID:		097a17c3-6c43ed7b-3f2db1ff-35b37660
% LTXE-SLIDE-TITLE:		Breaking #67890: Redesign FluidTemplateDataProcessorInterface to DataProcessorInterface
% LTXE-SLIDE-REFERENCE:	Breaking-67890-RedesignFluidTemplateDataProcessorInterfaceToDataProcessorInterface.rst
% ------------------------------------------------------------------------------

\begin{frame}[fragile]
	\frametitle{Extbase \& Fluid}
	\framesubtitle{\texttt{dataProcessing} bei FLUIDTEMPLATE}

	% decrease font size for code listing
	\lstset{basicstyle=\tiny\ttfamily}

	\begin{itemize}

		\item Mit TYPO3 CMS 7.3 wurde die Option \texttt{dataProcessing} beim Content-Objekt \texttt{FLUIDTEMPLATE} eingeführt

		\item Hierfür ändert sich das zu implementierende Interface von \texttt{FluidTemplateDataProcessorInterface} in \texttt{DataProcessorInterface} und damit auch die Methode \texttt{process()}

			\begin{lstlisting}
				public function process(
				  ContentObjectRenderer $cObj,
				  array $contentObjectConfiguration,
				  array $processorConfiguration,
				  array $processedData
				);
			\end{lstlisting}

	\end{itemize}

	\breakingchange

\end{frame}

% ------------------------------------------------------------------------------
