% ------------------------------------------------------------------------------
% TYPO3 CMS 7.4 - What's New - Chapter "Deprecated Functions" (German Version)
%
% @author	Patrick Lobacher <patrick@lobacher.de> and Michael Schams <schams.net>
% @license	Creative Commons BY-NC-SA 3.0
% @link		http://typo3.org/download/release-notes/whats-new/
% @language	German
% ------------------------------------------------------------------------------
% LTXE-CHAPTER-UID:		052872e9-9f10b97b-dcf0f34b-0bb90d27
% LTXE-CHAPTER-NAME:	Deprecated Functions
% ------------------------------------------------------------------------------

\section{Veraltete/Entfernte Funktionen}
\begin{frame}[fragile]
	\frametitle{Veraltete/Entfernte Funktionen}

	\begin{center}\huge{Kapitel 6:}\end{center}
	\begin{center}\huge{\color{typo3darkgrey}\textbf{Veraltete und entfernte Funktionen}}\end{center}

\end{frame}

% ------------------------------------------------------------------------------
% LTXE-SLIDE-START
% LTXE-SLIDE-UID:		ebe40c2e-32da4e43-67f18e6c-866b0496
% LTXE-SLIDE-TITLE:		Deprecation: #67991 - Removed EXT:cms (1)
% LTXE-SLIDE-REFERENCE:	Deprecation-67991-RemovedExtCms.rst
% ------------------------------------------------------------------------------

\begin{frame}[fragile]
	\frametitle{Veraltete/Entfernte Funktionen}
	\framesubtitle{Systemextension \texttt{cms} entfernt (1)}

	% decrease font size for code listing
	\lstset{basicstyle=\tiny\ttfamily}

	\begin{itemize}

		\item Die Systemextension \texttt{cms} wurde entfernt

		\item Entwickler von Extensions sollten prüfen, ob Abhängigkeit zu \texttt{cms} in der Datei \texttt{ext\_emconf.php}
			vorhanden sind und diese ggf. korrigieren

			\begin{lstlisting}
				[...]
				'constraints' => array(
				  'depends' => array(
				    // 'cms' => ' ... ',           <= FALSCH!
				    'typo3' => '7.0.0-7.99.99',
				  ),
				),
				[...]
			\end{lstlisting}

		\item Die meiste Funktionalität wurde zur Systemextension \texttt{frontend} migriert
			(daher müssen ggf. Referenzen zu Sprachdateien angepasst werden, siehe folgende Slide)

	\end{itemize}

\end{frame}

% ------------------------------------------------------------------------------
% LTXE-SLIDE-START
% LTXE-SLIDE-UID:		4e4332da-8e6c67f1-0496866b-ebe40c2e
% LTXE-SLIDE-TITLE:		Deprecation: #67991 - Removed EXT:cms (2)
% LTXE-SLIDE-REFERENCE:	Deprecation-67991-RemovedExtCms.rst
% ------------------------------------------------------------------------------

\begin{frame}[fragile]
	\frametitle{Veraltete/Entfernte Funktionen}
	\framesubtitle{Systemextension \texttt{cms} entfernt (2)}

	% decrease font size for code listing
	\lstset{basicstyle=\tiny\ttfamily}

	\begin{itemize}

		\item Notwendige Anpassungen der Referenzen zu Sprachdateien:

			% hint for translators:
			% please translate words "old" and "new" in listing below,
			% but DO NOT USE special characters, like German umlauts, etc.
			% (special characters inside "lstlisting" break LaTeX).

			\begin{lstlisting}
				ALT: typo3/sysext/cms/web_info/locallang.xlf
				NEU: typo3/sysext/frontend/Resources/Private/Language/locallang_webinfo.xlf
			\end{lstlisting}
			\vspace{-0.3cm}
			\begin{lstlisting}
				ALT: typo3/sysext/cms/locallang_ttc.xlf
				NEU: typo3/sysext/frontend/Resources/Private/Language/locallang_ttc.xlf
			\end{lstlisting}
			\vspace{-0.3cm}
			\begin{lstlisting}
				ALT: typo3/sysext/cms/locallang_tca.xlf
				NEU: typo3/sysext/frontend/Resources/Private/Language/locallang_tca.xlf
			\end{lstlisting}
			\vspace{-0.3cm}
			\begin{lstlisting}
				ALT: typo3/sysext/cms/layout/locallang_db_new_content_el.xlf
				NEU: typo3/sysext/backend/Resources/Private/Language/locallang_db_new_content_el.xlf
			\end{lstlisting}
			\vspace{-0.3cm}
			\begin{lstlisting}
				ALT: typo3/sysext/cms/layout/locallang.xlf
				NEU: typo3/sysext/backend/Resources/Private/Language/locallang_layout.xlf
			\end{lstlisting}
			\vspace{-0.3cm}
			\begin{lstlisting}
				ALT: typo3/sysext/cms/layout/locallang_mod.xlf
				NEU: typo3/sysext/backend/Resources/Private/Language/locallang_mod.xlf
			\end{lstlisting}
			\vspace{-0.3cm}
			\begin{lstlisting}
				ALT: typo3/sysext/cms/locallang_csh_webinfo.xlf
				NEU: typo3/sysext/frontend/Resources/Private/Language/locallang_csh_webinfo.xlf
			\end{lstlisting}
			\vspace{-0.3cm}
			\begin{lstlisting}
				ALT: typo3/sysext/cms/locallang_csh_weblayout.xlf
				NEU: typo3/sysext/frontend/Resources/Private/Language/locallang_csh_weblayout.xlf
			\end{lstlisting}

	\end{itemize}

\end{frame}

% ------------------------------------------------------------------------------
% LTXE-SLIDE-START
% LTXE-SLIDE-UID:		ae6fc887-9aeff493-158130de-3cdbf052
% LTXE-SLIDE-TITLE:		Deprecation: #68074 - Deprecate getPageRenderer() methods
% LTXE-SLIDE-REFERENCE:	Deprecation-68074-DeprecateGetPageRenderer.rst
% ------------------------------------------------------------------------------

\begin{frame}[fragile]
	\frametitle{Veraltete/Entfernte Funktionen}
	\framesubtitle{PageRenderer ist veraltet}

	% decrease font size for code listing
	\lstset{basicstyle=\tiny\ttfamily}

	\begin{itemize}
		\item Die folgenden PageRenderer-Methoden wurden als veraltet deklariert:

			\begin{lstlisting}
				TYPO3\CMS\Backend\Controller\BackendController::getPageRenderer()
				TYPO3\CMS\Backend\Template\DocumentTemplate::getPageRenderer()
				TYPO3\CMS\Backend\Template\FrontendDocumentTemplate::getPageRenderer()
				TYPO3\CMS\Frontend\Controller\TypoScriptFrontendController::getPageRenderer()
			\end{lstlisting}

		\item Stattdessen ist nun folgender Code zu verwenden, um eine Instanz des PageRenderers zu erhalten:

			\begin{lstlisting}
				\TYPO3\CMS\Core\Utility\GeneralUtility::makeInstance(\TYPO3\CMS\Core\Page\PageRenderer::class)
			\end{lstlisting}

	\end{itemize}

\end{frame}

% ------------------------------------------------------------------------------
% LTXE-SLIDE-START
% LTXE-SLIDE-UID:		a58c2f4f-3eb806e5-d256078a-70ed73c0
% LTXE-SLIDE-TITLE:		Deprecation #68098 and #68122
% LTXE-SLIDE-REFERENCE:	Deprecation-68098-GeneralUtilityMethods.rst
% LTXE-SLIDE-REFERENCE:	Deprecation-68122-GeneralUtilityReadLLfile.rst
% ------------------------------------------------------------------------------
% Deprecation: #68098 - Deprecate GeneralUtility methods
% Deprecation: #68122 - Deprecate GeneralUtility::readLLfile

\begin{frame}[fragile]
	\frametitle{Veraltete/Entfernte Funktionen}
	\framesubtitle{Veraltete \texttt{GeneralUtility}-Methoden}

	% decrease font size for code listing
	\lstset{basicstyle=\tiny\ttfamily}

	\begin{itemize}
		\item Die folgenden \texttt{GeneralUtility}-Methoden wurden als veraltet deklariert und werden in
			TYPO3 CMS version 8 entfernt:

			\begin{lstlisting}
				GeneralUtility::modifyHTMLColor()
				GeneralUtility::modifyHTMLColorAll()
				GeneralUtility::isBrokenEmailEnvironment()
				GeneralUtility::normalizeMailAddress()
				GeneralUtility::formatForTextarea()
				GeneralUtility::getThisUrl()
				GeneralUtility::cleanOutputBuffers()
				GeneralUtility::readLLfile()
			\end{lstlisting}

		\item Methode \texttt{readLLfile()} kann durch folgenden Code ersetzt werden:

			\begin{lstlisting}
				/** @var $languageFactory \TYPO3\CMS\Core\Localization\LocalizationFactory */
				$languageFactory = GeneralUtility::makeInstance(
				  \TYPO3\CMS\Core\Localization\LocalizationFactory::class
				);
				$languageFactory->getParsedData($fileToParse, $language, $renderCharset, $errorMode);
			\end{lstlisting}

	\end{itemize}

\end{frame}

% ------------------------------------------------------------------------------
% LTXE-SLIDE-START
% LTXE-SLIDE-UID:		b0c4d95f-d9fd698f-2d9dfdec-42aaf1af
% LTXE-SLIDE-TITLE:		Breaking: #39721 - Prototype.js and Scriptaculous removed
% LTXE-SLIDE-REFERENCE:	Breaking-39721-PrototypejsAndScriptaculousRemoved.rst
% ------------------------------------------------------------------------------

\begin{frame}[fragile]
	\frametitle{Veraltete/Entfernte Funktionen}
	\framesubtitle{JavaScript Bibliotheken entfernt}

	\begin{itemize}

		\item Die JavaScript-Bibliotheken \texttt{prototype.js} und \texttt{scriptaculous} wurden entfernt.
			Somit haben die folgenden TypoScript-Eigenschaften keine Funktion mehr:

			\begin{itemize}
				\item \texttt{page.javascriptLibs.Prototype}
				\item \texttt{page.javascriptLibs.Scriptaculous.*}
			\end{itemize}

		\item Im ViewHelper \texttt{be.container} liefern die entsprechenden Attribute Fehler:
			\begin{lstlisting}
				<f:be.container loadPrototype="false" loadScriptaculous="false" scriptaculousModule="someModule,someOtherModule">
			\end{lstlisting}

		\item Stattdessen wird empfohlen \texttt{jQuery} und \texttt{RequireJS} zu verwenden
			(die im Backend bereits standarmäßig geladen werden)

	\end{itemize}

\end{frame}

% ------------------------------------------------------------------------------
% LTXE-SLIDE-START
% LTXE-SLIDE-UID:		1375cbd0-05287ba0-3235df98-19473bf0
% LTXE-SLIDE-TITLE:		Deprecation: #68183 - typo3/mod.php
% LTXE-SLIDE-REFERENCE:	Deprecation-68183-Typo3modphp.rst
% ------------------------------------------------------------------------------

\begin{frame}[fragile]
	\frametitle{Veraltete/Entfernte Funktionen}
	\framesubtitle{\texttt{init.php}, \texttt{mod.php} und \texttt{ajax.php} sind veraltet}

	% decrease font size for code listing
	%\lstset{basicstyle=\tiny\ttfamily}

	\begin{itemize}

		\item Da alle nicht benötigten Dateien aus \texttt{typo3} aufgeräumt werden sollen,
			wurden die Dateien \texttt{init.php}, \texttt{mod.php} und \texttt{ajax.php}
			als veraltet markiert

		\item Will man eigene Init Entry Points verwenden, so geht dies über den folgenden Code:

			\begin{lstlisting}
				call_user_func(function() {
				  $classLoader = require __DIR__ . '/vendor/autoload.php';
				  (new \TYPO3\CMS\Backend\Http\Application($classLoader))->run();
				});
			\end{lstlisting}

		\item Anstelle des Zugriffs auf \texttt{mod.php} verwendet man nun:

			\begin{lstlisting}
				BackendUtility::getModuleUrl()
			\end{lstlisting}

	\end{itemize}

\end{frame}

% ------------------------------------------------------------------------------
% LTXE-SLIDE-START
% LTXE-SLIDE-UID:		749d4c59-1ec6087a-45020fbd-2f7b8646
% LTXE-SLIDE-TITLE:		Deprecation: #67737 - TCA: Drop additional palette
% LTXE-SLIDE-REFERENCE:	Deprecation-67737-TcaDropAdditionalPalette.rst
% ------------------------------------------------------------------------------

\begin{frame}[fragile]
	\frametitle{Veraltete/Entfernte Funktionen}
	\framesubtitle{TCA: Zusätzliche Palette entfernt}

	% decrease font size for code listing
	\lstset{basicstyle=\tiny\ttfamily}

	\begin{itemize}

		\item Der \texttt{showitem} String des TCA-Schlüssels \texttt{types} sah die Möglichkeit vor,
			eine zusätzliche Palette zu definieren. Diese wurde nach dem Hauptfeld gerendert

		\item Jenes wurde nun entfernt und in die normale Paletten-Definition migriert

		\item Bisher:

			\begin{lstlisting}
				'types' => array(
				  'aType' => array(
				    'showitem' => 'aField;aLabel;anAdditionalPaletteName',
				  ),
				),
			\end{lstlisting}

		\item Neu:

			\begin{lstlisting}
				'types' => array(
				  'aType' => array(
				    'showitem' => 'aField;aLabel, --palette--;;anAdditionalPaletteName',
				  ),
				),
			\end{lstlisting}

	\end{itemize}

\end{frame}

% ------------------------------------------------------------------------------
% LTXE-SLIDE-START
% LTXE-SLIDE-UID:		6398e32e-7f467ac7-463d9b47-bc0fd47c
% LTXE-SLIDE-TITLE:		Breaking: #67577, #67646 and #67824
% LTXE-SLIDE-REFERENCE: Breaking-67577-RteEnabledFlagHandling.rst
% LTXE-SLIDE-REFERENCE: Breaking-67646-LibraryInclusionInFrontend.rst
% LTXE-SLIDE-REFERENCE: Breaking-67824-Typo3ExtFolderRemoved.rst
% ------------------------------------------------------------------------------
% Breaking: #67577 - rte_enabled and flag handling
% Breaking: #67646 - PHP library inclusion in frontend removed
% Breaking: #67824 - typo3/ext folder removed

\begin{frame}[fragile]
	\frametitle{Veraltete/Entfernte Funktionen}
	\framesubtitle{Diverse Änderungen (1)}

	\begin{itemize}

		\item Die Content-Objekte "Text" und "Text mit Bild" hatten bisher eine Checkbox "RTE enabled".
			Diese wurde, zusammen mit der dazugehörigen TCA-Option \texttt{flag}, entfernt.

		\item Die folgenden TypoScript-Optionen zum Einbinden von PHP-Dateien wurden entfernt:

			\begin{itemize}
				\item \texttt{config.includeLibrary}
				\item \texttt{config.includeLibs}
			\end{itemize}

		\item Das Verzeichnis \texttt{typo3/ext} wurden entfernt\newline
			\small
				(nicht aber die Möglichkeit, globale Extensions zu verwenden: das Verzeichnis kann manuell angelegt werden)
			\normalsize

	\end{itemize}

\end{frame}

% ------------------------------------------------------------------------------
% LTXE-SLIDE-START
% LTXE-SLIDE-UID:		463d9b47-bc0fd47c-6398e32e-7f467ac7
% LTXE-SLIDE-TITLE:		Breaking: #68001
% LTXE-SLIDE-REFERENCE: Breaking-68001-RemovedExtJSCoreAndExtJSAdapters.rst
% ------------------------------------------------------------------------------
% Breaking: #68001 - Removed ExtJS Core and ExtJS Adapters
% Breaking: #68020 - Dropped DisableBigButtons

\begin{frame}[fragile]
	\frametitle{Veraltete/Entfernte Funktionen}
	\framesubtitle{Diverse Änderungen (2)}

	\begin{itemize}

		\item ExtCore (ein schlanker ExtJS Adapter) wurde entfernt und damit die folgenden TypoScript-Optionen:

			\begin{itemize}
				\item \texttt{page.javascriptLibs.ExtCore.*}
				\item \texttt{page.javascriptLibs.ExtJs.*}
			\end{itemize}

			Außerdem die entsprechende Option im \texttt{<f:be.container>}-ViewHelper

		\item Die sogenannten "BigButtons" ("Edit Page Properties", "Move Page",...) wurden entfernt
			und mit ihnen die TSconfig-Einstellung \texttt{mod.we\_layout.disableBigButtons}

	\end{itemize}

\end{frame}

% ------------------------------------------------------------------------------
% LTXE-SLIDE-START
% LTXE-SLIDE-UID:		d3a774ba-c1fd32c6-099aa85f-96b6a2a2
% LTXE-SLIDE-TITLE:		Breaking: #68020, #68131, #65790 and #67506
% LTXE-SLIDE-REFERENCE: Breaking-68020-DroppedDisableBigButtons.rst
% LTXE-SLIDE-REFERENCE: Breaking-68131-StreamlineErrorAndExceptionHandling.rst
% LTXE-SLIDE-REFERENCE: Deprecation-65790-PagesStoragePidDeprecated.rst
% LTXE-SLIDE-REFERENCE: Deprecation-67506-DeprecateIconUtilitygetIcon.rst
% ------------------------------------------------------------------------------
% Breaking: #68131 - Streamline error and exception handling
% Deprecation: #65790 - Remove pages.storage_pid and logic
% Deprecation: #67506 - Deprecate IconUtility::getIcon

\begin{frame}[fragile]
	\frametitle{Veraltete/Entfernte Funktionen}
	\framesubtitle{Diverse Änderungen (3)}

	\begin{itemize}

		\item Die Konfiguration für das Error- und Exception-Handling kann nun nicht mehr in der Datei
			\texttt{ext\_localconf.php} der Extension überschrieben werden, sondern muss in einer der Dateien
			\texttt{LocalConfiguration.php} oder \texttt{AdditionalConfiguration.php} gesetzt werden.

		\item Das Feld "General Record Storage Page" welches die Storage-PID für die Seite aufgenommen hat,
			wurde entfernt. Stattdessen muss man die Storage-PID nun per TypoScript (oder FlexForm) setzen.

		\item Die Funktion \texttt{IconUtility::getIcon()} wurde als veraltet gekennzeichnet - stattdessen
			verwendet man \texttt{IconUtility::getSpriteIconForRecord()}

	\end{itemize}

\end{frame}

% ------------------------------------------------------------------------------
