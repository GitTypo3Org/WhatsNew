% ------------------------------------------------------------------------------
% TYPO3 CMS 7.4 - What's New - Chapter "TypoScript Data Processors" (English Version)
%
% @author	Michael Schams <schams.net>
% @license	Creative Commons BY-NC-SA 3.0
% @link		http://typo3.org/download/release-notes/whats-new/
% @language	English
% ------------------------------------------------------------------------------
% LTXE-CHAPTER-UID:		c12ee814-839a93b0-c35138aa-688d84fd
% LTXE-CHAPTER-NAME:	TypoScript Data Processors
% ------------------------------------------------------------------------------

\section{TSconfig \& TypoScript: Data Processors}
\begin{frame}[fragile]
	\frametitle{TSconfig \& TypoScript: Data Processors}

	\begin{center}\huge{Capitolo 3:}\end{center}
	\begin{center}\huge{\color{typo3darkgrey}\textbf{TSconfig \& TypoScript: Data Processors}}\end{center}

\end{frame}

% ------------------------------------------------------------------------------
% LTXE-SLIDE-START
% LTXE-SLIDE-UID:		1255a00a-5c5c1a24-cddd9208-5c3caefb
% LTXE-SLIDE-ORIGIN:	093a9669-ea06d193-c4f990b9-329ebf9b English
% LTXE-SLIDE-ORIGIN:	73029207-22e86360-370a7baa-f6abe0c8 German
% LTXE-SLIDE-TITLE:		Feature: #67658 - Introduce DataProcessors
% LTXE-SLIDE-REFERENCE:	Feature-67658-IntroduceDataProcessorsForSplittingValues.rst
% ------------------------------------------------------------------------------

\begin{frame}[fragile]
	\frametitle{TypoScript Data Processors}
	\framesubtitle{Introduzione Data Processors}

	% decrease font size for code listing
	\lstset{basicstyle=\tiny\ttfamily}

	\begin{itemize}

		\item I seguenti Data Processors sono stati introdotti; essi permettono una lavorazione flessibile di 
			liste separate da virgole, array, file, etc.:

			\begin{itemize}
				\item \texttt{SplitProcessor}
				\item \texttt{CommaSeparatedValueProcessor}
				\item \texttt{FilesProcessor}
				\item \texttt{GalleryProcessor}
				\item \texttt{DatabaseQueryProcessor}
			\end{itemize}

		\item Vedi:
			\texttt{TYPO3\textbackslash
				CMS\textbackslash
				Frontend\textbackslash
				DataProcessing}

	\end{itemize}

\end{frame}

% ------------------------------------------------------------------------------
% LTXE-SLIDE-START
% LTXE-SLIDE-UID:		207eee34-d3c39d12-1714470e-506d8394
% LTXE-SLIDE-ORIGIN:	c77c07c9-47dc7fca-c98a4b9a-8582e3b3 English
% LTXE-SLIDE-ORIGIN:	a3886ad1-c35410c2-44630a18-6a98dfa6 German
% LTXE-SLIDE-TITLE:		Feature: #67658 - SplitProcessor
% LTXE-SLIDE-REFERENCE:	Feature-67658-IntroduceDataProcessorsForSplittingValues.rst
% ------------------------------------------------------------------------------

\begin{frame}[fragile]
	\frametitle{TypoScript Data Processors}
	\framesubtitle{\texttt{SplitProcessor}}

	% decrease font size for code listing
	\lstset{basicstyle=\tiny\ttfamily}

	\begin{itemize}

		\item Il "SplitProcessor" permette di dividere una valore separato da delimitatore in un array

			\begin{lstlisting}
				page.10 = FLUIDTEMPLATE
				page.10.file = EXT:site_default/Resources/Private/Template/Default.html
				page.10.dataProcessing.2 = TYPO3\CMS\Frontend\DataProcessing\SplitProcessor
				page.10.dataProcessing.2 {
				  if.isTrue.field = bodytext
				  delimiter = ,
				  fieldName = bodytext
				  removeEmptyEntries = 1
				  filterIntegers = 1
				  filterUnique = 1
				  as = keywords
				}
			\end{lstlisting}

		\item Possibile uso in Fluid:

			\begin{lstlisting}
				<f:for each="{keywords}" as="keyword">
				  <li>Keyword: {keyword}</li>
				</f:for>
			\end{lstlisting}

	\end{itemize}

\end{frame}

% ------------------------------------------------------------------------------
% LTXE-SLIDE-START
% LTXE-SLIDE-UID:		c67b5f35-701420a2-18197e36-61b0e6d2
% LTXE-SLIDE-ORIGIN:	7057a94e-e3bdd575-c4fd9cdf-f9690b04 English
% LTXE-SLIDE-ORIGIN:	c2272a2c-2d27b8ce-6b172e55-30ccbe34 German
% LTXE-SLIDE-TITLE:		Feature: #67658 - CommaSeparatedValueProcessor (1)
% LTXE-SLIDE-REFERENCE:	Feature-67658-IntroduceDataProcessorsForSplittingValues.rst
% ------------------------------------------------------------------------------

\begin{frame}[fragile]
	\frametitle{TypoScript Data Processors}
	\framesubtitle{\texttt{CommaSeparatedValueProcessor} (1)}

	% decrease font size for code listing
	\lstset{basicstyle=\tiny\ttfamily}

	\begin{itemize}

		\item Il "CommaSeparatedValueProcessor" suddivide un valore separato da delimitatore in un array bi-dimensionale:

			\begin{lstlisting}
				page.10 = FLUIDTEMPLATE
				page.10.file = EXT:site_default/Resources/Private/Template/Default.html
				page.10.dataProcessing.4 = TYPO3\CMS\Frontend\DataProcessing\CommaSeparatedValueProcessor
				page.10.dataProcessing.4 {
				  if.isTrue.field = bodytext
				  fieldName = bodytext
				  fieldDelimiter = |
				  fieldEnclosure =
				  maximumColumns = 2
				  as = table
				}
			\end{lstlisting}

		\item Utile per elaborare file CSV ad esempio o insiemi di dati \texttt{tt\_content} del CType "table"

			\vspace{0.2cm}
			\small
				Vedi un esempio di utilizzo in Fluid nella slide seguente
			\normalsize

	\end{itemize}

\end{frame}

% ------------------------------------------------------------------------------
% LTXE-SLIDE-START
% LTXE-SLIDE-UID:		cd606c50-078f02da-9020c309-0827190a
% LTXE-SLIDE-ORIGIN:	a3133675-b3a6a744-59c470b3-6a5348c7 English
% LTXE-SLIDE-ORIGIN:	be3430cc-6b172a2c-b8ce2d27-2e55c227 German
% LTXE-SLIDE-TITLE:		Feature: #67658 - CommaSeparatedValueProcessor (2)
% LTXE-SLIDE-REFERENCE:	Feature-67658-IntroduceDataProcessorsForSplittingValues.rst
% ------------------------------------------------------------------------------

\begin{frame}[fragile]
	\frametitle{TypoScript Data Processors}
	\framesubtitle{\texttt{CommaSeparatedValueProcessor} (2)}

	% decrease font size for code listing
	\lstset{basicstyle=\tiny\ttfamily}

	\begin{itemize}

		\item Possibile utilizzo in Fluid:

			\begin{lstlisting}
				<table>
				  <f:for each="{table}" as="columns">
				    <tr>
				      <f:for each="{columns}" as="column">
				        <td>
				          {column}
				        </td>
				      </f:for>
				    <tr>
				  </f:for>
				</table>
			\end{lstlisting}

	\end{itemize}

\end{frame}

% ------------------------------------------------------------------------------
% LTXE-SLIDE-START
% LTXE-SLIDE-UID:		6b92105d-33d666ff-078dfe6d-929f7752
% LTXE-SLIDE-ORIGIN:	c5bbdde9-1ab95545-65d21bce-332c2d11 English
% LTXE-SLIDE-ORIGIN:	09e5e821-ffca91f9-0d4f3bac-85bfc17b German
% LTXE-SLIDE-TITLE:		Feature: #67662 - DataProcessor for files (1)
% LTXE-SLIDE-REFERENCE:	Feature-67662-DataProcessorForFiles.rst
% ------------------------------------------------------------------------------

\begin{frame}[fragile]
	\frametitle{TypoScript Data Processors}
	\framesubtitle{\texttt{FilesProcessor} (1)}

	% decrease font size for code listing
	\lstset{basicstyle=\tiny\ttfamily}

	\begin{itemize}

		\item Il "FilesProcessor" gestisce riferimenti di file, file o file dentro una directory o collezione
			da utilizzare per l'output di frontend

			\begin{lstlisting}
				tt_content.image.20 = FLUIDTEMPLATE
				tt_content.image.20 {
				  file = EXT:myextension/Resources/Private/Templates/ContentObjects/Image.html
				  dataProcessing.10 = TYPO3\CMS\Frontend\DataProcessing\FilesProcessor
				  dataProcessing.10 {
				    references.fieldName = image
				    references.table = tt_content
				    files = 21,42
				    collections = 13,14
				    folders = 1:introduction/images/,1:introduction/posters/
				    folders.recursive = 1
				    sorting = description
				    sorting.direction = descending
				    as = myfiles
				  }
				}
			\end{lstlisting}

			\small
				Vedi un esempio di utilizzo in Fluid nella slide seguente
			\normalsize

	\end{itemize}

\end{frame}

% ------------------------------------------------------------------------------
% LTXE-SLIDE-START
% LTXE-SLIDE-UID:		8b1a8b6d-2b53379e-e614fba0-5f5f35d5
% LTXE-SLIDE-ORIGIN:	aebad6d6-cf790b17-d385fe19-b48a6016 English
% LTXE-SLIDE-ORIGIN:	ffca91f9-e82109e5-0d4f3bac-85bfc17b German
% LTXE-SLIDE-TITLE:		Feature: #67662 - DataProcessor for files (2)
% LTXE-SLIDE-REFERENCE:	Feature-67662-DataProcessorForFiles.rst
% ------------------------------------------------------------------------------

\begin{frame}[fragile]
	\frametitle{TypoScript Data Processors}
	\framesubtitle{\texttt{FilesProcessor} (2)}

	% decrease font size for code listing
	\lstset{basicstyle=\tiny\ttfamily}

	\begin{itemize}

		\item Possibile utilizzo in Fluid:

			\begin{lstlisting}
				<ul>
				  <f:for each="{myfiles}" as="file">
				    <li>
				      <a href="{file.publicUrl}">{file.name}</a>
				    </li>
				  </f:for>
				</ul>
			\end{lstlisting}

	\end{itemize}

\end{frame}

% ------------------------------------------------------------------------------
% LTXE-SLIDE-START
% LTXE-SLIDE-UID:		a8ab53a5-5407bd10-0322645a-12cd8444
% LTXE-SLIDE-ORIGIN:	cb6f86cd-767463a3-fff3c768-a6b6755b English
% LTXE-SLIDE-ORIGIN:	4e375ebd-6c1cb855-d9f32eac-bfb1754e German
% LTXE-SLIDE-TITLE:		Feature: #67663 - Introduce DataProcessor for media galleries
% LTXE-SLIDE-REFERENCE:	Feature-67663-IntroduceDataProcessorForMediaGalleries.rst
% ------------------------------------------------------------------------------

\begin{frame}[fragile]
	\frametitle{TypoScript Data Processors}
	\framesubtitle{\texttt{GalleryProcessor}}

	% decrease font size for code listing
	\lstset{basicstyle=\tiny\ttfamily}

	\begin{itemize}

		% Note for translators: make sure, the following bullet point does not exceed ONE line.
		% If it wraps and uses two lines, the code example below does not fit on the page.
		% If not avoidable, split the content and create a second slide for the code.

		\item Il "GalleryProcessor" calcola la dimensione massima di un set di file

			\begin{lstlisting}
				tt_content.text_media.20 = FLUIDTEMPLATE
				tt_content.image.20 {
				  file = EXT:myextension/Resources/Private/Templates/ContentObjects/Image.html
				  dataProcessing {
				    10 = TYPO3\CMS\Frontend\DataProcessing\FilesProcessor
				    20 = TYPO3\CMS\Frontend\DataProcessing\GalleryProcessor
				    20 {
				      filesProcessedDataKey = files
				      mediaOrientation.field = imageorient
				      numberOfColumns.field = imagecols
				      equalMediaHeight.field = imageheight
				      equalMediaWidth.field = imagewidth
				      maxGalleryWidth = 1000
				      maxGalleryWidthInText = 1000
				      columnSpacing = 0
				      borderEnabled.field = imageborder
				      borderWidth = 0
				      borderPadding = 10
				      as = gallery
				    }
				  }
				}
			\end{lstlisting}

	\end{itemize}

\end{frame}

% ------------------------------------------------------------------------------
% LTXE-SLIDE-START
% LTXE-SLIDE-UID:		70bdb0ea-dd46395b-f03b6a2a-49b4387d
% LTXE-SLIDE-ORIGIN:	61607e79-8b231b33-2b9cc07d-ada762f3 English
% LTXE-SLIDE-ORIGIN:	773e4cfa-a49bd28b-fe31681a-eae68427 German
% LTXE-SLIDE-TITLE:		Feature: #68094 - Database Query DataProcessor
% LTXE-SLIDE-REFERENCE:	Feature-68094-DatabaseQueryDataProcessor.rst
% ------------------------------------------------------------------------------

\begin{frame}[fragile]
	\frametitle{TypoScript Data Processors}
	\framesubtitle{\texttt{DatabaseQueryProcessor} (1)}

	% decrease font size for code listing
	\lstset{basicstyle=\tiny\ttfamily}

	\begin{itemize}

		\item Il "DatabaseQueryProcessor" può essere usato per recuperare dati dal database

			\begin{lstlisting}
				tt_content.mycontent.20 = FLUIDTEMPLATE
				tt_content.mycontent.20 {
				  file = EXT:myextension/Resources/Private/Templates/ContentObjects/MyContent.html
				  dataProcessing.10 = TYPO3\CMS\Frontend\DataProcessing\DatabaseQueryProcessor
				  dataProcessing.10 {
				    if.isTrue.field = records
				    table = tt_address
				    colPos = 1
				    pidInList = 13,14
				    as = myrecords
				    dataProcessing {
				      10 = TYPO3\CMS\Frontend\DataProcessing\FilesProcessor
				      10 {
				        references.fieldName = image
				      }
				    }
				  }
				}
			\end{lstlisting}

			% Note for translators: make sure, the following sentence appears on the slide.
			% If the bullet point above wraps and uses two lines, the code example becomes too
			% long and pushes the sentence off the slide. If not avoidable, simply remove the
			% sentence (our readers will realize that there is a code example on the next slide).

			\vspace{-0.2cm}
			\small
				Vedi un esempio di utilizzo in Fluid nella slide seguente
			\normalsize

	\end{itemize}

\end{frame}

% ------------------------------------------------------------------------------
% LTXE-SLIDE-START
% LTXE-SLIDE-UID:		cdf5a096-bcaa8c26-1548d130-50ee2053
% LTXE-SLIDE-ORIGIN:	ee266111-2964afae-9cfcfd3f-3264b90b English
% LTXE-SLIDE-ORIGIN:	bd28ba49-e42ae687-68fe311a-cfa773e4 German
% LTXE-SLIDE-TITLE:		Feature: #68094 - Database Query DataProcessor
% LTXE-SLIDE-REFERENCE:	Feature-68094-DatabaseQueryDataProcessor.rst
% ------------------------------------------------------------------------------
 
\begin{frame}[fragile]
	\frametitle{TypoScript Data Processors}
	\framesubtitle{\texttt{DatabaseQueryProcessor} (2)}

	% decrease font size for code listing
	\lstset{basicstyle=\tiny\ttfamily}

	\begin{itemize}

		\item Possibile utilizzo in Fluid:

			\begin{lstlisting}
				<ul>
				  <f:for each="{myrecords}" as="record">
				    <li>
				      <f:image image="{record.files.0}" ></f:image>
				      <a href="{record.data.www}">{record.data.first_name} {record.data.last_name}</a>
				    </li>
				  </f:for>
				</ul>
			\end{lstlisting}

	\end{itemize}

\end{frame}

% ------------------------------------------------------------------------------
