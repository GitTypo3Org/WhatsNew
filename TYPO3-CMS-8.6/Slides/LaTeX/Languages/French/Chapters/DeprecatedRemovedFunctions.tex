% ------------------------------------------------------------------------------
% TYPO3 CMS 8.6 - What's New - Chapter "Fonctions dépréciées" (French Version)
%
% @author	Michael Schams <schams.net>
% @license	Creative Commons BY-NC-SA 3.0
% @link		http://typo3.org/download/release-notes/whats-new/
% @language	French
% ------------------------------------------------------------------------------
% LTXE-CHAPTER-UID:		3f842373-9262b8d3-f9c8de76-cf29ce17
% LTXE-CHAPTER-NAME:	Fonctions dépréciées
% ------------------------------------------------------------------------------

\section{Fonctions dépréciées et retirées}
\begin{frame}[fragile]
	\frametitle{Fonctions dépréciées et retirées}

	\begin{center}\huge{Chapitre 5~:}\end{center}
	\begin{center}\huge{\color{typo3darkgrey}\textbf{Fonctions dépréciées et retirées}}\end{center}

\end{frame}

% ------------------------------------------------------------------------------
% LTXE-SLIDE-START
% LTXE-SLIDE-UID:		d7687b20-b8f723c9-61671d12-c6049052
% LTXE-SLIDE-ORIGIN:	f6dccaab-1b20a7cb-3b708187-74c96db9 English
% LTXE-SLIDE-TITLE:		!Breaking: #78988 - Remove optional Fluid TypoScript template
% ------------------------------------------------------------------------------

\begin{frame}[fragile]
	\frametitle{Fonctions dépréciées et retirées}
	\framesubtitle{Gabarit TypoScript Fluid optionnel retiré}

	% decrease font size for code listing
	\lstset{basicstyle=\tiny\ttfamily}

	\begin{itemize}
		\item Le fichier de gabarit statique «~Fluid: (Optional) default
			ajax configuration (fluid)~» se voulait être un exemple/une démonstration de
			l'usage des Widget Fluid en FE. Il est obsolète et donc retiré.
		\item Plutôt, incluez les fichiers requis manuellement dans votre gabarit TypoScript~:

			\begin{lstlisting}
				page.includeJSLibs {
				  jquery = https://code.jquery.com/jquery-3.1.1.slim.min.js
				  jquery.external = 1
				  jquery.integrity = sha256-/SIrNqv8h6QGKDuNoLGA4iret+kyesCkHGzVUUV0shc=
				  jqueryUi = https://code.jquery.com/ui/1.12.1/jquery-ui.min.js
				  jqueryUi.external = 1
				  jqueryUi.integrity = sha256-VazP97ZCwtekAsvgPBSUwPFKdrwD3unUfSGVYrahUqU=
				}

				page.includeCSSLibs {
				  jqueryUI = https://code.jquery.com/ui/1.12.1/themes/smoothness/jquery-ui.css
				  jqueryUi.external = 1
				}
			\end{lstlisting}

	\end{itemize}

\end{frame}

% ------------------------------------------------------------------------------
% LTXE-SLIDE-START
% LTXE-SLIDE-UID:		4b1830df-9362ba09-39dd8d4b-302f8dde
% LTXE-SLIDE-ORIGIN:	2f81893a-60617439-c59f6e53-1c79a41f English
% LTXE-SLIDE-TITLE:		!Breaking: #79109 - Lowlevel VersionsCommand parameters changed
% ------------------------------------------------------------------------------

\begin{frame}[fragile]
	\frametitle{Fonctions dépréciées et retirées}
	\framesubtitle{Changement des paramètres de la commande lowlevel versions (1/2)}

	\begin{itemize}
		\item La commande CLI de \texttt{EXT:lowlevel} pour afficher et nettoyer les versions (de
			\texttt{EXT:version} / \texttt{EXT:workspaces}) est migrée en commande Symfony Console.

		\item La commande était précédemment \texttt{./typo3/cli\_dispatch.phpsh lowlevel\_cleaner versions}.
			Elle est maintenant \texttt{./typo3/sysext/core/bin/typo3 cleanup:versions} et permet de spécifier
			les options suivantes~:

			\begin{itemize}
				\item \texttt{-v} et \texttt{-vv} pour afficher des informations détaillées sur les
					enregistrements affectés
				\item \texttt{-}\texttt{-pid=23} ou \texttt{-p=23} pour les versions de la page d'identifiant 23
					(sinon "0" est utilisé)
			\end{itemize}

	\end{itemize}

	\small\textit{Suite diapositive suivante}\normalsize

\end{frame}

% ------------------------------------------------------------------------------
% LTXE-SLIDE-START
% LTXE-SLIDE-UID:		d6c50c43-77e73b8b-f3225dd8-8b47c930
% LTXE-SLIDE-ORIGIN:	2104e84b-c80621c1-0774f43b-33d86b6b English
% LTXE-SLIDE-TITLE:		!Breaking: #79109 - Lowlevel VersionsCommand parameters changed
% ------------------------------------------------------------------------------

\begin{frame}[fragile]
	\frametitle{Fonctions dépréciées et retirées}
	\framesubtitle{Changement des paramètres de la commande lowlevel versions (2/2)}

	\small\textit{Suite}\normalsize

	\begin{itemize}
		\item ...
			\begin{itemize}\vspace{-0.1cm}
				\item \texttt{-}\texttt{-depth=4} ou \texttt{-d=4} pour spécifier le degré d'exploration de
					l'arborescence des pages
				\item \texttt{-}\texttt{-dry-run} pour afficher les changements qui seront effectués
				\item \texttt{-}\texttt{-action=nameofaction} pour spécifier l'action de nettoyage.
					L'une de ces actions est possible~:
					\begin{itemize}
						\item \texttt{versions\_in\_live}~: Supprimer les enregistrements versionnés dans l'espace live
						\item \texttt{published\_versions}~: Suppr. les versions des enregistrements publiés
						\item \texttt{invalid\_workspace}~: Déplacer les enregistrements associés à un espace non existant
						dans l'espace de travail live
						\item \texttt{unused\_placeholders}~: Retirer les enregistrements de substitution qui ne sont plus
						utilisés dans la base
					\end{itemize}
			\end{itemize}
	\end{itemize}

\end{frame}

% ------------------------------------------------------------------------------
% LTXE-SLIDE-START
% LTXE-SLIDE-UID:		541a9fd3-c6cd647a-d23ef401-11096c22
% LTXE-SLIDE-ORIGIN:	35714a8f-ad99f4e7-ccda0780-c0aa44aa English
% LTXE-SLIDE-TITLE:		Default Layouts for Fluid Styled Content Changed
% LTXE-SLIDE-REFERENCE:	!Breaking: #79622 - Default layouts for Fluid Styled Content changed
% ------------------------------------------------------------------------------

\begin{frame}[fragile]
	\frametitle{Fonctions dépréciées et retirées}
	\framesubtitle{Dispositions par défaut de Fluid Styled Content changées}

	% decrease font size for code listing
	\lstset{basicstyle=\tiny\ttfamily}

	\begin{itemize}
		\item Les dispositions des éléments de contenu de Fluid Styled Content sont changées
			afin d'être maintenables et davantage flexibles.

		\item Les dispositions précédemment disponibles \texttt{ContentFooter}, \texttt{HeaderFooter}
			et \texttt{HeaderContentFooter} sont retirées et remplacées par une seule disposition
			\texttt{Default} qui est plus flexible.

			\begin{lstlisting}
				$GLOBALS['TCA']['tt_content']['columns']['CType']['config']['default'] = 'textmedia';
				$GLOBALS['TCA']['tt_content']['columns']['CType']['config']['default'] = 'header';
			\end{lstlisting}

	\end{itemize}

\end{frame}

% ------------------------------------------------------------------------------
% LTXE-SLIDE-START
% LTXE-SLIDE-UID:		f32536bd-2a54aa6f-8987b01d-508b7a4d
% LTXE-SLIDE-ORIGIN:	0e36fc45-799c2e27-3f058488-8173f554 English
% LTXE-SLIDE-TITLE:		TypoScript Standard Header (1/2)
% LTXE-SLIDE-REFERENCE:	!Breaking: #79622 - TypoScript Standard Header has been removed from Fluid Styled Content
% ------------------------------------------------------------------------------

\begin{frame}[fragile]
	\frametitle{Fonctions dépréciées et retirées}
	\framesubtitle{Ligne de titre TypoScript standard (1/2)}

	\begin{itemize}
		\item La définition standard du rendu TypoScript de la ligne de titre,
			\texttt{lib.stdHeader}, avait été introduite dans CSS Styled Content
			afin d'être référencée par les différents éléments de contenu pour
			simplifier la maintenance.

		\item Pour Fluid Styled Content, un contournement de compatibilité avec CMS 7
			était introduit pour simplifier la migration. Cependant, seul le rendu du
			titre est effectué et les cadres sont manquants. Des options supplémentaires
			sont aussi nécessaires pour inclure le rendu d'un élément de contenu lorsque
			sa disposition n'était pas implémentée correctement.

	\end{itemize}

\end{frame}

% ------------------------------------------------------------------------------
% LTXE-SLIDE-START
% LTXE-SLIDE-UID:		9eb70d6b-5b8ddf68-6016a37a-17d67d66
% LTXE-SLIDE-ORIGIN:	ca67c35f-cf7489b1-dd2ae6a2-f3557c53 English
% LTXE-SLIDE-TITLE:		TypoScript Standard Header (2/2)
% LTXE-SLIDE-REFERENCE:	!Breaking: #79622 - TypoScript Standard Header has been removed from Fluid Styled Content
% ------------------------------------------------------------------------------

\begin{frame}[fragile]
	\frametitle{Fonctions dépréciées et retirées}
	\framesubtitle{Ligne de titre TypoScript standard (2/2)}

	% decrease font size for code listing
	\lstset{basicstyle=\tiny\ttfamily}

	\begin{itemize}

		\item Nouvelle sortie~:

			\begin{lstlisting}
				tt_content.simple_content = COA
				tt_content.simple_content {
				  10 < lib.stdHeader
				  20 = TEXT
				  20.field = bodytext
				}

				<header>
				  <h1>Nunc vel libero dignissim</h1>
				</header>
				<p>
				  ...
				</p>
			\end{lstlisting}

	\end{itemize}

\end{frame}

% ------------------------------------------------------------------------------
% LTXE-SLIDE-START
% LTXE-SLIDE-UID:		47e3eee0-2dba40e5-73c81ebb-296bd30c
% LTXE-SLIDE-ORIGIN:	64df6081-eb1f0a13-7f8292d3-9c69a293 English
% LTXE-SLIDE-TITLE:		Miscellaneous (1/4)
% ------------------------------------------------------------------------------

\begin{frame}[fragile]
	\frametitle{Fonctions dépréciées et retirées}
	\framesubtitle{Divers (1/4)}

	% !Breaking: #78477 - Remove method FlashMessage->getMessageAsMarkup()
	% !Breaking: #79100 - ext:felogin: Remove default CSS
	% !Breaking: #79201 - EXT:form: Split TypoScript Includes
	% !Breaking: #79242 - Remove l10n_mode noCopy
	% !Breaking: #79243 - Remove l10n_mode mergeIfNotBlank
	% !Breaking: #79243 - Remove sys_language_softMergeIfNotBlank

	\begin{itemize}
		\item La méthode suivante est retirée~:\newline
			\small\texttt{FlashMessage->getMessageAsMarkup()}\normalsize
		\item \texttt{EXT:felogin} n'ajoute plus ses styles CSS par défaut car il
			pourrait casser le rendu frontend. Ex. lorsqu'un framework CSS est utilisé.
		\item La configuration TypoScript frontend spécifique de \texttt{EXT:form} n'est
			plus chargée automatiquement et doit être ajoutée manuellement par les inclusions
			statiques. Ce changement permet aux intégrateurs TYPO3 de décider plus facilement
			où le TypoScript de l'extension est ajoutée.
		\item L'option \texttt{noCopy} est retirée sans remplacement de la liste des valeurs
			possibles de la propriété TCA de colonne \texttt{l10n\_mode}.
		\item L'option \texttt{mergeIfNotBlank} est retirée sans remplacement de la liste des
			valeurs possibles de la propriété TCA de colonne \texttt{l10n\_mode}.

	\end{itemize}

\end{frame}


% ------------------------------------------------------------------------------
% LTXE-SLIDE-START
% LTXE-SLIDE-UID:		791204e6-ff34f0a7-114b4b49-be08a66b
% LTXE-SLIDE-ORIGIN:	9b7c4a92-17f663e5-b890e0b9-1b039f82 English
% LTXE-SLIDE-TITLE:		Miscellaneous (2/4)
% ------------------------------------------------------------------------------

\begin{frame}[fragile]
	\frametitle{Fonctions dépréciées et retirées}
	\framesubtitle{Divers (2/4)}

	% !Breaking: #79302 - Moved pages.url_scheme to compatibility7 extension
	% !Breaking: #79364 - Move page module function `QuickEdit` to compatibility7
	% !Breaking: #79622 - CSS Styled Content and TypoScript

	\begin{itemize}
		\item La configuration TypoScript \texttt{config.sys\_language\_softMergeIfNotBlank}
			est retirée sans remplacement. C'est un effet du retrait de l'option
			\texttt{mergeIfNotBlank} des valeurs possibles de la propriété TCA
			\texttt{l10n\_mode}.

		\item La fonctionnalité fournie par le champ de base de données \texttt{pages.url\_scheme}
			est déplacée dans l'extension \texttt{EXT:compatibility7}. Ce champ permet de forcer
			le protocole HTTP ou HTTPS d'une page par un contributeur dans les propriétés de page
			pour chaque page. Cependant, il est maintenant commun de s'assurer que l'ensemble
			du site (si un certificat x509 est mis en place) ou qu'une section (avec sous-pages)
			utilise le protocole HTTPS.

	\end{itemize}

\end{frame}

% ------------------------------------------------------------------------------
% LTXE-SLIDE-START
% LTXE-SLIDE-UID:		1140100a-5082ef88-b467deed-3246fa0e
% LTXE-SLIDE-ORIGIN:	6b0c8695-0ad6167d-2eda453e-5b16dd49 English
% LTXE-SLIDE-TITLE:		Miscellaneous (3/4)
% LTXE-SLIDE-REFERENCE:	!Deprecation: #77934 - Deprecate tt_content field select_key
% LTXE-SLIDE-REFERENCE:	!Deprecation: #78477 - Refactoring of FlashMessage rendering
% LTXE-SLIDE-REFERENCE:	!Deprecation: #79316 - Deprecate ArrayUtility::inArray()
% LTXE-SLIDE-REFERENCE:	!Deprecation: #79622 - Deprecation of CSS Styled Content
% ------------------------------------------------------------------------------

\begin{frame}[fragile]
	\frametitle{Fonctions dépréciées et retirées}
	\framesubtitle{Divers (3/4)}

	\begin{itemize}
		\item La fonctionnalité \texttt{QuickEdit} du module page est déplacée dans
			l'extension \texttt{EXT:compatibility7} et ne recevra pas d'évolution.\newline
			Voir \href{https://typo3.org/extensions/repository}{le dépôt d'extensions TYPO3 (TER)}.

		\item Afin de faire coïncider CSS Styled Content et Fluid Style Content, plusieurs
			options de CSS Styled Content sont retirées sans remplacement~:
			\texttt{TCA image\_compression}, \texttt{TCA image\_effects}, \texttt{TCA image\_noRows},
			\texttt{TypoScript IMAGE noRows}, \texttt{TypoScript IMAGE noCols},
			\texttt{TypoScript IMAGE noRowsStdWrap}, \texttt{TypoScript IMGTEXT captionAlign}

		\item Le champ \texttt{select\_key} de la table \texttt{tt\_content} n'est pas utilisé
			dans le noyau et donc retiré.

	\end{itemize}

\end{frame}

% ------------------------------------------------------------------------------
% LTXE-SLIDE-START
% LTXE-SLIDE-UID:		6ca52ccc-787fc802-b1129e3b-0bded775
% LTXE-SLIDE-ORIGIN:	3fe05cb6-a2648210-5486e0fa-93f0f02a English
% LTXE-SLIDE-TITLE:		Miscellaneous (4/4)
% LTXE-SLIDE-REFERENCE:	!Deprecation: #77934 - Deprecate tt_content field select_key
% LTXE-SLIDE-REFERENCE:	!Deprecation: #78477 - Refactoring of FlashMessage rendering
% LTXE-SLIDE-REFERENCE:	!Deprecation: #79316 - Deprecate ArrayUtility::inArray()
% LTXE-SLIDE-REFERENCE:	!Deprecation: #79622 - Deprecation of CSS Styled Content
% ------------------------------------------------------------------------------

\begin{frame}[fragile]
	\frametitle{Fonctions dépréciées et retirées}
	\framesubtitle{Divers (4/4)}

	\begin{itemize}

		\item Les méthodes et propriétés suivantes de \texttt{FlashMessage::class} sont
			marquées dépréciées~:

			\begin{itemize}
				\item \texttt{FlashMessage->classes}
				\item \texttt{FlashMessage->icons}
				\item \texttt{FlashMessage->getClass()}
				\item \texttt{FlashMessage->getIconName()}
			\end{itemize}

		\item La méthode \texttt{ArrayUtility::inArray()} est marquée dépréciée

		\item CSS Styled Content est déprécié\newline
			\small(sera retiré de TYPO3 CMS version 9)\normalsize

	\end{itemize}

\end{frame}

% ------------------------------------------------------------------------------
