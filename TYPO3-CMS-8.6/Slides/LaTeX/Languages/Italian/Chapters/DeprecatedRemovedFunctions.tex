% ------------------------------------------------------------------------------
% TYPO3 CMS 8.6 - What's New - Chapter "Deprecated Functions" (Italian Version)
%
% @author	Michael Schams <schams.net>
% @license	Creative Commons BY-NC-SA 3.0
% @link		http://typo3.org/download/release-notes/whats-new/
% @language	English
% ------------------------------------------------------------------------------
% LTXE-CHAPTER-UID:		3f842373-9262b8d3-f9c8de76-cf29ce17
% LTXE-CHAPTER-NAME:	Deprecated Functions
% ------------------------------------------------------------------------------

\section{Funzionalità deprecate/rimosse}
\begin{frame}[fragile]
	\frametitle{Funzionalità deprecate/rimosse}

	\begin{center}\huge{Capitolo 5:}\end{center}
	\begin{center}\huge{\color{typo3darkgrey}\textbf{Funzionalità deprecate/rimosse}}\end{center}

\end{frame}

% ------------------------------------------------------------------------------
% LTXE-SLIDE-START
% LTXE-SLIDE-UID:		c24a57a1-a9be5f6d-898a02a9-db19642b
% LTXE-SLIDE-ORIGIN:	f6dccaab-1b20a7cb-3b708187-74c96db9 English
% LTXE-SLIDE-TITLE:		!Breaking: #78988 - Remove optional Fluid TypoScript template
% ------------------------------------------------------------------------------

\begin{frame}[fragile]
	\frametitle{Funzionalità deprecate/rimosse}
	\framesubtitle{Rimosso Fluid Template TypoScript opzionale}

	% decrease font size for code listing
	\lstset{basicstyle=\tiny\ttfamily}

	\begin{itemize}
		\item L'inclusione del file statico "Fluid: (Optional) default ajax configuration (fluid)" è intesto
			come un esempio/dimostrazione su come utilizzare i widget Fluid in FE. Esso è obsoleto e quindi rimosso.
		\item Al suo posto vanno inclusi manualmente i file necessari nel template TypoScript:

			\begin{lstlisting}
				page.includeJSLibs {
				  jquery = https://code.jquery.com/jquery-3.1.1.slim.min.js
				  jquery.external = 1
				  jquery.integrity = sha256-/SIrNqv8h6QGKDuNoLGA4iret+kyesCkHGzVUUV0shc=
				  jqueryUi = https://code.jquery.com/ui/1.12.1/jquery-ui.min.js
				  jqueryUi.external = 1
				  jqueryUi.integrity = sha256-VazP97ZCwtekAsvgPBSUwPFKdrwD3unUfSGVYrahUqU=
				}

				page.includeCSSLibs {
				  jqueryUI = https://code.jquery.com/ui/1.12.1/themes/smoothness/jquery-ui.css
				  jqueryUi.external = 1
				}
			\end{lstlisting}

	\end{itemize}

\end{frame}

% ------------------------------------------------------------------------------
% LTXE-SLIDE-START
% LTXE-SLIDE-UID:		faa1635d-eca27275-fe2d9ea8-a43900bf
% LTXE-SLIDE-ORIGIN:	2f81893a-60617439-c59f6e53-1c79a41f English
% LTXE-SLIDE-TITLE:		!Breaking: #79109 - Lowlevel VersionsCommand parameters changed
% ------------------------------------------------------------------------------

\begin{frame}[fragile]
	\frametitle{Funzionalità deprecate/rimosse}
	\framesubtitle{Parametri modificati in Lowlevel VersionsCommand (1/2)}

	\begin{itemize}
		\item I comandi CLI esistenti in \texttt{EXT:lowlevel} per mostrare e pulire le versioni (da
			\texttt{EXT:version} / \texttt{EXT:workspaces}) sono stati migrati a comandi della Symfony Console.

		\item I comandi precedentemente disponibili via \texttt{./typo3/cli\_dispatch.phpsh lowlevel\_cleaner versions}
			sono ora disponibili via \texttt{./typo3/sysext/core/bin/typo3 cleanup:versions} e permettono di impostare
			le seguenti opzioni CLI:

			\begin{itemize}
				\item \texttt{-v} e \texttt{-vv} per visualizzare le informazioni più dettagliate sui record interessati
				\item \texttt{-}\texttt{-pid=23} o \texttt{-p=23} per trovare solamente versioni con id di pagina 23 (altrimenti è preso "0")
			\end{itemize}

	\end{itemize}

	\small\textit{Continua sulla slide seguente}\normalsize

\end{frame}

% ------------------------------------------------------------------------------
% LTXE-SLIDE-START
% LTXE-SLIDE-UID:		f1ce4048-df5948fa-357b0d49-c02adede
% LTXE-SLIDE-ORIGIN:	2104e84b-c80621c1-0774f43b-33d86b6b English
% LTXE-SLIDE-TITLE:		!Breaking: #79109 - Lowlevel VersionsCommand parameters changed
% ------------------------------------------------------------------------------

\begin{frame}[fragile]
	\frametitle{Funzionalità deprecate/rimosse}
	\framesubtitle{Parametri modificati in Lowlevel VersionsCommand (2/2)}

	\small\textit{Continuazione}\normalsize

	\begin{itemize}
		\item ...
			\begin{itemize}
				\item \texttt{-}\texttt{-depth=4} o \texttt{-d=4} per ripulire in modo ricorsivo fino ad un livello dell'alberatura di pagina
				\item \texttt{-}\texttt{-dry-run} visualizza solamente i record da modificare / eliminati
				\item \texttt{-}\texttt{-action=nameofaction} per pulire i record di versione,
					sono possibili le seguenti azioni
					\begin{itemize}
						\item \texttt{versions\_in\_live}: Cancella i record versionati del workspace live
						\item \texttt{published\_versions}: Cancella le versioni dei record pubblicati
						\item \texttt{invalid\_workspace}: Sposta i record in un ID di workspace non esistente nel workspace live 
						\item \texttt{unused\_placeholders}: Rimuove i placeholder che non sono più utilizzati nel database
					\end{itemize}
			\end{itemize}
	\end{itemize}

\end{frame}

% ------------------------------------------------------------------------------
% LTXE-SLIDE-START
% LTXE-SLIDE-UID:		56e63427-554fdca0-bf8fcbe6-df95fef6
% LTXE-SLIDE-ORIGIN:	35714a8f-ad99f4e7-ccda0780-c0aa44aa English
% LTXE-SLIDE-TITLE:		Default Layouts for Fluid Styled Content Changed
% LTXE-SLIDE-REFERENCE:	!Breaking: #79622 - Default layouts for Fluid Styled Content changed
% ------------------------------------------------------------------------------

\begin{frame}[fragile]
	\frametitle{Funzionalità deprecate/rimosse}
	\framesubtitle{Cambiati i layout di default per Fluid Styled Content}

	% decrease font size for code listing
	\lstset{basicstyle=\tiny\ttfamily}

	\begin{itemize}
		\item I layout degli elementi di contenuto di Fluid Styled Content sono stati modificati per avere
			una manutenzione migliore ed essere più flessibili
			
		\item I layout precedentemente disponibili \texttt{ContentFooter}, \texttt{HeaderFooter} e
			\texttt{HeaderContentFooter} sono stati rimossi e sostituiti dal singolo layout \texttt{Default} 
			che è più flessibile.

			\begin{lstlisting}
				$GLOBALS['TCA']['tt_content']['columns']['CType']['config']['default'] = 'textmedia';
				$GLOBALS['TCA']['tt_content']['columns']['CType']['config']['default'] = 'header';
			\end{lstlisting}

	\end{itemize}

\end{frame}

% ------------------------------------------------------------------------------
% LTXE-SLIDE-START
% LTXE-SLIDE-UID:		d09bffd0-a5da6dc8-b568c2fb-991225d2
% LTXE-SLIDE-ORIGIN:	0e36fc45-799c2e27-3f058488-8173f554 English
% LTXE-SLIDE-TITLE:		TypoScript Standard Header (1/2)
% LTXE-SLIDE-REFERENCE:	!Breaking: #79622 - TypoScript Standard Header has been removed from Fluid Styled Content
% ------------------------------------------------------------------------------

\begin{frame}[fragile]
	\frametitle{Funzionalità deprecate/rimosse}
	\framesubtitle{Intestazione TypoScript Standard (1/2)}

	\begin{itemize}
		\item La definizione dell'intestazione di rendering standard di TypoScript \texttt{lib.stdHeader} è stata introdotta in
			CSS Styled Content per poter farvi riferimento in più elementi di contenuto e facilitare la manutenzione.

		\item In Fluid Styled Content un workaround per la compatibilita al CMS 7 è stato introdotto per semplificare la migrazione.
			Tuttavia, esso gestisce solamente l'intestazione e manca di tutti i frame, e le opzioni addizionali sono necessarie
			per generare un output snello se il layout dell'elemento di contenuto non è implementato correttamente.

	\end{itemize}

\end{frame}

% ------------------------------------------------------------------------------
% LTXE-SLIDE-START
% LTXE-SLIDE-UID:		84236ba2-93cb27cc-8793a7b1-7091d20e
% LTXE-SLIDE-ORIGIN:	ca67c35f-cf7489b1-dd2ae6a2-f3557c53 English
% LTXE-SLIDE-TITLE:		TypoScript Standard Header (2/2)
% LTXE-SLIDE-REFERENCE:	!Breaking: #79622 - TypoScript Standard Header has been removed from Fluid Styled Content
% ------------------------------------------------------------------------------

\begin{frame}[fragile]
	\frametitle{Funzionalità deprecate/rimosse}
	\framesubtitle{Intestazione TypoScript Standard (2/2)}

	% decrease font size for code listing
	\lstset{basicstyle=\tiny\ttfamily}

	\begin{itemize}

		\item Output ora:

			\begin{lstlisting}
				tt_content.simple_content = COA
				tt_content.simple_content {
				  10 < lib.stdHeader
				  20 = TEXT
				  20.field = bodytext
				}

				<header>
				  <h1>Nunc vel libero dignissim</h1>
				</header>
				<p>
				  ...
				</p>
			\end{lstlisting}

	\end{itemize}

\end{frame}

% ------------------------------------------------------------------------------
% LTXE-SLIDE-START
% LTXE-SLIDE-UID:		8d9882c5-fffc075e-1ba204eb-8aadd53b
% LTXE-SLIDE-ORIGIN:	64df6081-eb1f0a13-7f8292d3-9c69a293 English
% LTXE-SLIDE-TITLE:		Miscellaneous (1/4)
% ------------------------------------------------------------------------------

\begin{frame}[fragile]
	\frametitle{Funzionalità deprecate/rimosse}
	\framesubtitle{Varie (1/4)}

	% !Breaking: #78477 - Remove method FlashMessage->getMessageAsMarkup()
	% !Breaking: #79100 - ext:felogin: Remove default CSS
	% !Breaking: #79201 - EXT:form: Split TypoScript Includes
	% !Breaking: #79242 - Remove l10n_mode noCopy
	% !Breaking: #79243 - Remove l10n_mode mergeIfNotBlank
	% !Breaking: #79243 - Remove sys_language_softMergeIfNotBlank

	\begin{itemize}
		\item I metodi seguenti sono stati rimossi:\newline
			\small\texttt{FlashMessage->getMessageAsMarkup()}\normalsize
		\item \texttt{EXT:felogin} non aggiunge più lo stile CSS di default perchè esso
			poteva rompere l'output di frontend, per esempio se era utilizzato un framework CSS.
		\item Il setup specifico di TypoScript per \texttt{EXT:form} non è più caricato automaticamente
			e deve essere aggiunto manualmente attraverso l'inclusione statica.
			Con questo cambiamento un integratore TYPO3 può facilmente decidere dove l'estensione TypoScript viene inclusa.
		\item L'impostazione \texttt{noCopy} è stata rimossa senza sostituzioni dalla lista 
			dei possibili valori della proprietà di colonna TCA \texttt{l10n\_mode}.
		\item L'impostazione \texttt{mergeIfNotBlank} è stata rimossa senza sostituzioni dalla lista
			dei possibili valori della proprietà di colonna TCA \texttt{l10n\_mode}.

	\end{itemize}

\end{frame}


% ------------------------------------------------------------------------------
% LTXE-SLIDE-START
% LTXE-SLIDE-UID:		399ebab8-3043d0e2-b3c22483-b217a7c2
% LTXE-SLIDE-ORIGIN:	9b7c4a92-17f663e5-b890e0b9-1b039f82 English
% LTXE-SLIDE-TITLE:		Miscellaneous (2/4)
% ------------------------------------------------------------------------------

\begin{frame}[fragile]
	\frametitle{Funzionalità deprecate/rimosse}
	\framesubtitle{Varie (2/4)}

	% !Breaking: #79302 - Moved pages.url_scheme to compatibility7 extension
	% !Breaking: #79364 - Move page module function `QuickEdit` to compatibility7
	% !Breaking: #79622 - CSS Styled Content and TypoScript

	\begin{itemize}
		\item L'impostazione TypoScript \texttt{config.sys\_language\_softMergeIfNotBlank}
			è stata rimossa senza sostituzioni. Questo è un risultato della rimozione
			dell'impostazione TCA \texttt{mergeIfNotBlank} dalla lista dei possibili valori per \texttt{l10n\_mode}.

		\item La funzionalità del campo del database \texttt{pages.url\_scheme} è stata spostata
			nell'estensione compatibility7. Il campo permette di forzare il protocollo HTTP o HTTPS
			per specifiche pagine da parte di un redattore nelle proprietà di pagina. 
			Tuttavia, oggi è comune garantire (se un certificato SSL è disponibile)
			l'uso di HTTPS per un intero sito o anche solamente per una determinata area
			(es. sottopagine) per forzare il protocollo.

	\end{itemize}

\end{frame}

% ------------------------------------------------------------------------------
% LTXE-SLIDE-START
% LTXE-SLIDE-UID:		d7d1a11b-68eb3ede-dd66c71f-0cb18dc6
% LTXE-SLIDE-ORIGIN:	6b0c8695-0ad6167d-2eda453e-5b16dd49 English
% LTXE-SLIDE-TITLE:		Miscellaneous (3/4)
% LTXE-SLIDE-REFERENCE:	!Deprecation: #77934 - Deprecate tt_content field select_key
% LTXE-SLIDE-REFERENCE:	!Deprecation: #78477 - Refactoring of FlashMessage rendering
% LTXE-SLIDE-REFERENCE:	!Deprecation: #79316 - Deprecate ArrayUtility::inArray()
% LTXE-SLIDE-REFERENCE:	!Deprecation: #79622 - Deprecation of CSS Styled Content
% ------------------------------------------------------------------------------

\begin{frame}[fragile]
	\frametitle{Funzionalità deprecate/rimosse}
	\framesubtitle{Varie (3/4)}

	\begin{itemize}
		\item La funzione \texttt{QuickEdit} nel modulo di pagina è stata spostata in
			\texttt{EXT:compatibility7} e non sarà sviluppata in futuro.\newline
			Vedi \href{https://typo3.org/extensions/repository}{TYPO3 Extension Repository (TER)}.

		\item Al fine di razionalizzare CSS Styled Content e Fluid Styled Content, molte opzioni
			di CSS Styled Content sono state rimosse senza sostituzione:
			\texttt{TCA image\_compression}, \texttt{TCA image\_effects}, \texttt{TCA image\_noRows},
			\texttt{TypoScript IMAGE noRows}, \texttt{TypoScript IMAGE noCols},
			\texttt{TypoScript IMAGE noRowsStdWrap}, \texttt{TypoScript IMGTEXT captionAlign}

		\item Il campo \texttt{select\_key} della tabella \texttt{tt\_content} non è usato nel core
			ed è stato rimosso.

	\end{itemize}

\end{frame}

% ------------------------------------------------------------------------------
% LTXE-SLIDE-START
% LTXE-SLIDE-UID:		dcd2f134-a5f4c18b-7ba8a547-5313de52
% LTXE-SLIDE-ORIGIN:	3fe05cb6-a2648210-5486e0fa-93f0f02a English
% LTXE-SLIDE-TITLE:		Miscellaneous (4/4)
% LTXE-SLIDE-REFERENCE:	!Deprecation: #77934 - Deprecate tt_content field select_key
% LTXE-SLIDE-REFERENCE:	!Deprecation: #78477 - Refactoring of FlashMessage rendering
% LTXE-SLIDE-REFERENCE:	!Deprecation: #79316 - Deprecate ArrayUtility::inArray()
% LTXE-SLIDE-REFERENCE:	!Deprecation: #79622 - Deprecation of CSS Styled Content
% ------------------------------------------------------------------------------

\begin{frame}[fragile]
	\frametitle{Funzionalità deprecate/rimosse}
	\framesubtitle{Varie (4/4)}

	\begin{itemize}

		\item I seguenti metodi e proprietà in \texttt{FlashMessage::class}
			sono stati segnati come deprecati:

			\begin{itemize}
				\item \texttt{FlashMessage->classes}
				\item \texttt{FlashMessage->icons}
				\item \texttt{FlashMessage->getClass()}
				\item \texttt{FlashMessage->getIconName()}
			\end{itemize}

		\item Il metodo \texttt{ArrayUtility::inArray()} è stato segnato come deprecato

		\item CSS Styled Content è ora deprecato\newline
			\small(sarà rimosso in TYPO3 CMS version 9)\normalsize

	\end{itemize}

\end{frame}

% ------------------------------------------------------------------------------
