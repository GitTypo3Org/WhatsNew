% ------------------------------------------------------------------------------
% TYPO3 CMS 8.6 - What's New - Chapter "In-Depth Changes" (Italian Version)
%
% @author	Michael Schams <schams.net>
% @license	Creative Commons BY-NC-SA 3.0
% @link		http://typo3.org/download/release-notes/whats-new/
% @language	English
% ------------------------------------------------------------------------------
% LTXE-CHAPTER-UID:		5ebcecbe-66abfa57-cf38bc00-aa637965
% LTXE-CHAPTER-NAME:	In-Depth Changes
% ------------------------------------------------------------------------------

\section{Modifiche rilevanti}
\begin{frame}[fragile]
	\frametitle{Modifiche rilevanti}

	\begin{center}\huge{Capitolo 3:}\end{center}
	\begin{center}\huge{\color{typo3darkgrey}\textbf{Modifiche rilevanti}}\end{center}

\end{frame}

% ------------------------------------------------------------------------------
% LTXE-SLIDE-START
% LTXE-SLIDE-UID:		d300fa50-d9ba70fc-e70cf258-8dccfdcd
% LTXE-SLIDE-ORIGIN:	5ae3ff92-e2f03af9-8f5caf79-b7e0aca1 English
% LTXE-SLIDE-TITLE:		Page Browser for scheduler tasks
% LTXE-SLIDE-REFERENCE:	!Feature: #12211 - Usability: Scheduler provide page browser to choose start page
% ------------------------------------------------------------------------------

\begin{frame}[fragile]
	\frametitle{Modifiche rilevanti}
	\framesubtitle{Page Browser per attività dello scheduler}

	\begin{itemize}
		\item Le attività dello scheduler che necessitano di \texttt{uid} di pagina possono aggiungere un bottone per il popup di scelta pagina.

		\item E' possibile aggiungere due nuovi campi nel \texttt{ValidatorTaskAdditionalFieldProvider}.

		\item Se il campo aggiuntivo \texttt{browser} è impostato a \texttt{page} lo
			\texttt{SchedulerModuleController} aggiunge un bottone per aprire il popup di scelta pagina.

			\begin{lstlisting}
				'browser' => 'page',
			\end{lstlisting}

		\item Nel campo \texttt{pageTitle} indicare il titolo della pagina che viene mostrata cliccando sul bottone di scelta.

			\begin{lstlisting}
				'pageTitle' => $pageTitle,
			\end{lstlisting}

	\end{itemize}

\end{frame}

% ------------------------------------------------------------------------------
% LTXE-SLIDE-START
% LTXE-SLIDE-UID:		47284c9b-99d5be8a-4deea2d7-e4cd6e6f
% LTXE-SLIDE-ORIGIN:	02a7fe32-ae3fd901-b04372a2-7d3726f4 English
% LTXE-SLIDE-TITLE:		Synchronized field values in localized records (1/2)
% LTXE-SLIDE-REFERENCE:	!Feature: #51291 - Synchronized field values in localized records
% ------------------------------------------------------------------------------

\begin{frame}[fragile]
	\frametitle{Modifiche rilevanti}
	\framesubtitle{Sincronizzazione dei valori dei campi nei record tradotti (1/2)}

	% decrease font size for code listing
	\lstset{basicstyle=\tiny\ttfamily}

	\begin{itemize}
		\item Il comportamento di sovrapposizione dei record tradotti è stato modificato per rendere la traduzione di riga indipendente.

		\item Precedentemente, se il campo \texttt{TCA} di una voce era impostatato a \texttt{l10n\_mode} \texttt{exclude} o
			\texttt{mergeIfNotBlank}, la sovrapposizione di traduzione registrata non conteneva valori ed esso era ricavato dal record della lingua principale.

	\end{itemize}

\end{frame}

% ------------------------------------------------------------------------------
% LTXE-SLIDE-START
% LTXE-SLIDE-UID:		775bc8db-0ce74388-06c6412f-d7dd7141
% LTXE-SLIDE-ORIGIN:	cf030e7c-03fb0ac4-9b9ff848-ac69581b English
% LTXE-SLIDE-TITLE:		Synchronized field values in localized records (2/2)
% LTXE-SLIDE-REFERENCE:	!Feature: #51291 - Synchronized field values in localized records
% ------------------------------------------------------------------------------

\begin{frame}[fragile]
	\frametitle{Modifiche rilevanti}
	\framesubtitle{Sincronizzazione dei valori dei campi nei record tradotti (2/2)}

	% decrease font size for code listing
	\lstset{basicstyle=\tiny\ttfamily}

	\begin{itemize}
		\item Ora è modificato, il \texttt{DataHandler} copia il valore nel record tradotto
			e li sincronizza se il record della lingua principale viene modificato.

			\begin{lstlisting}
				'columns' => [
				  ...
				  'header' => [
				    'label' => 'My header',
				    'config' => [
				      'type' => 'input',
				      'behaviour' => [
				        'allowLanguageSynchronization' => true,
				      ],
				    ],
				  ],
				],
			\end{lstlisting}

	\end{itemize}

\end{frame}

% ------------------------------------------------------------------------------
% LTXE-SLIDE-START
% LTXE-SLIDE-UID:		5d8f1e83-1060ac30-829d23aa-07d6d89e
% LTXE-SLIDE-ORIGIN:	18c4f826-440f3e00-a854e8e8-ddc7cbc0 English
% LTXE-SLIDE-TITLE:		Image Manipulation Tool (1/6)
% LTXE-SLIDE-REFERENCE:	!Feature: #75880 - Implement multiple cropping variants in image manipulation tool
% ------------------------------------------------------------------------------

\begin{frame}[fragile]
	\frametitle{Modifiche rilevanti}
	\framesubtitle{Tool di manipolazione immagini (1/6)}

	% decrease font size for code listing
	\lstset{basicstyle=\tiny\ttfamily}

	\begin{itemize}
		\item La funzionalità del TCA \texttt{imageManipulation} è ora in grado di gestire molteplici varianti di ritaglio, se configurato.

		\item E' possibile fornire un'area iniziale di ritaglio. Se nessuna area di ritaglio iniziale viene definita, l'area di ritaglio di default
			impostata sarà l'immagine completa.
		\item Le aree di ritaglio sono definite con numeri in virgola mobile. Le coordinate e le dimensioni vanno definite
			per questo motivo.

	\end{itemize}

\end{frame}

% ------------------------------------------------------------------------------
% LTXE-SLIDE-START
% LTXE-SLIDE-UID:		52b39938-b13e091f-79f463d0-c5b4e878
% LTXE-SLIDE-ORIGIN:	36ea74c0-3b92dbf6-e35c9e25-af0a0abf English
% LTXE-SLIDE-TITLE:		Image Manipulation Tool (2/6)
% LTXE-SLIDE-REFERENCE:	!Feature: #75880 - Implement multiple cropping variants in image manipulation tool
% ------------------------------------------------------------------------------

\begin{frame}[fragile]
	\frametitle{Modifiche rilevanti}
	\framesubtitle{Tool di manipolazione immagini (2/6)}

	% decrease font size for code listing
	\lstset{basicstyle=\tiny\ttfamily}

	\begin{itemize}

		\item L'esempio seguente configura due varianti di ritaglio,
			la prima con id "mobile", la seconda con id "desktop".
			La chiave dell'array definisce l'id, che sarà utilizzato durante
			il rendering dell'immagine nel view helper dell'immagine.

			\begin{lstlisting}
				'config' => [
				  'type' => 'imageManipulation',
				  'cropVariants' => [
				    'mobile' => [
				      'title' => 'Mobile',
				      'allowedAspectRatios' => [
				        '4:3' => [
				          'title' => '4:3',
				          'value' => 4 / 3
				        ],
				        ...
				      ],
				    ],
				    'desktop' => [
				      ...
				    ],
				  ],
				]
			\end{lstlisting}

	\end{itemize}

\end{frame}

% ------------------------------------------------------------------------------
% LTXE-SLIDE-START
% LTXE-SLIDE-UID:		72f22dce-c9fada71-efbbce67-4dbcabd5
% LTXE-SLIDE-ORIGIN:	27b20030-548f80c4-3540aabb-41a83498 English
% LTXE-SLIDE-TITLE:		Image Manipulation Tool (3/6)
% LTXE-SLIDE-REFERENCE:	!Feature: #75880 - Implement multiple cropping variants in image manipulation tool
% ------------------------------------------------------------------------------

\begin{frame}[fragile]
	\frametitle{Modifiche rilevanti}
	\framesubtitle{Tool di manipolazione immagini (3/6)}

	% decrease font size for code listing
	\lstset{basicstyle=\tiny\ttfamily}

	\begin{itemize}

		\item Il seguente esempio ha un'area di ritaglio iniziale della dimensione dell'immagine precedentemente
				ritagliata fornita di default.

			\begin{lstlisting}
				'config' => [
				  'type' => 'imageManipulation',
				  'cropVariants' => [
				    'mobile' => [
				      'title' => 'LLL:EXT:ext_key/Resources/Private/Language/locallang.xlf:imageManipulation.mobile',
				      'cropArea' => [
				        'x' => 0.1,
				        'y' => 0.1,
				        'width' => 0.8,
				        'height' => 0.8,
				      ],
				    ],
				  ],
				]
			\end{lstlisting}

	\end{itemize}

\end{frame}

% ------------------------------------------------------------------------------
% LTXE-SLIDE-START
% LTXE-SLIDE-UID:		e1cdab43-8b74e87d-2c6e1e1c-45ba8036
% LTXE-SLIDE-ORIGIN:	58cebeda-4ee109f4-5e9eb728-b7703b40 English
% LTXE-SLIDE-TITLE:		Image Manipulation Tool (4/6)
% LTXE-SLIDE-REFERENCE:	!Feature: #75880 - Implement multiple cropping variants in image manipulation tool
% ------------------------------------------------------------------------------

\begin{frame}[fragile]
	\frametitle{Modifiche rilevanti}
	\framesubtitle{Tool di manipolazione immagini (4/6)}

	% decrease font size for code listing
	\lstset{basicstyle=\tiny\ttfamily}

	\begin{itemize}
		\item Gli utenti possono configurare anche un'area di focus, quando configurato.
		\item L'area di focus è sempre dentro l'area ritagliata e definisce l'area dell'immagine
			che deve essere visibile perchè l'immagine abbia significato.

			\begin{lstlisting}
				'config' => [
				  'type' => 'imageManipulation',
				  'cropVariants' => [
				    'mobile' => [
				      'title' =>
				        'LLL:EXT:ext_key/Resources/Private/Language/locallang.xlf:imageManipulation.mobile',
				      'focusArea' => [
				        'x' => 1 / 3,
				        'y' => 1 / 3,
				        'width' => 1 / 3,
				        'height' => 1 / 3,
				      ],
				    ],
				  ],
				]
			\end{lstlisting}

	\end{itemize}

\end{frame}

% ------------------------------------------------------------------------------
% LTXE-SLIDE-START
% LTXE-SLIDE-UID:		aa6daa5d-8121b3c7-e444a4f8-6912f311
% LTXE-SLIDE-ORIGIN:	94d809e0-3369bf18-616710ee-93f6a9da English
% LTXE-SLIDE-TITLE:		Image Manipulation Tool (5/6)
% LTXE-SLIDE-REFERENCE:	!Feature: #75880 - Implement multiple cropping variants in image manipulation tool
% ------------------------------------------------------------------------------

\begin{frame}[fragile]
	\frametitle{Modifiche rilevanti}
	\framesubtitle{Tool di manipolazione immagini (5/6)}

	% decrease font size for code listing
	\lstset{basicstyle=\tiny\ttfamily}

	\begin{itemize}
		\item Molto spesso le immagini sono usate in un contesto, dove sono sovrapposte con altri elementi del DOM come ad esempio il titolo.
		\item Per dare un indicazione ai redattori, quando fanno un operazione di ritaglio, di quale area dell'immagine è influenzata 
			è possibile definire più aree di copertura.
		\item Queste aree sono mostrate all'interno dell'area di ritaglio. L'area di fuoco non potrà intersecarsi con nessuna delle aree di copertura.

			\begin{lstlisting}
				'config' => [
				  'type' => 'imageManipulation',
				  'coverAreas' => [
				    [
				      'x' => 0.05, 'y' => 0.85,
				      'width' => 0.9, 'height' => 0.1,
				    ],
				  ],
				]
			\end{lstlisting}

	\end{itemize}

\end{frame}

% ------------------------------------------------------------------------------
% LTXE-SLIDE-START
% LTXE-SLIDE-UID:		92a55fa2-011e5355-eb464901-739f0f5a
% LTXE-SLIDE-ORIGIN:	c9df3305-35544e3c-6a073be7-1cec40f4 English
% LTXE-SLIDE-TITLE:		Image Manipulation Tool (6/6)
% LTXE-SLIDE-REFERENCE:	!Feature: #75880 - Implement multiple cropping variants in image manipulation tool
% ------------------------------------------------------------------------------

\begin{frame}[fragile]
	\frametitle{Modifiche rilevanti}
	\framesubtitle{Tool di manipolazione immagini (6/6)}

	% decrease font size for code listing
	\lstset{basicstyle=\smaller\ttfamily}

	\begin{itemize}
		\item Per renderizzare le varianti di ritaglio, esse dovranno essere indicate come argomenti nel view helper dell'immagine:

			\begin{lstlisting}
				<f:image image="{data.image}" cropVariant="mobile" width="800" >
				</f:image>
			\end{lstlisting}

	\end{itemize}

\end{frame}

% ------------------------------------------------------------------------------
% LTXE-SLIDE-START
% LTXE-SLIDE-UID:		0abec121-820882a2-12acd924-2d5e69ae
% LTXE-SLIDE-ORIGIN:	a71d77d1-6a419c8d-0fbb35d0-e1ce57e4 English
% LTXE-SLIDE-TITLE:		Default Content Element Changed for Fluid Styled Content
% LTXE-SLIDE-REFERENCE:	!Breaking: #79622 - Default content element changed for Fluid Styled Content
% ------------------------------------------------------------------------------

\begin{frame}[fragile]
	\frametitle{Modifiche rilevanti}
	\framesubtitle{Elemento di contenuto predefinito modificato per Fluid Styled Content}

	% decrease font size for code listing
	\lstset{basicstyle=\tiny\ttfamily}

	\begin{itemize}
		\item L'elemento di contenuto predefinito è stato semplificato con CSS Styled Content ed è cambiato in "Text"
		\item Per ripristinare la configurazione è necessario impostare manualmente l'elemento di contenuto predefinito con il preferito.
			E' possibile farlo sovrascrivendo la configurazione nel file
			\texttt{Configuration/TCA/Overrides/tt\_content.php}.

			\begin{lstlisting}
				$GLOBALS['TCA']['tt_content']['columns']['CType']['config']['default'] = 'textmedia';
				$GLOBALS['TCA']['tt_content']['columns']['CType']['config']['default'] = 'header';
			\end{lstlisting}

	\end{itemize}

\end{frame}

% ------------------------------------------------------------------------------
% LTXE-SLIDE-START
% LTXE-SLIDE-UID:		59692355-14bb7b00-49597df2-4708220e
% LTXE-SLIDE-ORIGIN:	035f6a19-9be2f793-bc0ebb3e-c6a4cca4 English
% LTXE-SLIDE-TITLE:		TCA Changes (1/2)
% LTXE-SLIDE-REFERENCE:	!Deprecation: #79440 - TCA Changes
% ------------------------------------------------------------------------------

\begin{frame}[fragile]
	\frametitle{Modifiche rilevanti}
	\framesubtitle{Cambiamenti TCA (1/2)}

	\begin{itemize}

		\item Il \texttt{TCA} è cambiato nel livello dei campi.

		\item Quasi tutti i tipi di colonna sono interessati.

		\item In generale, la sottosezione \texttt{wizards} è sostituita da una combinazione di nuovi
			\texttt{renderType} e da una nuova serie di opzioni di configurazione.

		\item I wizards sono ora suddivisi in tre tipi differenti: \texttt{fieldInformation}, \texttt{fieldControl}
			e \texttt{fieldWizard}.

	\end{itemize}

\end{frame}
% ------------------------------------------------------------------------------
% LTXE-SLIDE-START
% LTXE-SLIDE-UID:		fb9c6093-36f330bc-2fcc58ef-97d0c971
% LTXE-SLIDE-ORIGIN:	55c8113e-b234daa0-e701f035-8e1b58d1 English
% LTXE-SLIDE-TITLE:		TCA Changes (2/2)
% LTXE-SLIDE-REFERENCE:	!Deprecation: #79440 - TCA Changes
% ------------------------------------------------------------------------------

\begin{frame}[fragile]
	\frametitle{Modifiche rilevanti}
	\framesubtitle{Cambiamenti TCA (2/2)}

	% decrease font size for code listing
	\lstset{basicstyle=\tiny\ttfamily}

	\begin{itemize}
		\item Esempio:

			\begin{lstlisting}
				'fieldControl' => [
				  'editPopup' => [
				    'disabled' => false,
				  ],
				  'addRecord' => [
				    'disabled' => false,
				    'options' => [
				      'setValue' => 'prepend',
				    ],
				  ],
				  'listModule' => [
				    'disabled' => false,
				  ],
				],
			\end{lstlisting}

		\item Potete trovare ulteriori dettagli su
			\href{https://docs.typo3.org/typo3cms/extensions/core/8-dev/singlehtml/Index.html#deprecation-79440-formengine-element-expansion}{docs.typo3.org}

	\end{itemize}

\end{frame}


% ------------------------------------------------------------------------------
% LTXE-SLIDE-START
% LTXE-SLIDE-UID:		15786d0e-8aa7f055-7b79c577-200cf0b5
% LTXE-SLIDE-ORIGIN:	da922107-1604619b-66f713fe-9508d9d7 English
% LTXE-SLIDE-TITLE:		Introduce Session Storage Framework
% LTXE-SLIDE-REFERENCE:	!Feature: #70316 - Introduce Session Storage Framework
% ------------------------------------------------------------------------------

\begin{frame}[fragile]
	\frametitle{Modifiche rilevanti}
	\framesubtitle{Introdotto Session Storage Framework}

	\begin{itemize}
		\item Un nuovo session storage framework è stato introdotto
		\item L'obiettivo di questo framework è di creare interoperabilità tra i diversi archivi di sessioni
			(chiamati "backends") come database, archiviazione file, Redis, etc.
		\item I seguenti backend di sessioni sono disponibili per impostazioni predefinite:

			\begin{itemize}
				\item
					\smaller\texttt{\textbackslash
						TYPO3\textbackslash
						CMS\textbackslash
						Core\textbackslash
						Session\textbackslash
						Backend\textbackslash
						DatabaseSessionBackend}

				\item
					\texttt{\textbackslash
						TYPO3\textbackslash
						CMS\textbackslash
						Core\textbackslash
						Session\textbackslash
						Backend\textbackslash
						RedisSessionBackend}
			\end{itemize}
	\end{itemize}

\end{frame}

% ------------------------------------------------------------------------------
% LTXE-SLIDE-START
% LTXE-SLIDE-UID:		e7907e42-ef4643a0-b63375d4-8bd695cd
% LTXE-SLIDE-ORIGIN:	2d65beca-277a2f35-a1bc5ae7-c71712aa English
% LTXE-SLIDE-TITLE:		CLI Support for T3D Imports
% LTXE-SLIDE-REFERENCE:	!Feature: #72749 - CLI support for T3D import
% ------------------------------------------------------------------------------

\begin{frame}[fragile]
	\frametitle{Modifiche rilevanti}
	\framesubtitle{Supporto CLI per importazioni T3D}

	\begin{itemize}
		\item \texttt{EXT:impexp} permette ora di importare file di dati (T3D o XML) tramite linea di comando
			interfacciandosi attraverso un comando Symfony.

		\item Utilizza:\newline
			\smaller
				\texttt{./typo3/sysext/core/bin/typo3 impexp:import [<options>] <file> <pageId>}
			\normalsize

		\item Opzioni:
			\begin{itemize}
				\item \texttt{-}\texttt{-updateRecords}: Forza l'aggiornamento di record esistenti
				\item \texttt{-}\texttt{-ignorePid}: Non corregge gli id di pagina dei record aggiornati
				\item \texttt{-}\texttt{-enableLog}: registra nel log tutte le azioni di database.
			\end{itemize}

	\end{itemize}

\end{frame}

% ------------------------------------------------------------------------------
% LTXE-SLIDE-START
% LTXE-SLIDE-UID:		8f3bccb4-cc6e559e-9891ff70-4ca398d0
% LTXE-SLIDE-ORIGIN:	38064844-53baa89a-7f73669e-2bb96cab English
% LTXE-SLIDE-TITLE:		Hook in typolink for Modification of Page Params
% LTXE-SLIDE-REFERENCE:	!Feature: #79121 - Implement hook in typolink for modification of page params
% ------------------------------------------------------------------------------

\begin{frame}[fragile]
	\frametitle{Modifiche rilevanti}
	\framesubtitle{Inserito Hook in \texttt{typolink} per la modifica dei parametri di pagina}

	% decrease font size for code listing
	\lstset{basicstyle=\tiny\ttfamily}

	\begin{itemize}
		\item Un nuovo hook è stato implementato in \texttt{ContentObjectRenderer::typoLink} per i link alle pagine.
			Con questo hook è possibile modificare la configurazione dei link, per esempio arricchendola con parametri
			in più o meta dati della pagina.

		\item Si può registrare un hook via:

			\begin{lstlisting}
				$GLOBALS['TYPO3_CONF_VARS']['SC_OPTIONS']['typolinkProcessing']
				  ['typolinkModifyParameterForPageLinks'][] = \Your\Namespace\Hooks\MyHook::class;
			\end{lstlisting}

		\item Usare:

			\begin{lstlisting}
				public function modifyPageLinkConfiguration(
				  array $linkConfiguration, array $linkDetails, array $pageRow) : array
				{
				  $linkConfiguration['additionalParams'] .= $pageRow['myAdditionalParamsField'];
				  return $linkConfiguration;
				}
			\end{lstlisting}

	\end{itemize}

\end{frame}

% ------------------------------------------------------------------------------
% LTXE-SLIDE-START
% LTXE-SLIDE-UID:		2ad4b8bc-34a39803-42238e28-6b10fc35
% LTXE-SLIDE-ORIGIN:	290e1c2e-e708ede1-82aaf127-e69b2338 English
% LTXE-SLIDE-TITLE:		Hook to Add Custom TypoScript Templates (1/2)
% LTXE-SLIDE-REFERENCE:	!Feature: #79140 - Add hook to add custom TypoScript templates
% ------------------------------------------------------------------------------

\begin{frame}[fragile]
	\frametitle{Modifiche rilevanti}
	\framesubtitle{Hook per aggiungere template TypoScript personalizzati (1/2)}

	% decrease font size for code listing
	\lstset{basicstyle=\tiny\ttfamily}

	\begin{itemize}
		\item Un nuovo hook in TemplateService permette di aggiungere o modificare template TypoScript esistenti.

		\item Si può ora registrare un hook tramite il seguente codice in un estensione, nel file \texttt{ext\_localconf.php}:

			\begin{lstlisting}
				$GLOBALS['TYPO3_CONF_VARS']['SC_OPTIONS']['Core/TypoScript/TemplateService']
				  ['runThroughTemplatesPostProcessing']
			\end{lstlisting}

		\item \texttt{EXT:my\_site/Classes/Hooks/TypoScriptHook.php} (1/2)

			\begin{lstlisting}
				namespace MyVendor\MySite\Hooks;
				class TypoScriptHook
				{
				  /**
				   * Hooks into TemplateService after
				   * @param array $parameters
				   * @param \TYPO3\CMS\Core\TypoScript\TemplateService $parentObject
				   * @return void
				   */
				...
			\end{lstlisting}

	\end{itemize}

\end{frame}

% ------------------------------------------------------------------------------
% LTXE-SLIDE-START
% LTXE-SLIDE-UID:		6a4a75be-89fee49b-f6f22813-24729d1b
% LTXE-SLIDE-ORIGIN:	d64368f4-0f5c43e3-92a546bf-75d50a61 English
% LTXE-SLIDE-TITLE:		Hook to Add Custom TypoScript Templates (2/2)
% LTXE-SLIDE-REFERENCE:	!Feature: #79140 - Add hook to add custom TypoScript templates
% ------------------------------------------------------------------------------

\begin{frame}[fragile]
	\frametitle{Modifiche rilevanti}
	\framesubtitle{Hook per aggiungere template TypoScript personalizzati (2/2)}

	% decrease font size for code listing
	\lstset{basicstyle=\tiny\ttfamily}

	\begin{itemize}
		\item \texttt{EXT:my\_site/Classes/Hooks/TypoScriptHook.php} (2/2)

			\begin{lstlisting}
				...
				  public function addCustomTypoScriptTemplate($parameters, $parentObject)
				  {
				    // Disable the inclusion of default TypoScript set via TYPO3_CONF_VARS
				    $parameters['isDefaultTypoScriptAdded'] = true;
				    // Disable the inclusion of ext_typoscript_setup.txt of all extensions
				    $parameters['processExtensionStatics'] = false;

				    // No template was found in rootline so far, so a custom "fake" sys_template record is added
				    if ($parentObject->outermostRootlineIndexWithTemplate === 0) {
				      $row = [
				        'uid' => 'my_site_template',
				        'config' =>
					      '<INCLUDE_TYPOSCRIPT: source="FILE:EXT:my_site/Configuration/TypoScript/site_setup.t3s">',
				        'root' => 1,
				        'pid' => 0
				      ];
				      $parentObject->processTemplate($row, 'sys_' . $row['uid'], 0, 'sys_' . $row['uid']);
				    }
				  }
				}
			\end{lstlisting}

	\end{itemize}

\end{frame}

% ------------------------------------------------------------------------------
% LTXE-SLIDE-START
% LTXE-SLIDE-UID:		4171b080-06c38af1-a101326f-885d6371
% LTXE-SLIDE-ORIGIN:	ad9f66e3-bcd548f4-8e14e04c-de29a79c English
% LTXE-SLIDE-TITLE:		Plugin Preview with Fluid
% LTXE-SLIDE-REFERENCE:	!Feature: #79225 - Plugin preview with Fluid
% ------------------------------------------------------------------------------
\begin{frame}[fragile]
	\frametitle{Modifiche rilevanti}
	\framesubtitle{Plugin Anteprima con Fluid}

	% decrease font size for code listing
	\lstset{basicstyle=\tiny\ttfamily}

	\begin{itemize}
		\item Il TSconfig di pagina per renderizzare un anteprima del contenuto singolo di un elemento nel backend è stato migliorato
			consentendo il rendering del plugin attraverso Fluid.

		\item Tutte le proprietà del record \texttt{tt\_content} sono disponibili direttamente nel template (es. UID via \{uid\})

		\item Qualsiasi dato dei campi flexform \texttt{pi\_flexform} è disponibile come array con la proprietà
			\texttt{pi\_flexform\_transformed}.

			\begin{lstlisting}
				mod.web_layout.tt_content.preview.list.simpleblog_bloglisting =
				  EXT:simpleblog/Resources/Private/Templates/Preview/SimpleblogPlugin.html
			\end{lstlisting}

	\end{itemize}

\end{frame}

% ------------------------------------------------------------------------------
% LTXE-SLIDE-START
% LTXE-SLIDE-UID:		11aa0920-dd606d39-964fbdda-9323bac2
% LTXE-SLIDE-ORIGIN:	245de018-04aed929-a3b3c380-6c822a14 English
% LTXE-SLIDE-TITLE:		Template Paths in BackendTemplateView
% LTXE-SLIDE-REFERENCE:	!Feature: #79124 - Allow overwriting of template paths in BackendTemplateView
% ------------------------------------------------------------------------------

\begin{frame}[fragile]
	\frametitle{Modifiche rilevanti}
	\framesubtitle{Percorsi dei template in BackendTemplateView}

	% decrease font size for code listing
	\lstset{basicstyle=\tiny\ttfamily}

	\begin{itemize}
		\item BackendTemplateView permette ora la sovrascrittura dei percorsi del template per aggiungere i propri percorsi di template,
			partial e layout in un BackendTemplateView basato su modulo di backend.

			\begin{lstlisting}
				$frameworkConfiguration =
				  $this->configurationManager->getConfiguration(
				    ConfigurationManagerInterface::CONFIGURATION_TYPE_FRAMEWORK
				  );
				$viewConfiguration = [
				  'view' => [
				    'templateRootPaths' => ['EXT:myext/Resources/Private/Backend/Templates'],
				    'partialRootPaths' => ['EXT:myext/Resources/Private/Backend/Partials'],
				    'layoutRootPaths' => ['EXT:myext/Resources/Private/Backend/Layouts'],
				  ],
				];
				$this->configurationManager->setConfiguration(
				  array_merge($frameworkConfiguration, $viewConfiguration)
				);
			\end{lstlisting}

	\end{itemize}

\end{frame}

% ------------------------------------------------------------------------------
% LTXE-SLIDE-START
% LTXE-SLIDE-UID:		b435970a-07c6750d-757b81fa-19c245d8
% LTXE-SLIDE-ORIGIN:	82514d33-f6c72709-2052880d-4d753fe8 English
% LTXE-SLIDE-TITLE:		Miscellaneous
% LTXE-SLIDE-REFERENCE:	!Feature: #78899 - TCA maxitems optional
% LTXE-SLIDE-REFERENCE:	!Feature: #79240 - Single cli user for cli commands
% ------------------------------------------------------------------------------

\begin{frame}[fragile]
	\frametitle{Modifiche rilevanti}
	\framesubtitle{Varie}

	\begin{itemize}
		\item La configurazione di \texttt{TCA} \texttt{maxitems} per i tipi \texttt{type=select} e \texttt{type=group}
			è ora un impostazione opzionale di default impostata ad un valore alto (99999) invece di 1 come prima.

		\item L'accesso alle funzionalità TYPO3 dalla linea di comando è stata semplificata. I singoli comandi non necessitano più
			di utenti dedicati nel database, tutti i comandi cli utilizzano l'utente \texttt{\_cli\_}.
			Questo utente è creato su richiesta dal framework, se non esiste alla prima chiamata di riga di comando.
			L'utente \texttt{\_cli\_} ha diritti di amministratore e non ha più necessità di diritti di accesso assegnati per
			svolgere attività specifiche come intervenire sui contenuti del database utilizzando il \texttt{DataHandler}.

	\end{itemize}

\end{frame}

% ------------------------------------------------------------------------------
