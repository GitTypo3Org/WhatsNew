% ------------------------------------------------------------------------------
% TYPO3 CMS 8.6 - What's New - Chapter "Extbase & Fluid" (English Version)
%
% @author	Michael Schams <schams.net>
% @license	Creative Commons BY-NC-SA 3.0
% @link		http://typo3.org/download/release-notes/whats-new/
% @language	English
% ------------------------------------------------------------------------------
% LTXE-CHAPTER-UID:		846d40ce-66fdc2ec-750dcf95-33ce93e0
% LTXE-CHAPTER-NAME:	Extbase & Fluid
% ------------------------------------------------------------------------------

\section{Extbase \& Fluid}
\begin{frame}[fragile]
	\frametitle{Extbase \& Fluid}

	\begin{center}\huge{Chapter 4:}\end{center}
	\begin{center}\huge{\color{typo3darkgrey}\textbf{Extbase \& Fluid}}\end{center}

\end{frame}

% ------------------------------------------------------------------------------
% LTXE-SLIDE-START
% LTXE-SLIDE-UID:		a9a5442b-0f7ff21f-9e8812b1-9531a9c1
% LTXE-SLIDE-TITLE:		Widget Identifier Extended
% LTXE-SLIDE-REFERENCE:	!Feature: #47006 - Extend the widget identifier with custom string
% ------------------------------------------------------------------------------

\begin{frame}[fragile]
	\frametitle{Extbase \& Fluid}
	\framesubtitle{Widget Identifier Extended}

	\begin{itemize}
		\item The parameter \texttt{customWidgetId} has been introduced for fluid widgets.
			This string is used in the widget identifier in addition to the \texttt{nextWidgetNumber}.

		\item The widget identifier is used to create the GET parameter names.

		\item A good value for the \texttt{customWidgetId} is \texttt{{contentObjectData.uid}} to ensure no collisions happen.

		\item Allows to use the same fluid widget more than once on one page in different content elements.

			\begin{lstlisting}
				<f:widget.paginate customWidgetId="{contentObjectData.uid}" ...>
				</f:widget.paginate>
			\end{lstlisting}

	\end{itemize}

\end{frame}

% ------------------------------------------------------------------------------
% LTXE-SLIDE-START
% LTXE-SLIDE-UID:		cbc1b52f-68132620-c5f08dbb-14723e1f
% LTXE-SLIDE-TITLE:		FlashMessagesViewHelper
% LTXE-SLIDE-REFERENCE:	!Breaking: #78477 - FlashMessageViewHelper no longer inherits from TagBasedViewHelper
% ------------------------------------------------------------------------------

\begin{frame}[fragile]
	\frametitle{Extbase \& Fluid}
	\framesubtitle{FlashMessagesViewHelper}

	% decrease font size for code listing
	\lstset{basicstyle=\tiny\ttfamily}

	\begin{itemize}
		\item The \texttt{FlashMessagesViewHelper} has been refactored and no longer inherits
			from the \texttt{TagBasedViewHelper}

		\item Remove the tag specific attributes and style the default output.
			If you need custom output use the possibility to render FlashMessages yourself, for example:

			\begin{lstlisting}
				<f:flashMessages as="flashMessages">
				  <dl class="messages">
				    <f:for each="{flashMessages}" as="flashMessage">
				      <dt>CODE!! {flashMessage.code}</dt>
				      <dd>MESSAGE:: {flashMessage.message}</dd>
				    </f:for>
				  </dl>
				</f:flashMessages>
			\end{lstlisting}

	\end{itemize}

\end{frame}

% ------------------------------------------------------------------------------
% LTXE-SLIDE-START
% LTXE-SLIDE-UID:		91358547-99cab1b1-208f0f24-a5737bee
% LTXE-SLIDE-TITLE:		Removal of Fluid Styled Content Menu ViewHelpers (1/3)
% LTXE-SLIDE-REFERENCE:	!Breaking: #79622 - Removal of Fluid Styled Content Menu ViewHelpers
% ------------------------------------------------------------------------------

\begin{frame}[fragile]
	\frametitle{Extbase \& Fluid}
	\framesubtitle{Removal of Fluid Styled Content Menu ViewHelpers (1/3)}

	\begin{itemize}
		\item Fetching data directly in the view is not recommended and
			the temporary solution of menu ViewHelpers has been replaced by
			its successor, the menu processor that is based on HMENU.

		\item Menu ViewHelpers have been moved to the \texttt{compatibility7}
			extension, and are replaced in the core menu content elements.

	\end{itemize}

\end{frame}

% ------------------------------------------------------------------------------
% LTXE-SLIDE-START
% LTXE-SLIDE-UID:		8515a53c-1e60545a-7e99c75e-2638a0b1
% LTXE-SLIDE-TITLE:		Removal of Fluid Styled Content Menu ViewHelpers (2/3)
% LTXE-SLIDE-REFERENCE:	!Breaking: #79622 - Removal of Fluid Styled Content Menu ViewHelpers
% ------------------------------------------------------------------------------

\begin{frame}[fragile]
	\frametitle{Extbase \& Fluid}
	\framesubtitle{Removal of Fluid Styled Content Menu ViewHelpers (2/3)}

	% decrease font size for code listing
	\lstset{basicstyle=\tiny\ttfamily}

	\begin{itemize}

		\item Before:

			\begin{lstlisting}
				tt_content.menu_subpages.dataProcessing {
				  10 = TYPO3\CMS\Frontend\DataProcessing\SplitProcessor
				  10 {
				    if.isTrue.field = pages
				    fieldName = pages
				    delimiter = ,
				    removeEmptyEntries = 1
				    filterIntegers = 1
				    filterUnique = 1
				    as = pageUids
				  }
				}

				<ce:menu.directory pageUids="{pageUids}" as="pages" levelAs="level">
				  <f:for each="{pages}" as="page">
				    ...
				  </f:for>
				</ce.menu.directory>
			\end{lstlisting}

	\end{itemize}

\end{frame}

% ------------------------------------------------------------------------------
% LTXE-SLIDE-START
% LTXE-SLIDE-UID:		53461585-875b27dd-f00c174b-9b459763
% LTXE-SLIDE-TITLE:		Removal of Fluid Styled Content Menu ViewHelpers (3/3)
% LTXE-SLIDE-REFERENCE:	!Breaking: #79622 - Removal of Fluid Styled Content Menu ViewHelpers
% ------------------------------------------------------------------------------

\begin{frame}[fragile]
	\frametitle{Extbase \& Fluid}
	\framesubtitle{Removal of Fluid Styled Content Menu ViewHelpers (3/3)}

	% decrease font size for code listing
	\lstset{basicstyle=\tiny\ttfamily}

	\begin{itemize}

		\item After:

			\begin{lstlisting}
				tt_content.menu_subpages.dataProcessing {
				  10 = TYPO3\CMS\Frontend\DataProcessing\MenuProcessor
				  10.special = directory
				  10.special.value.field = pages
				}

				<f:for each="{menu}" as="page">
				  ...
				</f:for>
			\end{lstlisting}

	\end{itemize}

\end{frame}

% ------------------------------------------------------------------------------
% LTXE-SLIDE-START
% LTXE-SLIDE-UID:		0bdb89f8-d302c7ef-2037711d-a3706e69
% LTXE-SLIDE-TITLE:		Fluid ViewHelper <f:variable>
% LTXE-SLIDE-REFERENCE:	!Feature: #79402 - New Fluid ViewHelper f:variable added
% ------------------------------------------------------------------------------

\begin{frame}[fragile]
	\frametitle{Extbase \& Fluid}
	\framesubtitle{New Fluid ViewHelper \texttt{f:variable}}

	\begin{itemize}

		\item A new ViewHelper \texttt{f:variable} has been added in Fluid 2.2.0,
			which is now minimum required dependency for TYPO3 CMS

		\item The ViewHelper allows variables to be assigned in the template:

			\begin{lstlisting}
				<f:variable name="myvariable">My variable's content</f:variable>
				<f:variable name="myvariable" value="My variable's content"></f:variable>
				{f:variable(name: 'myvariable', value: 'My variable\'s content')}
				{myoriginalvariable -> f:variable(name: 'mynewvariable')}
			\end{lstlisting}

	\end{itemize}

\end{frame}

% ------------------------------------------------------------------------------
% LTXE-SLIDE-START
% LTXE-SLIDE-UID:		c459d3a9-e6fb03fe-cc79880c-d86a18d1
% LTXE-SLIDE-TITLE:		New Default Layout for Fluid Styled Content (1/2)
% LTXE-SLIDE-REFERENCE:	!Feature: #79622 - New default layout for Fluid Styled Content
% ------------------------------------------------------------------------------

\begin{frame}[fragile]
	\frametitle{Extbase \& Fluid}
	\framesubtitle{New Default Layout for Fluid Styled Content (1/2)}

	\begin{itemize}
		\item Previously, there have been three layouts you could
			choose from when you were defining your own custom content
			elements or overriding an existing template.

		\item To provide a better maintainability and ease of use of overrides
			we are reducing these to a single layout that is named \texttt{Default}
			with all sections optional and fallbacks if the section is not set. Also
			we are introducing the "DropIn" concept.

		\item The \texttt{Default} layout consists of five predefined sections
			that can be utilized to shape the output for your content rendering.
			In most cases you will not have to care about other section than
			\texttt{Main}. The sections will be rendered in that exact ordering:
			\texttt{Before}, \texttt{Header}, \texttt{Main}, \texttt{Footer}, \texttt{After}
	\end{itemize}

\end{frame}

% ------------------------------------------------------------------------------
% LTXE-SLIDE-START
% LTXE-SLIDE-UID:		da3b6df5-a15241de-c724dfb6-765885d4
% LTXE-SLIDE-TITLE:		New Default Layout for Fluid Styled Content (2/2)
% LTXE-SLIDE-REFERENCE:	!Feature: #79622 - New default layout for Fluid Styled Content
% ------------------------------------------------------------------------------

\begin{frame}[fragile]
	\frametitle{Extbase \& Fluid}
	\framesubtitle{New Default Layout for Fluid Styled Content (2/2)}

	\begin{itemize}
		\item The sections \texttt{Before} and \texttt{After} are so called "DropIn" sections
		\item DropIns have been introduced to be able to place additional functionality to all content
			elements without overriding layouts or templates
		\item DropIns are basically placeholders/empty partials that are meant to be overridden if necessary
		\item DropIn Locations:
			\begin{itemize}
				\item \texttt{Resources/Private/Partials/DropIn/Before/All.html}
				\item \texttt{Resources/Private/Partials/DropIn/After/All.html}
			\end{itemize}
	\end{itemize}

\end{frame}

% ------------------------------------------------------------------------------