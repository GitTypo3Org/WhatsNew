% ------------------------------------------------------------------------------
% TYPO3 CMS 7.1 - What's New - Chapter "Extbase & Fluid" (German Version)
%
% @author	Patrick Lobacher <patrick@lobacher.de> and Michael Schams <schams.net>
% @license	Creative Commons BY-NC-SA 3.0
% @link		http://typo3.org/download/release-notes/whats-new/
% @language	German
% ------------------------------------------------------------------------------
% LTXE-CHAPTER-UID:		7b0b9a18-6df803cd-8b5e7778-87e587d9
% LTXE-CHAPTER-NAME:	Chapter: Extbase & Fluid
% ------------------------------------------------------------------------------

\section{Extbase \& Fluid}
\begin{frame}[fragile]
	\frametitle{Extbase \& Fluid}

	\begin{center}\huge{Kapitel 4:}\end{center}
	\begin{center}\huge{\color{typo3darkgrey}\textbf{Extbase \& Fluid}}\end{center}

\end{frame}

% ------------------------------------------------------------------------------
% LTXE-SLIDE-START
% LTXE-SLIDE-UID:		b5ee213c-b04df592-d31ed084-b94e0f58
% LTXE-SLIDE-ORIGIN:	dcd42ada-3bbb7cb1-a32f6fd0-025036bf English
% LTXE-SLIDE-TITLE:		PaginateViewHelper
% LTXE-SLIDE-REFERENCE:	Feature-34944-PaginateHandleNonQueryResultObjects.rst
% ------------------------------------------------------------------------------

\begin{frame}[fragile]
	\frametitle{Extbase \& Fluid}
	\framesubtitle{PaginateViewHelper}

	\begin{itemize}

		\item Der Paginate-ViewHelper unterstützt nun folgende Input-Werte:

			\begin{itemize}
				\item \texttt{QueryResultInterface}
				\item \texttt{ObjectStorage}
				\item \texttt{ArrayAccess}
				\item \texttt{array}
			\end{itemize}

		\item Beispiel:

			\begin{lstlisting}
				<f:widget.paginate objects="{blogs}" as="paginatedBlogs">
				  <f:for each="{paginatedBlogs}" as="blog">
				    <h4>{blog.title}</h4>
				  </f:for>
				</f:widget.paginate>
			\end{lstlisting}

	\end{itemize}

\end{frame}

% ------------------------------------------------------------------------------
% LTXE-SLIDE-START
% LTXE-SLIDE-UID:		44467f9a-abe64946-3da2bf4c-5bd094eb
% LTXE-SLIDE-ORIGIN:	bff3466a-fc0faf7d-466a0f7b-79ad4c18 English
% LTXE-SLIDE-TITLE:		ContainerViewHelper loads RequireJS modules
% LTXE-SLIDE-REFERENCE:	Feature-63913-AllowRequireJsModulesForContainerViewHelper.rst
% ------------------------------------------------------------------------------

\begin{frame}[fragile]
	\frametitle{Extbase \& Fluid}
	\framesubtitle{ContainerViewHelper lädt RequireJS Module}

	\begin{itemize}
		\item Der ContainerViewHelper kann RequireJS-Module via \texttt{includeRequireJsModules} Attribut laden

		\item Beispiel:

			\begin{lstlisting}
				<f:be.container pageTitle="Extension Module" loadJQuery="true"
				  includeRequireJsModules="{
				    0:'TYPO3/CMS/Extension/Module1',
				    1:'TYPO3/CMS/Extension/Module2',
				    2:'TYPO3/CMS/Extension/Module3',
				    3:'TYPO3/CMS/Extension/Module4'
				  }" >
			\end{lstlisting}

	\end{itemize}

\end{frame}

% ------------------------------------------------------------------------------
% LTXE-SLIDE-START
% LTXE-SLIDE-UID:		ec0810a7-c987f58a-9c42904e-8b7e0eba
% LTXE-SLIDE-ORIGIN:	5f4b177c-9a6847e1-d807cb68-16bffd06 English
% LTXE-SLIDE-TITLE:		Methode has() in ObjectAccess
% LTXE-SLIDE-REFERENCE:	Feature-56529-SupportHasInArrayObject.rst
% ------------------------------------------------------------------------------

\begin{frame}[fragile]
	\frametitle{Extbase \& Fluid}
	\framesubtitle{Methode \texttt{has()} im ObjectAccess}

	\begin{itemize}

		\item Für die Benutzung in Fluid, \texttt{object.property} und \texttt{object.isProperty}
			unterstützten bereits die folgenden Methoden:

			\begin{itemize}
				\item \texttt{isProperty()}
				\item \texttt{getProperty()}
			\end{itemize}

		\item Neu in TYPO3 CMS 7.1: \texttt{hasProperty()}
		\item Hier wird die Method \texttt{\$object->hasProperty()} aufgerufen,
			wenn \texttt{object.hasProperty} in Fluid benutzt wird

	\end{itemize}

\end{frame}

% ------------------------------------------------------------------------------
% LTXE-SLIDE-START
% LTXE-SLIDE-UID:		5d3c259c-f79dd1fe-4f272783-a63cf715
% LTXE-SLIDE-ORIGIN:	dacadcbe-2190f355-dc506bba-4276dede English
% LTXE-SLIDE-TITLE:		Upload multiple files with FormUpload-ViewHelper
% LTXE-SLIDE-REFERENCE:	Feature-47666-AttributeMulitpleForFormUploadViewhelper.rst
% ------------------------------------------------------------------------------

\begin{frame}[fragile]
	\frametitle{Extbase \& Fluid}
	\framesubtitle{Hochladen mehrerer Dateien im FormUpload-ViewHelper}

	\begin{itemize}

		\item Der FormUpload-Viewhelper unterstützt das neue Attribut \texttt{multiple},
			welches es ermöglicht, mehrere Dateien auf einmal zu übertragen

			\begin{lstlisting}
				<f:form.upload property="files" multiple="multiple" />
			\end{lstlisting}

			\vspace{0.2cm}

			\begingroup
				\color{red}
					Hinweis: es ist darauf zu achten, dass für das Property-Mapping
					ein eigener TypeConverter erstellt werden muss!
			\endgroup

	\end{itemize}

\end{frame}

% ------------------------------------------------------------------------------
