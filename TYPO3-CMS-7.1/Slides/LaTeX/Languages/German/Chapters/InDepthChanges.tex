% ------------------------------------------------------------------------------
% TYPO3 CMS 7.1 - What's New - Chapter "In-Depth Changes" (German Version)
%
% @author	Patrick Lobacher <patrick@lobacher.de> and Michael Schams <schams.net>
% @license	Creative Commons BY-NC-SA 3.0
% @link		http://typo3.org/download/release-notes/whats-new/
% @language	German
% ------------------------------------------------------------------------------
% LTXE-CHAPTER-UID:		7cb2d402-105890ae-a1632623-719adbb6
% LTXE-CHAPTER-NAME:	In-Depth Changes
% ------------------------------------------------------------------------------

\section{Änderungen im System}
\begin{frame}[fragile]
	\frametitle{Änderungen im System}

	\begin{center}\huge{Kapitel 3:}\end{center}
	\begin{center}\huge{\color{typo3darkgrey}\textbf{Änderungen im System}}\end{center}

\end{frame}

% ------------------------------------------------------------------------------
% LTXE-SLIDE-START
% LTXE-SLIDE-UID:		bd03e141-3842b75e-8a44920f-a8c139e5
% LTXE-SLIDE-ORIGIN:	505c827e-aab9a39a-612d0288-bb873878 English
% LTXE-SLIDE-TITLE:		Maximum chars in text element
% LTXE-SLIDE-REFERENCE:	Feature-24906-MaxForTextElement.rst
% ------------------------------------------------------------------------------

\begin{frame}[fragile]
	\frametitle{Änderungen im System}
	\framesubtitle{TCA: Maximum chars in text element}

	\begin{itemize}
		\item TCA-Typ \texttt{text} unterstützt nun das HTML5-Attribut \texttt{maxlength},
			um die maximale Anzahl der einzugebenden Zeichen zu beschränken
			\small(Hinweis: Zeilenumbrüche zählen hierbei als zwei Zeichen)\normalsize

			\begin{lstlisting}
				'teaser' => array(
				  'label' => 'Teaser',
				  'config' => array(
				    'type' => 'text',
				    'cols' => 60,
				    'rows' => 2,
				    'max' => '30' // <-- maxlength
				  )
				),
			\end{lstlisting}

			\small
				Es ist zu beachten, dass nicht alle Browser dieses Attribut unterstützen.\newline
				Siehe: \href{http://www.w3schools.com/tags/att_textarea_maxlength.asp}{Browserübersicht}
			\normalsize

	\end{itemize}

\end{frame}

% ------------------------------------------------------------------------------
% LTXE-SLIDE-START
% LTXE-SLIDE-UID:		9d56ca19-04a98896-ba0a973c-8a16e877
% LTXE-SLIDE-ORIGIN:	a3d28e8d-6d40bc8f-f60bb05f-d9e31bba English
% LTXE-SLIDE-TITLE:		New SplFileInfo implementation
% LTXE-SLIDE-REFERENCE:	Feature-60019-SplFileInfo-MimeTypeGuesser-hook.rst
% ------------------------------------------------------------------------------

\begin{frame}[fragile]
	\frametitle{Änderungen im System}
	\framesubtitle{New SplFileInfo implementation}

	% decrease font size for code listing
	\lstset{basicstyle=\smaller\ttfamily}

	\begin{itemize}

		\item Neue Klasse:
			\texttt{TYPO3\textbackslash
				CMS\textbackslash
				Core\textbackslash
				Type\textbackslash
				File\textbackslash
				FileInfo}

		\item Diese erweitert \texttt{SplFileInfo}, die wiederrum Meta-Informationen von Dateien ermittelt

			\begin{lstlisting}
				$fileIdentifier = '/tmp/foo.html';
				$fileInfo = GeneralUtility::makeInstance(
				  \TYPO3\CMS\Core\Type\File\FileInfo::class,
				  $fileIdentifier
				);
				echo $fileInfo->getMimeType(); // output: text/html
			\end{lstlisting}

		\item Entwickler können über folgenden Hook auf die Funktionalität zugreifen:

			\begin{lstlisting}
				$GLOBALS['TYPO3_CONF_VARS']['SC_OPTIONS']
				  [\TYPO3\CMS\Core\Type\File\FileInfo::class]['mimeTypeGuessers']
			\end{lstlisting}

	\end{itemize}

\end{frame}

% ------------------------------------------------------------------------------
% LTXE-SLIDE-START
% LTXE-SLIDE-UID:		cc63cd98-524ee6df-38430963-f7c7067c
% LTXE-SLIDE-ORIGIN:	5351c635-145cc752-9c930534-3a8b4e50 English
% LTXE-SLIDE-TITLE:		UserFunc in TCA Display Condition
% LTXE-SLIDE-REFERENCE:	Feature-62944-UserFuncAsDisplayCond.rst
% ------------------------------------------------------------------------------

\begin{frame}[fragile]
	\frametitle{Änderungen im System}
	\framesubtitle{UserFunc in TCA Display Condition}

	\begin{itemize}
		\item userFunc \texttt{displayCondition} ermöglicht es auf jeden erdenklichen Status und jede Condition zu prüfen
		\item Sollte irgendeine Situation nicht mit den existierenden Checks abgefangen werden können,
			ist es auch möglich, eigene Funktionen zu schreiben (diese müssen
			lediglich \texttt{TRUE/FALSE} zurückgeben, um das entsprechende
			TCA Field sichtbar zu machen oder zu verbergen)

			\begin{lstlisting}
				$GLOBALS['TCA']['tt_content']['columns']['bodytext']['displayCond'] =
				  'USER:Vendor\\Example\\User\\ElementConditionMatcher->
				    checkHeaderGiven:any:more:information';
			\end{lstlisting}

	\end{itemize}

\end{frame}

% ------------------------------------------------------------------------------
% LTXE-SLIDE-START
% LTXE-SLIDE-UID:		c9a5288c-f45aac1e-34606612-fdf6ec18
% LTXE-SLIDE-ORIGIN:	3ba78d86-ca95152a-5b7e29c9-7e6593da English
% LTXE-SLIDE-TITLE:		API for Twitter Bootstrap modals (1)
% LTXE-SLIDE-REFERENCE:	Feature-63729-ApiForBootstrapModals.rst
% ------------------------------------------------------------------------------

\begin{frame}[fragile]
	\frametitle{Änderungen im System}
	\framesubtitle{API für Twitter Bootstrap Modals (1)}

	% decrease font size for code listing
	\lstset{basicstyle=\smaller\ttfamily}

	\begin{itemize}

		\item Zwei neue API Methoden um Modal Popups zu erzeugen/entfernen:
			\begin{itemize}
				\item \texttt{TYPO3.Modal.confirm(title, content, severity, buttons)}
				\item \texttt{TYPO3.Modal.dismiss()}
			\end{itemize}

		\item Optionen \texttt{title} und \texttt{content} sind mindestens erforderlich
		\item Optionen \texttt{buttons.text} und \texttt{buttons.trigger} sind erforderlich, wenn \texttt{buttons} verwendet wird

		\item Beispiel 1:

			\begin{lstlisting}
				TYPO3.Modal.confirm(
				  'The title of the modal',         // title
				  'This the the body of the modal', // content
				  TYPO3.Severity.warning            // severity
				);
			\end{lstlisting}

	\end{itemize}

\end{frame}

% ------------------------------------------------------------------------------
% LTXE-SLIDE-START
% LTXE-SLIDE-UID:		d9657b5d-5cd56c65-d7266a27-401dd233
% LTXE-SLIDE-ORIGIN:	ca95152a-5b7e29c9-7e6593da-3ba78d86 English
% LTXE-SLIDE-TITLE:		API for Twitter Bootstrap modals (2)
% LTXE-SLIDE-REFERENCE:	Feature-63729-ApiForBootstrapModals.rst
% ------------------------------------------------------------------------------

\begin{frame}[fragile]
	\frametitle{Änderungen im System}
	\framesubtitle{API für Twitter Bootstrap Modals (2)}

	\begin{itemize}

		\item Beispiel 2:

		\begin{lstlisting}
			TYPO3.Modal.confirm('Warning', 'You may break the internet!',
			  TYPO3.Severity.warning,
			  [
			    {
			      text: 'Break it',
			      active: true,
			      trigger: function() { ... }
			    },
			    {
			      text: 'Abort!',
			      trigger: function() {
			        TYPO3.Modal.dismiss();
			      }
			    }
			  ]
			);
		\end{lstlisting}

	\end{itemize}

\end{frame}

% ------------------------------------------------------------------------------
% LTXE-SLIDE-START
% LTXE-SLIDE-UID:		b2acb5df-60b9a7dd-35cf697d-2a50d5db
% LTXE-SLIDE-ORIGIN:	4e7489f8-bced2d21-6e6b87d3-2700d161 English
% LTXE-SLIDE-TITLE:		JavaScript Storage API (1)
% LTXE-SLIDE-REFERENCE:	Feature-64031-JavaScript-Storage-API.rst
% ------------------------------------------------------------------------------

\begin{frame}[fragile]
	\frametitle{Änderungen im System}
	\framesubtitle{JavaScript Storage API (1)}

	\begin{itemize}
		\item Mittels JavaScript kann auf die BE User Konfiguration zugegriffen werden (\texttt{\$BE\_USER->uc}, einfache Key-Value Paare)
		\item Zusätzlich kann nun auch der HTML5 Standard \href{http://www.w3.org/TR/webstorage/}{localStorage}
			verwendet werden, um Daten (Client-seitig) im Browser des Benutzers
			zu speichern und auszulesen

		\item Zwei neue global TYPO3 Objekte:
			\begin{itemize}
				\item \texttt{top.TYPO3.Storage.Client}
				\item \texttt{top.TYPO3.Storage.Persistent}
			\end{itemize}

		\item Jedes Objekt hat folgende API Methoden:
			\begin{itemize}
				\item \texttt{get(key)}: Daten holen
				\item \texttt{set(key,value)}: Daten schreiben 
				\item \texttt{isset(key)}: Prüfen, ob \texttt{key} genutzt wird
				\item \texttt{clear()}: Löschen des Speichers
			\end{itemize}

	\end{itemize}

\end{frame}

% ------------------------------------------------------------------------------
% LTXE-SLIDE-START
% LTXE-SLIDE-UID:		98e2056a-d165b638-b1a1ee08-206bc15a
% LTXE-SLIDE-ORIGIN:	bced2d21-6e6b87d3-2700d161-4e7489f8 English
% LTXE-SLIDE-TITLE:		JavaScript Storage API (2)
% LTXE-SLIDE-REFERENCE:	Feature-64031-JavaScript-Storage-API.rst
% ------------------------------------------------------------------------------

\begin{frame}[fragile]
	\frametitle{Änderungen im System}
	\framesubtitle{JavaScript Storage API (2)}

	\begin{itemize}
		\item Beispiel:
			\begin{lstlisting}
				// get value of key 'startModule'
				var value = top.TYPO3.Storage.Persistent.get('startModule');

				// write value 'web_info' as key 'start_module'
				top.TYPO3.Storage.Persistent.set('startModule', 'web_info');
			\end{lstlisting}

	\end{itemize}

\end{frame}

% ------------------------------------------------------------------------------
% LTXE-SLIDE-START
% LTXE-SLIDE-UID:		ed863269-574e4061-d69f855e-6494c062
% LTXE-SLIDE-ORIGIN:	62879554-fac9feac-89f11d0e-afeefc65 English
% LTXE-SLIDE-TITLE:		Inline Rendering of Checkboxes
% LTXE-SLIDE-REFERENCE:	Feature-64190-FormEngineCheckboxElement.rst
% ------------------------------------------------------------------------------

\begin{frame}[fragile]
	\frametitle{Änderungen im System}
	\framesubtitle{Inline Rendering von Checkboxes}

	% decrease font size for code listing
	\lstset{basicstyle=\tiny\ttfamily}

	\begin{itemize}

		\item Die Konfiguration \texttt{inline} sorgt bei "cols" dafür, dass
			Checkboxen nebeneinander dargestellt werden, um Platz im
			Backend User Interface zu sparen

		\begin{lstlisting}
			'weekdays' => array(
			  'label' => 'Weekdays',
			  'config' => array(
			    'type' => 'check',
			    'items' => array(
			      array('Mo', ''),
			      array('Tu', ''),
			      array('We', ''),
			      array('Th', ''),
			      array('Fr', ''),
			      array('Sa', ''),
			      array('Su', '')
			    ),
			    'cols' => 'inline'
			  )
			),
			...
		\end{lstlisting}

	\end{itemize}

\end{frame}

% ------------------------------------------------------------------------------
% LTXE-SLIDE-START
% LTXE-SLIDE-UID:		e2aec9ec-ffb1f682-c491daf0-9f8791bc
% LTXE-SLIDE-ORIGIN:	75519e56-b1721347-754a017e-ac5d5e03 English
% LTXE-SLIDE-TITLE:		Content Object Registration
% LTXE-SLIDE-REFERENCE:	Feature-64386-ContentObjectRegistration.rst
% ------------------------------------------------------------------------------

\begin{frame}[fragile]
	\frametitle{Änderungen im System}
	\framesubtitle{Content Object Registration}

	\begin{itemize}

		\item Es wurde eine neue globale Option eingeführt, um cObjects wie
			\texttt{TEXT} zu registrieren bzw. zu erweitern

		\item Eine Liste aller verfügbaren cObjects ist verfügbar als:

			\begin{lstlisting}
				$GLOBALS['TYPO3_CONF_VARS']['FE']['ContentObjects']
			\end{lstlisting}

		\item Beispiel: ein neues cObject \texttt{EXAMPLE} registrieren

			\begin{lstlisting}
				$GLOBALS['TYPO3_CONF_VARS']['FE']['ContentObjects']['EXAMPLE'] =
				  Vendor\MyExtension\ContentObject\ExampleContentObject::class;
			\end{lstlisting}

		\item Die registrierte Klasse muss von der folgenden Klasse ableiten:

			\small
				\texttt{TYPO3\textbackslash
					CMS\textbackslash
					Frontend\textbackslash
					ContentObject\textbackslash
					AbstractContentObject}
			\normalsize

		\item Idealerweise speichert man seine Datei im Verzeichnis\newline
			\small
				\texttt{typo3conf/myextension/Classes/ContentObject/}
			\normalsize\newline
			um für zukünftige Autoload-Funktionen vorbereitet zu sein

	\end{itemize}

\end{frame}

% ------------------------------------------------------------------------------
% LTXE-SLIDE-START
% LTXE-SLIDE-UID:		96a6085f-e37b39f6-045f7e74-9057bc68
% LTXE-SLIDE-ORIGIN:	2a4acfc9-da01c536-7ff147af-7cd94754 English
% LTXE-SLIDE-TITLE:		Hooks and Signals (1)
% LTXE-SLIDE-REFERENCE:	Feature-52131-HookForPageRepositoryInit.rst
% ------------------------------------------------------------------------------

\begin{frame}[fragile]
	\frametitle{Änderungen im System}
	\framesubtitle{Hooks und Signals (1)}

	\begin{itemize}

		\item Neuer Hook wurde am Ende von \texttt{PageRepository->init()} hinzugefügt,
			mit dem die Sichtbarkeit von Seiten beeinflusst werden kann

		\item Der Hook kann wie folgt registriert werden:
			\begin{lstlisting}
				$GLOBALS['TYPO3_CONF_VARS']['SC_OPTIONS']
				   [\TYPO3\CMS\Frontend\Page\PageRepository::class]['init']
			\end{lstlisting}

		\item Die Hook-Klasse muss das folgende Interface implementieren:
			\begin{lstlisting}
				\TYPO3\CMS\Frontend\Page\PageRepositoryInitHookInterface
			\end{lstlisting}

	\end{itemize}

\end{frame}

% ------------------------------------------------------------------------------
% LTXE-SLIDE-START
% LTXE-SLIDE-UID:		e3b429c5-aee2105d-fd5f39cc-3b3041af
% LTXE-SLIDE-ORIGIN:	da01c536-7ff147af-7cd94754-2a4acfc9 English
% LTXE-SLIDE-TITLE:		Hooks and Signals (2)
% LTXE-SLIDE-REFERENCE: Feature-58929-FooterHookInPageLayoutView.rst
% ------------------------------------------------------------------------------

\begin{frame}[fragile]
	\frametitle{Änderungen im System}
	\framesubtitle{Hooks und Signals (2)}

	\begin{itemize}

		\item Neuer Hook wurde zu PageLayoutView hinzugefügt, um das
			Rendering des Footers von Inhaltselementen im Backend
			manipulieren zu können

		\item Beispiel:
			\begin{lstlisting}
				$GLOBALS['TYPO3_CONF_VARS']['SC_OPTIONS']
				  ['cms/layout/class.tx_cms_layout.php']['tt_content_drawFooter'];
			\end{lstlisting}

		\item Die Hook-Klasse muss das folgende Interface implementieren:
			\begin{lstlisting}
				\TYPO3\CMS\Backend\View\PageLayoutViewDrawFooterHookInterface
			\end{lstlisting}

	\end{itemize}

\end{frame}

% ------------------------------------------------------------------------------
% LTXE-SLIDE-START
% LTXE-SLIDE-UID:		94204dd4-1cceeca7-704fafad-69e77731
% LTXE-SLIDE-ORIGIN:	babf7cab-2216fff5-9bf5848c-f76386df English
% LTXE-SLIDE-TITLE:		Hooks and Signals (3)
% LTXE-SLIDE-REFERENCE: Feature-61725-AddHookToBackendUtilityCountVersionsOfRecordsOnPage.rst
% ------------------------------------------------------------------------------

\begin{frame}[fragile]
	\frametitle{Änderungen im System}
	\framesubtitle{Hooks und Signals (3)}

	\begin{itemize}

		\item Es wurde ein Hook als Post-Prozessor zu
			\small
				\texttt{BackendUtility::countVersionsOfRecordsOnPage}
			\normalsize
			hinzugefügt

		\item Dieser wird z.B. verwendet, um Workspace-Zustände im Seitenbaum zu visualisieren
		\item Der Hook kann wie folgt registriert werden:

			\begin{lstlisting}
				$GLOBALS['TYPO3_CONF_VARS']['SC_OPTIONS']
				  ['t3lib/class.t3lib_befunc.php']['countVersionsOfRecordsOnPage'][] =
				  'My\Package\HookClass->hookMethod';
			\end{lstlisting}

	\end{itemize}

\end{frame}

% ------------------------------------------------------------------------------
% LTXE-SLIDE-START
% LTXE-SLIDE-UID:		9191d0cf-f2223cd6-9049ee0d-7c573ec4
% LTXE-SLIDE-ORIGIN:	2216fff5-9bf5848c-f76386df-babf7cab English
% LTXE-SLIDE-TITLE:		Hooks and Signals (4)
% LTXE-SLIDE-REFERENCE:	Feature-61711-SignalAtVeryEndOfDataPreprocessorFetchRecord.rst
% ------------------------------------------------------------------------------

\begin{frame}[fragile]
	\frametitle{Änderungen im System}
	\framesubtitle{Hooks und Signals (4)}

	\begin{itemize}

		\item Neues Signal wurde am Ende der Methode \texttt{DataPreprocessor::fetchRecord()} hinzugefügt
		\item Jenes kann dazu verwendet werden, um das Array \texttt{regTableItems\_data}
			zu manipulieren, damit die manipulierten Daten in TCEForms angezeigt
			werden können

			\begin{lstlisting}
				$this->getSignalSlotDispatcher()->dispatch(
				  \TYPO3\CMS\Backend\Form\DataPreprocessor::class,
				  'fetchRecordPostProcessing',
				  array($this)
				);
			\end{lstlisting}

	\end{itemize}

\end{frame}

% ------------------------------------------------------------------------------
% LTXE-SLIDE-START
% LTXE-SLIDE-UID:		5187f2fb-9dea8a55-ca95bd94-85423108
% LTXE-SLIDE-ORIGIN:	016bf36a-e935e00f-54bfa8f7-d0d2c16d English
% LTXE-SLIDE-TITLE:		Hooks and Signals (5)
% LTXE-SLIDE-REFERENCE:	Feature-62960-SignalForMailerInitialization.rst
% ------------------------------------------------------------------------------

\begin{frame}[fragile]
	\frametitle{In-Depth Changes}
	\framesubtitle{Hooks und Signals (5)}

	% decrease font size for code listing
	\lstset{basicstyle=\tiny\ttfamily}

	\begin{itemize}

		\item Neues Signal wurde eingeführt, um zusätzlichen Code bei der
			Registrierung des Mailer-Objekts auszuführen (z.B. Swift
			Mailer Plugins)

			\begin{lstlisting}
				$signalSlotDispatcher = \TYPO3\CMS\Core\Utility\GeneralUtility::makeInstance(
				  \TYPO3\CMS\Extbase\SignalSlot\Dispatcher::class
				);

				$signalSlotDispatcher->connect(
				  \TYPO3\CMS\Core\Mail\Mailer::class,
				  'postInitializeMailer',
				  \Vendor\Package\Slots\MailerSlot::class,
				  'registerPlugin'
				);
		\end{lstlisting}

	\end{itemize}

\end{frame}

% ------------------------------------------------------------------------------
% LTXE-SLIDE-START
% LTXE-SLIDE-UID:		0b221b9d-be411e9d-98ccd7e5-88f40a4a
% LTXE-SLIDE-ORIGIN:	e935e00f-54bfa8f7-d0d2c16d-016bf36a English
% LTXE-SLIDE-TITLE:		Multiple UID in PageRepository::getMenu()
% LTXE-SLIDE-REFERENCE: Feature-64257-MultipleUidInPageRepositoryGetMenu.rst
% ------------------------------------------------------------------------------

\begin{frame}[fragile]
	\frametitle{Änderungen im System}
	\framesubtitle{Multiple UID in \texttt{PageRepository::getMenu()}}

	\begin{itemize}


		\item Methode \texttt{PageRepository::getMenu()} kann nun auch ein
			Array aufnehmen, um meherer Root-Seiten zu definieren

			\begin{lstlisting}
				$pageRepository = new \TYPO3\CMS\Frontend\Page\PageRepository();
				$pageRepository->init(FALSE);
				$rows = $pageRepository->getMenu(array(2, 3));
			\end{lstlisting}

	\end{itemize}

\end{frame}

% ------------------------------------------------------------------------------
