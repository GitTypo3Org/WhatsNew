% ------------------------------------------------------------------------------
% TYPO3 CMS 6.2 LTS - What's New - Chapter "Package Management" (German Version)
%
% @author	Michael Schams <schams.net>
% @license	Creative Commons BY-NC-SA 3.0
% @link		http://typo3.org/download/release-notes/whats-new/
% @language	German
% ------------------------------------------------------------------------------
% Chapter: Package Management
% ------------------------------------------------------------------------------

\section{Package Management}
\begin{frame}[fragile]
	\frametitle{Package Management}

	\begin{center}\huge{Kapitel 5:}\end{center}
	\begin{center}\huge{\color{typo3darkgrey}\textbf{Package Management}}\end{center}

\end{frame}

% ------------------------------------------------------------------------------
% Package Manager
% ------------------------------------------------------------------------------
% http://wiki.typo3.org/Blueprints/Packagemanager
% http://forge.typo3.org/issues/47018
% http://forge.typo3.org/issues/52737

\begin{frame}[fragile]
	\frametitle{Package Management}
	\framesubtitle{Package Manager}

	\begin{itemize}
		\item Der \textbf{Package Manager} von TYPO3 Flow wurde zu TYPO3 CMS portiert
		\item Planung wurde bereits während der TYPO3 CMS 6.1 Entwicklung begonnen
		\item Vereinheitlicht die Paket-Formate
		\item Extensions sind damit nur eine spezielle Art von "Packages"
		\item Hauptziele des Projekts:

			\begin{itemize}
				\item Saubere API für Package Management
				\item Unterstützung von Vendor Namespace
				\item Unterstützung von Composer Packages
				\item Unterstützung von Flow Packages
				\item Überarbeitung des Autoloaders
			\end{itemize}

	\end{itemize}

\end{frame}

% ------------------------------------------------------------------------------
% Package Manager Integration
% ------------------------------------------------------------------------------

\begin{frame}[fragile]
	\frametitle{Package Management}
	\framesubtitle{Integration des Package Managers}

	\begin{itemize}
		\item Entfernung des Schlüssels \small\texttt{\$TYPO3\_CONF['EXT']['extListArray']}\normalsize\space
			aus der Datei \small\texttt{typo3conf/LocalConfiguration.php}\normalsize

		\item Inhalt der Datei \small\texttt{LocalConfiguration.php}\normalsize\space wird umkopiert nach
			\smaller\texttt{typo3conf/LocalConfiguration.beforePackageStatesMigration.php}\normalsize

		\item Datei \small\texttt{typo3conf/PackageStates.php}\normalsize\space enthält:

			\begin{itemize}
				\item Status der Packages (aktiv/inaktiv)
				\item Positionen der Extensions im Filesystem
			\end{itemize}

		\item Extensions in folgenden Verzeichnissen werden automatisch erkannt:
	\end{itemize}

	\begin{columns}[T]
		\begin{column}{.5\textwidth}
			\advance\leftskip+1.6cm
			\small
				\texttt{typo3/sysext/}\newline
				\texttt{typo3/ext/}\newline
				\texttt{typo3/contrib/}\newline
			\normalsize
		\end{column}
		\begin{column}{.5\textwidth}
			\small
				\texttt{typo3conf/ext/}\newline
				\texttt{Packages/} \emph{(rekursiv)}\newline
			\normalsize
		\end{column}
	\end{columns}

\end{frame}

% ------------------------------------------------------------------------------
% Package Manager Integration
% ------------------------------------------------------------------------------

\begin{frame}[fragile]
	\frametitle{Package Management}
	\framesubtitle{Integration des Package Managers}

	\begin{itemize}

		\item Jede Extension erhält zwei zusätzliche Dateien:

			\begin{itemize}
				\item \texttt{composer.json}
				\item \texttt{Classes/Package.php}
			\end{itemize}

		\item Wenn die Extension "required" ist, wird jenes über ein
			\small\texttt{protected}\normalsize\space Flag in der Datei
			\small\texttt{composer.json}\normalsize\space gesetzt

		\item Wenn die Extension "required" ist, wird jendes über die Eigenschaft
			\small\texttt{protected \$protected}\normalsize\space in der Datei
			\small\texttt{Classes/Package.php}\normalsize\space gesetzt

		\item Fehlt die Datei \small\texttt{PackageStates.php}\normalsize, wird sie neu
			erstellt und enthält Extensions, bei denen die Eigenschaft auf \texttt{TRUE} steht

		\item Autoloader bekommt ein eigenes Caching-Backend

		\item Weitere Informationen:\newline
			\url{http://wiki.typo3.org/Blueprints/Packagemanager}

	\end{itemize}

\end{frame}

% ------------------------------------------------------------------------------
% Examples
% ------------------------------------------------------------------------------

\begin{frame}[fragile]
	\frametitle{Package Management}
	\framesubtitle{Integration des Package Managers}

	Beispiel: \texttt{typo3conf/PackageManager.php}

	\lstset{
		basicstyle=\tiny\ttfamily
		% basicstyle=\fontsize{5}{7}\selectfont\ttfamily
	}

	\begin{lstlisting}
		return array ('packages' =>
		    array (
		      'core' =>
		        array (
		          'manifestPath' => '',
		          'composerName' => 'typo3/cms/core',
		          'state' => 'active',
		          'packagePath' => 'typo3/sysext/core/',
		          'classesPath' => 'Classes/',
		        ),
		      'workspaces' =>
		        array (
		          'manifestPath' => '',
		          'composerName' => 'typo3/cms/workspaces',
		          'state' => 'inactive',
		          'packagePath' => 'typo3/sysext/workspaces/',
		          'classesPath' => 'Classes/',
		        ),
		      ...
		    ),
		    'version' => 4,
		);
	\end{lstlisting}

\end{frame}

% ------------------------------------------------------------------------------
% Examples
% ------------------------------------------------------------------------------

\begin{frame}[fragile]
	\frametitle{Package Management}
	\framesubtitle{Integration des Package Managers}

	Beispiel: \texttt{composer.json}

	\lstset{
		basicstyle=\tiny\ttfamily
	}

	\begin{lstlisting}
		{
		  "name": "typo3/cms-indexed-search",
		  "type": "typo3-cms-framework",
		  "description": "TYPO3 Core",
		  "homepage": "http://typo3.org",
		  "license": ["GPL-2.0+"],
		  "version": "6.2.0",
		  "require": {
		    "typo3/cms-core": "*"
		  },
		  "replace": {
		    "indexed_search": "*"
		  }
		}
	\end{lstlisting}

\end{frame}

% ------------------------------------------------------------------------------
% Miscellaneous
% ------------------------------------------------------------------------------

\begin{frame}[fragile]
	\frametitle{Package Management}
	\framesubtitle{Integration des Package Managers}

	\lstset{
		basicstyle=\smaller\ttfamily
	}

	\begin{itemize}
		\item Packages können auch zur Laufzeit aktiviert werden:
			\smaller\texttt{\$GLOBALS['TYPO3\_CONF\_VARS']['EXT']['runtimeActivatedPackages'] = array(}\space\textit{packageKey}\space\texttt{);}\normalsize

		\item Der Schlüssel wird unmittlbar nach der Initialisierung des Package Management aktiviert
	\end{itemize}

\end{frame}

% ------------------------------------------------------------------------------

