% ------------------------------------------------------------------------------
% TYPO3 CMS 6.2 LTS - What's New - Chapter "Package Management" (Spanish Version)
%
% @author	Sergio Catalá <sergio.catala@e-net.info>
% @author	Michel Mix <mmix@autistici.org>
% @license	Creative Commons BY-NC-SA 3.0
% @link		http://typo3.org/download/release-notes/whats-new/
% @language	Spanish
% ------------------------------------------------------------------------------
% Chapter: Package Management
% ------------------------------------------------------------------------------

\section{Gestión de Paquetes}
\begin{frame}[fragile]
	\frametitle{Gestión de Paquetes}

	\begin{center}\huge{Capítulo 5:}\end{center}
	\begin{center}\huge{\color{typo3darkgrey}\textbf{Gestión de Paquetes}}\end{center}

\end{frame}

% ------------------------------------------------------------------------------
% Package Manager
% ------------------------------------------------------------------------------
% http://wiki.typo3.org/Blueprints/Packagemanager
% http://forge.typo3.org/issues/47018
% http://forge.typo3.org/issues/52737

\begin{frame}[fragile]
	\frametitle{Gestión de Paquetes}
	\framesubtitle{Gestor de Paquetes}

	\begin{itemize}
		\item \textbf{Gestor de Paquetes} de TYPO3 Flow portado a TYPO3 CMS
		\item Desarrollo/exploración comenzó durante el desarrollo de TYPO3 CMS 6.1
		\item Este proyecto pretende armonizar los formatos de paquetes
		\item Las extensiones de TYPO3 CMS son soló un tipo especial de "Paquetes"
		\item Objetivos principales del proyecto:

			\begin{itemize}
				\item API adecuada para la Gestión de Paquetes
				\item Soporte para Espacio de Nombres de Empresas
				\item Soporte para el Paquete Composer
				\item Soporte para el Paquete Flow
				\item Refactorización del Autocargador
			\end{itemize}

	\end{itemize}

\end{frame}

% ------------------------------------------------------------------------------
% Package Manager Integration
% ------------------------------------------------------------------------------

\begin{frame}[fragile]
	\frametitle{Gestión de Paquetes}
	\framesubtitle{Integración del Gestor de Paquetes (1)}

	\begin{itemize}
		\item Eliminación de \texttt{\$TYPO3\_CONF['EXT']['extListArray']} del archivo \newline
			\smaller\texttt{typo3conf/LocalConfiguration.php}\normalsize

		\item Contenido antiguo del archivo \small\texttt{typo3conf/LocalConfiguration.php} copiado a\normalsize\newline
			\smaller\texttt{typo3conf/LocalConfiguration.beforePackageStatesMigration.php}\normalsize

		\item Archivo \texttt{typo3conf/PackageStates.php} contiene:

			\begin{itemize}
				\item estado del paquete (activo/inactivo)
				\item ubicación de la extensión en el sistema de archivos
			\end{itemize}
	\end{itemize}
\end{frame}

% ------------------------------------------------------------------------------
% Package Manager Integration
% ------------------------------------------------------------------------------

\begin{frame}[fragile]
	\frametitle{Gestión de Paquetes}
	\framesubtitle{Integración del Gestor de Paquetes (2)}

	\begin{itemize}
		\item Las extensiones en los siguientes directorios se detectan automáticamente:

			\begin{itemize}
				\item \texttt{typo3/sysext/}
				\item \texttt{typo3/ext/}
				\item \texttt{typo3/contrib/}
				\item \texttt{typo3conf/ext/}
				\item \texttt{Packages/} \emph{(recursivo)}
			\end{itemize}
	\end{itemize}
\end{frame}

% ------------------------------------------------------------------------------
% Package Manager Integration
% ------------------------------------------------------------------------------

\begin{frame}[fragile]
	\frametitle{Gestión de Paquetes}
	\framesubtitle{Integración del Gestor de Paquetes (3)}

	\begin{itemize}

		\item Dos nuevos archivos (adicionales) en el directorio de la extensión:

			\begin{itemize}
				\item \texttt{composer.json}
				\item \texttt{Classes/Package.php}
			\end{itemize}

		\item Si se requiere una extensión, un flag \texttt{protected \$protected}\newline
			se puede configurar en el fichero \texttt{composer.json}

		\item Si el archivo \texttt{PackageStates.php} no existe, será (re)creado,\newline
			conteniendo todas las extensiones, que tienen la propiedad anterior a \texttt{TRUE}

		\item El autocargador recibe su propio servidor de caché

		\item Más información:\newline
			\url{http://wiki.typo3.org/Blueprints/Packagemanager}

	\end{itemize}

\end{frame}

% ------------------------------------------------------------------------------
% Examples
% ------------------------------------------------------------------------------

\begin{frame}[fragile]
	\frametitle{Gestión de Paquetes}
	\framesubtitle{Integración del Gestor de Paquetes (4)}

	Ejemplo: \texttt{typo3conf/PackageManager.php}

	\lstset{
		basicstyle=\tiny\ttfamily
		% basicstyle=\fontsize{5}{7}\selectfont\ttfamily
	}

	\begin{lstlisting}
		return array ('packages' =>
		    array (
		      'core' =>
		        array (
		          'manifestPath' => '',
		          'composerName' => 'typo3/cms/core',
		          'state' => 'active',
		          'packagePath' => 'typo3/sysext/core/',
		          'classesPath' => 'Classes/',
		        ),
		      'workspaces' =>
		        array (
		          'manifestPath' => '',
		          'composerName' => 'typo3/cms/workspaces',
		          'state' => 'inactive',
		          'packagePath' => 'typo3/sysext/workspaces/',
		          'classesPath' => 'Classes/',
		        ),
		      ...
		    ),
		    'version' => 4,
		);
	\end{lstlisting}

\end{frame}

% ------------------------------------------------------------------------------
% Examples
% ------------------------------------------------------------------------------

\begin{frame}[fragile]
	\frametitle{Gestión de Paquetes}
	\framesubtitle{Integración del Gestor de Paquetes (5)}

	Ejemplo: \texttt{composer.json}

	\lstset{
		basicstyle=\tiny\ttfamily
	}

	\begin{lstlisting}
		{
		  "name": "typo3/cms-indexed-search",
		  "type": "typo3-cms-framework",
		  "description": "TYPO3 Core",
		  "homepage": "http://typo3.org",
		  "license": ["GPL-2.0+"],
		  "version": "6.2.0",
		  "require": {
		    "typo3/cms-core": "*"
		  },
		  "replace": {
		    "indexed_search": "*"
		  }
		}
	\end{lstlisting}

\end{frame}

% ------------------------------------------------------------------------------
% Miscellaneous
% (slide added in March 2014)
% ------------------------------------------------------------------------------

\begin{frame}[fragile]
	\frametitle{Gestión de Paquetes}
	\framesubtitle{Integración del Gestor de Paquetes (6)}

	\lstset{
		basicstyle=\smaller\ttfamily
	}

	\begin{itemize}
		\item Los paquetes pueden también ser activados en tiempo de ejecución usando la clave:
			\smaller\texttt{\$GLOBALS['TYPO3\_CONF\_VARS']['EXT']['runtimeActivatedPackages'] = array(}\space\textit{packageKey}\space\texttt{);}\normalsize

		\item Esta clave se activa inmediatamente tras la inicialización del Gestor de Paquetes

	\end{itemize}

\end{frame}

% ------------------------------------------------------------------------------

