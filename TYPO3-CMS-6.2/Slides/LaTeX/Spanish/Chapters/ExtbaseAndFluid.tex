% ------------------------------------------------------------------------------
% TYPO3 CMS 6.2 LTS - What's New - Chapter "Extbase & Fluid" (Spanish Version)
%
% @author	Sergio Catalá <sergio.catala@e-net.info>
% @author	Michel Mix <mmix@autistici.org>
% @license	Creative Commons BY-NC-SA 3.0
% @link		http://typo3.org/download/release-notes/whats-new/
% @language	Spanish
% ------------------------------------------------------------------------------
% Chapter: Extbase & Fluid
% ------------------------------------------------------------------------------

\section{Extbase y Fluid}
\begin{frame}[fragile]
	\frametitle{Extbase y Fluid}

	\begin{center}\huge{Capítulo 8:}\end{center}
	\begin{center}\huge{\color{typo3darkgrey}\textbf{Extbase y Fluid}}\end{center}

\end{frame}

% ------------------------------------------------------------------------------
% ObjectManager->getScope()
% ------------------------------------------------------------------------------
% http://forge.typo3.org/issues/48766

\begin{frame}[fragile]
	\frametitle{Extbase y Fluid}
	\framesubtitle{ObjectManager->getScope()}

	\lstset{
		basicstyle=\tiny\ttfamily
	}

	\begin{itemize}
		\item Método \texttt{ObjectManager->getScope()} determina\newline
			si una clase es de tipo \textbf{prototype} o \textbf{singleton}

			\begin{lstlisting}
				/**
				 * @var \TYPO3\CMS\Extbase\Object\ObjectManagerInterface
				 * @inject
				 */
				protected $objectManager;

				$this->objectManager->getScope($propertyTargetClassName) === \TYPO3\CMS
				\Extbase\Object\Container\Container::SCOPE_PROTOTYPE

				$this->objectManager->getScope($propertyTargetClassName) === \TYPO3\CMS
				\Extbase\Object\Container\Container::SCOPE_SINGLETON
			\end{lstlisting}

	\end{itemize}

\end{frame}

% ------------------------------------------------------------------------------
% Page type for URIs with format
% ------------------------------------------------------------------------------
% http://forge.typo3.org/issues/27498

\begin{frame}[fragile]
	\frametitle{Extbase y Fluid}
	\framesubtitle{Tipo de página para URI}

	\lstset{
		basicstyle=\tiny\ttfamily
	}

	\begin{itemize}
		\item Atributo personalizado de tipo de página ya no es necesario en enlaces, al generar un formato especial

			\smaller\textbf{TYPO3 < 6.2:}\normalsize
			\begin{lstlisting}
				<f:link.action arguments="{blog: blog}" pageType="{settings.plaintextPageType}"
				  format="txt">[plaintext]</f:link.action></li>
			\end{lstlisting}

		\item Nueva opción en TypoScript \texttt{formatToPageTypeMapping} permite una asignación global:

			\begin{lstlisting}
				plugin.tx_myextension {
				  view.formatToPageTypeMapping {
				    txt = 99
				    pdf = 123
				  }
				}
			\end{lstlisting}

			\smaller\textbf{TYPO3 >= 6.2:}\normalsize
			\begin{lstlisting}
				<f:link.action arguments="{blog: blog}"
				  format="txt">[plaintext]</f:link.action></li>
			\end{lstlisting}

	\end{itemize}

\end{frame}

% ------------------------------------------------------------------------------
% Object Type Converter
% ------------------------------------------------------------------------------
% http://forge.typo3.org/issues/48548

\begin{frame}[fragile]
	\frametitle{Extbase y Fluid}
	\framesubtitle{Convertidor del Tipo del Objeto (1)}

	\lstset{
		basicstyle=\tiny\ttfamily
	}

	\begin{itemize}
		\item Asigna vectores a objetos no persistentes
		\item Útil si necesita objetos transicionales construidos a partir de argumentos de solicitud
		\item Algunos ejemplos en las siguientes diapositivas...

	\end{itemize}

\end{frame}

% ------------------------------------------------------------------------------
% Object Type Converter
% ------------------------------------------------------------------------------
% http://forge.typo3.org/issues/48548

\begin{frame}[fragile]
	\frametitle{Extbase y Fluid}
	\framesubtitle{Convertidor del Tipo del Objeto (2)}

	\lstset{
		basicstyle=\tiny\ttfamily
	}

	\smaller\textbf{GET request}\normalsize
	\begin{lstlisting}
		http://example.com/index.php?id=299
		  &tx_myextension[action]=list
		  &tx_myextension[controller]=Entity
		  &tx_myextension[demand][title]=foo
		  &tx_myextension[demand][relation]=1
	\end{lstlisting}

	\smaller\textbf{Entity controller: \texttt{initializeListAction()}}\normalsize
	\begin{lstlisting}
		use [Vendor]\myextension\Domain\Dto\Demand;
		public function initializeListAction() {
		  /**
		   * @var PropertyMappingConfiguration $demandConfiguration
		   */
		  $demandConfiguration = $this->arguments['demand']->getPropertyMappingConfiguration();
		  $demandConfiguration->allowAllProperties()->forProperty('relation')->allowAllProperties()->
		    setTypeConverterOption(
		      'TYPO3\\CMS\\Extbase\\Property\\TypeConverter\\PersistentObjectConverter',
		      PersistentObjectConverter::CONFIGURATION_CREATION_ALLOWED,
		      TRUE
		  );
		}
	\end{lstlisting}

\end{frame}

% ------------------------------------------------------------------------------
% Object Type Converter
% ------------------------------------------------------------------------------
% http://forge.typo3.org/issues/48548

\begin{frame}[fragile]
	\frametitle{Extbase y Fluid}
	\framesubtitle{Convertidor del Tipo del Objeto (3)}

	\lstset{
		basicstyle=\tiny\ttfamily
	}

	\smaller\textbf{Controlador Entidad: \texttt{listAction()}}\normalsize
	\begin{lstlisting}
		use [Vendor]\myextension\Domain\Dto\Demand;
		/**
		 * @var PropertyMappingConfiguration $demandConfiguration
		 */
		public function listAction(Demand $demand = NULL) {
		  $entities = $this->entityRepository->findAll();
		  $this->view->assign('entities', $entities);
		}
	\end{lstlisting}

	\smaller\textbf{Modelo: \texttt{[Vendor]\textbackslash myextension\textbackslash Domain\textbackslash Dto\textbackslash Demand.php}}\normalsize
	\begin{lstlisting}
		namespace [Vendor]\myextension\Domain\Dto;
		use [Vendor]\myextension\Domain\Model\Relation;
		class Demand {
		  protected $relation;
		  /**
		   * @param \TYPO3Friends\MapperExample\Domain\Model\Relation $relation
		   */
		  public function setRelation($relation) {
		    $this->relation = $relation;
		  }
		}
	\end{lstlisting}

\end{frame}

% ------------------------------------------------------------------------------
% Chaining of set* functions
% ------------------------------------------------------------------------------
% http://forge.typo3.org/issues/47684

\begin{frame}[fragile]
	\frametitle{Extbase y Fluid}
	\framesubtitle{Encadenamiento de Funciones set* }

	\lstset{
		basicstyle=\tiny\ttfamily
	}

	\begin{itemize}
		\item Ahora los métodos \texttt{set*} pueden estar \emph{encadenados} dentro de la API Querysettings
		\item Incluye nuevas opciones introducidas con TYPO3 CMS 6.0:\newline
			\texttt{setIncludeDeleted} y \texttt{setIgnoreEnableFields}

			\begin{lstlisting}
				$query->getQuerySettings()
				  ->setRespectStoragePage(FALSE)
				  ->setRespectSysLanguage(FALSE)
				  ->setIgnoreEnableFields(TRUE)
				  ->setIncludeDeleted(TRUE);
			\end{lstlisting}
	\end{itemize}

\end{frame}

% ------------------------------------------------------------------------------
% returnRawQueryResult as argument
% ------------------------------------------------------------------------------
% http://forge.typo3.org/issues/51145

\begin{frame}[fragile]
	\frametitle{Extbase y Fluid}
	\framesubtitle{returnRawQueryResult Como Argumento}

	\lstset{
		basicstyle=\smaller\ttfamily
	}

	\begin{itemize}
		\item Resultado crudo de la consulta no más un método central,\newline
			sino como un argumento en el método: \texttt{execute()}
			\newline

			\smaller\textbf{TYPO3 < 6.2:}\normalsize\newline
			\lstinline!$query->getQuerySettings()->setReturnRawQueryResult(TRUE);!
			\newline

			\smaller\textbf{TYPO3 >= 6.2:}\normalsize\newline
			\lstinline!$query->execute(TRUE);!

	\end{itemize}

\end{frame}

% ------------------------------------------------------------------------------
% Recursive validation
% ------------------------------------------------------------------------------
% http://forge.typo3.org/issues/6893

\begin{frame}[fragile]
	\frametitle{Extbase y Fluid}
	\framesubtitle{Validación Recursiva}

	\begin{itemize}
		\item Extbase ahora utiliza validación recursiva (como en TYPO3 Flow)
		\item Esto significa que, cuando se crean los objetos anidades por el Mapeador de Propiedades,
			se validan los objetos dentro de una propiedad, así como los objetos de fuera\newline
			(en TYPO3 CMS < 6.2, sólo han sido validados los objetos de fuera)
		\item Además, los validadores permiten valores vacíos ahora
	\end{itemize}

	\breakingchange

	\smaller\begin{center} ¡Para hacer una propiedad obligatoria, \underline{tiene que agregar} el \textbf{NotEmptyValidator} explícitamente!\end{center}\normalsize

\end{frame}

% ------------------------------------------------------------------------------
% Application Context
% ------------------------------------------------------------------------------
% http://forge.typo3.org/issues/49988

\begin{frame}[fragile]
	\frametitle{Extbase y Fluid}
	\framesubtitle{Contexto de Aplicación}

	\begin{itemize}
		\item Acceso al Contexto de Aplicación actual en Extbase\newline
			(configurado como una variable de entorno \texttt{TYPO3\_CONTEXT} o en la Herramienta de Instalación)\newline

			\lstinline!\TYPO3\CMS\Core\Core\Bootstrap::getInstance()->getContext();!
			\lstinline!\TYPO3\CMS\Core\Utility\GeneralUtility::getContext();!

	\end{itemize}

\end{frame}

% ------------------------------------------------------------------------------
% ViewHelper: Image
% ------------------------------------------------------------------------------
% http://forge.typo3.org/issues/47552

\begin{frame}[fragile]
	\frametitle{Extbase y Fluid}
	\framesubtitle{ViewHelper: image}

	\begin{itemize}
		\item Fluid ViewHelper \textbf{image} con el atributo opcional \texttt{title}\newline

			\smaller\textbf{Ejemplo:}\normalsize\newline
			\lstinline!<f:image src="background.jpg" alt="Text" />!
			\newline

			\smaller\textbf{TYPO3 < 6.2:}\normalsize\newline
			\lstinline!<img src="background.jpg" alt="Text" title="Text" />!
			\newline

			\smaller\textbf{TYPO3 >= 6.2:}\normalsize\newline
			\lstinline!<img src="background.jpg" alt="Text" />!

	\end{itemize}

\end{frame}

% ------------------------------------------------------------------------------
% ViewHelper: textfield and textarea
% ------------------------------------------------------------------------------
% http://forge.typo3.org/issues/45960
% http://forge.typo3.org/issues/48689 (Added autofocus attribute to textfield and textarea)

\begin{frame}[fragile]
	\frametitle{Extbase y Fluid}
	\framesubtitle{ViewHelper: textfield y textarea}

	\begin{itemize}
		\item Argumentos \texttt{autofocus} y \texttt{placeholder} (argumento válido para HTML5) para Fluid ViewHelper \textbf{form.textarea} y \textbf{form.textfield}\newline

			\smaller\textbf{Ejemplo ("placeholder"):}\normalsize
			\begin{lstlisting}
				<f:form.textfield
				  id="powermail_field_{field.marker}"
				  ...
				  placeholder="{field.title -> vh:string.RawAndRemoveXss()}"
				  ...
				  name="field[{field.uid}]"
				  required="{field.mandatory}" />
			\end{lstlisting}

	\end{itemize}

\end{frame}

% ------------------------------------------------------------------------------
% ViewHelper: switch
% ------------------------------------------------------------------------------
% http://forge.typo3.org/issues/48653

\begin{frame}[fragile]
	\frametitle{Extbase y Fluid}
	\framesubtitle{ViewHelper: switch}

	\lstset{
		basicstyle=\smaller\ttfamily
	}

	\begin{itemize}
		\item Nuevo Fluid ViewHelper \textbf{switch} genera contenido según un determinado valor o expresión
		\item Se comporta similar a la declaración \texttt{switch()} en PHP

			\begin{lstlisting}
				<f:switch expression="{person.gender}">
				  <f:case value="male">Sr.</f:case>
				  <f:case value="female">Sra.</f:case>
					<f:case default="TRUE">Sra. o Sr.</f:case>
				</f:switch>
			\end{lstlisting}

		\item \textbf{\underline{Nota:}} ¡el uso excesivo de este ViewHelper es un indicador de un mal diseño! El ejemplo anterior también se podría lograr usando los parciales "\texttt{title.male.html}" y "\texttt{title.female.html}" y lo siguiente:

			\begin{lstlisting}
				<f:render partial="title.{person.gender}" />
			\end{lstlisting}

	\end{itemize}

\end{frame}

% ------------------------------------------------------------------------------
% ViewHelper: fileSize
% ------------------------------------------------------------------------------
% http://forge.typo3.org/issues/49139

\begin{frame}[fragile]
	\frametitle{Extbase y Fluid}
	\framesubtitle{ViewHelper: fileSize}

	\begin{itemize}
		\item Convierte un tamaño de archivo (entero) a una cadena legible\newline
	\end{itemize}

	\begin{columns}[T]

		\begin{column}{.5\textwidth}
			\advance\leftskip+0.8cm
			\smaller
				\textbf{Ejemplo 1} (fileSize = 1263616):\newline
				\texttt{fileSize -> f:format.bytes()}\newline
				\newline
				Output: "1234 KB"
			\normalsize
		\end{column}
		\begin{column}{.5\textwidth}
			\smaller
				\textbf{Ejemplo 2} (fileSize = 1263616):\newline
				\texttt{fileSize -> f:format.bytes(}\newline
				\texttt{
				decimals: 2,\newline
				decimalSeparator: '.',\newline
				thousandsSeparator: ','\newline
				)}\newline
				\newline
				Salida: "1,234.00 KB"
			\normalsize
		\end{column}

	\end{columns}

\end{frame}

% ------------------------------------------------------------------------------
% ViewHelper: format.date
% (slide added in March 2014)
% ------------------------------------------------------------------------------

\begin{frame}[fragile]
	\frametitle{Extbase y Fluid}
	\framesubtitle{ViewHelper: format.date}

	\lstset{
		basicstyle=\smaller\ttfamily
	}

	\begin{itemize}
		\item El valor por defecto del \textbf{format.date} es el valor configurado en la Herramienta de Instalación

			\lstinline!$GLOBALS['TYPO3_CONF_VARS']['SYS']['ddmmyy']!

		\item Si no se configura este valor, se usa "\texttt{Y-m-d}" (año, mes, día)

	\end{itemize}

\end{frame}

% ------------------------------------------------------------------------------
% ViewHelper: Backend Container
% ------------------------------------------------------------------------------
% http://forge.typo3.org/issues/49749

\begin{frame}[fragile]
	\frametitle{Extbase y Fluid}
	\framesubtitle{ViewHelper: be.container}

	\lstset{
		basicstyle=\smaller\ttfamily
	}

	\begin{itemize}
		\item Fluid ViewHelper backend container (\texttt{be.container}) rediseñado:\newline
			\smaller\texttt{typo3/sysext/fluid/Classes/ViewHelpers/Be/ContainerViewHelper.php}\normalsize\newline

			\smaller\textbf{Discontinuado:}\normalsize
			\begin{itemize}
				\item \texttt{\$addCssFile} (utiliza \texttt{\$includeCssFiles} en su lugar)
				\item \texttt{\$addJsFile} (utiliza \texttt{\$includeJsFiles} en su lugar)
			\end{itemize}

			\smaller\textbf{Nuevo:}\normalsize
			\begin{itemize}
				\item \texttt{\$loadJQuery}
				\item \texttt{\$includeCssFiles}
				\item \texttt{\$includeJsFiles}
				\item \texttt{\$addJsInlineLabels}
			\end{itemize}

	\end{itemize}

\end{frame}

% ------------------------------------------------------------------------------
% ViewHelper: button.icon
% ------------------------------------------------------------------------------

\begin{frame}[fragile]
	\frametitle{Extbase y Fluid}
	\framesubtitle{ViewHelper: button.icon}

	\lstset{
		basicstyle=\smaller\ttfamily
	}

	\begin{itemize}
		\item Fluid ViewHelper \textbf{button.icon} finalizado (era "experimental")
		\item Crea un icono de botón (opcionalmente con un enlace)

			\begin{lstlisting}
				<f:be.buttons.icon uri="{f:uri.action(action:'new')}"
				  icon="actions-document-new" title="Crea nuevo Foo" />

				<f:be.buttons.icon
				  icon="actions-document-new" title="Crea nuevo Foo" />
			\end{lstlisting}

		\item ¡El atributo \texttt{icon} acepta más de 310 valores!\newline

			\smaller
				Busca:\newline
				\texttt{\$GLOBALS['TBE\_STYLES']['spriteIconApi']['coreSpriteImageNames']}\newline
				...en archivo:\newline
				\texttt{typo3/systext/core/ext\_tables.php}
			\normalsize

	\end{itemize}

\end{frame}

% ------------------------------------------------------------------------------
% Option addQueryStringMethod
% ------------------------------------------------------------------------------
% http://forge.typo3.org/issues/11441
% http://forge.typo3.org/issues/35281

\begin{frame}[fragile]
	\frametitle{Extbase y Fluid}
	\framesubtitle{Opción addQueryStringMethod}

	\begin{itemize}
		\item Opción \texttt{addQueryString} sólo soporta argumentos \textbf{GET}\newline
			\small(que se añaden al enlace generado)\normalsize
		\item Argumentos \textbf{POST} (utilizado por Widgets) no funcionan con esta opción
		\item La nueva opción \texttt{addQueryStringMethod} dirige esta característica y permite definir qué métodos se deben tener en cuenta:\newline
			GET (predeterminada), POST, GET/POST or POST/GET
		\item Varios Fluid ViewHelpers soportan esta nueva opción:

			\begin{itemize}\smaller
				\item \texttt{link.action}
				\item \texttt{link.page}
				\item \texttt{uri.action}
				\item \texttt{uri.page}
				\item \texttt{widget.link}
				\item \texttt{widget.uri}
				\item \texttt{widget.paginate}
			\end{itemize}

	\end{itemize}

\end{frame}

% ------------------------------------------------------------------------------
% Fluid: Fallback path for templates
% ------------------------------------------------------------------------------

\begin{frame}[fragile]
	\frametitle{Extbase y Fluid}
	\framesubtitle{Fluid: Directorio alternativo para plantillas}

	\lstset{
		basicstyle=\tiny\ttfamily
	}

	\begin{itemize}
		\item Ahora Fluid soporta directorios "alternativos" para los templates, parciales y diseños:\newline
			\smaller\texttt{templateRootPaths}, \texttt{partialRootPaths}, \texttt{layoutRootPaths}\normalsize
		\item Índice más alto primero, luego iterar a través de menores índices, hasta que se encuentra la plantilla

			\begin{lstlisting}
				plugin.tx_myextension {
				  view {
				    templateRootPath = EXT:myextension/Resources/Private/Templates/
				  }
				}

				plugin.tx_myextension {
				  view {
				    templateRootPath >
				    templateRootPaths {
				      10 = fileadmin/myextension/Templates/
				      20 = EXT:myextension/Resources/Private/Templates/
				    }
				  }
				}
			\end{lstlisting}

	\end{itemize}

\end{frame}

% ------------------------------------------------------------------------------

