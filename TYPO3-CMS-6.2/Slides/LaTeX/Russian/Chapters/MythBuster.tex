% ------------------------------------------------------------------------------
% TYPO3 CMS 6.2 LTS - What's New - Chapter "MythBuster" (Russian Version)
%
% @author	Andrey Aksenov <aksenovaa@bk.ru>
% @license	Creative Commons BY-NC-SA 3.0
% @link		http://typo3.org/download/release-notes/whats-new/
% @language	Russian
% ------------------------------------------------------------------------------
% Chapter: MythBuster
% ------------------------------------------------------------------------------

\section{Мифы}
\begin{frame}[fragile]
	\frametitle{Мифы}

	\begin{center}\huge{Chapter 10:}\end{center}
	\begin{center}\huge{\color{typo3darkgrey}\textbf{TYPO3 CMS 6.2 LTS - мифы}}\end{center}

\end{frame}

% ------------------------------------------------------------------------------
% MythBuster
% ------------------------------------------------------------------------------

\begin{frame}[fragile]
	\frametitle{Мифы}
	\framesubtitle{Мифы о TYPO3 CMS 6.2}

	\begin{itemize}
		\item TYPO3 CMS 6.2 LTS будет последней версией TYPO3 CMS
			\tabto{9cm}\color{red}\textbf{\textrightarrow вранье!}\color{black}

			\smaller
				Несмотря на выпуск \href{http://neos.typo3.org}{TYPO3 Neos}, разработка TYPO3 CMS будет продолжена,
				и в будущем нас ждут и другие её версии.
			\normalsize

		\item Ядро TYPO3 в 6.x было полностью переписано
			\tabto{9cm}\color{red}\textbf{\textrightarrow вранье!}\color{black}

			\smaller
				В TYPO3 CMS 6.0 была представлена концепция области имен PHP, что привело к новым наименованиям классов. Но
				имеется слой совместимости, так что разработчики могут пользоваться старыми названиями классов в расширениях.
			\normalsize

		\item Расширения, разработанные для 4.5 не заработают в 6.2
			\tabto{9cm}\color{red}\textbf{\textrightarrow вранье!}\color{black}

			\smaller
				Ядро API изменилось далеко не полностью и имеет обратную совместимость, если использовать \href{http://forge.typo3.org/projects/typo3v4-core/wiki/CoreDevPolicy}{стратегию обновления}. Ядро TYPO3 CMS 6.2 все еще поддерживает большинство 
				расширений, написанных для 4.5, хотя они и потребуют небольших изменений.
			\normalsize

	\end{itemize}

\end{frame}

% ------------------------------------------------------------------------------
% MythBuster
% ------------------------------------------------------------------------------

\begin{frame}[fragile]
	\frametitle{Мифы}
	\framesubtitle{Мифы о TYPO3 CMS 6.2}

	\begin{itemize}
		\item TemplaVoila невозможно более использовать в TYPO3 6.2
			\tabto{9cm}\color{red}\textbf{\textrightarrow вранье!}\color{black}

			\smaller
				Сообщество работает над совместимой с TYPO3 CMS 6.2 версией TemplaVoila. Но TemplaVoila далее не будет развиваться, а разработчикам рекомендуется изучить другие методы шаблонирования для своих будущих проектов.
			\normalsize

		\item \texttt{tslib\_pibase} расширения не работают
			\tabto{9cm}\color{red}\textbf{\textrightarrow вранье!}\color{black}

			\smaller
				Класс \texttt{tslib\_pibase} все еще имеется в 6.2, но ввиду преобразований области именования называется иначе: \texttt{\textbackslash TYPO3\textbackslash CMS\textbackslash Frontend\textbackslash Plugin\textbackslash AbstractPlugin}.\newline
				Псевдоним класса отвечает за работоспособность по старому названию (слой совместимости).
			\normalsize

		\item Отсутствует возможность миграции записей DAM на 6.2 с FAL
			\tabto{9cm}\color{red}\textbf{\textrightarrow вранье!}\color{black}

			\smaller
				DAM не работает в TYPO3 6.x. Но в FAL имеется API, которая в состоянии воссоздать все то,
				что было возможно в DAM. Существует и расширение \href{https://github.com/fnagel/t3ext-dam_falmigration}{DAM-в-FAL-преобразующее}.
			\normalsize

	\end{itemize}

\end{frame}

% ------------------------------------------------------------------------------
% MythBuster
% ------------------------------------------------------------------------------

\begin{frame}[fragile]
	\frametitle{Мифы}
	\framesubtitle{Мифы о TYPO3 CMS 6.2}

	\begin{itemize}
		\item Возможно обновить 4.5 до 6.2 при помощи мастера обновления
			\tabto{9cm}\color{red}\textbf{\textrightarrow вранье!}\color{black}

			\smaller
				Ходят слухи, что проект "упрощенного перехода" даст супермастера автоматического обновления TYPO3 4.5 до 6.2.
				Но проект – это информация, документация, обнаружение несовместимости и т. п. То есть это поддержка
				разработчиков в процессе перехода.
			\normalsize

		\item TYPO3 6.2 требует значительно больших ресурсов оборудовния
			\tabto{9cm}\color{red}\textbf{\textrightarrow вранье!}\color{black}
			% \tabto{8.2cm}\color{red}\textbf{\textrightarrow частично правда :-)}\color{black}

			\smaller
				Ходят слухи, что  6.2 в 10 раз медленнее 4.5. Вообще, в большинстве задач производительность та же,
				что и на предыдущих версиях. \href{http://typo3.org/about/typo3-the-cms/system-requirements/}{Минимальные требования} для запуска TYPO3 не изменились. Но ввиду архитектурных изменений и новомодных технологий,
				системные администраторы должны быть готовы к обновлению оборудования (вспомните,
				TYPO3 4.5 увидела свет в январе 2011, три года назад).
			\normalsize

	\end{itemize}

\end{frame}

% ------------------------------------------------------------------------------

