% ------------------------------------------------------------------------------
% TYPO3 CMS 6.2 LTS - What's New - Chapter "Package Management" (Russian Version)
%
% @author	Andrey Aksenov <aksenovaa@bk.ru>
% @license	Creative Commons BY-NC-SA 3.0
% @link		http://typo3.org/download/release-notes/whats-new/
% @language	Russian
% ------------------------------------------------------------------------------
% Chapter: Package Management
% ------------------------------------------------------------------------------

\section{Управление пакетами}
\begin{frame}[fragile]
	\frametitle{Управление пакетами}

	\begin{center}\huge{Глава 5:}\end{center}
	\begin{center}\huge{\color{typo3darkgrey}\textbf{Управление пакетами}}\end{center}

\end{frame}

% ------------------------------------------------------------------------------
% Package Manager
% ------------------------------------------------------------------------------
% http://wiki.typo3.org/Blueprints/Packagemanager
% http://forge.typo3.org/issues/47018
% http://forge.typo3.org/issues/52737

\begin{frame}[fragile]
	\frametitle{Управление пакетами}
	\framesubtitle{Мастер пакетов/Package Manager}

	\begin{itemize}
		\item TYPO3 Flow \textbf{Package Manager} портирован в TYPO3 CMS
		\item Разработка/использование началось с разработки TYPO3 CMS 6.1
		\item Этот проект привносит гармонию в формат пакетов
		\item Расширения в TYPO3 CMS просто имеют специфичный тип "Пакеты"/"Packages"
		\item Цели основного проекта:

			\begin{itemize}
				\item Правильное API для управления пакетами
				\item Поддержка области имен разработчика (Vendor Namespace)
				\item Поддержка сочетания пакетов (Composer Package)
				\item Поддержка пакетов Flow
				\item Переработка автозагрузки
			\end{itemize}

	\end{itemize}

\end{frame}

% ------------------------------------------------------------------------------
% Package Manager Integration
% ------------------------------------------------------------------------------

\begin{frame}[fragile]
	\frametitle{Управление пакетами}
	\framesubtitle{Интеграция мастера пакетов}

	\begin{itemize}
		\item Удаление \texttt{\$TYPO3\_CONF['EXT']['extListArray']} из файла\newline
			\smaller\texttt{typo3conf/LocalConfiguration.php}\normalsize

		\item Старое содержимое файла \small\texttt{typo3conf/LocalConfiguration.php} копируется в
		to\normalsize\newline
			\smaller\texttt{typo3conf/LocalConfiguration.beforePackageStatesMigration.php}\normalsize

		\item Файл \texttt{typo3conf/PackageStates.php} содержит:

			\begin{itemize}
				\item статус пакета (active/inactive)
				\item местоположение расширения в файловой системе
			\end{itemize}

		\item Автоматически распознаются расширения в следующих директориях:

			\begin{itemize}
				\item \texttt{typo3/sysext/}
				\item \texttt{typo3/ext/}
				\item \texttt{typo3/contrib/}
				\item \texttt{typo3conf/ext/}
				\item \texttt{Packages/} \emph{(рекурсивно)}
			\end{itemize}

	\end{itemize}

\end{frame}

% ------------------------------------------------------------------------------
% Package Manager Integration
% ------------------------------------------------------------------------------

\begin{frame}[fragile]
	\frametitle{Управление пакетами}
	\framesubtitle{Интеграция мастера пакетов}

	\begin{itemize}

		\item В директории расширения появились два новых (дополнительных) файла:

			\begin{itemize}
				\item \texttt{composer.json}
				\item \texttt{Classes/Package.php}
			\end{itemize}

		\item Если расширение обязательно, нужно установить флаг \texttt{protected} в файле \texttt{composer.json}

		\item Если в файле \texttt{PackageStates.php} имеется ошибка, он будет воссоздан,\newline
			и будет содержать все расширения, с указанным выше свойством, установленным в \texttt{TRUE}

		\item Автозагрузчик получил собственный внутренний кеш

		\item Дополнительная информация:\newline
			\url{http://wiki.typo3.org/Blueprints/Packagemanager}

	\end{itemize}

\end{frame}

% ------------------------------------------------------------------------------
% Examples
% ------------------------------------------------------------------------------

\begin{frame}[fragile]
	\frametitle{Управление пакетами}
	\framesubtitle{Интеграция мастера пакетов}

	Пример: \texttt{typo3conf/PackageManager.php}

	\lstset{
		basicstyle=\tiny\ttfamily
		% basicstyle=\fontsize{5}{7}\selectfont\ttfamily
	}

	\begin{lstlisting}
		return array ('packages' =>
		    array (
		      'core' =>
		        array (
		          'manifestPath' => '',
		          'composerName' => 'typo3/cms/core',
		          'state' => 'active',
		          'packagePath' => 'typo3/sysext/core/',
		          'classesPath' => 'Classes/',
		        ),
		      'workspaces' =>
		        array (
		          'manifestPath' => '',
		          'composerName' => 'typo3/cms/workspaces',
		          'state' => 'inactive',
		          'packagePath' => 'typo3/sysext/workspaces/',
		          'classesPath' => 'Classes/',
		        ),
		      ...
		    ),
		    'version' => 4,
		);
	\end{lstlisting}

\end{frame}

% ------------------------------------------------------------------------------
% Examples
% ------------------------------------------------------------------------------

\begin{frame}[fragile]
	\frametitle{Управление пакетами}
	\framesubtitle{Интеграция мастера пакетов}

	Пример: \texttt{composer.json}

	\lstset{
		basicstyle=\tiny\ttfamily
	}

	\begin{lstlisting}
		{
		  "name": "typo3/cms-indexed-search",
		  "type": "typo3-cms-framework",
		  "description": "TYPO3 Core",
		  "homepage": "http://typo3.org",
		  "license": ["GPL-2.0+"],
		  "version": "6.2.0",
		  "require": {
		    "typo3/cms-core": "*"
		  },
		  "replace": {
		    "indexed_search": "*"
		  }
		}
	\end{lstlisting}

\end{frame}

% ------------------------------------------------------------------------------
% Miscellaneous
% (slide added in March 2014)
% ------------------------------------------------------------------------------

\begin{frame}[fragile]
	\frametitle{Управление пакетами}
	\framesubtitle{Интеграция мастера пакетов}

	\lstset{
		basicstyle=\smaller\ttfamily
	}

	\begin{itemize}
		\item Пакеты также могут быть задействованы во время выполнения, посредством ключа:
			\smaller\texttt{\$GLOBALS['TYPO3\_CONF\_VARS']['EXT']['runtimeActivatedPackages'] = array(}\space\textit{packageKey}\space\texttt{);}\normalsize

		\item Этот ключ активируется сразу после инициализации Управления пакетов

	\end{itemize}

\end{frame}

% ------------------------------------------------------------------------------

