% ------------------------------------------------------------------------------
% TYPO3 CMS 6.2 LTS - What's New - Chapter "Package Management" (Dutch Version)
%
% @author	Christiaan Wiesenekker <cwiesenekker@gmail.com>
% @author	Ric van Westhreenen <ric.vanwesthreenen@typo3.org>
% @license	Creative Commons BY-NC-SA 3.0
% @link		http://typo3.org/download/release-notes/whats-new/
% @language	Dutch
% ------------------------------------------------------------------------------
% Chapter: Package Management
% ------------------------------------------------------------------------------

\section{Package Management}
\begin{frame}[fragile]
	\frametitle{Package Management}

	\begin{center}\huge{Hoofdstuk 5:}\end{center}
	\begin{center}\huge{\color{typo3darkgrey}\textbf{Pakket Management}}\end{center}

\end{frame}

% ------------------------------------------------------------------------------
% Package Manager
% ------------------------------------------------------------------------------
% http://wiki.typo3.org/Blueprints/Packagemanager
% http://forge.typo3.org/issues/47018
% http://forge.typo3.org/issues/52737

\begin{frame}[fragile]
	\frametitle{Pakket Management}
	\framesubtitle{Package Manager}

	\begin{itemize}
		\item TYPO3 Flow's \textbf{Pakket Manager} overgezet naar TYPO3 CMS
		\item Ontwikkel/exploitatie gestart tijdens de ontwikkeling van TYPO3 CMS 6.1
		\item Het project richt zich op het harmoniseren van pakket formats
		\item Extensies in TYPO3 CMS zijn gewoon een speciaal type van de "Pakketen"
		\item Main project goals:

			\begin{itemize}
				\item Proper API for Package Management
				\item Vendor Namespace Support
				\item Composer Package Support
				\item Flow Package Support
				\item Autoloader Re-factoring
			\end{itemize}

	\end{itemize}

\end{frame}

% ------------------------------------------------------------------------------
% Package Manager Integration
% ------------------------------------------------------------------------------

\begin{frame}[fragile]
	\frametitle{Pakket Management}
	\framesubtitle{Pakket Manager Integratie}

	\begin{itemize}
		\item Het verwijderen van \texttt{\$TYPO3\_CONF['EXT']['extListArray']} van bestand\newline
			\smaller\texttt{typo3conf/LocalConfiguration.php}\normalsize

		\item Oude content van het bestand \small\texttt{typo3conf/LocalConfiguration.php} gekopieerd naar\normalsize\newline
			\smaller\texttt{typo3conf/LocalConfiguration.beforePackageStatesMigration.php}\normalsize

		\item Bestand \texttt{typo3conf/PackageStates.php} bevat:

			\begin{itemize}
				\item status van package (active/inactive)
				\item extensie locatie in bestands-systeem
			\end{itemize}

		\item Extenties in de volgende mappen worden automatisch gedetecteerd:

			\begin{itemize}
				\item \texttt{typo3/sysext/}
				\item \texttt{typo3/ext/}
				\item \texttt{typo3/contrib/}
				\item \texttt{typo3conf/ext/}
				\item \texttt{Packages/} \emph{(recursive)}
			\end{itemize}

	\end{itemize}

\end{frame}

% ------------------------------------------------------------------------------
% Package Manager Integration
% ------------------------------------------------------------------------------

\begin{frame}[fragile]
	\frametitle{Pakket Management}
	\framesubtitle{Pakket Manager Integratie}

	\begin{itemize}

		\item Twee nieuwe (extra) bestanden in de map van de extensties:

			\begin{itemize}
				\item \texttt{composer.json}
				\item \texttt{Classes/Package.php}
			\end{itemize}

		\item Als de extensie nodig is moet er een \texttt{protected} vlag \newline
			worden toegevoegd in \texttt{Composer.json}

		\item Als bestand \texttt{PackageStates.php} ontbreekt, wordt het (opnieuw)aangemaakt.\newline
			Inclusief alle extenies die bovesntaande property hebben ingesteld op \texttt{TRUE}

		\item Autoloader heeft zijn eigen gecachde backend

		\item Verdere informatie:\newline
			\url{http://wiki.typo3.org/Blueprints/Packagemanager}

	\end{itemize}

\end{frame}

% ------------------------------------------------------------------------------
% Examples
% ------------------------------------------------------------------------------

\begin{frame}[fragile]
	\frametitle{Pakket Management}
	\framesubtitle{Pakket Manager Integratie}

	Voorbeeld: \texttt{typo3conf/PackageManager.php}

	\lstset{
		basicstyle=\tiny\ttfamily
		% basicstyle=\fontsize{5}{7}\selectfont\ttfamily
	}

	\begin{lstlisting}
		return array ('packages' =>
		    array (
		      'core' =>
		        array (
		          'manifestPath' => '',
		          'composerName' => 'typo3/cms/core',
		          'state' => 'active',
		          'packagePath' => 'typo3/sysext/core/',
		          'classesPath' => 'Classes/',
		        ),
		      'workspaces' =>
		        array (
		          'manifestPath' => '',
		          'composerName' => 'typo3/cms/workspaces',
		          'state' => 'inactive',
		          'packagePath' => 'typo3/sysext/workspaces/',
		          'classesPath' => 'Classes/',
		        ),
		      ...
		    ),
		    'version' => 4,
		);
	\end{lstlisting}

\end{frame}

% ------------------------------------------------------------------------------
% Examples
% ------------------------------------------------------------------------------

\begin{frame}[fragile]
	\frametitle{Pakket Management}
	\framesubtitle{Pakket Manager Integratie}

	Voorbeeld: \texttt{composer.json}

	\lstset{
		basicstyle=\tiny\ttfamily
	}

	\begin{lstlisting}
		{
		  "name": "typo3/cms-indexed-search",
		  "type": "typo3-cms-framework",
		  "description": "TYPO3 Core",
		  "homepage": "http://typo3.org",
		  "license": ["GPL-2.0+"],
		  "version": "6.2.0",
		  "require": {
		    "typo3/cms/-ore": "*"
		  },
		  "replace": {
		    "indexed-search": "*"
		  }
		}
	\end{lstlisting}

\end{frame}

% ------------------------------------------------------------------------------
% Examples
% ------------------------------------------------------------------------------

%\begin{frame}[fragile]
%	\frametitle{Pakket Management}
%	\framesubtitle{Pakket Manager Integratie}
%
%	Voorbeeld: \texttt{Classes/Package.php}
%
%	\lstset{
%	}
%
%	\begin{lstlisting}
%		<?php
%		namespace TYPO3\CMS\Cms;
%		use TYPO3\CMS\Core\Package\Package as BasePackage;
%
%		/**
%		 * This is the CMS package
%		 */
%		class Package extends BasePackage {
%
%		  /**
%		   * @var boolean
%		   */
%		  protected $protected = TRUE;
%
%		}
%		?>
%	\end{lstlisting}
%
%\end{frame}

% ------------------------------------------------------------------------------

