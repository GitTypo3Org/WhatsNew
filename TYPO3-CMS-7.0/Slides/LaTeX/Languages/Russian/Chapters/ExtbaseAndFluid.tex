% ------------------------------------------------------------------------------
% TYPO3 CMS 7.0 - What's New - Chapter "Extbase & Fluid" (Russian Version)
%
% @author	Michael Schams <schams.net>
% @license	Creative Commons BY-NC-SA 3.0
% @link		http://typo3.org/download/release-notes/whats-new/
% @language	Russian
% ------------------------------------------------------------------------------
% LTXE-CHAPTER-UID:		ce3d9af8-d40a462f-d7743950-cc2de332
% LTXE-CHAPTER-NAME:	Extbase & Fluid
% ------------------------------------------------------------------------------

\section{Extbase и Fluid}
\begin{frame}[fragile]
	\frametitle{Extbase и Fluid}

	\begin{center}\huge{Глава 4:}\end{center}
	\begin{center}\huge{\color{typo3darkgrey}\textbf{Extbase и Fluid}}\end{center}

\end{frame}

% ------------------------------------------------------------------------------
% LTXE-SLIDE-START
% LTXE-SLIDE-UID:		6e58b151-7d45f79a-9096e87f-b7c8bbc6
% LTXE-SLIDE-ORIGIN:	aba42e5a-136183fe-58e36ebd-bfab3106 English
% LTXE-SLIDE-TITLE:		Template Path Fallback
% LTXE-SLIDE-REFERENCE:	http://forge.typo3.org/issues/61361
% LTXE-SLIDE-REFERENCE:	http://forge.typo3.org/issues/39868
% ------------------------------------------------------------------------------

\begin{frame}[fragile]
	\frametitle{Extbase и Fluid}
	\framesubtitle{Резервный путь к шаблону}

	\lstset{
		basicstyle=\tiny\ttfamily
	}

	\begin{itemize}
		\item \texttt{Fluid Standalone View / автономный режим}, как и объект TypoScript \texttt{FLUIDTEMPLATE} теперь поддерживает резервные пути

			\begin{lstlisting}
				page.10 = FLUIDTEMPLATE
				page.10.file = EXT:myextension/Resources/Private/Templates/Main.html
				page.10.partialRootPaths {
				  10 = EXT:myextension/Resources/Private/Partials
				  20 = EXT:fallback/Resources/Private/Partials
				}
			\end{lstlisting}

		\item Если используются новый и старый варианты (Например, \texttt{partialRootPaths} и \texttt{partialRootPath}),
			начальный путь параметра будет в первой позиции (index = 0)

	\end{itemize}

\end{frame}

% ------------------------------------------------------------------------------
% LTXE-SLIDE-START
% LTXE-SLIDE-UID:		3a2c9558-a0ce3777-3b105229-6c0dd84e
% LTXE-SLIDE-ORIGIN:	8deb15a5-bff46e9f-ad676c73-427ca720 English
% LTXE-SLIDE-TITLE:		Typolink ViewHelper
% LTXE-SLIDE-REFERENCE:	https://forge.typo3.org/issues/59396
% ------------------------------------------------------------------------------

\begin{frame}[fragile]
	\frametitle{Extbase и Fluid}
	\framesubtitle{Typolink ViewHelper / проектор}

	\lstset{
		basicstyle=\tiny\ttfamily
	}

	\begin{itemize}
		\item Новый проектор / ViewHelper Typolink может разбирать и анализировать строку \texttt{typolink}, созданную мастером ссылок и RTE

			\begin{lstlisting}
				<f:link.typolink parameter="{link}" target="_blank" class="ico-class" title="some title" additionalAttributes="{type:'button'}">
			\end{lstlisting}

			\texttt{ссылка} может включать:
			\begin{lstlisting}
				42 _blank - "This is the link title" &foo=bar
			\end{lstlisting}

			Вывод:
			\begin{lstlisting}
				<a href="index.php?id=42&foo=bar" title="This is the title" target="_blank" class="ico-class" type="button">
			\end{lstlisting}

			Замечание: требуется лишь \texttt{параметр}, остальное не обязательно

	\end{itemize}

\end{frame}

% ------------------------------------------------------------------------------
% LTXE-SLIDE-START
% LTXE-SLIDE-UID:		14249451-4b9a5fe2-ac0920d8-5b172506
% LTXE-SLIDE-ORIGIN:	143bb339-be7e8e0b-e354de35-3591d05a English
% LTXE-SLIDE-TITLE:		Generic data-* Attribute
% LTXE-SLIDE-REFERENCE:	https://forge.typo3.org/issues/61351
% ------------------------------------------------------------------------------


\begin{frame}[fragile]
	\frametitle{Extbase и Fluid}
	\framesubtitle{Общий атрибут data-*}

%	\lstset{
%		basicstyle=\tiny\ttfamily
%	}

	\begin{itemize}
		\item Все проекторы / ViewHelpers, выводящие теги HTML, теперь поддерживают HTML5 \texttt{data-*} атрибут
		\item Массив, переданный как \texttt{data} преобразуется и пары ключ/значение составят атрибуты:
			\texttt{data-\begingroup\color{typo3orange}key\endgroup
				="\begingroup\color{typo3orange}value\endgroup
				"}\newline

			Пример:
			\begin{lstlisting}
				<f:form.textfield data="{foo: 'bar', baz: 'foos'}" />
			\end{lstlisting}

			Вывод:
			\begin{lstlisting}
				<input data-foo="bar" data-baz="foos" ... />
			\end{lstlisting}
		
	\end{itemize}

\end{frame}

% ------------------------------------------------------------------------------
% LTXE-SLIDE-START
% LTXE-SLIDE-UID:		a93d038d-540424b3-1bfb7634-b2b0d010
% LTXE-SLIDE-ORIGIN:	c5e5ddfc-f68f277a-3e0bb2b1-623f3e49 English
% LTXE-SLIDE-TITLE:		Class Tag Values Via Reflection
% LTXE-SLIDE-REFERENCE:	https://forge.typo3.org/issues/60822
% ------------------------------------------------------------------------------

\begin{frame}[fragile]
	\frametitle{Extbase и Fluid}
	\framesubtitle{Значения класса тега через отражение / Reflection}

	\lstset{
		basicstyle=\tiny\ttfamily
	}

	\begin{itemize}
		\item Extbase Reflection сервис может возвращать теги и аннотации, добавленные к классу\newline

			Пример:
			\begin{lstlisting}
				/**
				 * @SomeClassAnnotation A value
				 */
				class MyClass {
				}
			\end{lstlisting}

			К аннотации можно получить доступ через:
			\begin{lstlisting}
				$service = new \TYPO3\CMS\Extbase\Reflection\ReflectionService();

				// Returns all tags and their values the specified class is tagged with
				$classValues = $service->getClassTagsValues('MyClass');

				// Returns the values of the specified class tag
				$classValue = $service->getClassTagValue('MyClass', 'SomeClassAnnotation');
			\end{lstlisting}

	\end{itemize}

\end{frame}

% ------------------------------------------------------------------------------
