% ------------------------------------------------------------------------------
% TYPO3 CMS 7.0 - What's New - Chapter "TypoScript" (German Version)
%
% @author	Patrick Lobacher <patrick@lobacher.de>
% @license	Creative Commons BY-NC-SA 3.0
% @link		http://typo3.org/download/release-notes/whats-new/
% @language	German
% ------------------------------------------------------------------------------
% LTXE-CHAPTER-UID:		d92ab5b9-9ffa4925-30a2e34f-0d6939e5
% LTXE-CHAPTER-NAME:	TypoScript
% ------------------------------------------------------------------------------

\section{TSconfig \& TypoScript}
\begin{frame}[fragile]
	\frametitle{TSconfig \& TypoScript}

	\begin{center}\huge{Kapitel 2:}\end{center}
	\begin{center}\huge{\color{typo3darkgrey}\textbf{TSconfig \& TypoScript}}\end{center}

\end{frame}

% ------------------------------------------------------------------------------
% LTXE-SLIDE-START
% LTXE-SLIDE-UID:		4bc3c9cc-59d06820-c6f719d4-6798606a
% LTXE-SLIDE-ORIGIN:	xxxxxxxx-xxxxxxxx-xxxxxxxx-xxxxxxxx
% LTXE-SLIDE-TITLE:		TSconfig Available to Link Checkers
% LTXE-SLIDE-REFERENCE:	https://forge.typo3.org/issues/54518
% ------------------------------------------------------------------------------

\begin{frame}[fragile]
	\frametitle{TSconfig \& TypoScript}
	\framesubtitle{TSconfig für Linkvalidator inkludieren}

	\begin{itemize}
		\item TSconfig Konfiguration für den Linkvalidator wird gelesen...

			\begin{itemize}
				\item entweder aus dem aktiven TSconfig des Backends,
				\item oder aus der Konfiguration des Scheduler-Tasks
			\end{itemize}

		\item Das folgende TSconfig kann vom Linkchecker ausgelesen werden

			\lstinline!mod.linkvalidator.mychecker.myvar = 1!

		\item Dort steht das TSconfig dann als \texttt{\$this->tsConfig} zur Verfügung
	\end{itemize}

\end{frame}

% ------------------------------------------------------------------------------
% LTXE-SLIDE-START
% LTXE-SLIDE-UID:		2cb2a467-28128785-d607951d-6da71df7
% LTXE-SLIDE-ORIGIN:	xxxxxxxx-xxxxxxxx-xxxxxxxx-xxxxxxxx
% LTXE-SLIDE-TITLE:		Links zu deaktivierten Datensätzen im Linkhandler melden
% LTXE-SLIDE-REFERENCE:	https://forge.typo3.org/issues/54519
% ------------------------------------------------------------------------------

\begin{frame}[fragile]
	\frametitle{TSconfig \& TypoScript}
	\framesubtitle{Links zu deaktivierten Datensätzen im Linkhandler melden}

	\begin{itemize}
		\item Bisher hat der Linkhandler lediglich gewarnt, wenn es Links zu gelöschten oder nicht existierenden Datensätzen gab
		\item Über die folgende TSconfig Einstellung kann nun auch eine Warnung aktiviert werden, wenn der Link auf einen deaktivierten Datensatz zeigt:

			\lstinline!mod.linkvalidator.linkhandler.reportHiddenRecords = 1!

	\end{itemize}

\end{frame}

% ------------------------------------------------------------------------------
% LTXE-SLIDE-START
% LTXE-SLIDE-UID:		fc40447a-69c717a6-a98ff939-1c8ed1a2
% LTXE-SLIDE-ORIGIN:	xxxxxxxx-xxxxxxxx-xxxxxxxx-xxxxxxxx
% LTXE-SLIDE-TITLE:		RTE: mehrere CSS Klassen
% LTXE-SLIDE-REFERENCE:	https://forge.typo3.org/issues/51905
% ------------------------------------------------------------------------------

\begin{frame}[fragile]
	\frametitle{TSconfig \& TypoScript}
	\framesubtitle{RTE: mehrere CSS Klassen}

	\begin{itemize}
		\item Um das Handling mit komplexen CSS Frameworks wie Twitter Bootstrap zu handhaben, muss es möglich sein, mehrere Klassen an ein Element zu vergeben
		\item Mit diesem neuen Feature muss der Autor nur noch einen Style auswählen, um dies zu erreichen, und nicht mehrere

			\begin{lstlisting}
				RTE.classes.[ *classname* ] {
				  .requires = list of CSS classes
				}
			\end{lstlisting}

	\end{itemize}

\end{frame}

% ------------------------------------------------------------------------------
% LTXE-SLIDE-START
% LTXE-SLIDE-UID:		4ba6854b-2a858198-9016aae5-39f8700d
% LTXE-SLIDE-ORIGIN:	xxxxxxxx-xxxxxxxx-xxxxxxxx-xxxxxxxx
% LTXE-SLIDE-TITLE:		RTE: nicht-selektierbare Klassen
% LTXE-SLIDE-REFERENCE:	https://forge.typo3.org/issues/58122
% ------------------------------------------------------------------------------

\begin{frame}[fragile]
	\frametitle{TSconfig \& TypoScript}
	\framesubtitle{RTE: nicht-selektierbare Klassen}

	\begin{itemize}
		\item Man kann nun Klassen als "nicht-selektierbar" im Style-Selektor des RTE konfigurieren

			\begin{lstlisting}
				// der Wert "1" bedeutet, class ist auswaehlbar
				// der Wert "0" bedeutet, class ist nicht auswaehlbar
				RTE.classes.[ *classname* ] {
				  .selectable = 1
				}
			\end{lstlisting}

	\end{itemize}

\end{frame}

% ------------------------------------------------------------------------------
% LTXE-SLIDE-START
% LTXE-SLIDE-UID:		fba34bad-dabea95b-4ed6b0af-d58e4c2e
% LTXE-SLIDE-ORIGIN:	xxxxxxxx-xxxxxxxx-xxxxxxxx-xxxxxxxx
% LTXE-SLIDE-TITLE:		RTE: mehrere CSS-Dateien einbinden
% LTXE-SLIDE-REFERENCE:	https://forge.typo3.org/issues/50039
% ------------------------------------------------------------------------------

\begin{frame}[fragile]
	\frametitle{TSconfig \& TypoScript}
	\framesubtitle{RTE: mehrere CSS-Dateien einbinden}

	\begin{itemize}
		\item Man kann nun mehrere CSS-Dateien in den RTE laden

			\begin{lstlisting}
				RTE.default.contentCSS {
				  file1 = fileadmin/rte_stylesheet1.css
				  file2 = fileadmin/rte_stylesheet2.css
				}
			\end{lstlisting}

		\item Gibt man keine CSS Datei an, so wird die Default-Datei geladen:\newline
			\texttt{typo3/sysext/rtehtmlarea/res/contentcss/default.css}
	\end{itemize}

\end{frame}

% ------------------------------------------------------------------------------
% LTXE-SLIDE-START
% LTXE-SLIDE-UID:		ae64ffca-43d35dbd-2e2eeaaf-9266d73c
% LTXE-SLIDE-ORIGIN:	xxxxxxxx-xxxxxxxx-xxxxxxxx-xxxxxxxx
% LTXE-SLIDE-TITLE:		Exception während Rendering erzeugen - Teil 1
% LTXE-SLIDE-REFERENCE:	https://forge.typo3.org/issues/47919
% ------------------------------------------------------------------------------

\begin{frame}[fragile]
	\frametitle{TSconfig \& TypoScript}
	\framesubtitle{Exception während Rendering erzeugen - Teil 1}

	\begin{itemize}
		\item Sobald Fehler im Rendering von einzelnen Inhaltselementen (Content Objects) (z.B. mittels \texttt{USER}) auftreten, wird eine Fehlermeldung erzeugt, die in TYPO3 CMS < 7.0 die gesamte Ausgabe zerstört
		\item In TYPO3 CMS 7.0 wurde ein Exception-Handling eingeführt, welches eine Fehlermeldung in die Ausgabe an der Stelle integriert, an der das Rendering stattgefunden hat
	\end{itemize}

\end{frame}

% ------------------------------------------------------------------------------
% LTXE-SLIDE-START
% LTXE-SLIDE-UID:		9eb5e70e-826968d1-b3e7c2d8-4db8fb44
% LTXE-SLIDE-ORIGIN:	xxxxxxxx-xxxxxxxx-xxxxxxxx-xxxxxxxx
% LTXE-SLIDE-TITLE:		Exception während Rendering erzeugen - Teil 2
% LTXE-SLIDE-REFERENCE:	https://forge.typo3.org/issues/47919
% ------------------------------------------------------------------------------

\begin{frame}[fragile]
	\frametitle{TSconfig \& TypoScript}
	\framesubtitle{Exception während Rendering erzeugen - Teil 2}
	
	\lstset{
		basicstyle=\tiny\ttfamily
	}
	
	\begin{lstlisting}
		# default exception handler (activated in context "production")
		config.contentObjectExceptionHandler = 1

		# configuration of a class for the exception handling
		config.contentObjectExceptionHandler =
		  TYPO3\CMS\Frontend\ContentObject\Exception\ProductionExceptionHandler

		# customised error message (show random error code)
		config.contentObjectExceptionHandler.errorMessage = Oops an error occurred. Code: %s

		# configuration of exception codes, which are not dealt with
		tt_content.login.20.exceptionHandler.ignoreCodes.10 = 1414512813

		# deactivation of exception handling for a specific plugins or content objects
		tt_content.login.20.exceptionHandler = 0

		# ignoreCodes and errorMessage can be configured globally...
		config.contentObjectExceptionHandler.errorMessage = Oops an error occurred. Code: %s
		config.contentObjectExceptionHandler.ignoreCodes.10 = 1414512813

		# ...or locally for individual content objects
		tt_content.login.20.exceptionHandler.errorMessage = Oops an error occurred. Code: %s
		tt_content.login.20.exceptionHandler.ignoreCodes.10 = 1414512813
	\end{lstlisting}

\end{frame}

% ------------------------------------------------------------------------------
