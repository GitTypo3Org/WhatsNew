% ------------------------------------------------------------------------------
% TYPO3 CMS 7.0 - What's New - Chapter "Extbase & Fluid" (German Version)
%
% @author	Patrick Lobacher <patrick@lobacher.de>
% @license	Creative Commons BY-NC-SA 3.0
% @link		http://typo3.org/download/release-notes/whats-new/
% @language	German
% ------------------------------------------------------------------------------
% LTXE-CHAPTER-UID:		7b0b9a18-6df803cd-8b5e7778-87e587d9
% LTXE-CHAPTER-NAME:	Chapter: Extbase & Fluid
% ------------------------------------------------------------------------------

\section{Extbase \& Fluid}
\begin{frame}[fragile]
	\frametitle{Extbase \& Fluid}

	\begin{center}\huge{Kapitel 4:}\end{center}
	\begin{center}\huge{\color{typo3darkgrey}\textbf{Extbase \& Fluid}}\end{center}

\end{frame}

% ------------------------------------------------------------------------------
% LTXE-SLIDE-START
% LTXE-SLIDE-UID:		1a790457-5076a002-b516de3a-428c21ba
% LTXE-SLIDE-ORIGIN:	xxxxxxxx-xxxxxxxx-xxxxxxxx-xxxxxxxx
% LTXE-SLIDE-TITLE:		Template Path Fallback
% LTXE-SLIDE-REFERENCE:	http://forge.typo3.org/issues/61361
% LTXE-SLIDE-REFERENCE:	http://forge.typo3.org/issues/39868
% ------------------------------------------------------------------------------

\begin{frame}[fragile]
	\frametitle{Extbase \& Fluid}
	\framesubtitle{Template Path Fallback}

	\lstset{
		basicstyle=\tiny\ttfamily
	}

	\begin{itemize}

		\item Sowohl \texttt{Fluid Standalone View} wie auch das TypoScript-Objekt
			\texttt{FLUIDTEMPLATE} unterstützen nun Fallback Pfade

			\begin{lstlisting}
				page.10 = FLUIDTEMPLATE
				page.10.file = EXT:myextension/Resources/Private/Templates/Main.html
				page.10.partialRootPaths {
				  10 = EXT:myextension/Resources/Private/Partials
				  20 = EXT:fallback/Resources/Private/Partials
				}
			\end{lstlisting}

		\item Verwendet man die alten Optionen (z.B. \texttt{partialRootPath})
			zusammen mit den neuen, so wird die alte auf die erste Position in der
			Fallback-Liste gestellt (Index = 0)

	\end{itemize}

\end{frame}

% ------------------------------------------------------------------------------
% LTXE-SLIDE-START
% LTXE-SLIDE-UID:		771b27ed-c281f176-6be2486d-1293c6aa
% LTXE-SLIDE-ORIGIN:	xxxxxxxx-xxxxxxxx-xxxxxxxx-xxxxxxxx
% LTXE-SLIDE-TITLE:		Typolink ViewHelper
% LTXE-SLIDE-REFERENCE:	https://forge.typo3.org/issues/59396
% ------------------------------------------------------------------------------

\begin{frame}[fragile]
	\frametitle{Extbase \& Fluid}
	\framesubtitle{Typolink ViewHelper}

	\lstset{
		basicstyle=\tiny\ttfamily
	}

	\begin{itemize}
		\item Es wurde ein Typolink-ViewHelper zugefügt, der beispielsweise Felder
			auswerten und darstellen kann, die über einen Link-Wizard im Backend
			erzeugt wurden

			\begin{lstlisting}
				<f:link.typolink parameter="{link}" target="_blank" class="ico-class" title="some title" additionalAttributes="{type:'button'}">
			\end{lstlisting}

			\texttt{link} enthält beispielsweise:
			\begin{lstlisting}
				42 _blank - "This is the link title" &foo=bar
			\end{lstlisting}

			Ausgabe:
			\begin{lstlisting}
				<a href="index.php?id=42&foo=bar" title="This is the title" target="_blank" class="ico-class" type="button">
			\end{lstlisting}

			Lediglich \texttt{parameter} wird benötigt, der Rest ist optional

	\end{itemize}

\end{frame}

% ------------------------------------------------------------------------------
% LTXE-SLIDE-START
% LTXE-SLIDE-UID:		ab38d6e0-425393a7-8aa00052-736fa89e
% LTXE-SLIDE-ORIGIN:	xxxxxxxx-xxxxxxxx-xxxxxxxx-xxxxxxxx
% LTXE-SLIDE-TITLE:		Allgemeine data-* Attribut
% LTXE-SLIDE-REFERENCE:	https://forge.typo3.org/issues/61351
% ------------------------------------------------------------------------------

\begin{frame}[fragile]
	\frametitle{Extbase \& Fluid}
	\framesubtitle{Allgemeine data-* Attribut}

	\begin{itemize}
		\item Alle ViewHelper, die HTML Tags ausgeben, unterstützen nun das HTML5 \texttt{data-*} Attribut

		\item Array, welches als \texttt{data} übergeben wird, wird ausgewertet und dessen
			Schlüssel-/Werte-Paar ergeben das Attribut:
			\texttt{data-\begingroup\color{typo3orange}key\endgroup
				="\begingroup\color{typo3orange}value\endgroup
				"}\newline

			Beispiel:
			\begin{lstlisting}
				<f:form.textfield data="{foo: 'bar', baz: 'foos'}" />
			\end{lstlisting}

			Ausgabe:
			\begin{lstlisting}
				<input data-foo="bar" data-baz="foos" ... />
			\end{lstlisting}
		
	\end{itemize}

\end{frame}

% ------------------------------------------------------------------------------
% LTXE-SLIDE-START
% LTXE-SLIDE-UID:		1b70073c-5f41975c-c782231d-eb17f3ef
% LTXE-SLIDE-ORIGIN:	xxxxxxxx-xxxxxxxx-xxxxxxxx-xxxxxxxx
% LTXE-SLIDE-TITLE:		Class Tag Values via Reflection
% LTXE-SLIDE-REFERENCE:	https://forge.typo3.org/issues/60822
% ------------------------------------------------------------------------------

\begin{frame}[fragile]
	\frametitle{Extbase \& Fluid}
	\framesubtitle{Class Tag Values via Reflection}

	\lstset{
		basicstyle=\tiny\ttfamily
	}

	\begin{itemize}
		\item Extbase Reflection-Service kann nun Tags bzw. Annotations zurückliefern,
			die zu einer Klasse zugefügt wurden\newline

			Beispiel:
			\begin{lstlisting}
				/**
				 * @SomeClassAnnotation A value
				 */
				class MyClass {
				}
			\end{lstlisting}

			Die Annotation kann dann wie folgt ermittelt werden:
			\begin{lstlisting}
				$service = new \TYPO3\CMS\Extbase\Reflection\ReflectionService();

				// Returns all tags and their values the specified class is tagged with
				$classValues = $service->getClassTagsValues('MyClass');

				// Returns the values of the specified class tag
				$classValue = $service->getClassTagValue('MyClass', 'SomeClassAnnotation');
			\end{lstlisting}

	\end{itemize}

\end{frame}

% ------------------------------------------------------------------------------
