% ------------------------------------------------------------------------------
% TYPO3 CMS 7.2 - What's New - Chapter "TypoScript" (German Version)
%
% @author	Patrick Lobacher <patrick@lobacher.de> and Michael Schams <schams.net>
% @license	Creative Commons BY-NC-SA 3.0
% @link		http://typo3.org/download/release-notes/whats-new/
% @language	German
% ------------------------------------------------------------------------------
% LTXE-CHAPTER-UID:		12518a77-2b90a173-4e3f8420-485e0497
% LTXE-CHAPTER-NAME:	TypoScript
% ------------------------------------------------------------------------------
% LTXE-SLIDE-START
% LTXE-SLIDE-UID:		56b025e6-cf380b11-c9d1c009-0b548d6a
% LTXE-SLIDE-TITLE:		Add flexible preview URL configuration (1)
% LTXE-SLIDE-REFERENCE:	Feature-66370-AddFlexiblePreviewUrlConfiguration.rst
% ------------------------------------------------------------------------------
\begin{frame}[fragile]
	\frametitle{TSconfig \& TypoScript}
	\framesubtitle{Konfigurierbarer Vorschau-Link (1)}

	% decrease font size for code listing
	\lstset{basicstyle=\tiny\ttfamily}

	\begin{itemize}
		\item Es ist nun möglich, die URL zur Vorschau einer Seite zu definieren,
			die durch den Button "Speichern \& Vorschau" aufgerufen wird.

		\item Damit kann man unterschiedliche Links für Blog- oder News-Datensätze,
			aber auch für Inhaltselemente generieren lassen.

			\begin{lstlisting}
				TCEMAIN.preview {
				  <table name> {
				    previewPageId = 123
				    useDefaultLanguageRecord = 0
				    fieldToParameterMap {
				      uid = tx_myext_pi1[showUid]
				    }
				    additionalGetParameters {
				      tx_myext_pi1[special] = HELLO
				    }
				  }
				}
			\end{lstlisting}

	\end{itemize}

\end{frame}

% ------------------------------------------------------------------------------
% LTXE-SLIDE-START
% LTXE-SLIDE-UID:		6822c376-37649cd8-0acc836a-b856fc61
% LTXE-SLIDE-TITLE:		Add flexible preview URL configuration (2)
% LTXE-SLIDE-REFERENCE:	Feature-66370-AddFlexiblePreviewUrlConfiguration.rst
% ------------------------------------------------------------------------------
\begin{frame}[fragile]
	\frametitle{TSconfig \& TypoScript}
	\framesubtitle{Konfigurierbarer Vorschau-Link (2)}

	\begin{itemize}
		\item \texttt{previewPageId}:\newline
			\smaller
				UID der Seite, die für den Preview verwendet werden soll\newline
				(ohne Angabe wird die aktuelle Seite verwendet)
			\normalsize
		\item \texttt{useDefaultLanguageRecord}:\newline
			\smaller
				definiert, ob übersetzte Datensätze die UID des Default-Datensatzes verwenden
				(standardmäßig ist jenes aktiviert, default: 1)
			\normalsize
		\item \texttt{fieldToParameterMap}:\newline
			\smaller
				Mapping (Key = Value) von Feldern des Datensatzes, die als GET-Parameter an den Link angehängt werden
			\normalsize
		\item \texttt{additionalGetParameters}:\newline
			\smaller
				wie \texttt{fieldToParameterMap}, aber für beliebige Paramater
			\normalsize
	\end{itemize}

\end{frame}

% ------------------------------------------------------------------------------
% LTXE-SLIDE-START
% LTXE-SLIDE-UID:		c5dbf3c7-655fc6e4-227a3bf6-f5d89fb1
% LTXE-SLIDE-TITLE:		Add RTE configuration property buttons.abbreviation.removeFieldsets
% LTXE-SLIDE-REFERENCE:	Feature-63040-AddRteConfigurationPropertyButtonsAbbreviationRemoveFieldsets
% ------------------------------------------------------------------------------
\begin{frame}[fragile]
	\frametitle{TSconfig \& TypoScript}
	\framesubtitle{RTE Konfiguration: Default-Target}

	\begin{itemize}
		\item Das Default-Target in der RTE Konfiguration ist nun im PageTSconfig abhängig vom Typ einstellbar\newline

			\small
				\texttt{buttons.link.[}
				\textit{type}
				\texttt{].properties.target.default = ...}
			\normalsize\newline

		\item Als "\textit{type}" sind folgende Werte zulässig:\newline
			\small
				(weitere können via Extensions eingebracht werden)
			\normalsize

			\begin{itemize}
				\item \texttt{page}
				\item \texttt{file}
				\item \texttt{url}
				\item \texttt{mail}
				\item \texttt{spec}
			\end{itemize}
	\end{itemize}

\end{frame}

% ------------------------------------------------------------------------------
% LTXE-SLIDE-START
% LTXE-SLIDE-UID:		fd957580-301e4a0a-ad5325f5-ffa024c3
% LTXE-SLIDE-TITLE:		Strip empty HTML tags in HtmlParser
% LTXE-SLIDE-REFERENCE:	Feature-20555-StripEmptyHtmlTags.rst
% ------------------------------------------------------------------------------
\begin{frame}[fragile]
	\frametitle{TSconfig \& TypoScript}
	\framesubtitle{Leere HTML-Tags im HTMLparser löschen}

	% decrease font size for code listing
	\lstset{basicstyle=\tiny\ttfamily}

	\begin{itemize}
		\item Es ist nun möglich, leere HTML-Tags im HTMLparser zu löschen
			\begin{lstlisting}
				stdWrap {
				   // Hier werden alle leeren HTML-Tags entfernt
				   HTMLparser.stripEmptyTags = 1
				   // Hier werden nur leere h2 und h3 Tags entfernt
				   HTMLparser.stripEmptyTags.tags = h2, h3
				}

				RTE.default.proc.entryHTMLparser_db {
				   stripEmptyTags = 1
				   stripEmptyTags.tags = p
				   stripEmptyTags.treatNonBreakingSpaceAsEmpty = 1
				}
			\end{lstlisting}

			\item Da der HtmlParser unbekannte Tags grundsätzlich entfernt,
				ist es ratsam, diese zunächst zu behalten:\newline
				\texttt{HTMLparser.keepNonMatchedTags = 1}

	\end{itemize}

\end{frame}

% ------------------------------------------------------------------------------
% LTXE-SLIDE-START
% LTXE-SLIDE-UID:		62960e76-c3dca953-fe5a1070-831633fc
% LTXE-SLIDE-TITLE:		Miscellaneous
% LTXE-SLIDE-REFERENCE:	Feature-63040-AddRteConfigurationPropertyButtonsAbbreviationRemoveFieldsets.rst
% LTXE-SLIDE-REFERENCE:	commit 31c62f9311ee3d33bf792d548cc4a5fac83aa3d0
% ------------------------------------------------------------------------------
\begin{frame}[fragile]
	\frametitle{TSconfig \& TypoScript}
	\framesubtitle{Diverses}

	\begin{itemize}
		\item Der Button für "Abkürzung" (engl. \textit{abbreviation}) im RTE kann nun
			in der PageTSconfig ausgeblendet werden (da nicht mehr HTML5 konform):

			\begin{lstlisting}
				# moegliche Wert sind:
				# acronym, definedAcronym, abbreviation, definedAbbreviation
				buttons.abbreviation.removeFieldsets = acronym,definedAcronym
			\end{lstlisting}

		\item Die Eigenschaft \texttt{inlineLanguageLabel} des Objekts \texttt{PAGE} kann nun
			auch mit \texttt{LLL:}-Referenzen umgehen

	\end{itemize}

\end{frame}

% ------------------------------------------------------------------------------
