% ------------------------------------------------------------------------------
% TYPO3 CMS 7.4 - What's New - Chapter "In-Depth Changes" (German Version)
%
% @author	Patrick Lobacher <patrick@lobacher.de> and Michael Schams <schams.net>
% @license	Creative Commons BY-NC-SA 3.0
% @link		http://typo3.org/download/release-notes/whats-new/
% @language	German
% ------------------------------------------------------------------------------
% LTXE-CHAPTER-UID:		9a69925b-ed88557b-d362988a-940ed520
% LTXE-CHAPTER-NAME:	In-Depth Changes
% ------------------------------------------------------------------------------
% LTXE-SLIDE-START
% LTXE-SLIDE-UID:		5cfc1d44-3d6f3d87-789957ab-655f357d
% LTXE-SLIDE-TITLE:		Breaking #65305: DriverInterface has been extended
% LTXE-SLIDE-REFERENCE:	Breaking-65305-AddFunctionsToDriverInterface.rst
% ------------------------------------------------------------------------------

\begin{frame}[fragile]
	\frametitle{Änderungen im System}
	\framesubtitle{Driver Interface}

	% decrease font size for code listing
	\lstset{basicstyle=\tiny\ttfamily}

	\begin{itemize}

		\item Zum \texttt{DriverInterface} wurden die folgenden Methoden hinzugefügt:

			\begin{itemize}
				\item \texttt{getFolderInFolder}
				\item \texttt{getFileInFolder}
			\end{itemize}

		\item Jeder eigene FAL-Driver muss daher diese beiden Methoden nachimplementieren:

			\begin{columns}[T]
				\begin{column}{.07\textwidth}
                \end{column}
				\begin{column}{.465\textwidth}

					\begin{lstlisting}
						public function getFoldersInFolder(
						  $folderIdentifier,
						  $start = 0,
						  $numberOfItems = 0,
						  $recursive = FALSE,
						  array $folderNameFilterCallbacks = array(),
						  $sort = '',
						  $sortRev = FALSE
						);
					\end{lstlisting}

                \end{column}
				\begin{column}{.465\textwidth}

					\begin{lstlisting}
						public function getFileInFolder(
						  $fileName,
						  $folderIdentifier
						);
					\end{lstlisting}

				\end{column}
			\end{columns}

	\end{itemize}

	\breakingchange

\end{frame}

% ------------------------------------------------------------------------------
% LTXE-SLIDE-START
% LTXE-SLIDE-UID:		b719be34-650cf0d2-13ac26aa-dd3a4bbd
% LTXE-SLIDE-TITLE:		Feature #22175: Support IEC/SI units in file size formatting
% LTXE-SLIDE-REFERENCE:	Feature-22175-SupportIecSiUnitsInFileSizeFormatting.rst
% ------------------------------------------------------------------------------

\begin{frame}[fragile]
	\frametitle{Änderungen im System}
	\framesubtitle{Unterstützung von IEC/SI-Keywords für Größen}

	% decrease font size for code listing
	%\lstset{basicstyle=\tiny\ttfamily}

	\begin{itemize}

		\item Die Formatierung von Größen unterstützt nun zwei Keywords, um die Einheiten festzulegen:

			\begin{itemize}
				\item \small\texttt{iec} (default)\newline
					\small(Basis: 2, Labels: \texttt{| Ki| Mi| Gi| Ti| Pi| Ei| Zi| Yi})\normalsize
				\item \small\texttt{si}\newline
					\small(Basis: 10, Labels: \texttt{| k| M| G| T| P| E| Z| Y})\normalsize
			\end{itemize}

		\item Gesetzt werden kann die Formatierung z.B. via TypoScript:\newline
			\texttt{bytes.labels = iec}

		\begin{lstlisting}
			echo GeneralUtility::formatSize(85123);
			// => Vorher "83.1 K"
			// => Nachher "83.13 Ki"
		\end{lstlisting}

	\end{itemize}

\end{frame}

% ------------------------------------------------------------------------------
% LTXE-SLIDE-START
% LTXE-SLIDE-UID:		c5766b9f-4acc2f8d-dcadf0be-299ed259
% LTXE-SLIDE-TITLE:		Feature #67293: Dependency ordering service (1)
% LTXE-SLIDE-REFERENCE:	Feature-67293-DependencyOrderingService.rst
% ------------------------------------------------------------------------------

\begin{frame}[fragile]
	\frametitle{Änderungen im System}
	\framesubtitle{Dependency Ordering Service (1)}

	\begin{itemize}

		\item Oftmals ist es notwendig eine sortierte Liste an Items zur Verfügung zu stellen,
			deren Einträge einerseits Abhängigkeiten haben und andererseits dazu verwendet werden,
			um Aktionen in eben dieser Reihenfolge auszuführen.

		\item Im Core findet jenes beispielsweise Verwendung bei:

			\begin{itemize}
				\item Reihenfolge der Hook-Ausführung,
				\item Ladereihenfolge von Extensions,
				\item Reihefolge der Anzeige von Menü-Einträgen,
				\item usw.
			\end{itemize}

		\item Eine Überarbeitung des bisherigen \texttt{DependencyResolver} stellt nun den
			\texttt{DependencyOrderingService} zur Verfügung

	\end{itemize}

\end{frame}

% ------------------------------------------------------------------------------
% LTXE-SLIDE-START
% LTXE-SLIDE-UID:		4acc2f8d-6b9fc576-f0bedcad-d259299e
% LTXE-SLIDE-TITLE:		Feature #67293: Dependency ordering service (2)
% LTXE-SLIDE-REFERENCE:	Feature-67293-DependencyOrderingService.rst
% ------------------------------------------------------------------------------

\begin{frame}[fragile]
	\frametitle{Änderungen im System}
	\framesubtitle{Dependency Ordering Service (2)}

	% decrease font size for code listing
	\lstset{basicstyle=\tiny\ttfamily}

	\begin{itemize}

		\item Anwendung:

			\begin{lstlisting}
				$GLOBALS['TYPO3_CONF_VARS']['EXTCONF']['someExt']['someHook'][<some id>] = [
				  'handler' => someClass::class,
				  'runBefore' => [ <some other ID> ],
				  'runAfter' => [ ... ],
				  ...
				];
			\end{lstlisting}

		\item Zum Beispiel:

			\begin{lstlisting}
				$hooks = $GLOBALS['TYPO3_CONF_VARS']['EXTCONF']['someExt']['someHook'];
				$sorted = GeneralUtility:makeInstance(DependencyOrderingService::class)->orderByDependencies(
				  $hooks, 'runBefore', 'runAfter'
				);
			\end{lstlisting}

	\end{itemize}

\end{frame}

% ------------------------------------------------------------------------------
% LTXE-SLIDE-START
% LTXE-SLIDE-UID:		3752346f-1d6aa26b-4b3dd1a7-f142d76b
% LTXE-SLIDE-TITLE:		Feature #56644: Hook for InlineRecordContainer::checkAccess()
% LTXE-SLIDE-REFERENCE:	Feature-56644-AddHookToInlineRecordContainerCheckAccess.rst
% ------------------------------------------------------------------------------

\begin{frame}[fragile]
	\frametitle{Änderungen im System}
	\framesubtitle{Hooks und Signals (1)}

	% decrease font size for code listing
	\lstset{basicstyle=\tiny\ttfamily}

	\begin{itemize}

		\item Ein neuer Hook wurde am Ende von \texttt{InlineRecordContainer::checkAccess} hinzugefügt,
			mit dem der Zugriff von Inline-Records geprüft werden kann

		\item Der Hook kann wie folgt registriert werden:

			\begin{lstlisting}
				$GLOBALS['TYPO3_CONF_VARS']['SC_OPTIONS']['t3lib/class.t3lib_tceforms_inline.php']
				  ['checkAccess'][] = 'My\\Package\\HookClass->hookMethod';
			\end{lstlisting}

	\end{itemize}

\end{frame}

% ------------------------------------------------------------------------------
% LTXE-SLIDE-START
% LTXE-SLIDE-UID:		158946bb-f31ee48e-ee6d0762-6cac7a12
% LTXE-SLIDE-TITLE:		Feature #59231: Hook for AbstractUserAuthentication::checkAuthentication()
% LTXE-SLIDE-REFERENCE:	Feature-59231-AddHookToAbstractUserAuthenticationCheckAuthentication.rst
% ------------------------------------------------------------------------------

\begin{frame}[fragile]
	\frametitle{Änderungen im System}
	\framesubtitle{Hooks und Signals (2)}

	% decrease font size for code listing
	\lstset{basicstyle=\tiny\ttfamily}

	\begin{itemize}

		\item Ein neuer Hook wurde am Ende von \texttt{AbstractUserAuthentication::checkAuthentication} hinzugefügt,
			mit dem man fehlgeschlagene Anmeldeversuche verarbeiten kann

		\item Standardmäßig wartet der Prozess 5 Sekunden nachdem eine Anmeldung fehlgeschlagen ist

		\item Über den Hook kann ein anderes Verhalten implementiert werden\newline
			(z.B. zur Abwehr von Brute Force Angriffen)

		\item Der Hook kann wie folgt registriert werden:

			\begin{lstlisting}
				$GLOBALS['TYPO3_CONF_VARS']['SC_OPTIONS']['t3lib/class.t3lib_userauth.php']
				  ['postLoginFailureProcessing'][] = 'My\\Package\\HookClass->hookMethod';
			\end{lstlisting}

	\end{itemize}

\end{frame}

% ------------------------------------------------------------------------------
% LTXE-SLIDE-START
% LTXE-SLIDE-UID:		cfa748d2-7c412900-462446d1-14dd41e4
% LTXE-SLIDE-TITLE:		Feature #67228 and #68047
% LTXE-SLIDE-REFERENCE:	Feature-67228-EmitSignalWhenAnIndexRecordIsMarkedAsMissing.rst
% LTXE-SLIDE-REFERENCE:	Feature-68047-EmitASignalForEachMappedObject.rst
% ------------------------------------------------------------------------------

\begin{frame}[fragile]
	\frametitle{Änderungen im System}
	\framesubtitle{Hooks und Signals (3)}

	\begin{itemize}

		\item Das neue Signal \texttt{recordMarkedAsMissing} wird ausgesendet, wenn der FAL Indexer auf einen
			\texttt{sys\_file} Eintrag stößt, dessen Datei im Dateisystem aber nicht auffindbar ist.
			Dabei wird die \texttt{sys\_file} UID übermittelt.

		\item Jenes kann in Extensions verwendet werden, die Dienste rund um das Datei-Management anbieten
			(wie beispielsweise Versionierung, Synchronisation, Recovery, usw.)

		\item Das Signal \texttt{afterMappingSingleRow} wird ausgesendet, wann immer der DataMapper ein Objekt erstellt

	\end{itemize}

\end{frame}

% ------------------------------------------------------------------------------
% LTXE-SLIDE-START
% LTXE-SLIDE-UID:		24a6aa34-aa43c7ad-29712401-32d010f3
% LTXE-SLIDE-TITLE:		Breaking #55759: HTML (e.g. double quotes) in link titles
% LTXE-SLIDE-REFERENCE:	Breaking-55759-HTMLInLinkTitlesNotWorkingAnymore.rst
% ------------------------------------------------------------------------------
% Breaking #55759: HTML (e.g. double quotes) in link titles

\begin{frame}[fragile]
	\frametitle{Änderungen im System}
	\framesubtitle{HTML in TypoLink-Titeln}

	% decrease font size for code listing
	\lstset{basicstyle=\tiny\ttfamily}

	\begin{itemize}

		\item Anführungszeichen in TypoLink-Titeln werden nun automatisch "escaped"
		\item Ein eventuell bereits existierendes Escaping wird daher nun falsch dargestellt:

			Aus\tabto{1.2cm}'\texttt{Some \&quot;special\&quot; title}'\newline
			wird\tabto{1.2cm}'\texttt{Some \&amp;quot;special\&amp;quot; title}'

		\item Es wird empfohlen, hier auf Escaping komplett zu verzichten, da sich TYPO3 nun darum kümmert

	\end{itemize}

	\breakingchange

\end{frame}

% ------------------------------------------------------------------------------
% LTXE-SLIDE-START
% LTXE-SLIDE-UID:		8c2cf004-beb3b185-c41a8d75-35b4410a
% LTXE-SLIDE-TITLE:		Breaking #56133 and #52705 - Miscellaneous (1)
% LTXE-SLIDE-REFERENCE:	Breaking-56133-NewBeUserPermissionFilesReplace.rst
% LTXE-SLIDE-REFERENCE:	Breaking-52705-DefaultLogConfigurationIsChanged.rst
% ------------------------------------------------------------------------------
% Feature #56133: New BE user permission "Files: replace"
% Breaking #52705: Default log configuration is changed

\begin{frame}[fragile]
	\frametitle{Änderungen im System}
	\framesubtitle{Diverse Änderungen (1)}

	% decrease font size for code listing
	\lstset{basicstyle=\tiny\ttfamily}

	\begin{itemize}

		\item Mit \texttt{Files->replace} gibt eine neue Berechtigung für Backend Benutzer,
			um Dateien im Modul Dateiliste zu ersetzen

		% note: item above is classified as a breaking change, but I can not see why (Michael)
		% \breackingchange

		\item Der Dateinamen des Logfiles, welches der FileWriter schreibt, ändert sich wie folgt:

			\begin{itemize}
				\item bisher:\tabto{1.2cm}\texttt{typo3temp/logs/typo3.log}
				\item neu:\tabto{1.2cm}\texttt{typo3temp/logs/typo3\_<hash>.log}
			\end{itemize}

			\small(der Wert \texttt{<hash>} berechnet sich aus dem Encryptionkey)\normalsize

		% note: item above is classified as a breaking change, but I can not see why (Michael)
		% \breackingchange

	\end{itemize}

\end{frame}

% ------------------------------------------------------------------------------
% LTXE-SLIDE-START
% LTXE-SLIDE-UID:		10f3aa43-c7ad32d0-29712401-aa3424a6
% LTXE-SLIDE-TITLE:		Miscellaneous (2)
% LTXE-SLIDE-REFERENCE:	unknown
% ------------------------------------------------------------------------------

\begin{frame}[fragile]
	\frametitle{Änderungen im System}
	\framesubtitle{Diverse Änderungen (2)}

	% decrease font size for code listing
	\lstset{basicstyle=\tiny\ttfamily}

	\begin{itemize}

		\item Die in Hooks verwendeten Klassen müssen ab sofort dem Autoloading-Mechanismus folgen
		\item Daher kann die Hook-Definition auch verkürzt werden:

		\begin{lstlisting}
			$GLOBALS['TYPO3_CONF_VARS']['SC_OPTIONS']['tce']['formevals']
			  [\TYPO3\CMS\Saltedpasswords\Evaluation\FrontendEvaluator::class] = '';
		\end{lstlisting}

	\end{itemize}

	\breakingchange

\end{frame}

% ------------------------------------------------------------------------------
