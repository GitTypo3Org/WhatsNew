% ------------------------------------------------------------------------------
% TYPO3 CMS 7.0 - What's New - Chapter "TypoScript" (English Version)
%
% @author	Michael Schams <schams.net>
% @license	Creative Commons BY-NC-SA 3.0
% @link		http://typo3.org/download/release-notes/whats-new/
% @language	English
% ------------------------------------------------------------------------------
% LTXE-CHAPTER-UID:		8ca51a26-174e47a4-c9321443-37caf28c
% LTXE-CHAPTER-NAME:	TypoScript
% ------------------------------------------------------------------------------

\section{TSconfig \& TypoScript}

% ------------------------------------------------------------------------------
% LTXE-SLIDE-START
% LTXE-SLIDE-UID:		4e11c84b-c70aab50-69aaa2c7-3e9a13ef
% LTXE-SLIDE-ORIGIN:	c65cc12d-2f8c9d53-37d6577f-8a2f8357 English
% LTXE-SLIDE-TITLE:		TSconfig Available to Link Checkers
% LTXE-SLIDE-REFERENCE:	https://forge.typo3.org/issues/54518
% ------------------------------------------------------------------------------

\begin{frame}[fragile]
	\frametitle{TSconfig \& TypoScript}
	\framesubtitle{TSConfig beschikbaar voor linkvalidatoren}

	\begin{itemize}
		\item TSconfig-configuratie wordt gelezen

			\begin{itemize}
				\item of vanuit de backend (als Linkvalidator wordt gebruikt)
				\item of vanuit de configuratie van de taakplanner
			\end{itemize}

		\item Voorbeeld: TSconfig, die kan worden gelezen door Linkchecker:

			\lstinline!mod.linkvalidator.mychecker.myvar = 1!

		\item TSconfig is dan beschikbaar als \texttt{\$this->tsConfig}
	\end{itemize}

\end{frame}

% ------------------------------------------------------------------------------
% LTXE-SLIDE-START
% LTXE-SLIDE-UID:		6f44d70b-75e06bf4-b0297b0f-d3a4460a
% LTXE-SLIDE-ORIGIN:	1d38328d-a7245bbe-1b380ab3-88ea4885 English
% LTXE-SLIDE-TITLE:		Linkcheck: Report Deleted Records
% LTXE-SLIDE-REFERENCE:	https://forge.typo3.org/issues/54519
% ------------------------------------------------------------------------------

\begin{frame}[fragile]
	\frametitle{TSconfig \& TypoScript}
	\framesubtitle{Linkcontrole: Raportage verwijderde records}

	\begin{itemize}
		\item In TYPO3 CMS < 7.0 waarschuwt linkhandler alleen over links naar niet-bestaande of verwijderde records
		\item Sinds TYPO3 CMS >=  7.0 schakelt de volgende TSconfig-instelling een waarschuwing in als links wijzen naar uitgeschakelde records

			\lstinline!mod.linkvalidator.linkhandler.reportHiddenRecords = 1!

	\end{itemize}

\end{frame}

% ------------------------------------------------------------------------------
% LTXE-SLIDE-START
% LTXE-SLIDE-UID:		e12de111-6c0ab5dc-8ec48e4d-7ae61018
% LTXE-SLIDE-ORIGIN:	a27d2c0a-410cbfc8-50f176e2-027ce416 English
% LTXE-SLIDE-TITLE:		RTE: Multiple CSS Classes Per Style
% LTXE-SLIDE-REFERENCE:	https://forge.typo3.org/issues/51905
% ------------------------------------------------------------------------------

\begin{frame}[fragile]
	\frametitle{TSconfig \& TypoScript}
	\framesubtitle{RTE: meerdere CSS-klassen per stijl}

	\begin{itemize}
		\item Moderne frameworks zoals Twitter Bootstrap vereisen meerdere CSS-klassen per HTML tag\newline
			\small Bijvoorbeeld: \texttt{<a class="btn btn-danger">Pas op</a>}\normalsize
		\item Meerdere CSS-klassen worden nu ondersteund waardoor redacteuren maar één stijl hoeven te kiezen

			\begin{lstlisting}
				RTE.classes.[ *classname* ] {
				  .requires = lijst met CSS-klassen
				}
			\end{lstlisting}

	\end{itemize}

\end{frame}

% ------------------------------------------------------------------------------
% LTXE-SLIDE-START
% LTXE-SLIDE-UID:		9000242f-1e42e904-3cbad951-7bcc6f6e
% LTXE-SLIDE-ORIGIN:	e11a6c09-1eb3eaf7-6cebbee9-febef6ba English
% LTXE-SLIDE-TITLE:		RTE: Configure CSS Class As Not-Selectable
% LTXE-SLIDE-REFERENCE:	https://forge.typo3.org/issues/58122
% LTXE-SLIDE-REFERENCE:	https://forge.typo3.org/issues/51905
% ------------------------------------------------------------------------------

\begin{frame}[fragile]
	\frametitle{TSconfig \& TypoScript}
	\framesubtitle{RTE: CSS-klasse als niet-selecteerbaar instellen}

	\begin{itemize}
		\item CSS-klassen kunnen nu als "niet-selecteerbaar" ingesteld worden

			\begin{lstlisting}
				// waarde "1" betekent: klas is selecteerbaar
				// waarde "0" maakt het niet-selecteerbaar
				RTE.classes.[ *classname* ] {
				  .selectable = 1
				}
			\end{lstlisting}

	\end{itemize}

\end{frame}

% ------------------------------------------------------------------------------
% LTXE-SLIDE-START
% LTXE-SLIDE-UID:		60bd2b3d-9d38929d-7e034804-9a247666
% LTXE-SLIDE-ORIGIN:	e607c36f-e1be31da-05ace0c5-991e6e42 English
% LTXE-SLIDE-TITLE:		RTE: Include Multiple CSS Files
% LTXE-SLIDE-REFERENCE:	https://forge.typo3.org/issues/50039
% ------------------------------------------------------------------------------

\begin{frame}[fragile]
	\frametitle{TSconfig \& TypoScript}
	\framesubtitle{RTE: meerdere CSS-bestanden insluiten}

	\begin{itemize}
		\item Er kunnen meerdere CSS-bestanden ingesloten worden

			\begin{lstlisting}
				RTE.default.contentCSS {
				  file1 = fileadmin/rte_stylesheet1.css
				  file2 = fileadmin/rte_stylesheet2.css
				}
			\end{lstlisting}

		\item Als niets ingesteld wordt is de standaard:\newline
			\texttt{typo3/sysext/rtehtmlarea/res/contentcss/default.css}

	\end{itemize}

\end{frame}

% ------------------------------------------------------------------------------
% LTXE-SLIDE-START
% LTXE-SLIDE-UID:		ee825eb2-4e06bd8a-27fdda30-54d28cd8
% LTXE-SLIDE-ORIGIN:	c9d0ad7d-2c5aee11-79878250-9e8ce110 English
% LTXE-SLIDE-TITLE:		Exception Handling When cObjects Are Rendered (1)
% LTXE-SLIDE-REFERENCE:	https://forge.typo3.org/issues/47919
% ------------------------------------------------------------------------------

\begin{frame}[fragile]
	\frametitle{TSconfig \& TypoScript}
	\framesubtitle{Afhandeling exceptie wanneer cObject-en worden gerenderd (1)}

	\begin{itemize}
		\item In TYPO3 CMS < 7.0 breekt de foutmelding als die ontstaat bij het renderen van inhoudsobjecten (bijv. \texttt{USER}) de hele frontend
		\item Sinds TYPO3 CMS >= 7.0 is afhandeling van excepties toegevoegd waarmee een bericht afgebeeld kan worden in plaats van het foute cObject
	\end{itemize}

%	*TODO* it would be so nice, if someone could create a screenshot of this scenario!
%
%	\begin{figure}\vspace*{-0.3cm}
%		\includegraphics[width=0.90\linewidth]{TypoScript/cObjectExceptionHandling.png}
%	\end{figure}

\end{frame}

% ------------------------------------------------------------------------------
% LTXE-SLIDE-START
% LTXE-SLIDE-UID:		28313dae-c6c48c42-c7762748-4ee8d79c
% LTXE-SLIDE-ORIGIN:	1132ffa6-a06b7448-0b59e149-a28c2d43 English
% LTXE-SLIDE-TITLE:		Exception Handling When cObjects Are Rendered (2)
% LTXE-SLIDE-REFERENCE:	https://forge.typo3.org/issues/47919
% ------------------------------------------------------------------------------

\begin{frame}[fragile]
	\frametitle{TSconfig \& TypoScript}
	\framesubtitle{Afhandeling exceptie wanneer cObject-en worden gerenderd (2)}
	
	\lstset{
		basicstyle=\tiny\ttfamily
	}
	
	\begin{lstlisting}
		# standaard exceptie-afhandeling (actief in context "production")
		config.contentObjectExceptionHandler = 1

		# configuratie van een klasse voor exceptie-afhandeling
		config.contentObjectExceptionHandler =
		  TYPO3\CMS\Frontend\ContentObject\Exception\ProductionExceptionHandler

		# eigen foutmelding (toon willekeurige foutcode)
		config.contentObjectExceptionHandler.errorMessage = Oeps, dat ging fout. Code: %s

		# configuratie van exceptiecodes die niet afgehandeld worden
		tt_content.login.20.exceptionHandler.ignoreCodes.10 = 1414512813

		# deactivatie van exceptie-afhandeling voor specifieke plug-ins of inhoudobjecten
		tt_content.login.20.exceptionHandler = 0

		# ignoreCodes en errorMessage kunnen globaal ingesteld worden...
		config.contentObjectExceptionHandler.errorMessage = Oeps, er ging iets fout. Code: %s
		config.contentObjectExceptionHandler.ignoreCodes.10 = 1414512813

		# ...of lokaal voor specifieke inhoudsobjecten
		tt_content.login.20.exceptionHandler.errorMessage = Oeps, er ging iets fout. Code: %s
		tt_content.login.20.exceptionHandler.ignoreCodes.10 = 1414512813
	\end{lstlisting}

\end{frame}

% ------------------------------------------------------------------------------
