% ------------------------------------------------------------------------------
% TYPO3 CMS 7.5 - What's New - Chapter "Extbase & Fluid" (Dutch Version)
%
% @author	Michael Schams <schams.net>
% @license	Creative Commons BY-NC-SA 3.0
% @link		http://typo3.org/download/release-notes/whats-new/
% @language	Dutch
% ------------------------------------------------------------------------------
% LTXE-CHAPTER-UID:		c9b2e403-e06bb691-ce202054-149c5089
% LTXE-CHAPTER-NAME:	Extbase & Fluid
% ------------------------------------------------------------------------------

\section{Extbase \& Fluid}

% ------------------------------------------------------------------------------
% LTXE-SLIDE-START
% LTXE-SLIDE-UID:		8a5ffe70-70b97a80-46c05b87-e1aa7854
% LTXE-SLIDE-ORIGIN:	8bf4a04e-9453b098-84fa7616-1df8ea8f English
% LTXE-SLIDE-ORIGIN:	7528691a-d33af3f4-edab28ae-ba7c8016 German
% LTXE-SLIDE-TITLE:		Severity-filtering for FlashMessageQueue
% LTXE-SLIDE-REFERENCE:	Feature-7098-SeverityFilteringForFlashMessageQueue.rst
% ------------------------------------------------------------------------------

\begin{frame}[fragile]
	\frametitle{Extbase \& Fluid}
	\framesubtitle{Filter voor FlashMessageQueue}

	\begin{itemize}

		\item Tot TYPO3 CMS 7.5 werden \underline{alle} berichten van de 
			FlashMessageQueue opgehaald of verwijderd

		\item Vanaf TYPO3 CMS 7.5 kan men de berichten filteren op severity (ernst):

			\begin{lstlisting}
				FlashMessageQueue::getAllMessages($severity);
				FlashMessageQueue::getAllMessagesAndFlush($severity);
				FlashMessageQueue::removeAllFlashMessagesFromSession($severity);
				FlashMessageQueue::clear($severity);
			\end{lstlisting}

	\end{itemize}

\end{frame}

% ------------------------------------------------------------------------------
% LTXE-SLIDE-START
% LTXE-SLIDE-UID:		6b0ed855-46db049c-7c643543-0a0f771b
% LTXE-SLIDE-ORIGIN:	1728dca6-0c88a636-5dc7cc00-dfa3266f English
% LTXE-SLIDE-ORIGIN:	484e03e4-cb7828c2-8bf31f48-cabf004a German
% LTXE-SLIDE-TITLE:		Query support for BETWEEN added
% LTXE-SLIDE-REFERENCE:	Feature-47812-QuerySupportForBETWEENAdded.rst
% ------------------------------------------------------------------------------

\begin{frame}[fragile]
	\frametitle{Extbase \& Fluid}
	\framesubtitle{Ondersteuning voor "\texttt{between}" in query's}

	\begin{itemize}

		\item Er is ondersteuning toegevoegd voor het gebruik van \texttt{between} 
			in de Extbase Query object

		\item Dit heeft geen voordelen voor de prestaties aangezien databases
			"\texttt{between}" intern sowieso al converteren naar:\newline
			\texttt{min <= expr AND expr <= max}

		\item De nieuwe Extbase-feature repliceert het gedrag van de databases door gebruik te 
			maken van een logische \texttt{AND}-conditie, dus dit werkt in alle databases

			\begin{lstlisting}
				$query->matching(
				  $query->between('uid', 3, 5)
				);
			\end{lstlisting}

	\end{itemize}

\end{frame}

% ------------------------------------------------------------------------------
% LTXE-SLIDE-START
% LTXE-SLIDE-UID:		b01bb0db-c1288692-7bb6fc06-9a0caa2b
% LTXE-SLIDE-ORIGIN:	0f2d87b1-c907b8a6-44612e51-e84e6490 English
% LTXE-SLIDE-ORIGIN:	12b78f4c-7291d788-ea865da7-e6501a2e German
% LTXE-SLIDE-TITLE:		Added support for multiple FlashMessage queues
% LTXE-SLIDE-REFERENCE:	Feature-64726-UsingArbitraryFlashmessageQueues.rst
% ------------------------------------------------------------------------------

\begin{frame}[fragile]
	\frametitle{Extbase \& Fluid}
	\framesubtitle{Meerdere FlashMessageQueues}

	\begin{itemize}

		\item Het is nu mogelijk om meerdere FlashMessageQueues te implementeren:

			\begin{lstlisting}
				$queueIdentifier = 'myQueue';
				$this->controllerContext->getFlashMessageQueue($queueIdentifier);
			\end{lstlisting}

		\item Dit kan als volgt in Fluid worden gebruikt:

			\begin{lstlisting}
				<f:flashMessages queueIdentifier="myQueue" ></f:flashMessages>
			\end{lstlisting}

	\end{itemize}

\end{frame}

% ------------------------------------------------------------------------------
% LTXE-SLIDE-START
% LTXE-SLIDE-UID:		17566034-bbc4a816-f0e32436-b59fb4c5
% LTXE-SLIDE-ORIGIN:	6915482b-a7f5bd30-f04b3dc6-d94a5f2a English
% LTXE-SLIDE-ORIGIN:	2f1002c1-6bde766c-3b1ad500-ab9ccf0c German
% LTXE-SLIDE-TITLE:		Introduced MediaViewHelper (1)
% LTXE-SLIDE-REFERENCE:	Feature-66366-IntroducedMediaViewHelper.rst
% ------------------------------------------------------------------------------

\begin{frame}[fragile]
	\frametitle{Extbase \& Fluid}
	\framesubtitle{MediaViewHelper (1)}

	% decrease font size for code listing
	\lstset{basicstyle=\tiny\ttfamily}

	\begin{itemize}

		\item Voor de eenvoudige weergave van video, audio en andere bestandstypen met een Renderer, 
			is de \texttt{MediaViewHelper} beschikbaar gemaakt

		\item \texttt{MediaViewHelper} kijkt eerst of er een Renderer beschikbaar is voor het betreffende 
			bestand. Zo niet, dan wordt er teruggevallen op een image-tag

		\item Voorbeelden:

			\begin{lstlisting}
				<code title="Image Object">
				  <f:media file="{file}" width="400" height="375" ></f:media>
				</code>

				<output>
				  <img alt="alt set in image record" src="fileadmin/_processed_/323223424.png"
				    width="396" height="375" />
				</output>
			\end{lstlisting}

	\end{itemize}

\end{frame}

% ------------------------------------------------------------------------------
% LTXE-SLIDE-START
% LTXE-SLIDE-UID:		90b947de-1e91e943-dbeeb594-49923f5f
% LTXE-SLIDE-ORIGIN:	8bd8314d-631b7ae9-3102867f-804046fa English
% LTXE-SLIDE-ORIGIN:	73fe507d-772b6011-7457429b-4039dd5c German
% LTXE-SLIDE-TITLE:		Introduced MediaViewHelper (2)
% LTXE-SLIDE-REFERENCE:	Feature-66366-IntroducedMediaViewHelper.rst
% ------------------------------------------------------------------------------

\begin{frame}[fragile]
	\frametitle{Extbase \& Fluid}
	\framesubtitle{MediaViewHelper (2)}

	% decrease font size for code listing
	\lstset{basicstyle=\tiny\ttfamily}

	\begin{itemize}

		\item Voorbeelden (vervolg):

			\begin{lstlisting}
				<code title="MP4 Video Object">
				  <f:media file="{file}" width="400" height="375" ></f:media>
				</code>

				<output>
				  <video width="400" height="375" controls>
				    <source src="fileadmin/user_upload/my-video.mp4" type="video/mp4">
				  </video>
				</output>

				<code title="MP4 Video Object with loop and autoplay option set">
				  <f:media file="{file}" width="400" height="375"
				    additionalConfig="{loop: '1', autoplay: '1'}" ></f:media>
				</code>

				<output>
				  <video width="400" height="375" controls loop>
				    <source src="fileadmin/user_upload/my-video.mp4" type="video/mp4">
				  </video>
				</output>
			\end{lstlisting}

	\end{itemize}

\end{frame}

% ------------------------------------------------------------------------------
% LTXE-SLIDE-START
% LTXE-SLIDE-UID:		d2e83be9-60fc295d-1320bbf2-d5bbdf84
% LTXE-SLIDE-ORIGIN:	5d858cf5-81f4cf18-1b13e093-6a00c98a English
% LTXE-SLIDE-ORIGIN:	2abe5d2e-c1c7e93b-c86559da-81196fdb German
% LTXE-SLIDE-TITLE:		Adopt form to support the Extbase/ Fluid MVC stack (1)
% LTXE-SLIDE-REFERENCE:	Feature-69401-AdoptFormToSupportTheExtbaseFluidMVCStack.rst
% ------------------------------------------------------------------------------

\begin{frame}[fragile]
	\frametitle{Extbase \& Fluid}
	\framesubtitle{Systemextensie \texttt{form} (1)}

	\begin{itemize}

		\item Systemextensie \texttt{form} (inclusief het eigen datamodel, de controllers, validatie 
			en sjablonen) is aangepast aan het MVC-model van Extbase/Fluid

		\item Doordat de weergave nu volledig op Fluid is gebaseerd, is het eenvoudiger 
			om eigen sjablonen en ViewHelpers te gebruiken

		\item Elk form-element heeft een eigen Partial, die kan worden geconfigureerd met de 
			TypoScript-optie \texttt{partialPath = ...}

	\end{itemize}

\end{frame}

% ------------------------------------------------------------------------------
% LTXE-SLIDE-START
% LTXE-SLIDE-UID:		018fe5d3-85346731-80a136cb-0caa2632
% LTXE-SLIDE-ORIGIN:	20c7837c-5b841f21-fb3ead22-265aebe4 English
% LTXE-SLIDE-ORIGIN:	d45facb0-73989d92-ffbc7c2b-46f583ba German
% LTXE-SLIDE-TITLE:		Adopt form to support the Extbase/ Fluid MVC stack (2)
% LTXE-SLIDE-REFERENCE:	Feature-69401-AdoptFormToSupportTheExtbaseFluidMVCStack.rst
% ------------------------------------------------------------------------------

\begin{frame}[fragile]
	\frametitle{Extbase \& Fluid}
	\framesubtitle{Systemextensie \texttt{form} (2)}

	\begin{itemize}

		\item Er zijn 3 nieuwe ViewHelpers beschikbaar:

			\begin{itemize}
				\item \texttt{AggregateSelectOptionsViewHelper} (voor optgroup-tags)
				\item \texttt{SelectViewHelper} (voor optgroup-tags)
				\item \texttt{PlainMailViewHelper} (voor het renderen van e-mail in platte tekst)
			\end{itemize}

		\item Bovendien zijn er 3 Views:

			\begin{itemize}
				\item \texttt{show} (het formulier zelf)
				\item \texttt{confirmation} (de bevestigingspagina)
				\item \texttt{postProcessor}/\texttt{mail} (de e-mail)
			\end{itemize}

		\item Sjabloonpaden en de zichtbaarheid van velden kunnen voor elke View afzonderlijk 
			worden ingesteld

	\end{itemize}

\end{frame}

% ------------------------------------------------------------------------------
% LTXE-SLIDE-START
% LTXE-SLIDE-UID:		9c8a9046-03487a8e-cb5a601e-6baecf4d
% LTXE-SLIDE-ORIGIN:	20b68c91-1b501686-f97f7b55-0822d9d1 English
% LTXE-SLIDE-ORIGIN:	cd99ea8c-e6234f0f-caf559ed-c3f825c0 German
% LTXE-SLIDE-TITLE:		Add annotation for CLI only commands
% LTXE-SLIDE-REFERENCE:	Feature-68746-AddAnnotationForCLIOnlyCommands.rst
% ------------------------------------------------------------------------------

\begin{frame}[fragile]
	\frametitle{Extbase \& Fluid}
	\framesubtitle{Annotatie \texttt{@cli}}

	\begin{itemize}

		\item Door gebruik te maken van de nieuwe annotatie \texttt{@cli}, kan worden aangegeven dat commando's in 
			Extbase-CommandControllers alleen gebruikt mogen worden vanaf de command-line-interface

		\item Deze commando's kunnen niet als taak worden geselecteerd in de scheduler

		\item Typische use-cases zijn commando's zoals \texttt{extbase:help:help}

	\end{itemize}

\end{frame}

% ------------------------------------------------------------------------------
