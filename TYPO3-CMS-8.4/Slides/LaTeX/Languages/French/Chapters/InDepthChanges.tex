% ------------------------------------------------------------------------------
% TYPO3 CMS 8.4 - What's New - Chapter "In-Depth Changes" (French Version)
%
% @author	Michael Schams <schams.net>
% @license	Creative Commons BY-NC-SA 3.0
% @link		http://typo3.org/download/release-notes/whats-new/
% @language	French
% ------------------------------------------------------------------------------
% LTXE-CHAPTER-UID:		5ebcecbe-66abfa57-cf38bc00-aa637965
% LTXE-CHAPTER-NAME:	Changements en profondeur
% ------------------------------------------------------------------------------

\section{Changements en profondeur}
\begin{frame}[fragile]
	\frametitle{Changements en profondeur}

	\begin{center}\huge{Chapitre 3~:}\end{center}
	\begin{center}\huge{\color{typo3darkgrey}\textbf{Changements en profondeur}}\end{center}

\end{frame}

% ------------------------------------------------------------------------------
% LTXE-SLIDE-START
% LTXE-SLIDE-UID:		45992640-cfa74b17-df372e88-043fe686
% LTXE-SLIDE-ORIGIN:	bde270e6-ffef8544-ea472ed5-89ba8c3d English
% LTXE-SLIDE-TITLE:		#52877: Remove ExtJS Viewport
% ------------------------------------------------------------------------------

\begin{frame}[fragile]
	\frametitle{Changements en profondeur}
	\framesubtitle{Retrait de ExtJS}

	\begin{itemize}
		\item Le composant ExtJS \texttt{TYPO3.Viewport} est retiré
		\item \texttt{Ext.layout} et \texttt{Ext.Viewport} ne sont plus utilisés dans le backend
		\item La fonctionnalité est réimplémentée à l'aide d'une solution utilisant le JavaScript natif, jQuery et CSS
		\item Les composants de notification ExtJS \texttt{TYPO3.Window} et \texttt{TYPO3.Dialog} sont retirés
		\item Parties/tâches restantes du retrait complet d'ExtJS~:

		\begin{itemize}
			\item arborescence des pages
			\item fonctionnalité de glissé-déposé de l'extension form
			\item fonctionnalité ExtDirect
		\end{itemize}

	\end{itemize}

\end{frame}

% ------------------------------------------------------------------------------
% LTXE-SLIDE-START
% LTXE-SLIDE-UID:		c8897bf5-04a52c56-ec9bc539-680dbd8c
% LTXE-SLIDE-ORIGIN:	11a9285f-18bdbfac-88ece0c7-b3f2276a English
% LTXE-SLIDE-TITLE:		Doctrine DBAL
% ------------------------------------------------------------------------------

\begin{frame}[fragile]
	\frametitle{Changements en profondeur}
	\framesubtitle{Doctrine DBAL}

	\begin{itemize}
		\item Progrès supplémentaires dans la migration des appels à la base de données du noyau TYPO3 vers Doctrine DBAL
		\item La persistence Extbase est complètement contruite sur le QueryBuilder de Doctrine DBAL
		\item \texttt{EXT:dbal} et \texttt{EXT:adodb} sont retirés du noyau de TYPO3\newline
			\smaller
				Si une extension utilise l'ancienne API \texttt{TYPO3\_DB} pour requêter des tables non-MySQL,
				ces extensions peuvent être installées depuis le TER.
			\normalsize

		\item L'accès rapide à \texttt{TYPO3\_DB} est retiré dans la majorité des classes PHP du noyau de TYPO3\newline
			\smaller
				(utiliser \texttt{\$GLOBALS[TYPO3\_DB]} reste possible mais découragé)
			\normalsize

	\end{itemize}

\end{frame}

% ------------------------------------------------------------------------------
% LTXE-SLIDE-START
% LTXE-SLIDE-UID:		1d9094f4-a73cab9e-c5205bed-c5ca3c58
% LTXE-SLIDE-ORIGIN:	bb5c02d9-8dddd379-e73500a9-9534588b English
% LTXE-SLIDE-TITLE:		#77900: TYPO3 CMS supports TypeScript (1)
% ------------------------------------------------------------------------------

\begin{frame}[fragile]
	\frametitle{Changements en profondeur}
	\framesubtitle{Support de TypeScript (1)}

	\begin{itemize}
		\item \textbf{TypeScript} est intégré au noyau TYPO3 pour la prise en charge interne de JavaScript
		\item TypeScript est un langage de programmation libre et ouvert développé et maintenu par Microsoft
		\item C'est un surensemble strict de JavaScript, pouvant compiler du JavaScript
		\item Plus de détails~: \url{https://www.typescriptlang.org}
		\item Une tâche grunt compile chaque fichier TypeScript (.ts) en fichier JavaScript (.js) et produit un module AMD
	\end{itemize}

	\small
		Note~: tous les modules AMD actuellement dans TYPO3 CMS doivent être portés vers TypeScript
		afin d'assurer la prise en charge long-terme de JavaScript.
		L'objectif est de migrer tous les modules AMD en TypeScript avant le sortie de CMS 8 LTS.
	\normalsize

\end{frame}

% ------------------------------------------------------------------------------
% LTXE-SLIDE-START
% LTXE-SLIDE-UID:		375ce472-01a6dade-27151f45-66c981ca
% LTXE-SLIDE-ORIGIN:	d467c514-583b44e5-03e4927f-38ba0ad1 English
% LTXE-SLIDE-TITLE:		#77900: TYPO3 CMS supports TypeScript (2)
% ------------------------------------------------------------------------------

\begin{frame}[fragile]
	\frametitle{Changements en profondeur}
	\framesubtitle{Support de TypeScript (2)}

	Les règles les plus importantes de TypeScript sont définies dans un ensemble de règles vérifiées par le
	Linter JavaScript~:

	\begin{itemize}\smaller
		\item Toujours définir les types et ceux de retour, même si TypeScript fourni un type par défaut
		\item Portée des variables~: préférer \texttt{let} au lieu de \texttt{var}
		\item Les propriétés optionnelles dans les interfaces ne sont pas permises pour le noyau
		\item Une interface n'étendera jamais une classe.
		\item Iterables~: utiliser \texttt{for (i of list)} au lieu de \texttt{for (i in list)}
		\item Utiliser le mot-clé \texttt{implements}, même si TypeScript ne le requière pas
		\item Toute classe ou interface doit être déclarée avec «~export~» pour assurer la
			réutilisation ou l'export d'une instance d'objet pour le code existant qui ne peut être mis à jour.
	\end{itemize}

	\small(certaines règles ne peuvent cependant pas être vérifiées par le Linter)\normalsize

\end{frame}

% ------------------------------------------------------------------------------
% LTXE-SLIDE-START
% LTXE-SLIDE-UID:		1683d2b1-593ad140-6e52d3fa-d7e05510
% LTXE-SLIDE-ORIGIN:	2e0d0d9e-44771297-efe747f4-89c8b960 English
% LTXE-SLIDE-TITLE:		#38496: Shortcuts take all URL parameters into account
% ------------------------------------------------------------------------------

\begin{frame}[fragile]
	\frametitle{Changements en profondeur}
	\framesubtitle{Paramètres d'URL dans les raccourcis}

	\begin{itemize}
		\item Les raccourcis prennent en compte l'ensemble des paramètres.
		\item Exemple~:

			\begin{itemize}
				\item La page d'UID 2 est un raccourci vers la page d'UID 1
				\item Configuration TypoScript définie~: \texttt{config.linkVars = L}
			\end{itemize}

		\item \underline{Ancien} comportement~:\newline
			\smaller
				\tabto{0.5cm}\texttt{http://example.com?id=2\&L=1\&customparam=X}\newline
				redirige vers~:\newline
				\tabto{0.5cm}\texttt{http://example.com?id=1\&L=1}
			\normalsize

		\item \underline{Nouveau} comportement~:\newline
			\smaller
				\tabto{0.5cm}\texttt{http://example.com?id=2\&L=1\&customparam=X}\newline
				redirige vers~:\newline
				\tabto{0.5cm}\texttt{http://example.com?id=1\&L=1\&customparam=X}
			\normalsize

	\end{itemize}

\end{frame}

% ------------------------------------------------------------------------------
% LTXE-SLIDE-START
% LTXE-SLIDE-UID:		a60d9f7f-2ab0b63e-7a04f221-338cb20a
% LTXE-SLIDE-ORIGIN:	3b3fcb91-081940b0-2ec33de4-ef56f577 English
% LTXE-SLIDE-TITLE:		#75031 and #75032: Fluidification of TypoScriptTemplate...ModuleFunctionController
% ------------------------------------------------------------------------------

\begin{frame}[fragile]
	\frametitle{Changements en profondeur}
	\framesubtitle{Migration vers Fluid}

	\begin{itemize}
		\item Le code HTML est migré de code PHP en modèle Fluid
		\item Affecte les méthodes~:\newline

			\smaller\texttt{TypoScriptTemplateInformationModuleFunctionController\newline
				->tableRow()}\normalsize\newline
			\smaller\texttt{TypoScriptTemplateConstantEditorModuleFunctionController\newline
				->displayExample()}\normalsize

		\item Appeler ces méthodes résulte en une erreur fatale.

	\end{itemize}

\end{frame}

% ------------------------------------------------------------------------------
% LTXE-SLIDE-START
% LTXE-SLIDE-UID:		ec545ec6-3a0f94ab-21e02949-c6d93879
% LTXE-SLIDE-ORIGIN:	5f2bab62-fc48a5b2-0f70e9b0-834e47ce English
% LTXE-SLIDE-TITLE:		PageRenderer and ResourceCompressor support EXT: syntax
% LTXE-SLIDE-REFERENCE:	#77589: EXT: syntax in PageRenderer and Compressor
% ------------------------------------------------------------------------------

\begin{frame}[fragile]
	\frametitle{Changements en profondeur}
	\framesubtitle{PageRenderer et Compressor}

	% decrease font size for code listing
	\lstset{basicstyle=\smaller\ttfamily}

	\begin{itemize}

		\item Les classes PHP PageRenderer et ResourceCompressor supportent la syntaxe
			\texttt{EXT:} pour référencer les fichiers JS et CSS dans les dossiers
			d'extension.\newline
			\textbf{Précédemment~:}

			\begin{lstlisting}
				$this->pageRenderer->addJsFile(
				  ExtensionManagementUtility::extRelPath('myextension') .
				  'Resources/Public/JavaScript/example.js'
				);
			\end{lstlisting}

			\textbf{Maintenant~:}

			\begin{lstlisting}
				$this->pageRenderer->addJsFile(
				  'EXT:myextension/Resources/Public/JavaScript/example.js'
				);
			\end{lstlisting}

	\end{itemize}

\end{frame}

% ------------------------------------------------------------------------------
% LTXE-SLIDE-START
% LTXE-SLIDE-UID:		a83c639a-9c8e59cc-76e4083f-d2acb606
% LTXE-SLIDE-ORIGIN:	081940b0-ef56f577-3b3fcb91-0b008194 English
% LTXE-SLIDE-TITLE:		Miscellaneous (1) (#77700, #77750, #77814 and #78222)
% LTXE-SLIDE-REFERENCE:	#77700: EXT:indexed_search_mysql merged into EXT:indexed_search
% LTXE-SLIDE-REFERENCE:	#77750: Return value of ContentObjectRenderer::exec_Query changed
% LTXE-SLIDE-REFERENCE:	#77814: Remove feature subsearch from indexed search
% LTXE-SLIDE-REFERENCE:	#78222: Extension autoload information is now in typo3conf/autoload
% ------------------------------------------------------------------------------

\begin{frame}[fragile]
	\frametitle{Changements en profondeur}
	\framesubtitle{Divers}

	\begin{itemize}

		\item EXT:indexed\_search\_mysql fusionné dans EXT:indexed\_search

		\item La fonctionnalité «~subsearch~» est retirée de EXT:indexed\_search\_mysql\newline
			\smaller
				(L'option TypoScript \texttt{plugin.tx\_indexedsearch.clearSearchBox} est retirée)
			\normalsize

		\item Le type de retour de \texttt{ContentObjectRenderer::exec\_Query()} est changé\newline
			\smaller
				(la valeur de retour est toujours du type
					\texttt{\textbackslash
						Doctrine\textbackslash
						DBAL\textbackslash
						Driver\textbackslash
						Statement})
			\normalsize

		\item Pour mieux indiquer que les informations d'auto-chargement ne sont pas
			dans un cache, les fichiers sont déplacés de \texttt{typo3temp/} vers
			\texttt{typo3conf/}\newline
			\smaller
				Note~: Les déployements TYPO3 qui ne profitent pas de composer, peuvent nécessiter
				des ajustements pour prendre en compte ce nouvel emplacement.
			\normalsize

	\end{itemize}

\end{frame}

% ------------------------------------------------------------------------------
