% ------------------------------------------------------------------------------
% TYPO3 CMS 8.4 - What's New - Chapter "In-Depth Changes" (Italian Version)
%
% @author	Michael Schams <schams.net>
% @license	Creative Commons BY-NC-SA 3.0
% @link		http://typo3.org/download/release-notes/whats-new/
% @language	English
% ------------------------------------------------------------------------------
% LTXE-CHAPTER-UID:		5ebcecbe-66abfa57-cf38bc00-aa637965
% LTXE-CHAPTER-NAME:	In-Depth Changes
% ------------------------------------------------------------------------------

\section{Modifiche rilevanti}
\begin{frame}[fragile]
	\frametitle{Modifiche rilevanti}

	\begin{center}\huge{Capitolo 3:}\end{center}
	\begin{center}\huge{\color{typo3darkgrey}\textbf{Modifiche rilevanti}}\end{center}

\end{frame}

% ------------------------------------------------------------------------------
% LTXE-SLIDE-START
% LTXE-SLIDE-UID:		8e54d4ca-0d08b0d3-00e86294-a5b10a72
% LTXE-SLIDE-ORIGIN:	bde270e6-ffef8544-ea472ed5-89ba8c3d English
% LTXE-SLIDE-TITLE:		#52877: Remove ExtJS Viewport
% ------------------------------------------------------------------------------

\begin{frame}[fragile]
	\frametitle{Modifiche rilevanti}
	\framesubtitle{Rimosso ExtJS}

	\begin{itemize}
		\item Il componente ExtJS \texttt{TYPO3.Viewport} è stato rimosso
		\item \texttt{Ext.layout} e \texttt{Ext.Viewport} non sono più utilizzati nelle visualizzazioni di backend
		\item Le funzionalità sono state reimplementate con codice JavaScript nativo, jQuery e soluzioni CSS
		\item I componenti di notifica ExtJS \texttt{TYPO3.Window} e \texttt{TYPO3.Dialog} sono stati rimossi
		\item Parti ancora rimaste per la completa rimozione di ExtJS:

		\begin{itemize}
			\item albero delle pagine
			\item funzionalità drag'n drop nell'estenzione form
			\item funzionalità ExtDirect
		\end{itemize}

	\end{itemize}

\end{frame}

% ------------------------------------------------------------------------------
% LTXE-SLIDE-START
% LTXE-SLIDE-UID:		2eb26d3c-ccae9bcb-be7ec85a-5a5487dc
% LTXE-SLIDE-ORIGIN:	11a9285f-18bdbfac-88ece0c7-b3f2276a English
% LTXE-SLIDE-TITLE:		Doctrine DBAL
% ------------------------------------------------------------------------------

\begin{frame}[fragile]
	\frametitle{Modifiche rilevanti}
	\framesubtitle{Doctrine DBAL}

	\begin{itemize}
		\item Ulteriori progressi sono stati fatti nella migrazione di tutte le chiamate del database dal core di TYPO3 attraverso Doctrine DBAL
		\item La persistenza Extbase è ora basata completamente su QueryBuilder di Doctrine DBAL
		\item \texttt{EXT:dbal} e \texttt{EXT:adodb} sono state rimosse dal core di TYPO3\newline
			\smaller
				Se un estensione terza utilizza le vecchie API \texttt{TYPO3\_DB} per interagire con tabelle di database non-MySQL,
				queste due estensioni devono essere installate dal TER.
			\normalsize

		\item Le funzionalità shorthand di \texttt{TYPO3\_DB} sono state rimosse per la maggior parte delle classi PHP di base in TYPO3\newline
			\smaller
				(l'uso di \texttt{\$GLOBALS[TYPO3\_DB]} è ancora possibile, ma sconsigliato)
			\normalsize

	\end{itemize}

\end{frame}

% ------------------------------------------------------------------------------
% LTXE-SLIDE-START
% LTXE-SLIDE-UID:		858aefbc-d575c0d9-d5e24f9a-3968e46e
% LTXE-SLIDE-ORIGIN:	bb5c02d9-8dddd379-e73500a9-9534588b English
% LTXE-SLIDE-TITLE:		#77900: TYPO3 CMS supports TypeScript (1)
% ------------------------------------------------------------------------------

\begin{frame}[fragile]
	\frametitle{Modifiche rilevanti}
	\framesubtitle{Supporto TypeScript (1)}

	\begin{itemize}
		\item \textbf{TypeScript} è stato introdotto nel core di TYPO3 core per la gestione interna di Javascript
		\item TypeScript è un linguaggio di programmazione libero e open source sviluppato e mantenuto da Microsoft
		\item Si tratta di un rigoroso superset di JavaScript, che può compilare JavaScript
		\item Maggiori dettagli: \url{https://www.typescriptlang.org}
		\item Un processo grunt compila ogni file TypeScript (.ts) in un file Javascript (.js) e produce un modulo AMD
	\end{itemize}

	\small
		Nota: tutti i moduli AMD attualmente in TYPO3 CMS dovranno essere migrati a TypeScript per garantire una futura gestione avanzata di JavaScript.
		L'obbiettivo è migrare tutti i moduli AMD a TypeScript prima del rilascio della versione CMS 8 LTS.
	\normalsize

\end{frame}

% ------------------------------------------------------------------------------
% LTXE-SLIDE-START
% LTXE-SLIDE-UID:		888acaae-7a06ba16-50124d7d-c2ae1958
% LTXE-SLIDE-ORIGIN:	d467c514-583b44e5-03e4927f-38ba0ad1 English
% LTXE-SLIDE-TITLE:		#77900: TYPO3 CMS supports TypeScript (2)
% ------------------------------------------------------------------------------

\begin{frame}[fragile]
	\frametitle{Modifiche rilevanti}
	\framesubtitle{Supporto TypeScript (2)}

	\begin{itemize}
		\item Le regole più importanti per TypeScript sono definite in un rulesets che è gestito da TypeScript Linter:

			\begin{itemize}
				\item Definire e restituire sempre un tipo, anche se TypeScript definisce un tipo di default
				\item Variabili di scoping: preferire \texttt{let} invece di \texttt{var}
				\item Le proprietà opzionali nelle interfacce non sono permesse per il core
				\item Un interfaccia non può mai estendere una classe
				\item Iterazioni: usare \texttt{for (i of list)} invece di \texttt{for (i in list)}
				\item Usare sempre \texttt{implements}, anche se TypeScript non lo richiede
				\item Ogni classe o interfaccia devono essere dichiarati con "export" per permettere il riuso o l'esportazione di un istanza dell'oggetto per il codice esistente e che non può essere aggiornato al momento.
			\end{itemize}

			\small(non tutte le regole sono ancora verificate da Linter)\normalsize

	\end{itemize}

\end{frame}

% ------------------------------------------------------------------------------
% LTXE-SLIDE-START
% LTXE-SLIDE-UID:		823f088f-60511ab1-5366ea5a-2307b6d4
% LTXE-SLIDE-ORIGIN:	2e0d0d9e-44771297-efe747f4-89c8b960 English
% LTXE-SLIDE-TITLE:		#38496: Shortcuts take all URL parameters into account
% ------------------------------------------------------------------------------

\begin{frame}[fragile]
	\frametitle{Modifiche rilevanti}
	\framesubtitle{Parametri URL nei Shortcuts}

	\begin{itemize}
		\item Gli Shortcuts prendono in considerazione tutti i parametri delle URL.
		\item Esempio:

			\begin{itemize}
				\item La pagina UID 2 è uno shortcut della pagina UID 1
				\item La configurazione TypoScript prevede: \texttt{config.linkVars = L}
			\end{itemize}

		\item Comortamento \underline{vecchio}:\newline
			\smaller
				\tabto{0.5cm}\texttt{http://example.com?id=2\&L=1\&customparam=X}\newline
				redirige a:\newline
				\tabto{0.5cm}\texttt{http://example.com?id=1\&L=1}
			\normalsize

		\item Comportamento \underline{Nuovo}:\newline
			\smaller
				\tabto{0.5cm}\texttt{http://example.com?id=2\&L=1\&customparam=X}\newline
				redirige a:\newline
				\tabto{0.5cm}\texttt{http://example.com?id=1\&L=1\&customparam=X}
			\normalsize

	\end{itemize}

\end{frame}

% ------------------------------------------------------------------------------
% LTXE-SLIDE-START
% LTXE-SLIDE-UID:		cadd3c2b-41a10d8f-0872802e-d7856a5d
% LTXE-SLIDE-ORIGIN:	3b3fcb91-081940b0-2ec33de4-ef56f577 English
% LTXE-SLIDE-TITLE:		#75031 and #75032: Fluidification of TypoScriptTemplate...ModuleFunctionController
% ------------------------------------------------------------------------------

\begin{frame}[fragile]
	\frametitle{Modifiche rilevanti}
	\framesubtitle{Fluidification}

	\begin{itemize}
		\item Il codice HTML è stato migrato da codice PHP code a Fluid template
		\item Metodi interessati:\newline

			\smaller\texttt{TypoScriptTemplateInformationModuleFunctionController\newline
				->tableRow()}\normalsize\newline
			\smaller\texttt{TypoScriptTemplateConstantEditorModuleFunctionController\newline
				->displayExample()}\normalsize

		\item La chiamata a questi metodi ora restituisce un fatal error

	\end{itemize}

\end{frame}

% ------------------------------------------------------------------------------
% LTXE-SLIDE-START
% LTXE-SLIDE-UID:		35c031d8-491879c3-be76c770-6b61c5d8
% LTXE-SLIDE-ORIGIN:	5f2bab62-fc48a5b2-0f70e9b0-834e47ce English
% LTXE-SLIDE-TITLE:		PageRenderer and ResourceCompressor support EXT: syntax
% LTXE-SLIDE-REFERENCE:	#77589: EXT: syntax in PageRenderer and Compressor
% ------------------------------------------------------------------------------

\begin{frame}[fragile]
	\frametitle{Modifiche rilevanti}
	\framesubtitle{PageRenderer e Compressor}

	% decrease font size for code listing
	\lstset{basicstyle=\smaller\ttfamily}

	\begin{itemize}

		\item Le classi PHP PageRenderer e ResourceCompressor ora supportano la sintassi
			\texttt{EXT:} per referenziare file JS e CSS dentro le directory delle estensioni.\newline
			\textbf{Prima:}

			\begin{lstlisting}
				$this->pageRenderer->addJsFile(
				  ExtensionManagementUtility::extRelPath('myextension') .
				  'Resources/Public/JavaScript/example.js'
				);
			\end{lstlisting}

			\textbf{Ora è possibile:}

			\begin{lstlisting}
				$this->pageRenderer->addJsFile(
				  'EXT:myextension/Resources/Public/JavaScript/example.js'
				);
			\end{lstlisting}

	\end{itemize}

\end{frame}

% ------------------------------------------------------------------------------
% LTXE-SLIDE-START
% LTXE-SLIDE-UID:		17824e4d-c20e6104-8dac04e4-c5202ad9
% LTXE-SLIDE-ORIGIN:	081940b0-ef56f577-3b3fcb91-0b008194 English
% LTXE-SLIDE-TITLE:		Miscellaneous (1) (#77700, #77750, #77814 and #78222)
% LTXE-SLIDE-REFERENCE:	#77700: EXT:indexed_search_mysql merged into EXT:indexed_search
% LTXE-SLIDE-REFERENCE:	#77750: Return value of ContentObjectRenderer::exec_Query changed
% LTXE-SLIDE-REFERENCE:	#77814: Remove feature subsearch from indexed search
% LTXE-SLIDE-REFERENCE:	#78222: Extension autoload information is now in typo3conf/autoload
% ------------------------------------------------------------------------------

\begin{frame}[fragile]
	\frametitle{Modifiche rilevanti}
	\framesubtitle{Varie (1)}

	\begin{itemize}

		\item EXT:indexed\_search\_mysql è stato fuso con EXT:indexed\_search

		\item La funzionalità "subsearch" è stata rimossa da EXT:indexed\_search\_mysql\newline
			\smaller
				(L'opzione TypoScript \texttt{plugin.tx\_indexedsearch.clearSearchBox} è stata rimossa)
			\normalsize

		\item Il tipo restituito da \texttt{ContentObjectRenderer::exec\_Query()} è cambiato\newline
			\smaller
				(Il valore restituito è ora
					\texttt{\textbackslash
						Doctrine\textbackslash
						DBAL\textbackslash
						Driver\textbackslash
						Statement}
					)
			\normalsize

		\item Per rendere intuitivo che le informazioni autoload non sono in cache, i file
			sono stati spostati da \texttt{typo3temp/} a \texttt{typo3conf/}\newline
			\smaller
				Nota: Le implementazioni TYPO3, che non utilizzano composer, probabilmente
				avranno bisogno di alcuni aggiustamenti per gestire la nuova posizione.
			\normalsize

	\end{itemize}

\end{frame}

% ------------------------------------------------------------------------------
