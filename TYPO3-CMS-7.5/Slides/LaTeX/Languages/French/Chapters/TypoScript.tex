% ------------------------------------------------------------------------------
% TYPO3 CMS 7.5 - What's New - Chapter "TypoScript" (French Version)
%
% @author	Michael Schams <schams.net>
% @license	Creative Commons BY-NC-SA 3.0
% @link		http://typo3.org/download/release-notes/whats-new/
% @language	French
% ------------------------------------------------------------------------------
% LTXE-CHAPTER-UID:		f607fb32-7f7b68d8-68d0e69a-16738006
% LTXE-CHAPTER-NAME:	TypoScript
% ------------------------------------------------------------------------------

\section{TSconfig \& TypoScript}
\begin{frame}[fragile]
	\frametitle{TSconfig \& TypoScript}

	\begin{center}\huge{Chapitre 2~:}\end{center}
	\begin{center}\huge{\color{typo3darkgrey}\textbf{TSconfig \& TypoScript}}\end{center}

\end{frame}

% ------------------------------------------------------------------------------
% LTXE-SLIDE-START
% LTXE-SLIDE-UID:		596ef197-483261b7-a6d47b0f-39b7e8ee
% LTXE-SLIDE-ORIGIN:	195a3fa3-207484a7-94e7ce76-c821394c English
% LTXE-SLIDE-ORIGIN:	c76757e2-fc02b94b-166a8168-d68774c9 German
% LTXE-SLIDE-TITLE:		Feature #16525: Add conditions to INCLUDE_TYPOSCRIPT
% LTXE-SLIDE-REFERENCE:	Feature-16525-AddConditionsToINCLUDE_TYPOSCRIPT.rst
% ------------------------------------------------------------------------------
\begin{frame}[fragile]
	\frametitle{TSconfig \& TypoScript}
	\framesubtitle{Conditions pour les inclusions TypoScript}

	% decrease font size for code listing
	\lstset{basicstyle=\tiny\ttfamily}

	\begin{itemize}

		\item La propriété supplémentaire «~condition~» est ajoutée à \texttt{INCLUDE\_TYPOSCRIPT}, permettant
			de n'inclure le fichier ou dossier seulement lorsque la condition est respectée

		\begin{lstlisting}
			// only include TypoScript, if current user is logged in:
			<INCLUDE_TYPOSCRIPT: source="FILE:EXT:my_extension/Configuration/TypoScript/feuser.ts"
			  condition="[loginUser = *]">

			// include TypoScript depending on application context:
			<INCLUDE_TYPOSCRIPT: source="FILE:EXT:my_extension/Configuration/TypoScript/staging.ts"
			  condition="applicationContext = /^Production\\/Staging\\/Server\\d+$/">
		\end{lstlisting}

	\end{itemize}

\end{frame}

% ------------------------------------------------------------------------------
% LTXE-SLIDE-START
% LTXE-SLIDE-UID:		75c6742c-5134c205-9df38c2d-a4b67554
% LTXE-SLIDE-ORIGIN:	4003d374-92104168-0eb97927-726037fa English
% LTXE-SLIDE-ORIGIN:	1bde9e8a-82352f02-c5c9f3db-b194cfb1 German
% LTXE-SLIDE-TITLE:		Feature #28243: Introduce TCA option to disable age display of dates per field
% LTXE-SLIDE-REFERENCE:	Feature-28243-IntroduceTcaOptionToDisableAgeDisplay.rst
% ------------------------------------------------------------------------------
\begin{frame}[fragile]
	\frametitle{TSconfig \& TypoScript}
	\framesubtitle{Option TCA~: Afficher l'âge}

	% decrease font size for code listing
	\lstset{basicstyle=\tiny\ttfamily}

	\begin{itemize}

		\item L'option TCA \texttt{disableAgeDisplay} désactive l'affichage de l'âge\newline
			\small
				(par exemple~: \texttt{"2015-08-30 (-27 jours)"})
			\normalsize

			\begin{lstlisting}
				$GLOBALS['TCA']['tt_content']['columns']['date']['config']['disableAgeDisplay'] = true;
			\end{lstlisting}

		\item Bien évidemment, le \texttt{type} du champ doit être \texttt{input}
			et \texttt{eval} doit être défini à \texttt{date}

	\end{itemize}

\end{frame}

% ------------------------------------------------------------------------------
% LTXE-SLIDE-START
% LTXE-SLIDE-UID:		add7a662-4e545957-afe9ace9-a27f414a
% LTXE-SLIDE-ORIGIN:	68c8a68b-f7e50bf3-dc67e0e5-d3a33fd8 English
% LTXE-SLIDE-ORIGIN:	c3aa0981-411fcabb-bd9dca10-16f42898 German
% LTXE-SLIDE-TITLE:		Feature #57632: Include inline language label files with TypoScript (1)
% LTXE-SLIDE-REFERENCE:	Feature-57632-AddInlineLanguageLabelFilesWithTypoScript.rst
% ------------------------------------------------------------------------------
\begin{frame}[fragile]
	\frametitle{TSconfig \& TypoScript}
	\framesubtitle{Configurer les libellés de langue exportés en TypoScript (1)}

	% decrease font size for code listing
	\lstset{basicstyle=\tiny\ttfamily}

	\begin{itemize}

		\item Les fichiers de langue XLF peuvent être lus et exportés dans un tableau

		\item Ceci permet d'accéder aux libellés de langue en JavaScript, par exemple

		\item Les trois paramètres optionnels suivants sont supportés~:

			\begin{itemize}
				\item \texttt{selectionPrefix}~:\newline
					seul les identifiants de libellé commençant par ce préfix sont pris en compte
				\item \texttt{stripFromSelectionName}~:\newline
					chaîne retirée des identifiants de libellé
				\item \texttt{errorMode}~:\newline
					mode d'erreur si le fichier n'est pas trouvé~:\newline
					0~: entrée syslog (par défaut), 1~: ignorer, 3~: lever une exception
			\end{itemize}

	\end{itemize}

\end{frame}

% ------------------------------------------------------------------------------
% LTXE-SLIDE-START
% LTXE-SLIDE-UID:		520b16a4-9502c0d8-a775b817-a7cec6d7
% LTXE-SLIDE-ORIGIN:	bb800987-b0e74193-c6ec6370-3b6a062d English
% LTXE-SLIDE-ORIGIN:	289816f4-411fcabb-0981cabb-bd9dca10 German
% LTXE-SLIDE-TITLE:		Feature #57632: Include inline language label files with TypoScript (2)
% LTXE-SLIDE-REFERENCE:	Feature-57632-AddInlineLanguageLabelFilesWithTypoScript.rst
% ------------------------------------------------------------------------------
\begin{frame}[fragile]
	\frametitle{TSconfig \& TypoScript}
	\framesubtitle{Configurer les libellés de langue exportés en TypoScript (2)}

	% decrease font size for code listing
	\lstset{basicstyle=\tiny\ttfamily}

	\begin{itemize}

		\item Exemple~:

			\begin{lstlisting}
				page = PAGE
				page.inlineLanguageLabelFiles {
				  someLabels = EXT:myExt/Resources/Private/Language/locallang.xlf
				  someLabels.selectionPrefix = idPrefix
				  someLabels.stripFromSelectionName = strip_me
				  someLabels.errorMode = 2
				}
			\end{lstlisting}

		\item Sortie~:

			\begin{lstlisting}
				<script type="text/javascript">
				/*<![CDATA[*/
				  var TYPO3 = TYPO3 || {};
				  TYPO3.lang = {"firstLabel":[{"source":"first Label","target":"erstes Label"}],
				  "secondLabel":[{"source":"second Label","target":"zweites Label"}]};
				/*]]>*/
				</script>
			\end{lstlisting}

	\end{itemize}

\end{frame}

% ------------------------------------------------------------------------------
% LTXE-SLIDE-START
% LTXE-SLIDE-UID:		75ffe218-9e875794-0dbeaf4b-16319429
% LTXE-SLIDE-ORIGIN:	bac61b19-18a4db7b-af13de6c-5dbed67b English
% LTXE-SLIDE-ORIGIN:	f14c90e2-bf35bf40-e38d74a7-f4a2e973 German
% LTXE-SLIDE-TITLE:		Feature #59144: Previewing workspace records using Page TSconfig
% LTXE-SLIDE-REFERENCE:	Feature-59144-PageTSconfigWorkspacePreview.rst
% ------------------------------------------------------------------------------
\begin{frame}[fragile]
	\frametitle{TSconfig \& TypoScript}
	\framesubtitle{Prévisualisation dans les espaces de travail par TSconfig}

	% decrease font size for code listing
	\lstset{basicstyle=\tiny\ttfamily}

	\begin{itemize}

		\item TYPO3 CMS génère des liens de prévisualisation seulement pour les tables \texttt{tt\_content},
			\texttt{pages} et \texttt{pages\_language\_overlay} par défaut

		\item La configuration est possible en TSconfig de page~:

			\begin{lstlisting}
				# use page 123 for previewing workspaces records (in general)
				options.workspaces.previewPageId = 123

				# use the pid field of each record for previewing (in general)
				options.workspaces.previewPageId = field:pid

				# use page 123 for previewing workspaces records (for table tx_myext_table)
				options.workspaces.previewPageId.tx_myext_table = 123

				# use the pid field of each record for previewing (for table tx_myext_table)
				options.workspaces.previewPageId.tx_myext_table = field:pid
			\end{lstlisting}

	\end{itemize}

\end{frame}

% ------------------------------------------------------------------------------
% LTXE-SLIDE-START
% LTXE-SLIDE-UID:		4d5bbdef-82f23801-383d0406-92d96932
% LTXE-SLIDE-ORIGIN:	c4b7358c-07059082-861a3b2d-929e1cc6 English
% LTXE-SLIDE-ORIGIN:	faeca558-5a675884-a1b77fae-36e316eb German
% LTXE-SLIDE-TITLE:		Feature #59591: Image quality definable per sourceCollection
% LTXE-SLIDE-REFERENCE:	Feature-59591-ImageQualityDefinablePerSourceCollection.rst
% ------------------------------------------------------------------------------
\begin{frame}[fragile]
	\frametitle{TSconfig \& TypoScript}
	\framesubtitle{Qualité des images dans \texttt{sourceCollection}}

	% decrease font size for code listing
	\lstset{basicstyle=\tiny\ttfamily}

	\begin{itemize}

		\item La spécification de la qualité des images des entrées \texttt{sourceCollection} est ajoutée

		\item L'option est prioritaire sur la configuration de l'Install Tool\newline
			(enregistrée dans le fichier \texttt{LocalConfiguration.php})

		\item Exemple~:

			\begin{lstlisting}
				# for small retina images
				tt_content.image.20.1.sourceCollection.smallRetina.quality = 80

				# for large retina images
				tt_content.image.20.1.sourceCollection.largeRetina.quality = 65
			\end{lstlisting}

	\end{itemize}

\end{frame}

% ------------------------------------------------------------------------------
% LTXE-SLIDE-START
% LTXE-SLIDE-UID:		28d9200d-62c27059-77342153-f32454c4
% LTXE-SLIDE-ORIGIN:	ad691a22-7ab387b6-6e00c0aa-e24efd91 English
% LTXE-SLIDE-ORIGIN:	0073160e-6c95f70d-fcaf008f-3fd84819 German
% LTXE-SLIDE-TITLE:		Feature #67880: Added count to split
% LTXE-SLIDE-REFERENCE:	Feature-67880-AddedCountToSplit.rst
% ------------------------------------------------------------------------------
% Note: feature #67880 has been introduced in TYPO3 CMS 7.4 already
% and we mentioned it in the last version of our What's New Slides.
% It is not clear, why this feature is included in the ChangeLog of
% 7.5 again (with a slightly different title), but we should have it
% in our Slides, too.

\begin{frame}[fragile]
	\frametitle{TSconfig \& TypoScript}
	\framesubtitle{Compte des éléments d'une liste}

	% decrease font size for code listing
	\lstset{basicstyle=\tiny\ttfamily}

	\begin{itemize}

		\item La propriété \texttt{returnCount} est ajoutée à la propriété stdWrap \texttt{split}

		\item Elle permet de récupérer le nombre d'éléments d'une liste ayant la virgule comme séparateur

		\item Par exemple, le code suivant retourne \texttt{9}~:

			\begin{lstlisting}
				1 = TEXT
				1 {
				  value = x,y,z,1,2,3,a,b,c
				  split.token = ,
				  split.returnCount = 1
				}
			\end{lstlisting}

	\end{itemize}

\end{frame}
% ------------------------------------------------------------------------------
% LTXE-SLIDE-START
% LTXE-SLIDE-UID:		d0e758f6-98c4050d-aa055323-c0c2273b
% LTXE-SLIDE-ORIGIN:	1a1b3a3f-a12510ef-003db5f4-2fc4f613 English
% LTXE-SLIDE-ORIGIN:	da2aa521-48abf012-e8b4b95f-0e3f3556 German
% LTXE-SLIDE-TITLE:		Feature #69602: Simplify handling of backend layouts in frontend (1)
% LTXE-SLIDE-REFERENCE:	Feature-69602-SimplifyHandlingOfBackendLayoutsInFrontend.rst
% ------------------------------------------------------------------------------

\begin{frame}[fragile]
	\frametitle{TSconfig \& TypoScript}
	\framesubtitle{Prise en charge des dispositions backend (1)}

	% decrease font size for code listing
	\lstset{basicstyle=\tiny\ttfamily}

	\begin{itemize}

		\item La prise en charge des dispositions backend a été simplifiée pour le frontend

		\item La nouvelle option \texttt{pagelayout} est utilisable en TypoScript

		\item Exemple~:

			\begin{lstlisting}
				page.10 = FLUIDTEMPLATE
				page.10 {
				  file.stdWrap.cObject = CASE
				  file.stdWrap.cObject {
				    key.data = pagelayout
				    default = TEXT
				    default.value = EXT:sitepackage/Resources/Private/Templates/Home.html
				    3 = TEXT
				    3.value = EXT:sitepackage/Resources/Private/Templates/1-col.html
				    4 = TEXT
				    4.value = EXT:sitepackage/Resources/Private/Templates/2-col.html
				  }
				}
			\end{lstlisting}

			\smaller
				(continue sur la page suivante)
			\normalsize

	\end{itemize}

\end{frame}

% ------------------------------------------------------------------------------
% LTXE-SLIDE-START
% LTXE-SLIDE-UID:		193396e5-67f2f6f7-b51bbd3e-69abd963
% LTXE-SLIDE-ORIGIN:	4054dec5-eb1750c8-20a10e74-0935b709 English
% LTXE-SLIDE-ORIGIN:	b95fe8b4-a521da2a-f01248ab-35560e3f German
% LTXE-SLIDE-TITLE:		Feature #69602: Simplify handling of backend layouts in frontend (2)
% LTXE-SLIDE-REFERENCE:	Feature-69602-SimplifyHandlingOfBackendLayoutsInFrontend.rst
% ------------------------------------------------------------------------------

\begin{frame}[fragile]
	\frametitle{TSconfig \& TypoScript}
	\framesubtitle{Prise en charge des dispositions backend (2)}

	% decrease font size for code listing
	\lstset{basicstyle=\tiny\ttfamily}

	\begin{itemize}

		\item …où \texttt{key.data = pagelayout} remplace le code suivant~:

			\begin{lstlisting}
				field = backend_layout
				ifEmpty.data = levelfield:-2,backend_layout_next_level,slide
				ifEmpty.ifEmpty = default
			\end{lstlisting}

	\end{itemize}

\end{frame}

% ------------------------------------------------------------------------------
% LTXE-SLIDE-START
% LTXE-SLIDE-UID:		b2baf95e-ef845db7-e02c75ee-21e270ee
% LTXE-SLIDE-ORIGIN:	07353cea-dac9032f-52a37a42-a261e38a English
% LTXE-SLIDE-ORIGIN:	6f1f7bb6-b50c019e-29c9be36-7d30c87d German
% LTXE-SLIDE-TITLE:		Feature #68756: Add config "base" to stdWrap
% LTXE-SLIDE-REFERENCE:	Feature-68756-AddConfigBaseToStdWrap.rst
% ------------------------------------------------------------------------------
\begin{frame}[fragile]
	\frametitle{TSconfig \& TypoScript}
	\framesubtitle{Divers}

	\begin{itemize}

		\item La fonction stdWrap \texttt{bytes} a été introduite par TYPO3 CMS 7.4

		\item La possibilité de définir la \texttt{base} est ajoutée par TYPO3 CMS 7.5,
			permettant de définir s'il on veut calculer avec la base 1000 ou 1024

			\begin{lstlisting}
				bytes.labels = " | K| M| G"
				bytes.base = 1000
			\end{lstlisting}

	\end{itemize}

\end{frame}

% ------------------------------------------------------------------------------
