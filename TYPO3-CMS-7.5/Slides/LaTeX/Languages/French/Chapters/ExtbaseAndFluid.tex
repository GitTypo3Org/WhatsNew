% ------------------------------------------------------------------------------
% TYPO3 CMS 7.5 - What's New - Chapter "Extbase & Fluid" (French Version)
%
% @author	Michael Schams <schams.net>
% @license	Creative Commons BY-NC-SA 3.0
% @link		http://typo3.org/download/release-notes/whats-new/
% @language	French
% ------------------------------------------------------------------------------
% LTXE-CHAPTER-UID:		c9b2e403-e06bb691-ce202054-149c5089
% LTXE-CHAPTER-NAME:	Extbase & Fluid
% ------------------------------------------------------------------------------

\section{Extbase \& Fluid}
\begin{frame}[fragile]
	\frametitle{Extbase \& Fluid}

	\begin{center}\huge{Chapitre 4~:}\end{center}
	\begin{center}\huge{\color{typo3darkgrey}\textbf{Extbase \& Fluid}}\end{center}

\end{frame}

% ------------------------------------------------------------------------------
% LTXE-SLIDE-START
% LTXE-SLIDE-UID:		e475b123-c36e28fa-8df5626b-ea235459
% LTXE-SLIDE-ORIGIN:	8bf4a04e-9453b098-84fa7616-1df8ea8f English
% LTXE-SLIDE-ORIGIN:	7528691a-d33af3f4-edab28ae-ba7c8016 German
% LTXE-SLIDE-TITLE:		Severity-filtering for FlashMessageQueue
% LTXE-SLIDE-REFERENCE:	Feature-7098-SeverityFilteringForFlashMessageQueue.rst
% ------------------------------------------------------------------------------

\begin{frame}[fragile]
	\frametitle{Extbase \& Fluid}
	\framesubtitle{Filtrage par sévérité dans FlashMessageQueue}

	\begin{itemize}

		\item Avant TYPO3 CMS 7.5, \underline{tous} les messages de la file (FlashMessageQueue) peuvent
			seulement être tous récupérés ou retirés

		\item À partir de TYPO3 CMS 7.5, les opérations peuvent être effectuées en filtrant par sévérité~:

			\begin{lstlisting}
				FlashMessageQueue::getAllMessages($severity);
				FlashMessageQueue::getAllMessagesAndFlush($severity);
				FlashMessageQueue::removeAllFlashMessagesFromSession($severity);
				FlashMessageQueue::clear($severity);
			\end{lstlisting}

	\end{itemize}

\end{frame}

% ------------------------------------------------------------------------------
% LTXE-SLIDE-START
% LTXE-SLIDE-UID:		4eb1827c-bda620fc-cadb33f0-8811693b
% LTXE-SLIDE-ORIGIN:	1728dca6-0c88a636-5dc7cc00-dfa3266f English
% LTXE-SLIDE-ORIGIN:	484e03e4-cb7828c2-8bf31f48-cabf004a German
% LTXE-SLIDE-TITLE:		Query support for BETWEEN added
% LTXE-SLIDE-REFERENCE:	Feature-47812-QuerySupportForBETWEENAdded.rst
% ------------------------------------------------------------------------------

\begin{frame}[fragile]
	\frametitle{Extbase \& Fluid}
	\framesubtitle{Ajout du support du filtre "\texttt{between}"}

	\begin{itemize}

		\item Le support de \texttt{between} a été ajouté à l'objet Query d'Extbase

		\item Il n'y a pas différence de performance car le SGBD converti
			"\texttt{between}" en interne de toute manière~: \texttt{min <= expr AND expr <= max}

		\item La nouvelle fonctionnalité Extbase réplique le comportement du SGBD en construisant
			une condition \texttt{AND}, supportant ainsi tous les SGBD

			\begin{lstlisting}
				$query->matching(
				  $query->between('uid', 3, 5)
				);
			\end{lstlisting}

	\end{itemize}

\end{frame}

% ------------------------------------------------------------------------------
% LTXE-SLIDE-START
% LTXE-SLIDE-UID:		4b627d8c-fb4aa3fe-17411907-c2dd97c1
% LTXE-SLIDE-ORIGIN:	0f2d87b1-c907b8a6-44612e51-e84e6490 English
% LTXE-SLIDE-ORIGIN:	12b78f4c-7291d788-ea865da7-e6501a2e German
% LTXE-SLIDE-TITLE:		Added support for multiple FlashMessage queues
% LTXE-SLIDE-REFERENCE:	Feature-64726-UsingArbitraryFlashmessageQueues.rst
% ------------------------------------------------------------------------------

\begin{frame}[fragile]
	\frametitle{Extbase \& Fluid}
	\framesubtitle{Files d'attente de FlashMessage multiples}

	\begin{itemize}

		\item Le support de multiples files d'attentes \texttt{FlashMessageQueues} est ajouté~:

			\begin{lstlisting}
				$queueIdentifier = 'myQueue';
				$this->controllerContext->getFlashMessageQueue($queueIdentifier);
			\end{lstlisting}

		\item L'accès avec Fluid fonctionne comme indiqué~:

			\begin{lstlisting}
				<f:flashMessages queueIdentifier="myQueue" ></f:flashMessages>
			\end{lstlisting}

	\end{itemize}

\end{frame}

% ------------------------------------------------------------------------------
% LTXE-SLIDE-START
% LTXE-SLIDE-UID:		61331697-64c97fa2-be0a473b-1085c581
% LTXE-SLIDE-ORIGIN:	6915482b-a7f5bd30-f04b3dc6-d94a5f2a English
% LTXE-SLIDE-ORIGIN:	2f1002c1-6bde766c-3b1ad500-ab9ccf0c German
% LTXE-SLIDE-TITLE:		Introduced MediaViewHelper (1)
% LTXE-SLIDE-REFERENCE:	Feature-66366-IntroducedMediaViewHelper.rst
% ------------------------------------------------------------------------------

\begin{frame}[fragile]
	\frametitle{Extbase \& Fluid}
	\framesubtitle{ViewHelper Media (1)}

	% decrease font size for code listing
	\lstset{basicstyle=\tiny\ttfamily}

	\begin{itemize}

		\item Afin de faciliter le rendu vidéo, audio ou d'autres fichiers avec
			une classe Renderer enregistrée en frontend, le \texttt{MediaViewHelper}
			a été implémenté

		\item \texttt{MediaViewHelper} vérifie d'abord si un Renderer est présent pour
			le fichier donné - dans le cas contraire, il retourne à afficher une balise d'image

		\item Exemples~:

			\begin{lstlisting}
				<code title="Image Object">
				  <f:media file="{file}" width="400" height="375" ></f:media>
				</code>

				<output>
				  <img alt="alt set in image record" src="fileadmin/_processed_/323223424.png"
				    width="396" height="375" />
				</output>
			\end{lstlisting}

	\end{itemize}

\end{frame}

% ------------------------------------------------------------------------------
% LTXE-SLIDE-START
% LTXE-SLIDE-UID:		095758a3-84d54781-8584a6f2-43ced391
% LTXE-SLIDE-ORIGIN:	8bd8314d-631b7ae9-3102867f-804046fa English
% LTXE-SLIDE-ORIGIN:	73fe507d-772b6011-7457429b-4039dd5c German
% LTXE-SLIDE-TITLE:		Introduced MediaViewHelper (2)
% LTXE-SLIDE-REFERENCE:	Feature-66366-IntroducedMediaViewHelper.rst
% ------------------------------------------------------------------------------

\begin{frame}[fragile]
	\frametitle{Extbase \& Fluid}
	\framesubtitle{ViewHelper Media (2)}

	% decrease font size for code listing
	\lstset{basicstyle=\tiny\ttfamily}

	\begin{itemize}

		\item Exemples (suite)~:

			\begin{lstlisting}
				<code title="MP4 Video Object">
				  <f:media file="{file}" width="400" height="375" ></f:media>
				</code>

				<output>
				  <video width="400" height="375" controls>
				    <source src="fileadmin/user_upload/my-video.mp4" type="video/mp4">
				  </video>
				</output>

				<code title="MP4 Video Object with loop and autoplay option set">
				  <f:media file="{file}" width="400" height="375"
				    additionalConfig="{loop: '1', autoplay: '1'}" ></f:media>
				</code>

				<output>
				  <video width="400" height="375" controls loop>
				    <source src="fileadmin/user_upload/my-video.mp4" type="video/mp4">
				  </video>
				</output>
			\end{lstlisting}

	\end{itemize}

\end{frame}

% ------------------------------------------------------------------------------
% LTXE-SLIDE-START
% LTXE-SLIDE-UID:		8e56511d-fb03114b-d2dcdf4a-68d5ce70
% LTXE-SLIDE-ORIGIN:	5d858cf5-81f4cf18-1b13e093-6a00c98a English
% LTXE-SLIDE-ORIGIN:	2abe5d2e-c1c7e93b-c86559da-81196fdb German
% LTXE-SLIDE-TITLE:		Adopt form to support the Extbase/ Fluid MVC stack (1)
% LTXE-SLIDE-REFERENCE:	Feature-69401-AdoptFormToSupportTheExtbaseFluidMVCStack.rst
% ------------------------------------------------------------------------------

\begin{frame}[fragile]
	\frametitle{Extbase \& Fluid}
	\framesubtitle{Extension système \texttt{form} (1)}

	\begin{itemize}

		\item L'extension système \texttt{form} (incluant les modèles personnalisés, les contrôleurs,
			la validation, les vues et l'apparence) a été transformée pour supporter la pile MVC
			Extbase/Fluid

		\item Ceci permet une meilleure personnalisation et le contrôle des comportements et de
			la sortie en modifiant des modèles Fluid ou en utilisant des view helper personnalisés

		\item Chaque élément de formulaire utilise sa propre Partial, pouvant être configuré
			par l'option TypoScript \texttt{partialPath = …}

	\end{itemize}

\end{frame}

% ------------------------------------------------------------------------------
% LTXE-SLIDE-START
% LTXE-SLIDE-UID:		9eafffb2-f3a13f09-099427b8-14216f3a
% LTXE-SLIDE-ORIGIN:	20c7837c-5b841f21-fb3ead22-265aebe4 English
% LTXE-SLIDE-ORIGIN:	d45facb0-73989d92-ffbc7c2b-46f583ba German
% LTXE-SLIDE-TITLE:		Adopt form to support the Extbase/ Fluid MVC stack (2)
% LTXE-SLIDE-REFERENCE:	Feature-69401-AdoptFormToSupportTheExtbaseFluidMVCStack.rst
% ------------------------------------------------------------------------------

\begin{frame}[fragile]
	\frametitle{Extbase \& Fluid}
	\framesubtitle{Extension système \texttt{form} (2)}

	\begin{itemize}

		\item Les trois ViewHelpers suivants existent~:

			\begin{itemize}
				\item \texttt{AggregateSelectOptionsViewHelper} (for optgroup tags)
				\item \texttt{SelectViewHelper} (for optgroup tags)
				\item \texttt{PlainMailViewHelper} (to render plain text mails)
			\end{itemize}

		\item Ainsi que ces trois vues~:

			\begin{itemize}
				\item \texttt{show} (the form itself)
				\item \texttt{confirmation} (the confirmation page)
				\item \texttt{postProcessor}/\texttt{mail} (the email)
			\end{itemize}

		\item Le chemin des modèles et la visibilité des champs est personnalisable
			pour chaque vue individuellement

	\end{itemize}

\end{frame}

% ------------------------------------------------------------------------------
% LTXE-SLIDE-START
% LTXE-SLIDE-UID:		c9c47ff6-6086714b-3078d415-5fade306
% LTXE-SLIDE-ORIGIN:	20b68c91-1b501686-f97f7b55-0822d9d1 English
% LTXE-SLIDE-ORIGIN:	cd99ea8c-e6234f0f-caf559ed-c3f825c0 German
% LTXE-SLIDE-TITLE:		Add annotation for CLI only commands
% LTXE-SLIDE-REFERENCE:	Feature-68746-AddAnnotationForCLIOnlyCommands.rst
% ------------------------------------------------------------------------------

\begin{frame}[fragile]
	\frametitle{Extbase \& Fluid}
	\framesubtitle{Annotation \texttt{@cli}}

	\begin{itemize}

		\item En utilisant la nouvelle annotation \texttt{@cli}, les commandes d'un
		 	CommandController Extbase peuvent être marquées pour l'usage en ligne de commande
		 	uniquement

		\item Ces commandes sont exclues de la liste proposée dans le planificateur de tâche

		\item Les cas d'usage typiques sont les commandes comme \texttt{extbase:help:help}, par exemple

	\end{itemize}

\end{frame}

% ------------------------------------------------------------------------------
