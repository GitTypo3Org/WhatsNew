% ------------------------------------------------------------------------------
% TYPO3 CMS 7.5 - What's New - Chapter "TypoScript" (Dutch Version)
%
% @author	Michael Schams <schams.net>
% @license	Creative Commons BY-NC-SA 3.0
% @link		http://typo3.org/download/release-notes/whats-new/
% @language	Dutch
% ------------------------------------------------------------------------------
% LTXE-CHAPTER-UID:		f607fb32-7f7b68d8-68d0e69a-16738006
% LTXE-CHAPTER-NAME:	TypoScript
% ------------------------------------------------------------------------------

\section{TSconfig \& TypoScript}
\begin{frame}[fragile]
	\frametitle{TSconfig \& TypoScript}

	\begin{center}\huge{Hoofdstuk 2:}\end{center}
	\begin{center}\huge{\color{typo3darkgrey}\textbf{TSconfig \& TypoScript}}\end{center}

\end{frame}

% ------------------------------------------------------------------------------
% LTXE-SLIDE-START
% LTXE-SLIDE-UID:		878b8362-7f219ece-269ef9d4-82182694
% LTXE-SLIDE-ORIGIN:	195a3fa3-207484a7-94e7ce76-c821394c English
% LTXE-SLIDE-ORIGIN:	c76757e2-fc02b94b-166a8168-d68774c9 German
% LTXE-SLIDE-TITLE:		Feature #16525: Add conditions to INCLUDE_TYPOSCRIPT
% LTXE-SLIDE-REFERENCE:	Feature-16525-AddConditionsToINCLUDE_TYPOSCRIPT.rst
% ------------------------------------------------------------------------------
\begin{frame}[fragile]
	\frametitle{TSconfig \& TypoScript}
	\framesubtitle{Voorwaardelijke TypoScript-includes}

	% decrease font size for code listing
	\lstset{basicstyle=\tiny\ttfamily}

	\begin{itemize}

		\item \texttt{INCLUDE\_TYPOSCRIPT} heeft nu een extra (optionele) eigenschap "condition". 
			Alleen als aan de voorwaarde is voldaan, wordt het bestand (of de map) gebruikt.

		\begin{lstlisting}
			// TypoScript alleen laden als de gebruiker is ingelogd:
			<INCLUDE_TYPOSCRIPT: source="FILE:EXT:my_extension/Configuration/TypoScript/feuser.ts"
			  condition="[loginUser = *]">

			// TypoScript alleen laden wanneer de applicationContext overeenkomt:
			<INCLUDE_TYPOSCRIPT: source="FILE:EXT:my_extension/Configuration/TypoScript/staging.ts"
			  condition="applicationContext = /^Production\\/Staging\\/Server\\d+$/">
		\end{lstlisting}

	\end{itemize}

\end{frame}

% ------------------------------------------------------------------------------
% LTXE-SLIDE-START
% LTXE-SLIDE-UID:		8d543a51-56e40705-a103f16b-ddb1d6c0
% LTXE-SLIDE-ORIGIN:	4003d374-92104168-0eb97927-726037fa English
% LTXE-SLIDE-ORIGIN:	1bde9e8a-82352f02-c5c9f3db-b194cfb1 German
% LTXE-SLIDE-TITLE:		Feature #28243: Introduce TCA option to disable age display of dates per field
% LTXE-SLIDE-REFERENCE:	Feature-28243-IntroduceTcaOptionToDisableAgeDisplay.rst
% ------------------------------------------------------------------------------
\begin{frame}[fragile]
	\frametitle{TSconfig \& TypoScript}
	\framesubtitle{TCA-optie om de ouderdom te verbergen}

	% decrease font size for code listing
	\lstset{basicstyle=\tiny\ttfamily}

	\begin{itemize}

		\item Met TCA-optie \texttt{disableAgeDisplay} kan de ouderdom verborgen worden
			\small
				(bijvoorbeeld: \texttt{"30-08-2015 (-27 dagen)"})
			\normalsize

			\begin{lstlisting}
				$GLOBALS['TCA']['tt_content']['columns']['date']['config']['disableAgeDisplay'] = true;
			\end{lstlisting}

		\item Als voorwaarde moet het \texttt{type} een \texttt{input}-veld zijn
			en \texttt{eval} moet \texttt{date} hebben

	\end{itemize}

\end{frame}

% ------------------------------------------------------------------------------
% LTXE-SLIDE-START
% LTXE-SLIDE-UID:		9d50444a-52809a9a-8dd4f73c-6dafbf9f
% LTXE-SLIDE-ORIGIN:	68c8a68b-f7e50bf3-dc67e0e5-d3a33fd8 English
% LTXE-SLIDE-ORIGIN:	c3aa0981-411fcabb-bd9dca10-16f42898 German
% LTXE-SLIDE-TITLE:		Feature #57632: Include inline language label files with TypoScript (1)
% LTXE-SLIDE-REFERENCE:	Feature-57632-AddInlineLanguageLabelFilesWithTypoScript.rst
% ------------------------------------------------------------------------------
\begin{frame}[fragile]
	\frametitle{TSconfig \& TypoScript}
	\framesubtitle{Inline taallabels met TypoScript (1)}

	% decrease font size for code listing
	\lstset{basicstyle=\tiny\ttfamily}

	\begin{itemize}

		\item XLF-taalbestanden kunnen worden ingelezen en geconverteerd in een array

		\item Dat maakt het bijvoorbeeld mogelijk om taallabels in JavaScript te gebruiken

		\item De volgende 3 optionele parameters worden ondersteund:

			\begin{itemize}
				\item \texttt{selectionPrefix}:\newline
					gebruik alleen labels waarvan de index met deze tekst begint
				\item \texttt{stripFromSelectionName}:\newline
					tekst die van elke index wordt verwijderd
				\item \texttt{errorMode}:\newline
					foutafhandeling wanneer het bestand niet wordt gevonden:\newline
					0: log (standaard), 1: negeren, 3: gooi een exception
			\end{itemize}

	\end{itemize}

\end{frame}

% ------------------------------------------------------------------------------
% LTXE-SLIDE-START
% LTXE-SLIDE-UID:		785c7ae9-9c81a43d-3a2cbada-f58be0a6
% LTXE-SLIDE-ORIGIN:	bb800987-b0e74193-c6ec6370-3b6a062d English
% LTXE-SLIDE-ORIGIN:	289816f4-411fcabb-0981cabb-bd9dca10 German
% LTXE-SLIDE-TITLE:		Feature #57632: Include inline language label files with TypoScript (2)
% LTXE-SLIDE-REFERENCE:	Feature-57632-AddInlineLanguageLabelFilesWithTypoScript.rst
% ------------------------------------------------------------------------------
\begin{frame}[fragile]
	\frametitle{TSconfig \& TypoScript}
	\framesubtitle{Inline taallabels met TypoScript (2)}

	% decrease font size for code listing
	\lstset{basicstyle=\tiny\ttfamily}

	\begin{itemize}

		\item Voorbeeld:

			\begin{lstlisting}
				page = PAGE
				page.inlineLanguageLabelFiles {
				  someLabels = EXT:myExt/Resources/Private/Language/locallang.xlf
				  someLabels.selectionPrefix = idPrefix
				  someLabels.stripFromSelectionName = strip_me
				  someLabels.errorMode = 2
				}
			\end{lstlisting}

		\item Uitvoer:

			\begin{lstlisting}
				<script type="text/javascript">
				/*<![CDATA[*/
				  var TYPO3 = TYPO3 || {};
				  TYPO3.lang = {"firstLabel":[{"source":"first Label","target":"erstes Label"}],
				  "secondLabel":[{"source":"second Label","target":"zweites Label"}]};
				/*]]>*/
				</script>
			\end{lstlisting}

	\end{itemize}

\end{frame}

% ------------------------------------------------------------------------------
% LTXE-SLIDE-START
% LTXE-SLIDE-UID:		8d65602f-03629e13-0e19ee0b-94fee225
% LTXE-SLIDE-ORIGIN:	bac61b19-18a4db7b-af13de6c-5dbed67b English
% LTXE-SLIDE-ORIGIN:	f14c90e2-bf35bf40-e38d74a7-f4a2e973 German
% LTXE-SLIDE-TITLE:		Feature #59144: Previewing workspace records using Page TSconfig
% LTXE-SLIDE-REFERENCE:	Feature-59144-PageTSconfigWorkspacePreview.rst
% ------------------------------------------------------------------------------
\begin{frame}[fragile]
	\frametitle{TSconfig \& TypoScript}
	\framesubtitle{Workspacevoorbeelden via TSconfig}

	% decrease font size for code listing
	\lstset{basicstyle=\tiny\ttfamily}

	\begin{itemize}

		\item Standaard genereert TYPO3 CMS in de workspace alleen voorbeeldpagina's 
			voor de tabellen \texttt{tt\_content}, \texttt{pages} en \texttt{pages\_language\_overlay}

		\item Dat kan nu worden aangepast met PageTSconfig:

			\begin{lstlisting}
				# gebruik pagina 123 voor het voorbeeld (alle tabellen)
				options.workspaces.previewPageId = 123

				# gebruik het veld pid voor het voorbeeld (alle tabellen)
				options.workspaces.previewPageId = field:pid

				# gebruik pagina 123 voor het voorbeeld (tabel tx_myext_table)
				options.workspaces.previewPageId.tx_myext_table = 123

				# gebruik het veld pid voor het voorbeeld (tabel tx_myext_table)
				options.workspaces.previewPageId.tx_myext_table = field:pid
			\end{lstlisting}

	\end{itemize}

\end{frame}

% ------------------------------------------------------------------------------
% LTXE-SLIDE-START
% LTXE-SLIDE-UID:		ad1fa555-d589e30d-fa8b431c-b4551b5d
% LTXE-SLIDE-ORIGIN:	c4b7358c-07059082-861a3b2d-929e1cc6 English
% LTXE-SLIDE-ORIGIN:	faeca558-5a675884-a1b77fae-36e316eb German
% LTXE-SLIDE-TITLE:		Feature #59591: Image quality definable per sourceCollection
% LTXE-SLIDE-REFERENCE:	Feature-59591-ImageQualityDefinablePerSourceCollection.rst
% ------------------------------------------------------------------------------
\begin{frame}[fragile]
	\frametitle{TSconfig \& TypoScript}
	\framesubtitle{Afbeeldingskwaliteit per \texttt{sourceCollection}}

	% decrease font size for code listing
	\lstset{basicstyle=\tiny\ttfamily}

	\begin{itemize}

		\item De afbeeldingskwaliteit van elke \texttt{sourceCollection} kan worden aangepast

		\item Deze instelling heeft voorrang op de configuratie in de Installatie-module
			(die wordt opgeslagen in bestand \texttt{LocalConfiguration.php})

		\item Voorbeeld:

			\begin{lstlisting}
				# voor kleine retina-afbeeldingen
				tt_content.image.20.1.sourceCollection.smallRetina.quality = 80

				# voor grote retina-afbeeldingen
				tt_content.image.20.1.sourceCollection.largeRetina.quality = 65
			\end{lstlisting}

	\end{itemize}

\end{frame}

% ------------------------------------------------------------------------------
% LTXE-SLIDE-START
% LTXE-SLIDE-UID:		d8b9cb7e-22e25b7c-750dd473-21f03e53
% LTXE-SLIDE-ORIGIN:	ad691a22-7ab387b6-6e00c0aa-e24efd91 English
% LTXE-SLIDE-ORIGIN:	0073160e-6c95f70d-fcaf008f-3fd84819 German
% LTXE-SLIDE-TITLE:		Feature #67880: Added count to split
% LTXE-SLIDE-REFERENCE:	Feature-67880-AddedCountToSplit.rst
% ------------------------------------------------------------------------------
% Note: feature #67880 has been introduced in TYPO3 CMS 7.4 already
% and we mentioned it in the last version of our What's New Slides.
% It is not clear, why this feature is included in the ChangeLog of
% 7.5 again (with a slightly different title), but we should have it
% in our Slides, too.

\begin{frame}[fragile]
	\frametitle{TSconfig \& TypoScript}
	\framesubtitle{Aantal elementen in een lijst}

	% decrease font size for code listing
	\lstset{basicstyle=\tiny\ttfamily}

	\begin{itemize}

		\item Een nieuwe eigenschap \texttt{returnCount} is toegevoegd aan stdWrap-functie \texttt{split}

		\item Dit maakt het mogelijk om het aantal elementen in een kommagescheiden lijst te tellen

		\item De volgende code heeft bijvoorbeeld \texttt{9} als uitvoer:

			\begin{lstlisting}
				1 = TEXT
				1 {
				  value = x,y,z,1,2,3,a,b,c
				  split.token = ,
				  split.returnCount = 1
				}
			\end{lstlisting}

	\end{itemize}

\end{frame}
% ------------------------------------------------------------------------------
% LTXE-SLIDE-START
% LTXE-SLIDE-UID:		06628099-a9ae860c-4710d99d-6bf7f156
% LTXE-SLIDE-ORIGIN:	1a1b3a3f-a12510ef-003db5f4-2fc4f613 English
% LTXE-SLIDE-ORIGIN:	da2aa521-48abf012-e8b4b95f-0e3f3556 German
% LTXE-SLIDE-TITLE:		Feature #69602: Simplify handling of backend layouts in frontend (1)
% LTXE-SLIDE-REFERENCE:	Feature-69602-SimplifyHandlingOfBackendLayoutsInFrontend.rst
% ------------------------------------------------------------------------------

\begin{frame}[fragile]
	\frametitle{TSconfig \& TypoScript}
	\framesubtitle{Afhandeling van Backend Layouts (1)}

	% decrease font size for code listing
	\lstset{basicstyle=\tiny\ttfamily}

	\begin{itemize}

		\item De afhandeling van backend layouts voor de frontend is vereenvoudigd

		\item In TypoScript kan de nieuwe optie \texttt{pagelayout} worden gebruikt

		\item Voorbeeld:

			\begin{lstlisting}
				page.10 = FLUIDTEMPLATE
				page.10 {
				  file.stdWrap.cObject = CASE
				  file.stdWrap.cObject {
				    key.data = pagelayout
				    default = TEXT
				    default.value = EXT:sitepackage/Resources/Private/Templates/Home.html
				    3 = TEXT
				    3.value = EXT:sitepackage/Resources/Private/Templates/1-col.html
				    4 = TEXT
				    4.value = EXT:sitepackage/Resources/Private/Templates/2-col.html
				  }
				}
			\end{lstlisting}

			\smaller
				(vervolg op volgende pagina)
			\normalsize

	\end{itemize}

\end{frame}

% ------------------------------------------------------------------------------
% LTXE-SLIDE-START
% LTXE-SLIDE-UID:		71866a04-032beb1d-051794ef-c25ce686
% LTXE-SLIDE-ORIGIN:	4054dec5-eb1750c8-20a10e74-0935b709 English
% LTXE-SLIDE-ORIGIN:	b95fe8b4-a521da2a-f01248ab-35560e3f German
% LTXE-SLIDE-TITLE:		Feature #69602: Simplify handling of backend layouts in frontend (2)
% LTXE-SLIDE-REFERENCE:	Feature-69602-SimplifyHandlingOfBackendLayoutsInFrontend.rst
% ------------------------------------------------------------------------------

\begin{frame}[fragile]
	\frametitle{TSconfig \& TypoScript}
	\framesubtitle{Afhandeling van Backend Layouts (2)}

	% decrease font size for code listing
	\lstset{basicstyle=\tiny\ttfamily}

	\begin{itemize}

		\item Daarbij vervangt \texttt{key.data = pagelayout} de volgende code:

			\begin{lstlisting}
				field = backend_layout
				ifEmpty.data = levelfield:-2,backend_layout_next_level,slide
				ifEmpty.ifEmpty = default
			\end{lstlisting}

	\end{itemize}

\end{frame}

% ------------------------------------------------------------------------------
% LTXE-SLIDE-START
% LTXE-SLIDE-UID:		f5f58213-a60d3b96-28ff99ac-0fdeb14d
% LTXE-SLIDE-ORIGIN:	07353cea-dac9032f-52a37a42-a261e38a English
% LTXE-SLIDE-ORIGIN:	6f1f7bb6-b50c019e-29c9be36-7d30c87d German
% LTXE-SLIDE-TITLE:		Feature #68756: Add config "base" to stdWrap
% LTXE-SLIDE-REFERENCE:	Feature-68756-AddConfigBaseToStdWrap.rst
% ------------------------------------------------------------------------------
\begin{frame}[fragile]
	\frametitle{TSconfig \& TypoScript}
	\framesubtitle{Diversen}

	\begin{itemize}

		\item In TYPO3 CMS 7.4 werd de stdWrap-functie \texttt{bytes} geïntroduceerd

		\item In TYPO3 CMS 7.5 is aan \texttt{bytes} de eigenschap \texttt{base} toegevoegd,
			die het mogelijk maakt om bij de bestandsgrootte te rekenen met 1000 of 1024

			\begin{lstlisting}
				bytes.labels = " | K| M| G"
				bytes.base = 1000
			\end{lstlisting}

	\end{itemize}

\end{frame}

% ------------------------------------------------------------------------------
