% ------------------------------------------------------------------------------
% TYPO3 CMS 7.5 - What's New - Chapter "In-Depth Changes" (Dutch Version)
%
% @author	Michael Schams <schams.net>
% @license	Creative Commons BY-NC-SA 3.0
% @link		http://typo3.org/download/release-notes/whats-new/
% @language	Dutch
% ------------------------------------------------------------------------------
% LTXE-CHAPTER-UID:		b8f1c138-2fb2f5ce-15d1ffc0-e80171e2
% LTXE-CHAPTER-NAME:	In-Depth Changes
% ------------------------------------------------------------------------------

\section{Systeemwijzigingen}
\begin{frame}[fragile]
	\frametitle{Systeemwijzigingen}

	\begin{center}\huge{Hoofdstuk 3:}\end{center}
	\begin{center}\huge{\color{typo3darkgrey}\textbf{Systeemwijzigingen}}\end{center}

\end{frame}

% ------------------------------------------------------------------------------
% LTXE-SLIDE-START
% LTXE-SLIDE-UID:		04cbd98c-9b47030e-1104fa95-3f9c9c77
% LTXE-SLIDE-ORIGIN:	16e557e2-fa6a5035-5b28d66e-27c0c626 English
% LTXE-SLIDE-ORIGIN:	03be70bc-886235ca-10f0cef5-6b33fe60 German
% LTXE-SLIDE-TITLE:		Fluid-based Content Elements Introduced (1)
% LTXE-SLIDE-REFERENCE:	Feature-38732-Fluid-basedContentElementsIntroduced.rst
% ------------------------------------------------------------------------------

\begin{frame}[fragile]
	\frametitle{Systeemwijzigingen}
	\framesubtitle{Contentelementen gebaseerd op Fluid (1)}

	\begin{itemize}

		\item De nieuwe systeemextensie \textbf{"Fluid-based Content Elements"} is geïmplementeerd

		\item Hierbij worden i.p.v. TypoScript nu Fluid-sjablonen gebruikt voor de weergave van contentelementen

		\item Dat kan op termijn een alternatief worden voor \textit{CSS Styled Content}

		\item Voeg de volgende statische sjablonen toe om hiervan gebruik te maken:

			\begin{itemize}
				\item Content Elements (\texttt{fluid\_styled\_content})
				\item Content Elements CSS (optional) (\texttt{fluid\_styled\_content})
			\end{itemize}

	\end{itemize}

\end{frame}



% ------------------------------------------------------------------------------
% LTXE-SLIDE-START
% LTXE-SLIDE-UID:		22ef1aa5-1b2ff432-d591022f-cfc58ca9
% LTXE-SLIDE-ORIGIN:	d8383008-e3bbb7e2-91720bea-3b93db62 English
% LTXE-SLIDE-ORIGIN:	bb7124ec-4658414b-757ca8ca-f0addee3 German
% LTXE-SLIDE-TITLE:		Fluid-based Content Elements Introduced (2)
% LTXE-SLIDE-REFERENCE:	Feature-38732-Fluid-basedContentElementsIntroduced.rst
% ------------------------------------------------------------------------------

\begin{frame}[fragile]
	\frametitle{Systeemwijzigingen}
	\framesubtitle{Contentelementen gebaseerd op Fluid (2)}

	% decrease font size for code listing
	\lstset{basicstyle=\tiny\ttfamily}

	\begin{itemize}

		\item Ook moet de volgende PageTSconfig-template worden toegevoegd aan de pagina-eigenschappen:\newline
			\small
				\texttt{Fluid-based Content Elements (fluid\_styled\_content)}
			\normalsize

		\item De paden naar de sjablonen kunnen met TypoScript worden gewijzigd:

			\begin{lstlisting}
				lib.fluidContent.templateRootPaths.50 = EXT:site_example/Resources/Private/Templates/
				lib.fluidContent.partialRootPaths.50 = EXT:site_example/Resources/Private/Partials/
				lib.fluidContent.layoutRootPaths.50 = EXT:site_example/Resources/Private/Layouts/
			\end{lstlisting}

	\end{itemize}

\end{frame}

% ------------------------------------------------------------------------------
% LTXE-SLIDE-START
% LTXE-SLIDE-UID:		8e0e3b7f-a013bf47-ef5f4cc7-112ecdbc
% LTXE-SLIDE-ORIGIN:	d25f0c05-31c928e4-e7439714-d87f883b English
% LTXE-SLIDE-ORIGIN:	04cd65d3-e013813d-d2eac450-7d3a79a8 German
% LTXE-SLIDE-TITLE:		Fluid-based Content Elements Introduced (3)
% LTXE-SLIDE-REFERENCE:	Feature-38732-Fluid-basedContentElementsIntroduced.rst
% LTXE-SLIDE-REFERENCE:	Important-67954-MigrateCTypesTextImageAndTextpicToTextmedia.rst
% ------------------------------------------------------------------------------

\begin{frame}[fragile]
	\frametitle{Systeemwijzigingen}
	\framesubtitle{Contentelementen gebaseerd op Fluid (3)}

	\begin{itemize}

		\item Migratie van \textit{CSS Styled Content} naar \textit{Fluid-based Content Elements}:

			\begin{itemize}

				\item Deactiveer extensie \texttt{css\_styled\_content}

				\item Activeer extensie \texttt{fluid\_styled\_content}

				\item Gebruik de Upgrade Wizard in de Installatie-module om de contentelementen
					\texttt{text}, \texttt{image} en \texttt{textpic} naar \texttt{textmedia} te migreren

			\end{itemize}
	\end{itemize}

	\vspace{1.4cm}

	\begingroup
		\color{red}
			\small
				\underline{Let op:} De ontwikkeling van \textit{"Fluid-based Content Elements"} bevindt zich nog in een vroeg stadium en mogelijk werkt de extensie tot TYPO3 CMS 7 LTS nog niet goed. Ook is het nog mogelijk dat er conflicten zijn met \textit{CSS Styled Content}.
			\normalsize
	\endgroup

\end{frame}


% ------------------------------------------------------------------------------
% LTXE-SLIDE-START
% LTXE-SLIDE-UID:		477030ca-5ef45a4d-dad03f8b-dc7c381e
% LTXE-SLIDE-ORIGIN:	89b8c8c2-3a458bcf-5d61b025-6b4fb470 English
% LTXE-SLIDE-ORIGIN:	8807e1a0-2b3fadac-ac842970-b85e5130 German
% LTXE-SLIDE-TITLE:		Add SELECTmmQuery method to DatabaseConnection
% LTXE-SLIDE-REFERENCE:	Feature-19494-AddSELECTmmQueryMethodToDatabaseConnection.rst
% ------------------------------------------------------------------------------

\begin{frame}[fragile]
	\frametitle{Systeemwijzigingen}
	\framesubtitle{Methode SELECTmmQuery}

	% decrease font size for code listing
	\lstset{basicstyle=\tiny\ttfamily}

	\begin{itemize}

		\item De nieuwe methode \texttt{SELECT\_mm\_query} is toegevoegd aan klasse \texttt{DatabaseConnection}

		\item Gekopieerd uit \texttt{exec\_SELECT\_mm\_query} om het bouwen en uitvoeren van M:M-query's van elkaar te scheiden.

		\item Dit maakt het mogelijk om in de database abstraction layer een query op te bouwen

			\begin{lstlisting}
				$query = SELECT_mm_query('*', 'table1', 'table1_table2_mm', 'table2', 'AND table1.uid = 1',
				'', 'table1.title DESC');
			\end{lstlisting}

	\end{itemize}

\end{frame}

% ------------------------------------------------------------------------------
% LTXE-SLIDE-START
% LTXE-SLIDE-UID:		10c42ea7-19e93240-6a40b0f8-4bbe3977
% LTXE-SLIDE-ORIGIN:	f2c4f1a9-6cd020c0-518650d6-2a31d0de English
% LTXE-SLIDE-ORIGIN:	398fdcb2-1cc9a862-42e59a09-e4ed65ba German
% LTXE-SLIDE-TITLE:		Scheduler task to optimize database tables
% LTXE-SLIDE-REFERENCE:	Feature-25341-SchedulerTaskToOptimizeDatabaseTables.rst
% ------------------------------------------------------------------------------

\begin{frame}[fragile]
	\frametitle{Systeemwijzigingen}
	\framesubtitle{Optimaliseren van MySQL-databasetabellen}

	\begin{itemize}

		\item Aan de scheduler is een nieuwe taak toegevoegd die het MySQL-commando \texttt{OPTIMIZE TABLE}
			uitvoert op de geselecteerde databasetabellen

		\item Dit commando ordent de fysieke opslag van de database en de bijbehorende index
			om de benodigde opslagruimte te reduceren en de I/O-efficiëntie te verbeteren

		\item De volgende type tabellen worden ondersteund:\newline
			\texttt{MyISAM}, \texttt{InnoDB} en \texttt{ARCHIVE}

		\item Het uitvoeren van de taak op DBAL en andere databases wordt \underline{niet} ondersteund
			aangezien het commando alleen werkt in MySQL.

	\end{itemize}

	% Note to translators: if this does not fit on one slide, you could try to change
	% \small to \smaller or shorten the sentences below or the bullet points above.

	\begingroup
		\color{red}
			\smaller
				\underline{Let op:} het optimaliseren van tabellen is een intensief I/O-proces.
				Bovendien worden in MySQL < 5.6.17 de tabellen gelocked, wat invloed kan hebben op de website.
			\normalsize
	\endgroup

\end{frame}

% ------------------------------------------------------------------------------
% LTXE-SLIDE-START
% LTXE-SLIDE-UID:		2bee24da-6282f51d-40dcbec9-42f3a2de
% LTXE-SLIDE-ORIGIN:	f0da3ed6-ab7b2518-402338db-a1842865 English
% LTXE-SLIDE-ORIGIN:	f9cbeb69-d08efdaf-a8c00d48-353f667e German
% LTXE-SLIDE-TITLE:		Improved handling of online media (1)
% LTXE-SLIDE-REFERENCE:	Feature-61799-ImprovedHandlingOfOnlineMedia.rst
% ------------------------------------------------------------------------------

\begin{frame}[fragile]
	\frametitle{Systeemwijzigingen}
	\framesubtitle{Ondersteuning van online media (1)}

	\begin{itemize}

		\item Standaard worden nu externe online media ondersteund

		\item Zo is er bijvoorbeeld ondersteuning ingebouwd voor YouTube- en Vimeovideo's

		\item Aan het contentelement \textbf{"Text \& Media"} kunnen URL's worden toegevoegd

		\item Een bijbehorende helper-klasse haalt de metadata op en zorgt dat er
			(indien beschikbaar) een voorafbeelding wordt getoond

	\end{itemize}

\end{frame}

% ------------------------------------------------------------------------------
% LTXE-SLIDE-START
% LTXE-SLIDE-UID:		7271c93a-457bcdb8-e4f6c8c8-f21c2e67
% LTXE-SLIDE-ORIGIN:	485f95a8-a9ec5940-e9aa7322-688630cb English
% LTXE-SLIDE-ORIGIN:	76e87d6a-cd1ebbf9-9a4d49ec-49cc569f German
% LTXE-SLIDE-TITLE:		Improved handling of online media (2)
% LTXE-SLIDE-REFERENCE:	Feature-61799-ImprovedHandlingOfOnlineMedia.rst
% ------------------------------------------------------------------------------

\begin{frame}[fragile]
	\frametitle{Systeemwijzigingen}
	\framesubtitle{Ondersteuning van online media (2)}

	Voor de URL's worden de volgende schrijfwijzen ondersteund:
	\vspace{0.4cm}

	\begin{columns}[T]
		\begin{column}{.52\textwidth}
			\smaller
				\tabto{0.2cm}\textbf{\underline{YouTube:}}\newline
				\tabto{0.2cm}\texttt{youtu.be/<code>}\newline
				\tabto{0.2cm}\texttt{www.youtube.com/watch?v=<code>}\newline
				\tabto{0.2cm}\texttt{www.youtube.com/v/<code>}\newline
				\tabto{0.2cm}\texttt{www.youtube-nocookie.com/v/<code>}\newline
				\tabto{0.2cm}\texttt{www.youtube.com/embed/<code>}\newline
		\end{column}
		\begin{column}{.48\textwidth}
			\vspace{-0.25cm}\smaller
				\textbf{\underline{Vimeo:}}\newline
				\texttt{vimeo.com/<code>}\newline
				\texttt{player.vimeo.com/video/<code>}\newline
		\end{column}
	\end{columns}

\end{frame}

% ------------------------------------------------------------------------------
% LTXE-SLIDE-START
% LTXE-SLIDE-UID:		f8fe56e1-92956904-a4aa7782-b17bce41
% LTXE-SLIDE-ORIGIN:	dc79335c-d9e51543-ff5759e0-5005bff8 English
% LTXE-SLIDE-ORIGIN:	aef8746a-e1babbc9-4dd74eec-cf379dae German
% LTXE-SLIDE-TITLE:		Improved handling of online media (3)
% LTXE-SLIDE-REFERENCE:	Feature-61799-ImprovedHandlingOfOnlineMedia.rst
% ------------------------------------------------------------------------------

\begin{frame}[fragile]
	\frametitle{Systeemwijzigingen}
	\framesubtitle{Ondersteuning van online media (3)}

	% decrease font size for code listing
	\lstset{basicstyle=\tiny\ttfamily}

	\begin{itemize}

		\item In Fluid kunnen de elementen op de volgende manier worden gebruikt:

		\begin{lstlisting}
			<!-- gebruik js api en no-cookie-support voor YouTube-video's -->
			<f:media file="{file}" additionalConfig="{enablejsapi:1, 'no-cookie': true}" ></f:media>

			<!-- toon titel en uploader van YouTube- of Vimeo-video voor het afspelen -->
			<f:media file="{file}" additionalConfig="{showinfo:1}" ></f:media>
		\end{lstlisting}

		\item Mogelijke instellingen voor YouTube-video's:\newline
			\small
				\texttt{autoplay}, \texttt{controls}, \texttt{loop}, \texttt{enablejsapi}, \texttt{showinfo}, \texttt{no-cookie}
			\normalsize

		\item Mogelijke instellingen voor Vimeo-video's:\newline
			\small
				\texttt{autoplay}, \texttt{loop}, \texttt{showinfo}
			\normalsize
	\end{itemize}

\end{frame}

% ------------------------------------------------------------------------------
% LTXE-SLIDE-START
% LTXE-SLIDE-UID:		53f86721-7ddb741d-92dadac8-4414746d
% LTXE-SLIDE-ORIGIN:	2b41f269-eaa80c0c-89bcd0f0-ee66248b English
% LTXE-SLIDE-ORIGIN:	9ed71aaa-a7615392-78f0ac0c-f9bd7400 German
% LTXE-SLIDE-TITLE:		Improved handling of online media (4)
% LTXE-SLIDE-REFERENCE:	Feature-61799-ImprovedHandlingOfOnlineMedia.rst
% ------------------------------------------------------------------------------

\begin{frame}[fragile]
	\frametitle{Systeemwijzigingen}
	\framesubtitle{Ondersteuning van online media (4)}

	% decrease font size for code listing
	\lstset{basicstyle=\tiny\ttfamily}

	\begin{itemize}

		\item Om gebruik te maken van een eigen online mediaservice, is er een
			\small\texttt{OnlineMediaHelper}\normalsize-klasse nodig die de
			\small\texttt{OnlineMediaHelperInterface}\normalsize\space implementeert en een
			\small\texttt{FileRenderer}\normalsize-klasse die de
			\small\texttt{FileRendererInterface}\normalsize\space implementeert.

			\begin{lstlisting}
				// registreer een eigen online videoservice (de gebruikte sleutel is de extensienaam)
				$GLOBALS['TYPO3_CONF_VARS']['SYS']['OnlineMediaHelpers']['myvideo'] =
				  \MyCompany\Myextension\Helpers\MyVideoHelper::class;

				$rendererRegistry = \TYPO3\CMS\Core\Resource\Rendering\RendererRegistry::getInstance();
				$rendererRegistry->registerRendererClass(
				  \MyCompany\Myextension\Rendering\MyVideoRenderer::class
				);

				// registreer een eigen mime-type voor video's
				$GLOBALS['TYPO3_CONF_VARS']['SYS']['FileInfo']['fileExtensionToMimeType']['myvideo'] =
				  'video/myvideo';

				// registreer een eigen bestandsextensie als toegestaan mediabestand
				$GLOBALS['TYPO3_CONF_VARS']['SYS']['mediafile_ext'] .= ',myvideo';
			\end{lstlisting}

	\end{itemize}

\end{frame}

% ------------------------------------------------------------------------------
% LTXE-SLIDE-START
% LTXE-SLIDE-UID:		0401bb08-d4b8869b-1f90cb0d-1933f4d3
% LTXE-SLIDE-ORIGIN:	0d9d73fc-2a68f137-f5ff5876-74aeef2b English
% LTXE-SLIDE-ORIGIN:	c675e3a8-9a429b0b-61a4a603-d5b0740f German
% LTXE-SLIDE-TITLE:		Unified Backend Routing
% LTXE-SLIDE-REFERENCE:	Feature-65493-BackendRouting.rst
% ------------------------------------------------------------------------------

\begin{frame}[fragile]
	\frametitle{Systeemwijzigingen}
	\framesubtitle{Backend Routing}

	% decrease font size for code listing
	\lstset{basicstyle=\tiny\ttfamily}

	\begin{itemize}

		\item Er is een nieuwe routingcomponent toegevoegd aan de TYPO3-backend die de aanroepen van de verschillende modulen binnen TYPO3 CMS afhandelt

		\item Routes kunnen worden gedefinieerd in de volgende klasse:\newline
			\small
				\texttt{Configuration/Backend/Routes.php}
			\normalsize

			\begin{lstlisting}
				return [
				  'myRouteIdentifier' => [
				    'path' => '/document/edit',
				    'controller' => Acme\MyExtension\Controller\MyExampleController::class . '::methodToCall'
				  ]
				];
			\end{lstlisting}

		\item De objecten die aan de methode worden meegegeven zijn conform de PSR-7-standaard:

			\begin{lstlisting}
				public function methodToCall(ServerRequestInterface $request, ResponseInterface $response) {
				  ...
				}
			\end{lstlisting}

	\end{itemize}

\end{frame}

% ------------------------------------------------------------------------------
% LTXE-SLIDE-START
% LTXE-SLIDE-UID:		ee830a93-ca24c027-a2e05228-5d63186e
% LTXE-SLIDE-ORIGIN:	a38b3e30-a34a3bb1-ee94992a-1c91db32 English
% LTXE-SLIDE-ORIGIN:	df56086a-498ae937-9b2bf393-a95828f3 German
% LTXE-SLIDE-TITLE:		Autoload definition can be provided in ext_emconf.php
% LTXE-SLIDE-REFERENCE:	Feature-68700-AutoloadDefinitionCanBeProvidedInExt_emconfphp.rst
% ------------------------------------------------------------------------------

\begin{frame}[fragile]
	\frametitle{Systeemwijzigingen}
	\framesubtitle{Autoload-definities in \texttt{ext\_emconf.php}}

	% decrease font size for code listing
	\lstset{basicstyle=\tiny\ttfamily}

	\begin{itemize}

		\item Extensies mogen nu meerdere PSR-4-definities toevoegen aan \texttt{ext\_emconf.php}

		\item Dit was al mogelijk in \texttt{composer.json}, maar met deze nieuwe functionaliteit
			hoeft er niet meer speciaal hiervoor een composer-bestand te worden toegevoegd

			\begin{lstlisting}
				$EM_CONF[$_EXTKEY] = array (
				  'title' => 'Extension Skeleton for TYPO3 CMS 7',
				  ...
				'autoload' =>
				  array(
				    'psr-4' => array(
				      'Helhum\\ExtScaffold\\' => 'Classes'
				    )
				  )
				);
			\end{lstlisting}

			\small
				(dit is de nieuwe aanbevolen manier om in TYPO3 klassen te registreren)
			\normalsize

	\end{itemize}

\end{frame}

% ------------------------------------------------------------------------------
% LTXE-SLIDE-START
% LTXE-SLIDE-UID:		e8c559b4-91886e8a-608f000a-4f0429b9
% LTXE-SLIDE-ORIGIN:	6d370930-d24dd6c2-54e2330d-b873a914 English
% LTXE-SLIDE-ORIGIN:	137a9fca-b6558571-346141d2-9304a223 German
% LTXE-SLIDE-TITLE:		Icon-Factory (1)
% LTXE-SLIDE-REFERENCE:	Feature-68741-IntroduceNewIconFactoryAsBaseForReplaceTheIconSkinningAPI.rst
% LTXE-SLIDE-REFERENCE:	Feature-69095-IntroduceIconStateForIconFactory.rst
% ------------------------------------------------------------------------------

\begin{frame}[fragile]
	\frametitle{Systeemwijzigingen}
	\framesubtitle{Nieuwe Icon-factory (1)}

	% decrease font size for code listing
	\lstset{basicstyle=\smaller\ttfamily}

	\begin{itemize}

		\item Logica voor het werken met iconen, icoonformaten en -overlays
			is nu gebundeld in de nieuwe klasse \texttt{IconFactory}

		\item De nieuwe Icon-factory vervangt de oude API voor iconen

		\item Alle iconen uit de core worden in de \texttt{IconRegistry}-klasse geregistreerd

		\item Om bestaande iconen te vervangen of om iconen aan de Icon-factory toe te voegen 
			moeten extensies gebruik maken van \texttt{IconRegistry::registerIcon()}:

			\begin{lstlisting}
				IconRegistry::registerIcon(
				  $identifier,
				  $iconProviderClassName,
				  array $options = array()
				);
			\end{lstlisting}

	\end{itemize}

\end{frame}

% ------------------------------------------------------------------------------
% LTXE-SLIDE-START
% LTXE-SLIDE-UID:		b19bef40-d35e4a54-dbb37c47-4d64e2a7
% LTXE-SLIDE-ORIGIN:	4e9f1524-c3c04972-bf6bd14b-19193ab2 English
% LTXE-SLIDE-ORIGIN:	6b59b4c7-21a7ff5c-4e67b53d-e4509355 German
% LTXE-SLIDE-TITLE:		Icon-Factory (2)
% LTXE-SLIDE-REFERENCE:	Feature-68741-IntroduceNewIconFactoryAsBaseForReplaceTheIconSkinningAPI.rst
% LTXE-SLIDE-REFERENCE:	Feature-69095-IntroduceIconStateForIconFactory.rst
% ------------------------------------------------------------------------------

\begin{frame}[fragile]
	\frametitle{Systeemwijzigingen}
	\framesubtitle{Nieuwe Icon-factory (2)}

	% decrease font size for code listing
	\lstset{basicstyle=\tiny\ttfamily}

	\begin{itemize}

		\item De TYPO3 CMS-core gebruikt 3 IconProviders:\newline
			\smaller
				\texttt{BitmapIconProvider}, \texttt{FontawesomeIconProvider} en \texttt{SvgIconProvider}
			\normalsize

		\item Voorbeeld:

			\begin{lstlisting}
				$iconFactory = GeneralUtility::makeInstance(IconFactory::class);
				$iconFactory->getIcon(
				  $identifier,
				  Icon::SIZE_SMALL,
				  $overlay,
				  IconState::cast(IconState::STATE_DEFAULT)
				)->render();
			\end{lstlisting}

		\item Geldige waarden voor \texttt{Icon::SIZE\_...} zijn:\newline
			\small\texttt{SIZE\_SMALL}, \texttt{SIZE\_DEFAULT} en \texttt{SIZE\_LARGE}\normalsize
			\vspace{0.4cm}

		\item Geldige waarden voor \texttt{Icon::STATE\_...} zijn:\newline
			\small\texttt{STATE\_DEFAULT} en \texttt{STATE\_DISABLED}\normalsize

	\end{itemize}

\end{frame}

% ------------------------------------------------------------------------------
% LTXE-SLIDE-START
% LTXE-SLIDE-UID:		b5bca02c-667bc205-d8570391-56e1cdf0
% LTXE-SLIDE-ORIGIN:	ffa90088-571dcf20-3d315854-a05814da English
% LTXE-SLIDE-ORIGIN:	21a7eeb5-b4c7ff5c-b53de450-4e679355 German
% LTXE-SLIDE-TITLE:		Icon-Factory (3)
% LTXE-SLIDE-REFERENCE:	Feature-68741-IntroduceNewIconFactoryAsBaseForReplaceTheIconSkinningAPI.rst
% LTXE-SLIDE-REFERENCE:	Feature-69095-IntroduceIconStateForIconFactory.rst
% ------------------------------------------------------------------------------

\begin{frame}[fragile]
	\frametitle{Systeemwijzigingen}
	\framesubtitle{Nieuwe Icon-factory (3)}

	% decrease font size for code listing
	\lstset{basicstyle=\tiny\ttfamily}

	\begin{itemize}

		\item In de TYPO3 CMS-core is een ViewHelper beschikbaar waarmee het eenvoudig is om iconen te gebruiken in Fluid:

			\begin{lstlisting}
				{namespace core = TYPO3\CMS\Core\ViewHelpers}

				<core:icon identifier="my-icon-identifier"></core:icon>

				<!-- use the "small" size if none given ->
				<core:icon identifier="my-icon-identifier"></core:icon>
				<core:icon identifier="my-icon-identifier" size="large"></core:icon>
				<core:icon identifier="my-icon-identifier" overlay="overlay-identifier"></core:icon>

				<core:icon identifier="my-icon-identifier" size="default" overlay="overlay-identifier">
				</core:icon>

				<core:icon identifier="my-icon-identifier" size="large" overlay="overlay-identifier">
				</core:icon>
			\end{lstlisting}

	\end{itemize}

\end{frame}

% ------------------------------------------------------------------------------
% LTXE-SLIDE-START
% LTXE-SLIDE-UID:		dededf1f-30a5361d-383c993c-9d7b6a49
% LTXE-SLIDE-ORIGIN:	10739ce3-88cfb3f6-d8364643-424cbf3c English
% LTXE-SLIDE-ORIGIN:	d9de708f-474f2f36-c104b01c-0d23a352 German
% LTXE-SLIDE-TITLE:		Signal for pre processing linkvalidator records
% LTXE-SLIDE-REFERENCE:	Feature-52217-SignalForPreProcessingLinkvalidatorRecords.rst
% ------------------------------------------------------------------------------

\begin{frame}[fragile]
	\frametitle{Systeemwijzigingen}
	\framesubtitle{Hooks en signals}

	% decrease font size for code listing
	\lstset{basicstyle=\tiny\ttfamily}

	\begin{itemize}

		\item Er is een nieuw signal toegevoegd aan LinkValidator, die een aanvullende verwerking
			na het initialiseren van een bepaald record mogelijk maakt
			\newline
			\small
				(bijv. om gegevens uit de pluginconfiguratie in te laden)
			\normalsize

		\item Registreer de signal als volgt in \texttt{ext\_localconf.php}:

			\begin{lstlisting}
				$signalSlotDispatcher = \TYPO3\CMS\Core\Utility\GeneralUtility::makeInstance(
				  \TYPO3\CMS\Extbase\SignalSlot\Dispatcher::class
				);

				$signalSlotDispatcher->connect(
				  \TYPO3\CMS\Linkvalidator\LinkAnalyzer::class,
				  'beforeAnalyzeRecord',
				  \Vendor\Package\Slots\RecordAnalyzerSlot::class,
				  'beforeAnalyzeRecord'
				);
			\end{lstlisting}

	\end{itemize}

\end{frame}

% ------------------------------------------------------------------------------
% LTXE-SLIDE-START
% LTXE-SLIDE-UID:		5aae6048-911a83e1-42eb1df1-d512b80a
% LTXE-SLIDE-ORIGIN:	8b50edb3-2363f2a9-3120a711-db8b6ede English
% LTXE-SLIDE-ORIGIN:	bcb8ef73-b8669a17-102da209-5f5c9adc German
% LTXE-SLIDE-TITLE:		Replaced JumpURL features with hooks (1)
% LTXE-SLIDE-REFERENCE:	Breaking-52156-ReplaceJumpUrlWithHooks.rst
% ------------------------------------------------------------------------------

\begin{frame}[fragile]
	\frametitle{Systeemwijzigingen}
	\framesubtitle{Systeemextensie JumpUrl (1)}

	% decrease font size for code listing
	\lstset{basicstyle=\tiny\ttfamily}

	\begin{itemize}

		\item Het genereren en verwerken van JumpURL's is verplaatst naar de nieuwe systeemextensie \texttt{jumpurl}

		\item Er zijn nieuwe hooks beschikbaar die het mogelijk maken om eigen URL's te genereren
			en te verwerken (zie volgende pagina)

	\end{itemize}

	\breakingchange

\end{frame}

% ------------------------------------------------------------------------------
% LTXE-SLIDE-START
% LTXE-SLIDE-UID:		cade78ab-02592900-47ffd305-b81512b9
% LTXE-SLIDE-ORIGIN:	68003b46-065b81f8-5206141c-77a35b88 English
% LTXE-SLIDE-ORIGIN:	ecb8fef3-69b86a17-02da2109-9adc5f5c German
% LTXE-SLIDE-TITLE:		Replaced JumpURL features with hooks (2)
% LTXE-SLIDE-REFERENCE:	Breaking-52156-ReplaceJumpUrlWithHooks.rst
% ------------------------------------------------------------------------------

\begin{frame}[fragile]
	\frametitle{Systeemwijzigingen}
	\framesubtitle{Systeemextensie JumpUrl (2)}

	% decrease font size for code listing
	\lstset{basicstyle=\tiny\ttfamily}

	\begin{itemize}

		\item Hook 1: manipuleren van een \textbf{URL} tijdens het maken van een link

			\begin{lstlisting}
				$GLOBALS['TYPO3_CONF_VARS']['SC_OPTIONS']['urlProcessing']['urlHandlers']
				  ['myext_myidentifier']['handler'] = \Company\MyExt\MyUrlHandler::class;

				// klasse moet UrlHandlerInterface implementeren:
				class MyUrlHandler implements \TYPO3\CMS\Frontend\Http\UrlHandlerInterface {
				  ...
				}
			\end{lstlisting}

		\item Hook 2: verwerken van \textbf{links}

			\begin{lstlisting}
				$GLOBALS['TYPO3_CONF_VARS']['SC_OPTIONS']['urlProcessing']['urlProcessors']
				  ['myext_myidentifier']['processor'] = \Company\MyExt\MyUrlProcessor::class;

				// klasse moet UrlProcessorInterface implementeren:
				class MyUrlProcessor implements \TYPO3\CMS\Frontend\Http\UrlProcessorInterface {
				  ...
				}
			\end{lstlisting}

	\end{itemize}

\end{frame}

% ------------------------------------------------------------------------------
% LTXE-SLIDE-START
% LTXE-SLIDE-UID:		50038aac-e61b633c-bcc11161-2d99affb
% LTXE-SLIDE-ORIGIN:	ff50feed-0f2dfe5a-1f960c51-e493d3b0 English
% LTXE-SLIDE-ORIGIN:	08a70f0f-528daf11-ee485c29-d9707f25 German
% LTXE-SLIDE-TITLE:		Colored Output for CLI Calls on Errors
% LTXE-SLIDE-REFERENCE:	Feature-67056-AddOptionToDisableMoveButtonsTCAGroupType.rst
% LTXE-SLIDE-REFERENCE:	Feature-67875-OverrideCategoryRegistryEntry.rst
% LTXE-SLIDE-REFERENCE:	Feature-68804-ColoredOutputForCLI-relevantErrorMessages.rst
% ------------------------------------------------------------------------------

\begin{frame}[fragile]
	\frametitle{Systeemwijzigingen}
	\framesubtitle{Command-line-interface (CLI)}

	% decrease font size for code listing
	\lstset{basicstyle=\tiny\ttfamily}

	\begin{itemize}

		\item Wanneer vanaf de opdrachtregel \texttt{typo3/cli\_dispatch.phpsh} wordt aangeroepen
			zonder of met een verkeerde parameter, dan wordt er een gekleurde foutmelding getoond

		\item Extbase-commandcontrollers kunnen nu ook in submappen van de \texttt{Command}-map
			bewaard worden

		\item Voorbeeld:\newline

			De controller in bestand:\newline
			\smaller\texttt{my\_ext/Classes/Command/Hello/WorldCommandController.php}\normalsize\newline
			...kan als volgt vanaf de opdrachtregel worden aangeroepen:\newline
			\smaller\texttt{typo3/cli\_dispatch.sh extbase my\_ext:hello:world <arguments>}\normalsize

	\end{itemize}

\end{frame}

% ------------------------------------------------------------------------------
% LTXE-SLIDE-START
% LTXE-SLIDE-UID:		e6f524d9-9e539eae-b56cbf3c-cbad8f52
% LTXE-SLIDE-ORIGIN:	df540c68-edd41657-8854e46a-a716dccb English
% LTXE-SLIDE-ORIGIN:	a708daf1-e48f525c-7010fe7f-2529d9e4 German
% LTXE-SLIDE-TITLE:		Miscellaneous (1)
% LTXE-SLIDE-REFERENCE:	Feature-68804-ColoredOutputForCLI-relevantErrorMessages.rst
% LTXE-SLIDE-REFERENCE:	Feature-69512-SupportTyposcriptFilesAsTextFileType.rst
% LTXE-SLIDE-REFERENCE:	Feature-69543-IntroducedGLOBALSTYPO3_CONF_VARSSYSmediafile_ext.rst
% ------------------------------------------------------------------------------

\begin{frame}[fragile]
	\frametitle{Systeemwijzigingen}
	\framesubtitle{Diversen (1)}

	\begin{itemize}

		\item De verplaatsknoppen van TCA-type \texttt{group} kun nu expliciet worden uitgezet
			met de optie \texttt{hideMoveIcons = TRUE}

		\item Methode \texttt{makeCategorizable} is uitgebreid met een nieuwe parameter
			\texttt{override} waarmee een nieuwe categorieconfiguratie kan worden ingesteld
			voor een reeds geregistreerde tabel/veld-combinatie

		\item Voorbeeld:

			\begin{lstlisting}
				\TYPO3\CMS\Core\Utility\ExtensionManagementUtility::makeCategorizable(
				  'css_styled_content', 'tt_content', 'categories', array(), TRUE
				);
			\end{lstlisting}

			\small
				Met de laatste parameter (hier: \texttt{TRUE}) wordt het overschrijven geforceerd
				(standaardwaarde is \texttt{FALSE}).
			\normalsize

	\end{itemize}

\end{frame}

% ------------------------------------------------------------------------------
% LTXE-SLIDE-START
% LTXE-SLIDE-UID:		c47ee6b8-1b2c7bea-90c2f64b-4eda2e41
% LTXE-SLIDE-ORIGIN:	5a6db8b0-ff039479-8ba12ed0-d50d8a0f English
% LTXE-SLIDE-ORIGIN:	d52d6418-554d801f-352b85cb-54046d28 German
% LTXE-SLIDE-TITLE:		Miscellaneous (2)
% LTXE-SLIDE-REFERENCE:	Feature-69730-IntroduceUniqueIdGenerator.rst
% LTXE-SLIDE-REFERENCE:	Important-68758-CommandControllersAllowedInSubfolders.rst
% ------------------------------------------------------------------------------

\begin{frame}[fragile]
	\frametitle{Systeemwijzigingen}
	\framesubtitle{Diversen (2)}

	% decrease font size for code listing
	%\lstset{basicstyle=\tiny\ttfamily}

	\begin{itemize}

		\item Er is een nieuwe functie die een unieke ID genereert:

			\begin{lstlisting}
				$uniqueId = \TYPO3\CMS\Core\Utility\StringUtility::getUniqueId('Prefix');
			\end{lstlisting}

		\item Bestandsextensie \texttt{.typoscript} is toegevoegd aan de lijst van
			geldige extensies voor tekstbestanden

		\item Een nieuwe configuratie-optie definieert de bestandsextensies van mediabestanden

			\begin{lstlisting}
				$GLOBALS['TYPO3_CONF_VARS']['SYS']['mediafile_ext'] =
				  'gif,jpg,jpeg,bmp,png,pdf,svg,ai,mov,avi';
			\end{lstlisting}

	\end{itemize}

	\breakingchange

\end{frame}

% ------------------------------------------------------------------------------
