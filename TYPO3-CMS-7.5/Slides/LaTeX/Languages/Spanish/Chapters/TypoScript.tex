% ------------------------------------------------------------------------------
% TYPO3 CMS 7.5 - What's New - Chapter "TypoScript" (Spanish Version)
%
% @author	Michael Schams <schams.net>
% @license	Creative Commons BY-NC-SA 3.0
% @link		http://typo3.org/download/release-notes/whats-new/
% @language	Spanish
% ------------------------------------------------------------------------------
% LTXE-CHAPTER-UID:		f607fb32-7f7b68d8-68d0e69a-16738006
% LTXE-CHAPTER-NAME:	TypoScript
% ------------------------------------------------------------------------------

\section{TSconfig \& TypoScript}
\begin{frame}[fragile]
	\frametitle{TSconfig \& TypoScript}

	\begin{center}\huge{Capítulo 2:}\end{center}
	\begin{center}\huge{\color{typo3darkgrey}\textbf{TSconfig \& TypoScript}}\end{center}

\end{frame}

% ------------------------------------------------------------------------------
% LTXE-SLIDE-START
% LTXE-SLIDE-UID:		ae5be59b-d599eeca-29a897ed-10ec052d
% LTXE-SLIDE-ORIGIN:	195a3fa3-207484a7-94e7ce76-c821394c English
% LTXE-SLIDE-ORIGIN:	c76757e2-fc02b94b-166a8168-d68774c9 German
% LTXE-SLIDE-TITLE:		Feature #16525: Add conditions to INCLUDE_TYPOSCRIPT
% LTXE-SLIDE-REFERENCE:	Feature-16525-AddConditionsToINCLUDE_TYPOSCRIPT.rst
% ------------------------------------------------------------------------------
\begin{frame}[fragile]
	\frametitle{TSconfig \& TypoScript}
	\framesubtitle{Condiciones para Includes TypoScript}

	% decrease font size for code listing
	\lstset{basicstyle=\tiny\ttfamily}

	\begin{itemize}

		\item \texttt{INCLUDE\_TYPOSCRIPT} tiene una propiedad extra (opcional) "condition" ahora, que
			incluye el fichero/directorio sólo, si la condición se cumple

		\begin{lstlisting}
			// only include TypoScript, if current user is logged in:
			<INCLUDE_TYPOSCRIPT: source="FILE:EXT:my_extension/Configuration/TypoScript/feuser.ts"
			  condition="[loginUser = *]">

			// include TypoScript depending on application context:
			<INCLUDE_TYPOSCRIPT: source="FILE:EXT:my_extension/Configuration/TypoScript/staging.ts"
			  condition="applicationContext = /^Production\\/Staging\\/Server\\d+$/">
		\end{lstlisting}

	\end{itemize}

\end{frame}

% ------------------------------------------------------------------------------
% LTXE-SLIDE-START
% LTXE-SLIDE-UID:		3510daf2-948e7368-8ceb2f5d-3a343c2d
% LTXE-SLIDE-ORIGIN:	4003d374-92104168-0eb97927-726037fa English
% LTXE-SLIDE-ORIGIN:	1bde9e8a-82352f02-c5c9f3db-b194cfb1 German
% LTXE-SLIDE-TITLE:		Feature #28243: Introduce TCA option to disable age display of dates per field
% LTXE-SLIDE-REFERENCE:	Feature-28243-IntroduceTcaOptionToDisableAgeDisplay.rst
% ------------------------------------------------------------------------------
\begin{frame}[fragile]
	\frametitle{TSconfig \& TypoScript}
	\framesubtitle{Opción TCA: Mostrar Offset de Fecha}

	% decrease font size for code listing
	\lstset{basicstyle=\tiny\ttfamily}

	\begin{itemize}

		\item Opción TCA \texttt{disableAgeDisplay} deshabilita la visualización de la edad\newline
			\small
				(por ejemplo: \texttt{"2015-08-30 (-27 días)"})
			\normalsize

			\begin{lstlisting}
				$GLOBALS['TCA']['tt_content']['columns']['date']['config']['disableAgeDisplay'] = true;
			\end{lstlisting}

		\item Como una precondición, \texttt{tipo} del campo tiene que ser \texttt{input}
			y \texttt{eval} tiene que configurarse a \texttt{date}

	\end{itemize}

\end{frame}

% ------------------------------------------------------------------------------
% LTXE-SLIDE-START
% LTXE-SLIDE-UID:		fd681ef1-0e79016c-ce792a22-91f8a7de
% LTXE-SLIDE-ORIGIN:	68c8a68b-f7e50bf3-dc67e0e5-d3a33fd8 English
% LTXE-SLIDE-ORIGIN:	c3aa0981-411fcabb-bd9dca10-16f42898 German
% LTXE-SLIDE-TITLE:		Feature #57632: Include inline language label files with TypoScript (1)
% LTXE-SLIDE-REFERENCE:	Feature-57632-AddInlineLanguageLabelFilesWithTypoScript.rst
% ------------------------------------------------------------------------------
\begin{frame}[fragile]
	\frametitle{TSconfig \& TypoScript}
	\framesubtitle{Ficheros de Etiquetas de Lenguaje Inline con TypoScript (1)}

	% decrease font size for code listing
	\lstset{basicstyle=\tiny\ttfamily}

	\begin{itemize}

		\item Ficheros de lenguaje XLF pueden leerse y convertirse en un vector inline

		\item Esto habilita el acceso a etiquetas de lenguaje en JavaScript por ejemplo

		\item Se soportan los siguientes tres parámetros opcionales:

			\begin{itemize}
				\item \texttt{selectionPrefix}:\newline
					sólo se incluirán claves de etiquetas que empiecen con este prefijo
				\item \texttt{stripFromSelectionName}:\newline
					cadena que será eliminada de cualquier clave de etiqueta incluida
				\item \texttt{errorMode}:\newline
					modo de error si no puede encontrrse el fichero:\newline
					0: entrada syslog (por defecto), 1: ignorar, 3: lanzar una excepción
			\end{itemize}

	\end{itemize}

\end{frame}

% ------------------------------------------------------------------------------
% LTXE-SLIDE-START
% LTXE-SLIDE-UID:		4c880607-7aecfc44-89f8dae0-944f4adc
% LTXE-SLIDE-ORIGIN:	bb800987-b0e74193-c6ec6370-3b6a062d English
% LTXE-SLIDE-ORIGIN:	289816f4-411fcabb-0981cabb-bd9dca10 German
% LTXE-SLIDE-TITLE:		Feature #57632: Include inline language label files with TypoScript (2)
% LTXE-SLIDE-REFERENCE:	Feature-57632-AddInlineLanguageLabelFilesWithTypoScript.rst
% ------------------------------------------------------------------------------
\begin{frame}[fragile]
	\frametitle{TSconfig \& TypoScript}
	\framesubtitle{Ficheros de Etiquetas de Lenguaje Inline con TypoScript (2)}

	% decrease font size for code listing
	\lstset{basicstyle=\tiny\ttfamily}

	\begin{itemize}

		\item Ejemplo:

			\begin{lstlisting}
				page = PAGE
				page.inlineLanguageLabelFiles {
				  someLabels = EXT:myExt/Resources/Private/Language/locallang.xlf
				  someLabels.selectionPrefix = idPrefix
				  someLabels.stripFromSelectionName = strip_me
				  someLabels.errorMode = 2
				}
			\end{lstlisting}

		\item Salida:

			\begin{lstlisting}
				<script type="text/javascript">
				/*<![CDATA[*/
				  var TYPO3 = TYPO3 || {};
				  TYPO3.lang = {"firstLabel":[{"source":"first Label","target":"erstes Label"}],
				  "secondLabel":[{"source":"second Label","target":"zweites Label"}]};
				/*]]>*/
				</script>
			\end{lstlisting}

	\end{itemize}

\end{frame}

% ------------------------------------------------------------------------------
% LTXE-SLIDE-START
% LTXE-SLIDE-UID:		ec95ab22-e2fac342-60c1fbef-aaf4a536
% LTXE-SLIDE-ORIGIN:	bac61b19-18a4db7b-af13de6c-5dbed67b English
% LTXE-SLIDE-ORIGIN:	f14c90e2-bf35bf40-e38d74a7-f4a2e973 German
% LTXE-SLIDE-TITLE:		Feature #59144: Previewing workspace records using Page TSconfig
% LTXE-SLIDE-REFERENCE:	Feature-59144-PageTSconfigWorkspacePreview.rst
% ------------------------------------------------------------------------------
\begin{frame}[fragile]
	\frametitle{TSconfig \& TypoScript}
	\framesubtitle{Previsualización de Workspace por TSconfig}

	% decrease font size for code listing
	\lstset{basicstyle=\tiny\ttfamily}

	\begin{itemize}

		\item TYPO3 CMS genera enlaces de previsualización sólo para tablas \texttt{tt\_content}, \texttt{pages} y
			\texttt{pages\_language\_overlay} por defecto

		\item Esto puede configurarse usando PageTSconfig ahora:

			\begin{lstlisting}
				# use page 123 for previewing workspaces records (in general)
				options.workspaces.previewPageId = 123

				# use the pid field of each record for previewing (in general)
				options.workspaces.previewPageId = field:pid

				# use page 123 for previewing workspaces records (for table tx_myext_table)
				options.workspaces.previewPageId.tx_myext_table = 123

				# use the pid field of each record for previewing (or table tx_myext_table)
				options.workspaces.previewPageId.tx_myext_table = field:pid
			\end{lstlisting}

	\end{itemize}

\end{frame}

% ------------------------------------------------------------------------------
% LTXE-SLIDE-START
% LTXE-SLIDE-UID:		9c9f2e41-fa79e3fa-7da1ea95-171eea77
% LTXE-SLIDE-ORIGIN:	c4b7358c-07059082-861a3b2d-929e1cc6 English
% LTXE-SLIDE-ORIGIN:	faeca558-5a675884-a1b77fae-36e316eb German
% LTXE-SLIDE-TITLE:		Feature #59591: Image quality definable per sourceCollection
% LTXE-SLIDE-REFERENCE:	Feature-59591-ImageQualityDefinablePerSourceCollection.rst
% ------------------------------------------------------------------------------
\begin{frame}[fragile]
	\frametitle{TSconfig \& TypoScript}
	\framesubtitle{Caldad de Imagen de \texttt{sourceCollection}}

	% decrease font size for code listing
	\lstset{basicstyle=\tiny\ttfamily}

	\begin{itemize}

		\item Puede configurarse ahora la calidad de imagen de cada entrada \texttt{sourceCollection}

		\item Este ajuste toma precedencia sobre la configuración en la Herramienta de Instalación\newline
			(almacenado en el fichero \texttt{LocalConfiguration.php})

		\item Ejemplo:

			\begin{lstlisting}
				# for small retina images
				tt_content.image.20.1.sourceCollection.smallRetina.quality = 80

				# for large retina images
				tt_content.image.20.1.sourceCollection.largeRetina.quality = 65
			\end{lstlisting}

	\end{itemize}

\end{frame}

% ------------------------------------------------------------------------------
% LTXE-SLIDE-START
% LTXE-SLIDE-UID:		e476701a-35ff16c1-d5eab4c1-dcf308c7
% LTXE-SLIDE-ORIGIN:	ad691a22-7ab387b6-6e00c0aa-e24efd91 English
% LTXE-SLIDE-ORIGIN:	0073160e-6c95f70d-fcaf008f-3fd84819 German
% LTXE-SLIDE-TITLE:		Feature #67880: Added count to split
% LTXE-SLIDE-REFERENCE:	Feature-67880-AddedCountToSplit.rst
% ------------------------------------------------------------------------------
% Note: feature #67880 has been introduced in TYPO3 CMS 7.4 already
% and we mentioned it in the last version of our What's New Slides.
% It is not clear, why this feature is included in the ChangeLog of
% 7.5 again (with a slightly different title), but we should have it
% in our Slides, too.

\begin{frame}[fragile]
	\frametitle{TSconfig \& TypoScript}
	\framesubtitle{Contar Elementos en una Lista}

	% decrease font size for code listing
	\lstset{basicstyle=\tiny\ttfamily}

	\begin{itemize}

		\item Ha sido añadida una nueva propiedad \texttt{returnCount} a la propiedad stdWrap \texttt{split}

		\item Esto permite contar el número de elementos en una lista separada por comas

		\item El siguiente código devuelve \texttt{9} por ejemplo:

			\begin{lstlisting}
				1 = TEXT
				1 {
				  value = x,y,z,1,2,3,a,b,c
				  split.token = ,
				  split.returnCount = 1
				}
			\end{lstlisting}

	\end{itemize}

\end{frame}
% ------------------------------------------------------------------------------
% LTXE-SLIDE-START
% LTXE-SLIDE-UID:		dd28cddd-38b848ed-ea0269cf-c5aa2324
% LTXE-SLIDE-ORIGIN:	1a1b3a3f-a12510ef-003db5f4-2fc4f613 English
% LTXE-SLIDE-ORIGIN:	da2aa521-48abf012-e8b4b95f-0e3f3556 German
% LTXE-SLIDE-TITLE:		Feature #69602: Simplify handling of backend layouts in frontend (1)
% LTXE-SLIDE-REFERENCE:	Feature-69602-SimplifyHandlingOfBackendLayoutsInFrontend.rst
% ------------------------------------------------------------------------------

\begin{frame}[fragile]
	\frametitle{TSconfig \& TypoScript}
	\framesubtitle{Manejo de Backend Layouts (1)}

	% decrease font size for code listing
	\lstset{basicstyle=\tiny\ttfamily}

	\begin{itemize}

		\item Se ha simplificado el manejo de backend layouts para el frontend

		\item Puede usarse una nueva opción \texttt{pagelayout} en TypoScript

		\item Ejemplo:

			\begin{lstlisting}
				page.10 = FLUIDTEMPLATE
				page.10 {
				  file.stdWrap.cObject = CASE
				  file.stdWrap.cObject {
				    key.data = pagelayout
				    default = TEXT
				    default.value = EXT:sitepackage/Resources/Private/Templates/Home.html
				    3 = TEXT
				    3.value = EXT:sitepackage/Resources/Private/Templates/1-col.html
				    4 = TEXT
				    4.value = EXT:sitepackage/Resources/Private/Templates/2-col.html
				  }
				}
			\end{lstlisting}

			\smaller
				(continúa en la siguiente página)
			\normalsize

	\end{itemize}

\end{frame}

% ------------------------------------------------------------------------------
% LTXE-SLIDE-START
% LTXE-SLIDE-UID:		02f093aa-c73ed574-2d18efa9-f1e4bef2
% LTXE-SLIDE-ORIGIN:	4054dec5-eb1750c8-20a10e74-0935b709 English
% LTXE-SLIDE-ORIGIN:	b95fe8b4-a521da2a-f01248ab-35560e3f German
% LTXE-SLIDE-TITLE:		Feature #69602: Simplify handling of backend layouts in frontend (2)
% LTXE-SLIDE-REFERENCE:	Feature-69602-SimplifyHandlingOfBackendLayoutsInFrontend.rst
% ------------------------------------------------------------------------------

\begin{frame}[fragile]
	\frametitle{TSconfig \& TypoScript}
	\framesubtitle{Manejo de Backend Layouts (2)}

	% decrease font size for code listing
	\lstset{basicstyle=\tiny\ttfamily}

	\begin{itemize}

		\item ...donde \texttt{key.data = pagelayout} reemplaza el siguiente código:

			\begin{lstlisting}
				field = backend_layout
				ifEmpty.data = levelfield:-2,backend_layout_next_level,slide
				ifEmpty.ifEmpty = default
			\end{lstlisting}

	\end{itemize}

\end{frame}

% ------------------------------------------------------------------------------
% LTXE-SLIDE-START
% LTXE-SLIDE-UID:		5454160e-5b82ea68-cd106d8c-e5758258
% LTXE-SLIDE-ORIGIN:	07353cea-dac9032f-52a37a42-a261e38a English
% LTXE-SLIDE-ORIGIN:	6f1f7bb6-b50c019e-29c9be36-7d30c87d German
% LTXE-SLIDE-TITLE:		Feature #68756: Add config "base" to stdWrap
% LTXE-SLIDE-REFERENCE:	Feature-68756-AddConfigBaseToStdWrap.rst
% ------------------------------------------------------------------------------
\begin{frame}[fragile]
	\frametitle{TSconfig \& TypoScript}
	\framesubtitle{Miscelánea}

	\begin{itemize}

		\item Ha sido introducida la función stdWrap \texttt{bytes} en TYPO3 CMS 7.4

		\item La habilidad para configurar la \texttt{base} ha sido añadida en TYPO3 CMS 7.5,
			que permite definir si usar una base de 1000 o 1024 con la que calcular

			\begin{lstlisting}
				bytes.labels = " | K| M| G"
				bytes.base = 1000
			\end{lstlisting}

	\end{itemize}

\end{frame}

% ------------------------------------------------------------------------------
