% ------------------------------------------------------------------------------
% TYPO3 CMS 8.5 - What's New - Chapter "Extbase & Fluid" (French Version)
%
% @author	Michael Schams <schams.net>
% @license	Creative Commons BY-NC-SA 3.0
% @link		http://typo3.org/download/release-notes/whats-new/
% @language	French
% ------------------------------------------------------------------------------
% LTXE-CHAPTER-UID:		846d40ce-66fdc2ec-750dcf95-33ce93e0
% LTXE-CHAPTER-NAME:	Extbase & Fluid
% ------------------------------------------------------------------------------

\section{Extbase \& Fluid}
\begin{frame}[fragile]
	\frametitle{Extbase \& Fluid}

	\begin{center}\huge{Chapitre 4~:}\end{center}
	\begin{center}\huge{\color{typo3darkgrey}\textbf{Extbase \& Fluid}}\end{center}

\end{frame}

% ------------------------------------------------------------------------------
% LTXE-SLIDE-START
% LTXE-SLIDE-UID:		c5986329-fb36f473-a5430d2e-6ba49fb1
% LTXE-SLIDE-TITLE:		#78116: Extbase support for Doctrine's native DBAL Statement and QueryBuilder
% ------------------------------------------------------------------------------
\begin{frame}[fragile]
	\frametitle{Extbase \& Fluid}
	\framesubtitle{Doctrine DBAL}

	% decrease font size for code listing
	\lstset{basicstyle=\tiny\ttfamily}

	\begin{itemize}
		\item La fonctionalité de requête SQL directe supporte aussi des objets
			QueryBuilder et instances de
			\texttt{\textbackslash
			Doctrine\textbackslash
			DBAL\textbackslash
			Statement} comme requête préparée
		\item L'exemple suivant fonctionne dans n'importe quel dépôt Extbase utilisant les
		 	requêtes natives Doctrine DBAL~:

			\begin{lstlisting}
				$connection = $this->objectManager->get(ConnectionPool::class)->getConnectionForTable('mytable');
				$statement = $this->objectManager->get(
				  \Doctrine\DBAL\Statement::class,
				  'SELECT * FROM mytable WHERE uid=? OR title=?',
				  $connection
				);

				$query = $this->createQuery();
				$query->statement($statement, [$uid, $title]);
			\end{lstlisting}
	\end{itemize}

\end{frame}
% ------------------------------------------------------------------------------
% LTXE-SLIDE-START
% LTXE-SLIDE-UID:		8d31d436-4397d4fd-aa0eed2e-df798aa3
% LTXE-SLIDE-TITLE:		#78002: Enforce cHash argument for Extbase actions
% ------------------------------------------------------------------------------

\begin{frame}[fragile]
	\frametitle{Extbase \& Fluid}
	\framesubtitle{Argument \texttt{cHash}}

	\begin{itemize}
		\item Les URIs ciblant des actions Extbase nécessitent maintenant un \texttt{cHash} valide par défaut\newline
			(actions mise en cache ou non)
		\item Le comportement se désactive pour toutes les actions en utilisant l'option extbase
			\texttt{requireCHashArgumentForActionArguments}
	\end{itemize}

\end{frame}
% ------------------------------------------------------------------------------
% LTXE-SLIDE-START
% LTXE-SLIDE-UID:		50143a1e-53a17c8c-83ffe7ea-5e88cba0
% LTXE-SLIDE-TITLE:		#29399: OptionViewHelper and OptgroupViewHelper
% ------------------------------------------------------------------------------

\begin{frame}[fragile]
	\frametitle{Extbase \& Fluid}
	\framesubtitle{Contenu du ViewHelper \texttt{f:form.select}}

	% decrease font size for code listing
	\lstset{basicstyle=\tiny\ttfamily}

	\begin{itemize}
		\item Deux ViewHelper sont introduits pour permettre la définition manuelle des
			options et groupes de \texttt{f:form.select} comme contenu de la balise du
			champ liste

			\begin{itemize}
				\item \texttt{OptionViewHelper}
				\item \texttt{OptgroupViewHelper}
			\end{itemize}

		\item Exemple~:

			\begin{lstlisting}
				<f:form.select name="myproperty">
				  <f:form.select.option value="1">Option one</f:form.select.option>
				  <f:form.select.option value="2">Option two</f:form.select.option>
				  <f:form.select.optgroup>
				    <f:form.select.option value="3">Grouped option one</f:form.select.option>
				    <f:form.select.option value="4">Grouped option twi</f:form.select.option>
				  </f:form.select.optgroup>
				</f:form.select>
			\end{lstlisting}

		\end{itemize}

\end{frame}


% ------------------------------------------------------------------------------
% LTXE-SLIDE-START
% LTXE-SLIDE-UID:		d36b9935-fc7d9e60-742a3f42-ef15e856
% LTXE-SLIDE-TITLE:		#78415: Global Fluid ViewHelper Namespace
% ------------------------------------------------------------------------------
\begin{frame}[fragile]
	\frametitle{Extbase \& Fluid}
	\framesubtitle{Espaces de noms globaux des ViewHelper Fluid}

	\begin{itemize}
		\item Les espaces de noms globaux des ViewHelper Fluid peuvent être configurés~:\newline
			\smaller
				\texttt{\$GLOBALS['TYPO3\_CONF\_VARS']['SYS']['fluid']['namespaces']}
			\normalsize
		\item Permet aux espaces de nom d'être manipulés avec la configuration du site
		\item Avantages~:

			\begin{itemize}
				\item Les paquets de ViewHelper tiers peuvent manipuler l'espace de noms
					Fluid global \texttt{f:}
				\item Les paquets de ViewHelper tiers peuvent inscrire leurs nouveaux
					espaces de noms globaux comme requis
				\item Les développeurs de template peuvent utiliser ces espaces de noms globaux
					sans les importer préalablement et les utiliser dans tous les template Fluid
					quelque soit le contexte
			\end{itemize}

	\end{itemize}

\end{frame}


% ------------------------------------------------------------------------------
% LTXE-SLIDE-START
% LTXE-SLIDE-UID:		e16e8cfd-26f35f09-23b473a2-442ff1e2
% LTXE-SLIDE-TITLE:		#78842: FLUIDTEMPLATE can mimic an actual Extbase web request
% ------------------------------------------------------------------------------
\begin{frame}[fragile]
	\frametitle{Extbase \& Fluid}
	\framesubtitle{\texttt{FLUIDTEMPLATE} peut simuler les Web Requests Extbase}

	% decrease font size for code listing
	\lstset{basicstyle=\small\ttfamily}

	\begin{itemize}
		\item L'élément de contenu \texttt{FLUIDTEMPLATE} peut simuler une requête web Extbase
		\item Permet entre autre d'accéder aux données soumises~:

			\begin{lstlisting}
				$view->getRenderingContext()
				  ->getControllerContext()
				  ->getRequest()
				  ->getArguments();
			\end{lstlisting}

	\end{itemize}

\end{frame}




% ------------------------------------------------------------------------------
