% ------------------------------------------------------------------------------
% TYPO3 CMS 8.5 - What's New - Chapter "TypoScript" (Dutch Version)
%
% @author	Michael Schams <schams.net>
% @license	Creative Commons BY-NC-SA 3.0
% @link		http://typo3.org/download/release-notes/whats-new/
% @language	English
% ------------------------------------------------------------------------------
% LTXE-CHAPTER-UID:		3a9852ea-e2360d9d-1ff5eec1-a7de3f9f
% LTXE-CHAPTER-NAME:	TypoScript
% ------------------------------------------------------------------------------

\section{TSconfig \& TypoScript}
\begin{frame}[fragile]
	\frametitle{TSconfig \& TypoScript}

	\begin{center}\huge{Hoofdstuk 2:}\end{center}
	\begin{center}\huge{\color{typo3darkgrey}\textbf{TSconfig \& TypoScript}}\end{center}

\end{frame}

% ------------------------------------------------------------------------------
% LTXE-SLIDE-START
% LTXE-SLIDE-UID:		c410c002-5b37ba1d-36a2abd1-c9cb6fb2
% LTXE-SLIDE-ORIGIN:	aee7c947-a271d451-cb640e43-101a5543 English
% LTXE-SLIDE-TITLE:		#78549: Override New Page Creation Wizard via page TSconfig
% ------------------------------------------------------------------------------

\begin{frame}[fragile]
	\frametitle{TSconfig \& TypoScript}
	\framesubtitle{Nieuwe assistent pagina aanmaken}

	\begin{itemize}
		\item In vorige versies van TYPO3 CMS was het mogelijk om de "Assistent nieuwe pagina aanmaken"
		 	te overschrijven met eigen scripts:\newline
			\small
				\texttt{mod.web\_list.newPageWiz.overrideWithExtension = myextension}
			\normalsize
		\item De nieuwe manier van het afhandelen van ingangen en eigen scripts is gebouwd met modules/routes
		 	en de bovenstaande optie is verwijderd
		\item De volgende TSconfig-optie kan in plaats hiervan gebruikt worden:
			\small
				\texttt{mod.newPageWizard.override = my\_custom\_module}
			\normalsize

		\item In plaats van het instellen van een extensie-key moet er nu een module of
		 	route gespecificeerd worden

	\end{itemize}

\end{frame}
% ------------------------------------------------------------------------------
% LTXE-SLIDE-START
% LTXE-SLIDE-UID:		05cf8a69-f162fcfb-e5502986-a52948e4
% LTXE-SLIDE-ORIGIN:	c56cb2e8-f50d7eed-e93dcb9e-ab7be017 English
% LTXE-SLIDE-TITLE:		#73626: Number of search results is configurable now
% ------------------------------------------------------------------------------
\begin{frame}[fragile]
	\frametitle{TSconfig \& TypoScript}
	\framesubtitle{Aantal zoekresultaten}

	\begin{itemize}
		\item Het maximale aantal zoekresultaten kan in TypoScript ingesteld worden:\newline
			\texttt{plugin.tx\_indexedsearch.settings.blind.numberOfResults}
		\item Deze instelling slaat een lijst met waardes op
		\item Als het aantal resultaten wordt vermeld in het request en overeenkomt met een van
			de ingestelde waarden dan wordt het gebruikt
		\item Als het niet wordt meegestuurd of niet overeenkomt met de ingestelde waarden dan
		 	wordt de eerste van de lijst gebruikt
		\item Om compatibel te zijn met vorige versies is de standaard:\newline
			\texttt{10}, \texttt{25}, \texttt{50} and \texttt{100}
	\end{itemize}

\end{frame}
% ------------------------------------------------------------------------------
% LTXE-SLIDE-START
% LTXE-SLIDE-UID:		9a235253-f4ebf567-c9e4a64e-b4c78fcc
% LTXE-SLIDE-ORIGIN:	d9b9e73b-db575400-8276f4dd-810f4f4e English
% LTXE-SLIDE-TITLE:		#78672: Fluid data processor for menus (1)
% ------------------------------------------------------------------------------
\begin{frame}[fragile]
	\frametitle{TSconfig \& TypoScript}
	\framesubtitle{Fluid Data Processor voor menu's (1)}

	\begin{itemize}
		\item Menu processor gebruikt \texttt{HMENU} om de JSON-gecodeerde menu-tekenreeks te maken
			die gedecodeerd wordt en toegewezen aan \texttt{FLUIDTEMPLATE}
		\item Extra DataProcessing wordt ondersteund en op elk record toegepast
		\item Ondersteunde opties: \texttt{as}, \texttt{levels}, \texttt{expandAll}, \texttt{includeSpacer},
			\texttt{titleField}
			(zie \href{https://docs.typo3.org/typo3cms/TyposcriptReference/ContentObjects/Hmenu/Index.html}{TyposcriptReference} voor meer opties)
	\end{itemize}

\end{frame}

% ------------------------------------------------------------------------------
% LTXE-SLIDE-START
% LTXE-SLIDE-UID:		434c39b6-cad55c2d-f415d022-6acedee3
% LTXE-SLIDE-ORIGIN:	03e756a0-07740056-a2009898-d6159f4b English
% LTXE-SLIDE-TITLE:		#78672: Fluid data processor for menus (2)
% ------------------------------------------------------------------------------
\begin{frame}[fragile]
	\frametitle{TSconfig \& TypoScript}
	\framesubtitle{Fluid Data Processor voor menu's (2)}

	% decrease font size for code listing
	\lstset{basicstyle=\tiny\ttfamily}

	\begin{itemize}
		\item Voorbeeld TypoScript-configuratie:

			\begin{lstlisting}
				10 = TYPO3\CMS\Frontend\DataProcessing\MenuProcessor
				10 {
				  special = list
				  special.value.field = pages
				  levels = 7
				  as = menu
				  expandAll = 1
				  includeSpacer = 1
				  titleField = nav_title // title
				  dataProcessing {
				    10 = TYPO3\CMS\Frontend\DataProcessing\FilesProcessor
				    10 {
				      references.fieldName = media
				    }
				  }
				}
			\end{lstlisting}

	\end{itemize}

\end{frame}

% ------------------------------------------------------------------------------
% LTXE-SLIDE-START
% LTXE-SLIDE-UID:		23ed6f7c-876e1ec2-b140989e-273040de
% LTXE-SLIDE-ORIGIN:	4f6c99f3-cca1739f-205be2f4-3a0e3326 English
% LTXE-SLIDE-TITLE:		#72050: encapsLines does not render duplicate paragraphs anymore
% ------------------------------------------------------------------------------
\begin{frame}[fragile]
	\frametitle{TSconfig \& TypoScript}
	\framesubtitle{TypoScript functie \texttt{\_encapsLines}}

	\begin{itemize}
		\item TypoScript functie \texttt{\_encapsLines} maakte twee paragrafen voor elke lege regeleinde
		 	aan het eide van de content. Dit is gerepareerd.

		\item De wijziging beïnvloedt mogelijk de weergave in de frontend als meerdere lege regeleindes
			in de RTE-tekst zitten. De laatste paragraaf is niet langer dubbel gerenderd in de frontend
			sinds TYPO3 CMS versie 8.5.

	\end{itemize}

\end{frame}

% ------------------------------------------------------------------------------
