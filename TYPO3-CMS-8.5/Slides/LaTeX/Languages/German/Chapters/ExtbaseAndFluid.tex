% ------------------------------------------------------------------------------
% TYPO3 CMS 8.5 - What's New - Chapter "Extbase & Fluid" (German Version)
%
% @author	Michael Schams <schams.net>
% @license	Creative Commons BY-NC-SA 3.0
% @link		http://typo3.org/download/release-notes/whats-new/
% @language	English
% ------------------------------------------------------------------------------
% LTXE-CHAPTER-UID:		846d40ce-66fdc2ec-750dcf95-33ce93e0
% LTXE-CHAPTER-NAME:	Extbase & Fluid
% ------------------------------------------------------------------------------

\section{Extbase \& Fluid}
\begin{frame}[fragile]
	\frametitle{Extbase \& Fluid}

	\begin{center}\huge{Kapitel 4:}\end{center}
	\begin{center}\huge{\color{typo3darkgrey}\textbf{Extbase \& Fluid}}\end{center}

\end{frame}

% ------------------------------------------------------------------------------
% LTXE-SLIDE-START
% LTXE-SLIDE-UID:		506eeccc-87f8f148-a064b4e2-121d876f
% LTXE-SLIDE-ORIGIN:	c5986329-fb36f473-a5430d2e-6ba49fb1 English
% LTXE-SLIDE-TITLE:		#78116: Extbase support for Doctrine's native DBAL Statement and QueryBuilder
% ------------------------------------------------------------------------------
\begin{frame}[fragile]
	\frametitle{Extbase \& Fluid}
	\framesubtitle{Doctrine DBAL}

	% decrease font size for code listing
	\lstset{basicstyle=\tiny\ttfamily}

	\begin{itemize}
		\item Die Möglichkeit direkt SQL-Queries abzusetzen, unterstützt nun auch QueryBuilder-Objekte und Instanzen von 
			\texttt{\textbackslash
			Doctrine\textbackslash
			DBAL\textbackslash
			Statement} als "Prepared Statements"
		\item Das folgende Beispiel zeigt die Anwendung über eine Repository und mittels einem nativem Doctrine DBAL Statement:

			\begin{lstlisting}
				$connection = $this->objectManager->get(ConnectionPool::class)->getConnectionForTable('mytable');
				$statement = $this->objectManager->get(
				  \Doctrine\DBAL\Statement::class,
				  'SELECT * FROM mytable WHERE uid=? OR title=?',
				  $connection
				);

				$query = $this->createQuery();
				$query->statement($statement, [$uid, $title]);
			\end{lstlisting}
	\end{itemize}

\end{frame}
% ------------------------------------------------------------------------------
% LTXE-SLIDE-START
% LTXE-SLIDE-UID:		76b41536-ec795de9-6197748f-5d429230
% LTXE-SLIDE-ORIGIN:	8d31d436-4397d4fd-aa0eed2e-df798aa3 English
% LTXE-SLIDE-TITLE:		#78002: Enforce cHash argument for Extbase actions
% ------------------------------------------------------------------------------

\begin{frame}[fragile]
	\frametitle{Extbase \& Fluid}
	\framesubtitle{\texttt{cHash} Argument}

	\begin{itemize}
		\item URLS zu Extbase-Actions benötigen nun per default einen gültigen\texttt{cHash}\newline
			(für gechachte und ungecachte Actions)
		\item Dieses Verhalten kann über den Typoscript \texttt{feature} Switch abgeschaltet werden:
			\texttt{requireCHashArgumentForActionArguments}
	\end{itemize}

\end{frame}
% ------------------------------------------------------------------------------
% LTXE-SLIDE-START
% LTXE-SLIDE-UID:		80792903-9dd8a7ed-355eca78-479cea5e
% LTXE-SLIDE-ORIGIN:	50143a1e-53a17c8c-83ffe7ea-5e88cba0 English
% LTXE-SLIDE-TITLE:		#29399: OptionViewHelper and OptgroupViewHelper
% ------------------------------------------------------------------------------

\begin{frame}[fragile]
	\frametitle{Extbase \& Fluid}
	\framesubtitle{Inhalt für ViewHelper \texttt{f:form.select}}

	% decrease font size for code listing
	\lstset{basicstyle=\tiny\ttfamily}

	\begin{itemize}
		\item Es wurden zwei neue ViewHelper eingeführt, welche die manuelle Definition aller Optionen und Optionen-Gruppen für den \texttt{f:form.select} ViewHelper als Tag-Inhalt zulässt

			\begin{itemize}
				\item \texttt{OptionViewHelper}
				\item \texttt{OptgroupViewHelper}
			\end{itemize}

		\item Beispiel:

			\begin{lstlisting}
				<f:form.select name="myproperty">
				  <f:form.select.option value="1">Option one</f:form.select.option>
				  <f:form.select.option value="2">Option two</f:form.select.option>
				  <f:form.select.optgroup>
				    <f:form.select.option value="3">Grouped option one</f:form.select.option>
				    <f:form.select.option value="4">Grouped option twi</f:form.select.option>
				  </f:form.select.optgroup>
				</f:form.select>
			\end{lstlisting}

		\end{itemize}

\end{frame}


% ------------------------------------------------------------------------------
% LTXE-SLIDE-START
% LTXE-SLIDE-UID:		45ba8a94-0ddf9140-b871dc35-f0040ec2
% LTXE-SLIDE-ORIGIN:	d36b9935-fc7d9e60-742a3f42-ef15e856 English
% LTXE-SLIDE-TITLE:		#78415: Global Fluid ViewHelper Namespace
% ------------------------------------------------------------------------------
\begin{frame}[fragile]
	\frametitle{Extbase \& Fluid}
	\framesubtitle{Globaler Fluid ViewHelper Namespace}

	\begin{itemize}
		\item Der globale Fluid ViewHelper Namespace sind nun konfigurierbar:\newline
			\smaller
				\texttt{\$GLOBALS['TYPO3\_CONF\_VARS']['SYS']['fluid']['namespaces']}
			\normalsize
		\item Dies erlaubt die Manipulation der Namespaces auf der Ebene der Seitenkonfiguration
		\item Vorteile:

			\begin{itemize}
				\item Third-Party ViewHelper Pakete können nun den globalen Fluid Namespace \texttt{f:} manipulieren
				\item Third-Party ViewHelper Pakete können nun neue globale Namespaces registrieren
				\item Template-Entwickler können solche globale Namespace in allen Fluid Template (kontextunabhängig) verwenden, ohne diese vorher zu importieren.
			\end{itemize}

	\end{itemize}

\end{frame}


% ------------------------------------------------------------------------------
% LTXE-SLIDE-START
% LTXE-SLIDE-UID:		f3f97fb5-c9c96120-9b45756f-74d08679
% LTXE-SLIDE-ORIGIN:	e16e8cfd-26f35f09-23b473a2-442ff1e2 English
% LTXE-SLIDE-TITLE:		#78842: FLUIDTEMPLATE can mimic an actual Extbase web request
% ------------------------------------------------------------------------------
\begin{frame}[fragile]
	\frametitle{Extbase \& Fluid}
	\framesubtitle{\texttt{FLUIDTEMPLATE} kann Extbase Web Requests nachahmen}

	% decrease font size for code listing
	\lstset{basicstyle=\small\ttfamily}

	\begin{itemize}
		\item Das \texttt{FLUIDTEMPLATE} Content-Element kann nun einen Extbase Web Request nachahmen.
		\item Damit kann man z.B. auf übermittelte Daten zugreifen:

			\begin{lstlisting}
				$view->getRenderingContext()
				  ->getControllerContext()
				  ->getRequest()
				  ->getArguments();
			\end{lstlisting}

	\end{itemize}

\end{frame}




% ------------------------------------------------------------------------------
