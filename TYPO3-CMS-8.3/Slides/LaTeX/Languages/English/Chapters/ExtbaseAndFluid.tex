% ------------------------------------------------------------------------------
% TYPO3 CMS 8.3 - What's New - Chapter "Extbase & Fluid" (English Version)
%
% @author	Patrick Lobacher <patrick@lobacher.de> and Michael Schams <schams.net>
% @license	Creative Commons BY-NC-SA 3.0
% @link		http://typo3.org/download/release-notes/whats-new/
% @language	English
% ------------------------------------------------------------------------------
% LTXE-CHAPTER-UID:		846d40ce-66fdc2ec-750dcf95-33ce93e0
% LTXE-CHAPTER-NAME:	Extbase & Fluid
% ------------------------------------------------------------------------------

\section{Extbase \& Fluid}
\begin{frame}[fragile]
	\frametitle{Extbase \& Fluid}

	\begin{center}\huge{Chapter 4:}\end{center}
	\begin{center}\huge{\color{typo3darkgrey}\textbf{Extbase \& Fluid}}\end{center}

\end{frame}

% ------------------------------------------------------------------------------
% LTXE-SLIDE-START
% LTXE-SLIDE-UID:		06a9e027-1c0e09af-d41d1c4c-c0f07273
% LTXE-SLIDE-TITLE:		Feature: #76531 - Add IconForRecordViewHelper
% LTXE-SLIDE-REFERENCE:	Feature-76531-AddIconForRecordViewHelper.rst
% ------------------------------------------------------------------------------
\begin{frame}[fragile]
	\frametitle{Extbase \& Fluid}
	\framesubtitle{Add IconForRecordViewHelper}

	% decrease font size for code listin
	\lstset{basicstyle=\tiny\ttfamily}

	\begin{itemize}

		\item A new ViewHelper to render icons for records has been added

			\begin{lstlisting}
				<core:iconForRecord table="sys_template" row="{templateRecord}" ></core:iconForRecord>

				// output:
				<span class="t3js-icon icon icon-size-small icon-state-default icon-mimetypes-x-content-template"
				  data-identifier="mimetypes-x-content-template">
				  <span class="icon-markup">
				    <img src="/typo3/sysext/core/Resources/Public/Icons/T3Icons/mimetypes/mimetypes-x-content-template.svg" width="16" height="16">
				  </span>
				</span>
			\end{lstlisting}

	\end{itemize}

\end{frame}

% ------------------------------------------------------------------------------
% LTXE-SLIDE-START
% LTXE-SLIDE-UID:		c88b95e6-9808a38a-747272d5-19346080
% LTXE-SLIDE-TITLE:		Feature #76107: Add Fluid Interceptor Registration (1)
% LTXE-SLIDE-REFERENCE:	Feature-76107-AddFluidInterceptorRegistration.rst
% ------------------------------------------------------------------------------
\begin{frame}[fragile]
	\frametitle{Extbase \& Fluid}
	\framesubtitle{Add Fluid Interceptor Registration (1)}

	% decrease font size for code listin
	\lstset{basicstyle=\tiny\ttfamily}

	\begin{itemize}

		\item Interceptors in \textit{Fluid Standalone} were introduced to be able to change the template output

		\item The Fluid API already allows for registration of custom interceptors.
			Now it is possible to define custom interceptors by using the following option:

			\small
				\texttt{\$GLOBALS['TYPO3\_CONF\_VARS']['fluid']['interceptors']}
			\normalsize

		\item Interceptors registered here are added to the Fluid parser configuration

	\end{itemize}

\end{frame}

% ------------------------------------------------------------------------------
% LTXE-SLIDE-START
% LTXE-SLIDE-UID:		bec08e51-3b4dccca-f2060fd8-35d46d4b
% LTXE-SLIDE-TITLE:		Feature #76107: Add Fluid Interceptor Registration (2)
% LTXE-SLIDE-REFERENCE:	Feature-76107-AddFluidInterceptorRegistration.rst
% ------------------------------------------------------------------------------
\begin{frame}[fragile]
	\frametitle{Extbase \& Fluid}
	\framesubtitle{Add Fluid Interceptor Registration (2)}

	% decrease font size for code listin
	\lstset{basicstyle=\tiny\ttfamily}

	\begin{itemize}

		\item Register own interceptor to fluid parser configuration

			\begin{lstlisting}
				$GLOBALS['TYPO3_CONF_VARS']['SYS']['fluid']['interceptors']
			      [\TYPO3\CMS\Fluid\Core\Parser\Interceptor\DebugInterceptor::class] =
				  \TYPO3\CMS\Fluid\Core\Parser\Interceptor\DebugInterceptor::class;
			\end{lstlisting}

		\item Class code:

			\begin{lstlisting}
				use TYPO3Fluid\Fluid\Core\Parser\InterceptorInterface;
				use TYPO3Fluid\Fluid\Core\Parser\ParsingState;
				use TYPO3Fluid\Fluid\Core\Parser\SyntaxTree\NodeInterface;

				class DebugInterceptor implements InterceptorInterface
				{
				  public function process(NodeInterface $node, $interceptorPosition, ParsingState $parsingState)
				  {
				    return $node;
				  }

				  public function getInterceptionPoints()
				  {
				    return [];
				  }
				}
			\end{lstlisting}

	\end{itemize}

\end{frame}

% ------------------------------------------------------------------------------
