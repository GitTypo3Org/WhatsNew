% ------------------------------------------------------------------------------
% TYPO3 CMS 7.6 - What's New - Chapter "Extbase & Fluid" (German Version)
%
% @author	Patrick Lobacher <patrick@lobacher.de> and Michael Schams <schams.net>
% @license	Creative Commons BY-NC-SA 3.0
% @link		http://typo3.org/download/release-notes/whats-new/
% @language	German
% ------------------------------------------------------------------------------
% LTXE-CHAPTER-UID:		cfcf5377-f3e96f39-d1ba01e9-2f341646
% LTXE-CHAPTER-NAME:	Extbase & Fluid
% ------------------------------------------------------------------------------

\section{Extbase \& Fluid}
\begin{frame}[fragile]
	\frametitle{Extbase \& Fluid}

	\begin{center}\huge{Kapitel 4:}\end{center}
	\begin{center}\huge{\color{typo3darkgrey}\textbf{Extbase \& Fluid}}\end{center}

\end{frame}

% ------------------------------------------------------------------------------
% LTXE-SLIDE-START
% LTXE-SLIDE-UID:		78012a67-036705de-daefb092-e87adffa
% LTXE-SLIDE-TITLE:		Relations to the same table in Extbase
% LTXE-SLIDE-REFERENCE:	Feature-27057-RelationsToTheSameTableInExtbase.rst
% ------------------------------------------------------------------------------

\begin{frame}[fragile]
	\frametitle{Extbase \& Fluid}
	\framesubtitle{Relationen zu gleichen Tabellen}

	% decrease font size for code listing
	\lstset{basicstyle=\tiny\ttfamily}

	\begin{itemize}

		\item Es nun möglich ein Domain Model zu verwenden, in der ein Objekt eine
			Relation zu einem anderen Objekt der selben Klasse hat

			\begin{lstlisting}
				namespace \Vendor\Extension\Domain\Model;
				class A {
				  /**
				   * @var \Vendor\Extension\Domain\Model\A
				   */
				  protected $parent;
				}
			\end{lstlisting}

			\begin{lstlisting}
				namespace \Vendor\Extension\Domain\Model;
				class A {
				  /**
				   * @var \Vendor\Extension\Domain\Model\B
				   */
				  protected $x;

				  /**
				   * @var \Vendor\Extension\Domain\Model\B
				   */
				  protected $y;
				}
			\end{lstlisting}

	\end{itemize}

\end{frame}

% ------------------------------------------------------------------------------
% LTXE-SLIDE-START
% LTXE-SLIDE-UID:		0c279898-ab69247f-502eb397-6817d02b
% LTXE-SLIDE-TITLE:		Added absolute url option to uri.image and image viewHelper
% LTXE-SLIDE-REFERENCE:	Feature-64286-AddedAbsoluteUrlOptionToUriimageAndImageViewHelper.rst
% ------------------------------------------------------------------------------

\begin{frame}[fragile]
	\frametitle{Extbase \& Fluid}
	\framesubtitle{\texttt{absolute} Option für Image-ViewHelper}

	% decrease font size for code listing
	\lstset{basicstyle=\tiny\ttfamily}

	\begin{itemize}

		\item Der ImageViewhelper und Uri/ImageViewHelper haben nun eine Option \texttt{absolute},
			mit der eine absolute URL ausgegeben werden kann.

		\item Beispiel 1 (\texttt{ImageViewhelper}):

			\begin{lstlisting}
				<f:image image="{file}" width="400" height="375" absolute="1" ></f:image>

				// Ausgabe
				<img alt="alt set in image record"
				  src="http://www.mydomain.com/fileadmin/_processed_/323223424.png"
				  width="400" height="375" />
			\end{lstlisting}

		\item Beispiel 2 (\texttt{Uri/ImageViewHelper}):

			\begin{lstlisting}
				<f:uri.image image="{file}" width="400" height="375" absolute="1" ></f:uri>

				// Ausgabe
				http://www.mydomain.com/fileadmin/_processed_/323223424.png
			\end{lstlisting}

	\end{itemize}

\end{frame}

% ------------------------------------------------------------------------------
% LTXE-SLIDE-START
% LTXE-SLIDE-UID:		b880b0d8-83956e2a-0a594f16-7f4eb357
% LTXE-SLIDE-TITLE:		ViewHelper to strip whitespace between HTML tags
% LTXE-SLIDE-REFERENCE:	Feature-70170-ViewHelperToStripWhitespaceBetweenHTMLTags.rst
% ------------------------------------------------------------------------------

\begin{frame}[fragile]
	\frametitle{Extbase \& Fluid}
	\framesubtitle{ViewHelper um Whitespaces in HTML zu entfernen}

	\begin{itemize}

		\item Es wurde ein ViewHelper \texttt{spaceless} eingeführt, um überflüssige
			Leerzeichen zwischen HTML-Tages zu entfernen. Beispiel:

			\begin{lstlisting}
				<f:spaceless>
				<div>
				    <div>
				        <div>text

				text</div>
				</div>
				</div>
			\end{lstlisting}

		\item Ausgabe:

			\begin{lstlisting}
				<div><div><div>text

				text</div></div></div>
			\end{lstlisting}

	\end{itemize}

\end{frame}

% ------------------------------------------------------------------------------
% LTXE-SLIDE-START
% LTXE-SLIDE-UID:		84d3095f-75b5a68a-b9591a53-f34d8ecb
% LTXE-SLIDE-TITLE:		Respect rootLevel configuration in extbase queries
% LTXE-SLIDE-REFERENCE:	Breaking-63406-RespectRootlevelConfigurationinExtbaseQueries.rst
% ------------------------------------------------------------------------------

\begin{frame}[fragile]
	\frametitle{Extbase \& Fluid}
	\framesubtitle{RootLevel Konfiguration}

	% decrease font size for code listing
	\lstset{basicstyle=\small\ttfamily}

	\begin{itemize}

		\item Der RootLevel einer Tabelle kann nun im TCA konfiguriert werden\newline
			\small(damit wird festgelegt, wo die zugehörigen Datensätze gesucht werden)\normalsize

			\begin{itemize}
				\item \texttt{0}: Nur im Seitenbaum
				\item \texttt{1}: Nur auf der Root-Seite (PID 0)
				\item \texttt{-1}: In beiden
			\end{itemize}

		\item Zudem muss das TCA konfiguriert werden:\newline
			\tiny
				\texttt{\$GLOBALS['TCA']['tx\_myext\_domain\_model\_record']['ctrl']['rootLevel'] = -1;}
			\normalsize

	\end{itemize}

\end{frame}

% ------------------------------------------------------------------------------
