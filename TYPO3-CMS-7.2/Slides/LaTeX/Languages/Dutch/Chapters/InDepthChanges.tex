% ------------------------------------------------------------------------------
% TYPO3 CMS 7.2 - What's New - Chapter "In-Depth Changes" (English Version)
%
% @author	Michael Schams <schams.net>
% @license	Creative Commons BY-NC-SA 3.0
% @link		http://typo3.org/download/release-notes/whats-new/
% @language	English
% ------------------------------------------------------------------------------
% LTXE-CHAPTER-UID:		e28087f2-4bffcc7b-3c685cca-579f18a8
% LTXE-CHAPTER-NAME:	In-Depth Changes
% ------------------------------------------------------------------------------

\section{Wijzigingen nader bekeken}
\begin{frame}[fragile]
	\frametitle{Wijzingen nader bekeken}

	\begin{center}\huge{Chapter 3:}\end{center}
	\begin{center}\huge{\color{typo3darkgrey}\textbf{Wijzingen nader bekeken}}\end{center}

\end{frame}

% ------------------------------------------------------------------------------
% LTXE-SLIDE-START
% LTXE-SLIDE-UID:		f1b3e2e3-04b8a5d8-6e8a27ef-dba0728c
% LTXE-SLIDE-ORIGIN:	547708aa-92282fd1-6c27d618-4f13f86b English
% LTXE-SLIDE-TITLE:		Add SVG support
% LTXE-SLIDE-REFERENCE:	Feature-50136-AddSVGSupport.rst
% ------------------------------------------------------------------------------
\begin{frame}[fragile]
	\frametitle{Wijzingen nader bekeken}
	\framesubtitle{SVG-ondersteuning in core}

	\begin{itemize}
		\item TYPO3 CMS Core ondersteunt SVG-afbeeldingen ("Scalable Vector Graphics")

		\item Als een SVG-afbeelding wordt geschaald wordt een record met de nieuwe maten in
			\texttt{sys\_file\_processedfile} opgeslagen en geen nieuw bestand aangemaakt\newline
			\small(behalve als de afbeelding verder bewerkt wordt, bijv. afgesneden)\normalsize.

		\item Als ImageMagick/GraphicsMagick de maten niet kan vaststellen wordt teruggevallen
			op het uitlezen van de XML.

		\item SVG is ook toegevoegd aan de lijst met geldige afbeeldingsbestanden:\newline
			\texttt{\$GLOBALS['TYPO3\_CONF\_VARS']['GFX']['imagefile\_ext']}

	\end{itemize}

\end{frame}

% ------------------------------------------------------------------------------
% LTXE-SLIDE-START
% LTXE-SLIDE-UID:		f2cfbfa0-e61762ae-c9cd5761-d45a7699
% LTXE-SLIDE-ORIGIN:	bd03e141-3842b75e-8a44920f-a8c139e5 English
% LTXE-SLIDE-TITLE:		Add count methods and sort functionality to FAL drivers
% LTXE-SLIDE-REFERENCE:	Breaking-56746-AddCountMethodsAndSortFunctionalityToFalDrivers.rst
% ------------------------------------------------------------------------------
\begin{frame}[fragile]
	\frametitle{Wijzingen nader bekeken}
	\framesubtitle{Uitbreiding FAL-driver}

	% decrease font size for code listing
	\lstset{basicstyle=\tiny\ttfamily}

	\begin{itemize}
		\item Om de prestaties van de bestandslijst te verbeteren bij gebruik van opslag op
			niet-lokale systemen moet de FAL-driver zorgen voor sorteren en het vaststellen van
			het aantal bestanden/mappen. Twee nieuwe parameters \texttt{sort} en \texttt{sortRev} zijn
			toegevoegd hiervoor:

			\begin{lstlisting}
				public function getFilesInFolder($folderIdentifier, $start = 0, $numberOfItems = 0,
				  $recursive = FALSE, array $filenameFilterCallbacks = array(), $sort = '', $sortRev = FALSE);

				public function getFoldersInFolder($folderIdentifier, $start = 0, $numberOfItems = 0,
				  $recursive = FALSE, array $folderNameFilterCallbacks = array(), $sort = '', $sortRev = FALSE);
			\end{lstlisting}

		\item Daarnaast zijn er twee nieuwe functies:

			\begin{lstlisting}
				public function getFilesInFolderCount($folderIdentifier, $recursive = FALSE,
				  array $filenameFilterCallbacks = array());

				public function getFoldersInFolderCount($folderIdentifier, $recursive = FALSE,
				  array $folderNameFilterCallbacks = array());
			\end{lstlisting}

	\end{itemize}

\end{frame}

% ------------------------------------------------------------------------------
% LTXE-SLIDE-START
% LTXE-SLIDE-UID:		3b2f3305-587bd51a-e2b0d995-006e8d34
% LTXE-SLIDE-ORIGIN:	d0df6060-4b8d33a9-69adad59-422e9ada English
% LTXE-SLIDE-TITLE:		Introduce Backend Routing (1)
% LTXE-SLIDE-REFERENCE:	commit a08ce7238e583d1962edfe998e04c7d9c3d7c2ae
% ------------------------------------------------------------------------------
\begin{frame}[fragile]
	\frametitle{Wijzingen nader bekeken}
	\framesubtitle{API Backend-routering (1)}

	\begin{itemize}
		\item Een API voor backend routering is toegevoegd die de toegangspunten voor de backend
		 	afhandelt.
		\item De API is voor een groot deel compatible met de Symfony Routing Framework, waarop
			het is gebaseerd\newline
			\small(TYPO3 gebruikt momenteel ca. 20\%)\normalsize

		\item Drie klassen zorgen voor de functionaliteit:
			\begin{itemize}
				\item \texttt{class Route}:\tabto{3.6cm}bevat details over paden en opties
				\item \texttt{class Router}:\tabto{3.6cm}API om de route te detecteren
				\item \texttt{class UrlGenerator}:\tabto{3.6cm}bouwt de URL op
			\end{itemize}
	\end{itemize}

\end{frame}

% ------------------------------------------------------------------------------
% LTXE-SLIDE-START
% LTXE-SLIDE-UID:		d41fabf1-18f27fe4-9e302bcd-6fd9bd5e
% LTXE-SLIDE-ORIGIN:	fe1eec67-bbf31ed9-57b8e83a-3a922f0e English
% LTXE-SLIDE-TITLE:		Introduce Backend Routing (2)
% LTXE-SLIDE-REFERENCE:	commit a08ce7238e583d1962edfe998e04c7d9c3d7c2ae
% ------------------------------------------------------------------------------
\begin{frame}[fragile]
	\frametitle{Wijzingen nader bekeken}
	\framesubtitle{API Backend-routering (2)}

	\begin{itemize}

		\item Routes worden gedefinieerd in het volgende bestand of extensie:
			\texttt{Configuration/Backend/Routes.php}\newline
			(zie systeemextensie \texttt{backend} als voorbeeld)

		\item Verdere details over de API voor Backend-routering:\newline
			\small\url{http://wiki.typo3.org/Blueprints/BackendRouting}\normalsize

	\end{itemize}

\end{frame}

% ------------------------------------------------------------------------------
% LTXE-SLIDE-START
% LTXE-SLIDE-UID:		553e6585-5838262f-d7321b0f-d30fa659
% LTXE-SLIDE-ORIGIN:	9d56ca19-04a98896-ba0a973c-8a16e877 English
% LTXE-SLIDE-TITLE:		New system extension: mediace
% LTXE-SLIDE-REFERENCE:	Breaking-64719-MediaContentMovedToSystemExtension.rst
% LTXE-SLIDE-REFERENCE:	Breaking-65778-MediaWizardProviderMovedToSystemExtension.rst
% ------------------------------------------------------------------------------
\begin{frame}[fragile]
	\frametitle{Wijzingen nader bekeken}
	\framesubtitle{Nieuwe systeemextensie voor Media Contentelementen}

	\begin{itemize}

		\item Nieuwe systeemextensie "\texttt{mediace}" bevat de volgend cObjects:

			\begin{itemize}
				\item \texttt{MULTIMEDIA}
				\item \texttt{MEDIA}
				\item \texttt{SWFOBJECT}
				\item \texttt{FLOWPLAYER}
				\item \texttt{QTOBJECT}
			\end{itemize}

		\item Contentelementen \texttt{media} en \texttt{multimedia} zijn verplaatst naar
			deze systeemextensie, net zoals de "Media Wizard Provider"

		\item Deze extensie is \underline{\textbf{niet}} standaard geïnstalleerd!

	\end{itemize}

\end{frame}

% ------------------------------------------------------------------------------
% LTXE-SLIDE-START
% LTXE-SLIDE-UID:		78f456e2-53604120-fd67f276-e74a0006
% LTXE-SLIDE-ORIGIN:	cc63cd98-524ee6df-38430963-f7c7067c English
% LTXE-SLIDE-TITLE:		Composer vendor directory
% LTXE-SLIDE-REFERENCE:	Breaking-66001-ComposerVendorDirectoryChanged.rst
% ------------------------------------------------------------------------------
\begin{frame}[fragile]
	\frametitle{Wijzingen nader bekeken}
	\framesubtitle{Locatie van bibliotheken van derden}

	\begin{itemize}

		\item Bibliotheken van derden die door composer geïnstalleerd zijn staan onder \texttt{typo3/contrib/vendor}\newline
			\small
				(TYPO3 CMS < 7.2: in map \texttt{Packages/Libraries})
			\normalsize

		\item Hierdoor kan het inpakproces voor de publicatie van TYPO3 CMS als tar of zip een
		 	compleet werkende installatie samenstellen zonder \texttt{Packages/} voor bilbiotheken van
		 	derden mee te hoeven leveren

		\item Er kunnen problemen ontstaan bij installaties die via composer opgezet zijn en \texttt{phpunit}
			gebruiken tenzij composer afhankelijkheden opnieuw worden opgebouwd. Om dit op te lossen:

			\begin{lstlisting}
				# cd htdocs/
				# rm -rf typo3/contrib/vendor/ bin/ Packages/Libraries/ composer.lock
				# composer install
			\end{lstlisting}
	\end{itemize}

\end{frame}

% ------------------------------------------------------------------------------
% LTXE-SLIDE-START
% LTXE-SLIDE-UID:		2586fbbb-66ed49ef-661b5c7e-4b59a180
% LTXE-SLIDE-ORIGIN:	c9a5288c-f45aac1e-34606612-fdf6ec18 English
% LTXE-SLIDE-TITLE:		Introduce JavaScript notification API
% LTXE-SLIDE-REFERENCE:	Feature-66047-IntroduceJavascriptNotificationApi.rst
% ------------------------------------------------------------------------------
\begin{frame}[fragile]
	\frametitle{Wijzingen nader bekeken}
	\framesubtitle{JavaScript Notificaties}

	\begin{itemize}

		\item Een nieuwe JavaScript API voor Notificaties API is beschikbaar:
			\begin{lstlisting}
				// oud en deprecated:
				top.TYPO3.Flashmessages.display(TYPO3.Severity.notice)

				// nieuw en de enige juiste manier sinds TYPO3 CMS 7.2:
				top.TYPO3.Notification.notice(title, message)
    		\end{lstlisting}

		\item Overzicht API functies:\newline
			\small(parameter \texttt{duur} is optioneel en is standaard 5 seconde)\normalsize
			\begin{itemize}
				\item \normalsize\smaller\texttt{top.TYPO3.Notification.notice(titel, bericht, duur)}\normalsize
				\item \smaller\texttt{top.TYPO3.Notification.info(titel, bericht, duur)}\normalsize
				\item \smaller\texttt{top.TYPO3.Notification.success(titel, bericht, duur)}\normalsize
				\item \smaller\texttt{top.TYPO3.Notification.warning(titel, bericht, duur)}\normalsize
				\item \smaller\texttt{top.TYPO3.Notification.error(titel, bericht, duur)}\normalsize
			\end{itemize}

	\end{itemize}

\end{frame}

% ------------------------------------------------------------------------------
% LTXE-SLIDE-START
% LTXE-SLIDE-UID:		c6bc87fb-5916eb8b-8a32c834-73ef1f50
% LTXE-SLIDE-ORIGIN:	b2acb5df-60b9a7dd-35cf697d-2a50d5db English
% LTXE-SLIDE-TITLE:		System Information Dropdown (1)
% LTXE-SLIDE-REFERENCE:	Feature-65767-SystemInformationDropdown.rst
% ------------------------------------------------------------------------------
\begin{frame}[fragile]
	\frametitle{Wijzingen nader bekeken}
	\framesubtitle{Uitklapscherm systeeminformatie (1)}

	% decrease font size for code listing
	\lstset{basicstyle=\tiny\ttfamily}

	\begin{itemize}
		\item Extra systeeminformatie kan worden toegevoegd door een slot aan te maken

		\item Het slot moet geregistreerd worden in het bestand \texttt{ext\_localconf.php}:

			\begin{lstlisting}
				$signalSlotDispatcher = \TYPO3\CMS\Core\Utility\GeneralUtility::makeInstance(
				  \TYPO3\CMS\Extbase\SignalSlot\Dispatcher::class);

				$signalSlotDispatcher->connect(
				  \TYPO3\CMS\Backend\Backend\ToolbarItems\SystemInformationToolbarItem::class,
				  'getSystemInformation',
				  \Vendor\Extension\SystemInformation\Item::class,
				  'getItem'
				);
			\end{lstlisting}

	\end{itemize}

\end{frame}

% ------------------------------------------------------------------------------
% LTXE-SLIDE-START
% LTXE-SLIDE-UID:		04a71dff-ebc47109-409ebfea-e3d5cc9d
% LTXE-SLIDE-ORIGIN:	e0f13209-22e0ff8c-7ff23bf9-31b1bb52 English
% LTXE-SLIDE-TITLE:		System Information Dropdown (2)
% LTXE-SLIDE-REFERENCE:	Feature-65767-SystemInformationDropdown.rst
% ------------------------------------------------------------------------------
\begin{frame}[fragile]
	\frametitle{Wijzingen nader bekeken}
	\framesubtitle{Uitklapscherm systeeminformatie (2)}

	% decrease font size for code listing
	\lstset{basicstyle=\tiny\ttfamily}

	\begin{itemize}

		\item Extra systeeminformatie kan worden toegevoegd door een slot aan te maken

		\item Hiervoor is de klasse \texttt{Item} en de functie \texttt{getItem()} nodig in bestand
			\small
				\texttt{EXT:extension\textbackslash
					Classes\textbackslash
					SystemInformation\textbackslash
					Item.php}:
			\normalsize

		\begin{lstlisting}
			class Item {
			  public function getItem() {
			    return array(array(
			      'title' => 'Titel als muis erboven staat',
			      'value' => 'Beschrijving voor de lijstt',
			      'status' => SystemInformationHookInterface::STATUS_OK,
			      'count' => 4,
			      'icon' => \TYPO3\CMS\Backend\Utility\IconUtility::getSpriteIcon(
				    'extensions-example-information-icon')
			    ));
			  }
			}
		\end{lstlisting}

	\end{itemize}

\end{frame}

% ------------------------------------------------------------------------------
% LTXE-SLIDE-START
% LTXE-SLIDE-UID:		442e4de2-8f8842cb-af3f187c-a9ab49ef
% LTXE-SLIDE-ORIGIN:	7d27df49-260500ef-0fc78a89-6e80069b English
% LTXE-SLIDE-TITLE:		System Information Dropdown (3)
% LTXE-SLIDE-REFERENCE:	Feature-65767-SystemInformationDropdown.rst
% ------------------------------------------------------------------------------
\begin{frame}[fragile]
	\frametitle{Wijzingen nader bekeken}
	\framesubtitle{Uitklapscherm systeeminformatie (3)}

	% decrease font size for code listing
	\lstset{basicstyle=\tiny\ttfamily}

	\begin{itemize}
		\item Icoon \texttt{extensions-example-information-icon} moet geregistreerd worden in \texttt{ext\_localconf.php}:
		\begin{lstlisting}
			\TYPO3\CMS\Backend\Sprite\SpriteManager::addSingleIcons(
			  array(
			    'information-icon' => \TYPO3\CMS\Core\Utility\ExtensionManagementUtility::extRelPath(
			      $_EXTKEY) . 'Resources/Public/Images/Icons/information-icon.png'
			    ),
			   $_EXTKEY
			);
		\end{lstlisting}

	\end{itemize}

\end{frame}


% ------------------------------------------------------------------------------
% LTXE-SLIDE-START
% LTXE-SLIDE-UID:		58d6d077-f2b97f6e-0662bc45-8954f739
% LTXE-SLIDE-ORIGIN:	d7449c27-fd9676cf-d12be80b-c24a4536 English
% LTXE-SLIDE-TITLE:		System Information Dropdown (4)
% LTXE-SLIDE-REFERENCE:	Feature-65767-SystemInformationDropdown.rst
% ------------------------------------------------------------------------------
\begin{frame}[fragile]
	\frametitle{Wijzingen nader bekeken}
	\framesubtitle{Uitklapscherm systeeminformatie (4)}

	% decrease font size for code listing
	\lstset{basicstyle=\tiny\ttfamily}

	\begin{itemize}

		\item Berichten worden getoond onderaan het uitklapscherm

		\item Extensies kunnen een eigen slot hebben om de berichten te vullen:

		\begin{lstlisting}
			$signalSlotDispatcher = \TYPO3\CMS\Core\Utility\GeneralUtility::makeInstance(
			  \TYPO3\CMS\Extbase\SignalSlot\Dispatcher::class);

			$signalSlotDispatcher->connect(
			  \TYPO3\CMS\Backend\Backend\ToolbarItems\SystemInformationToolbarItem::class,
			  'loadMessages',
			  \Vendor\Extension\SystemInformation\Message::class,
			  'getMessage'
			);
		\end{lstlisting}

	\end{itemize}

\end{frame}

% ------------------------------------------------------------------------------
% LTXE-SLIDE-START
% LTXE-SLIDE-UID:		9c14895d-c87c92e2-b40044e1-a863beeb
% LTXE-SLIDE-ORIGIN:	ab3b46ce-c7d7a228-7f7888af-4312fcdd English
% LTXE-SLIDE-TITLE:		System Information Dropdown (5)
% LTXE-SLIDE-REFERENCE:	Feature-65767-SystemInformationDropdown.rst
% ------------------------------------------------------------------------------
\begin{frame}[fragile]
	\frametitle{Wijzingen nader bekeken}
	\framesubtitle{Uitklapscherm systeeminformatie (5)}

	% decrease font size for code listing
	\lstset{basicstyle=\tiny\ttfamily}

	\begin{itemize}

		\item Berichten worden getoond onderaan het uitklapscherm

		\item Dit vereist de klasse \texttt{Message} met de functie \texttt{getMessage()} in bestand
			\small
				\texttt{EXT:extension\textbackslash
					Classes\textbackslash
					SystemInformation\textbackslash
					Message.php}:
			\normalsize

		\begin{lstlisting}
			class Message {
			  public function getMessage() {
			    return array(array(
			      'status' => SystemInformationHookInterface::STATUS_OK,
			      'text' => 'Er ging iets fout. In de reports module is meer informatie.'
			    ));
			  }
			}
		\end{lstlisting}

	\end{itemize}

\end{frame}


% ------------------------------------------------------------------------------
% LTXE-SLIDE-START
% LTXE-SLIDE-UID:		440a48bf-8b7ade77-3253f946-1274610b
% LTXE-SLIDE-ORIGIN:	98e2056a-d165b638-b1a1ee08-206bc15a English
% LTXE-SLIDE-TITLE:		Add image cropping
% LTXE-SLIDE-REFERENCE:	Feature-65584-AddImageCropping.rst
% ------------------------------------------------------------------------------
\begin{frame}[fragile]
	\frametitle{Wijzingen nader bekeken}
	\framesubtitle{Configuratieopties voor het bewerken van afbeeldingen (1)}

	\begin{itemize}
		\item Nieuwe TypoScript configuratieopties:
			\begin{lstlisting}
				# afsnijden uitschakelen voor alle afbeeldingen
				tt_content.image.20.1.file.crop =

				# afsnijden overschrijven/instellen voor alle afbeeldingen
				# offsetX,offsetY,width,height
				tt_content.image.20.1.file.crop = 50,50,100,100
			\end{lstlisting}

		\item Fluid heeft ook ondersteuning voor afsnijdfuncties:
			\begin{lstlisting}
				# afsnijden uitschakelen voor alle afbeeldingen
				<f:image image="{imageObject}" crop="" ></f:image>

				# afsnijden overschrijven/instellen voor alle afbeeldingen
				# offsetX,offsetY,width,height
				<f:image image="{imageObject}" crop="50,50,100,100" ></f:image>
			\end{lstlisting}

	\end{itemize}

\end{frame}

% ------------------------------------------------------------------------------
% LTXE-SLIDE-START
% LTXE-SLIDE-UID:		2b09e34a-823cec13-2769eed7-a3414a65
% LTXE-SLIDE-ORIGIN:	e9cce085-6c300f1f-010de4ee-2c744704 English
% LTXE-SLIDE-TITLE:		Add image cropping (2)
% LTXE-SLIDE-REFERENCE:	Feature-65584-AddImageCropping.rst
% ------------------------------------------------------------------------------
\begin{frame}[fragile]
	\frametitle{Wijzingen nader bekeken}
	\framesubtitle{Configuratieopties voor het bewerken van afbeeldingen (2)}

	% decrease font size for code listing
	\lstset{basicstyle=\smaller\ttfamily}

	\begin{itemize}
		\item TCA kent ook de afsnijdfunctionaliteit:

			\begin{itemize}
				\item Column Type: \texttt{image\_manipulation}
				\item Config \texttt{file\_field}: string	\tabto{5.6cm}(default: \texttt{uid\_local})
				\item Config \texttt{enableZoom}: boolean	\tabto{5.6cm}(default: \texttt{FALSE})
				\item Config \texttt{allowedExtensions}: string\newline
					(default: \smaller\texttt{\$GLOBALS['TYPO3\_CONF\_VARS']['GFX']['imagefile\_ext']}\small)
				\item Config \texttt{ratios}: array, default:

					\begin{lstlisting}
						array(
						  '1.7777777777777777' => '16:9',
						  '1.3333333333333333' => '4:3',
						  '1' => '1:1',
						  'NaN' => 'Free'
						)
					\end{lstlisting}
			\end{itemize}

	\end{itemize}

\end{frame}

% ------------------------------------------------------------------------------
% LTXE-SLIDE-START
% LTXE-SLIDE-UID:		3cd2fc5c-4d7910b1-beede602-2a6d0201
% LTXE-SLIDE-ORIGIN:	ed863269-574e4061-d69f855e-6494c062 English
% LTXE-SLIDE-TITLE:		Additional params for HTMLparser userFunc
% LTXE-SLIDE-REFERENCE:	Feature-59712-HtmlParserAdditionalUserFuncParams.rst
% ------------------------------------------------------------------------------
\begin{frame}[fragile]
	\frametitle{Wijzingen nader bekeken}
	\framesubtitle{Extra parameters voor HTMLparser userFunc}

	% decrease font size for code listing
	\lstset{basicstyle=\tiny\ttfamily}

	\begin{itemize}

		\item Extra parameters kunnen doorgegeven worden aan een userFunc van de HTMLparser:

			\begin{lstlisting}
				myobj = TEXT
				myobj.value = <a href="/" class="myclass">MyText</a>
				myobj.HTMLparser.tags.a.fixAttrib.class {
				  userFunc = Tx\MyExt\Myclass->htmlUserFunc
				  userFunc.myparam = test
				}
			\end{lstlisting}

		\item In een extensie zjin deze parameters te bereiken via:

			\begin{lstlisting}
				function htmlUserFunc(array $params, HtmlParser $htmlParser) {
				  // $params['attributeValue'] bevat de waarde van het attribuut "myclass"
				  // $params['myparam'] bevat "test" in dit voorbeeld
				  ...
				}
			\end{lstlisting}

	\end{itemize}

\end{frame}

% ------------------------------------------------------------------------------
% LTXE-SLIDE-START
% LTXE-SLIDE-UID:		b171eef6-cf981466-24749657-83e4f494
% LTXE-SLIDE-ORIGIN:	b04f8d44-fd695c73-b768f3cb-fd6d8643 English
% LTXE-SLIDE-TITLE:		New Locking API (1)
% LTXE-SLIDE-REFERENCE:	Feature-47712-NewLockingAPI.rst
% ------------------------------------------------------------------------------
\begin{frame}[fragile]
	\frametitle{Wijzingen nader bekeken}
	\framesubtitle{API voor vergrendelingen (1)}

	\begin{itemize}

		\item Nieuwe API voor vergrendelingen is gemaakt die diverse blokkeringsmethodes bevat (SimpleFile, Semaphore, ...)
		\item Een blokkeringsmethode moet \small\texttt{LockingStrategyInterface}\normalsize implementeren:
		\begin{lstlisting}
			$lockFactory = GeneralUtility::makeInstance(LockFactory::class);
			$locker = $lockFactory->createLocker('someId');
			$locker->acquire() || die('Geen vergrendeling mogelijk.');
			...
			$locker->release();
		\end{lstlisting}

	\end{itemize}

\end{frame}

% ------------------------------------------------------------------------------
% LTXE-SLIDE-START
% LTXE-SLIDE-UID:		95e55274-bb1b4d77-3c79daec-2dea71db
% LTXE-SLIDE-ORIGIN:	79a00412-c72401e6-4d9667c9-0e15d3f4 English
% LTXE-SLIDE-TITLE:		New Locking API (2)
% LTXE-SLIDE-REFERENCE:	Feature-47712-NewLockingAPI.rst
% ------------------------------------------------------------------------------
\begin{frame}[fragile]
	\frametitle{Wijzingen nader bekeken}
	\framesubtitle{API voor vergrendelingen (2)}

	% decrease font size for code listing
	\lstset{basicstyle=\tiny\ttfamily}

	\begin{itemize}

		\item Sommige functies ondersteunen ook niet-blokkerende vergrendelingen:

		\begin{lstlisting}
			$lockFactory = GeneralUtility::makeInstance(LockFactory::class);
			$locker = $lockFactory->createLocker(
			  'someId',
			  LockingStrategyInterface::LOCK_CAPABILITY_SHARED |
			    LockingStrategyInterface::LOCK_CAPABILITY_NOBLOCK
			);
			try {
			  $result = $locker->acquire(LockingStrategyInterface::LOCK_CAPABILITY_SHARED | LockingStrategyInterface::LOCK_CAPABILITY_NOBLOCK);
			  catch (\RuntimeException $e) {
			  if ($e->getCode() === 1428700748) {
			    // een proces bezit de vergrendeling
			    // laten we ondertussen iets anders doen
			    ...
			  }
			}
			if ($result) {
			  $locker->release();
			}
		\end{lstlisting}

	\end{itemize}

\end{frame}

% ------------------------------------------------------------------------------
% LTXE-SLIDE-START
% LTXE-SLIDE-UID:		67c84cf6-99b7404d-db7f1c6d-b2dc89de
% LTXE-SLIDE-ORIGIN:	108cb4a7-6e006507-c25a91e7-3a867f09 English
% LTXE-SLIDE-TITLE:		Add signal after extension installation
% LTXE-SLIDE-REFERENCE:	commit 39e0fd0a2c919da270cb49223433bcf95febbb10
% ------------------------------------------------------------------------------
\begin{frame}[fragile]
	\frametitle{Wijzingen nader bekeken}
	\framesubtitle{Signaal installeren van extensie}

	% decrease font size for code listing
	\lstset{basicstyle=\tiny\ttfamily}

	\begin{itemize}

		\item Een nieuw signaal is beschikbaar in methode
			\smaller
				\texttt{\textbackslash
					TYPO3\textbackslash
					CMS\textbackslash
					Extensionmanager\textbackslash
					Utility\textbackslash
					InstallUtility::install()}
			\normalsize
			die wordt verstuurd zodra een extensie is geïnstalleer en alle imports/updates
			zijn afgerond

		\begin{lstlisting}
			// execution
			$this->emitAfterExtensionInstallSignal($extensionKey);

			// methode
			protected function emitAfterExtensionInstallSignal($extensionKey) {
			  $this->signalSlotDispatcher->dispatch(
			    __CLASS__,
			    'afterExtensionInstall',
			    array($extensionKey, $this)
			  );
			}
		\end{lstlisting}

	\end{itemize}

\end{frame}

% ------------------------------------------------------------------------------
% LTXE-SLIDE-START
% LTXE-SLIDE-UID:		2ca0a1b8-8e827603-573be36f-5a9f3a33
% LTXE-SLIDE-ORIGIN:	acb1ba95-1a401bef-549072a1-f32263a9 English
% LTXE-SLIDE-TITLE:		Registry for adding text extractor classes (1)
% LTXE-SLIDE-REFERENCE:	Feature-36743-FAL-TextExtractorRegistry.rst
% ------------------------------------------------------------------------------
\begin{frame}[fragile]
	\frametitle{Wijzingen nader bekeken}
	\framesubtitle{Registry voor uitlezen van tekst (1)}

	% decrease font size for code listing
	\lstset{basicstyle=\tiny\ttfamily}

	\begin{itemize}

		\item Meerdere tekstlezers kunnen geregistreerd worden om diverse bestandstypes (bijv. Office, PDF-bestanden, etc.)
		af te handelen

		\item TYPO3 core heeft een lezer voor gewone tekstbestanden

		\item Elke geregistreerde tekstlezer moet \texttt{TextExtractorInterface} implementeren

		\item ...en de volgende functies:\newline
			\texttt{canExtractText()}\newline
			\small
				bepaalt of het uitlezen van het bestand mogelijk is
			\normalsize
			\newline
			\texttt{extractText()}\newline
			\small
				geeft de inhoud van het bestand terug als tekst
			\normalsize

	\end{itemize}

\end{frame}

% ------------------------------------------------------------------------------
% LTXE-SLIDE-START
% LTXE-SLIDE-UID:		f2dc3e17-545f0899-ebef3b75-ab8598d5
% LTXE-SLIDE-ORIGIN:	acb1ba95-1a401bef-549072a1-f32263a9 English
% LTXE-SLIDE-TITLE:		Registry for adding text extractor classes (2)
% LTXE-SLIDE-REFERENCE:	Feature-36743-FAL-TextExtractorRegistry.rst
% ------------------------------------------------------------------------------
\begin{frame}[fragile]
	\frametitle{Wijzingen nader bekeken}
	\framesubtitle{Registry voor uitlezen van tekst (2)}

	% decrease font size for code listing
	\lstset{basicstyle=\tiny\ttfamily}

	\begin{itemize}

		\item Registratie tekstuitlezer in bestand \texttt{ext\_localconf.php}:

			\begin{lstlisting}
				$textExtractorRegistry = \TYPO3\CMS\Core\Resource\TextExtraction\TextExtractorRegistry::getInstance();
				$textExtractorRegistry->registerTextExtractor(
				  \TYPO3\CMS\Core\Resource\TextExtraction\PlainTextExtractor::class
				);
			\end{lstlisting}

		\item Gebruik als volgt:

			\begin{lstlisting}
				$textExtractorRegistry = \TYPO3\CMS\Core\Resource\TextExtraction\TextExtractorRegistry::getInstance();
				$extractor = $textExtractorRegistry->getTextExtractor($file);
				if($extractor !== NULL) {
				  $content = $extractor->extractText($file);
				}
			\end{lstlisting}
	\end{itemize}

\end{frame}

% ------------------------------------------------------------------------------
% LTXE-SLIDE-START
% LTXE-SLIDE-UID:		d1788b50-eed624d7-4f5ad368-4bea8a45
% LTXE-SLIDE-ORIGIN:	d9657b5d-5cd56c65-d7266a27-401dd233 English
% LTXE-SLIDE-TITLE:		Miscellaneous
% LTXE-SLIDE-REFERENCE:	Feature-66042-WebLibrariesLoadedViaBower.rst
% LTXE-SLIDE-REFERENCE:	Feature-63703-AddOptionToStopTask.rst
% LTXE-SLIDE-REFERENCE:	Feature-61463-AllowProcessedFoldersInDifferentStorage.rst
% LTXE-SLIDE-REFERENCE:	Feature-52693-TSFE-RequestedId.rst
% ------------------------------------------------------------------------------

\begin{frame}[fragile]
	\frametitle{Wijzingen nader bekeken}
	\framesubtitle{Allerlei}

	\begin{itemize}

		\item Web-bibliotheken (zoals Twitter Bootstrap, jQuery, Font Awesome, etc.) gebruiken
			"Bower" (\url{http://bower.io}) en zitten niet meer in de TYPO3 core Git repository
			\newline
			\small
				\texttt{\# bower install}	\tabto{3.4cm}voert installatie uit\newline
				\texttt{\# bower update}		\tabto{3.4cm}voert update uit\newline
			\normalsize
			(bestand \texttt{bower.json} zit in map \texttt{Build/})

		\item Taakplanner CLI heeft optie "\texttt{-s}" om een lopende taak te stoppen

		\item De map voor het bewerken van een niet-lokale opslag kan zich buiten de opslag bevinden
			(handig voor een alleen-legen opslag)

		\item De pagina ID van de oorspronkelijk opgevraagde pagina is uit te lezen:
			\texttt{\$TSFE->getRequestedId()}

	\end{itemize}

\end{frame}

% ------------------------------------------------------------------------------
