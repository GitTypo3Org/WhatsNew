% ------------------------------------------------------------------------------
% TYPO3 CMS 7.2 - What's New - Chapter "In-Depth Changes" (German Version)
%
% @author	Patrick Lobacher <patrick@lobacher.de> and Michael Schams <schams.net>
% @license	Creative Commons BY-NC-SA 3.0
% @link		http://typo3.org/download/release-notes/whats-new/
% @language	German
% ------------------------------------------------------------------------------
% LTXE-CHAPTER-UID:		e28087f2-4bffcc7b-3c685cca-579f18a8
% LTXE-CHAPTER-NAME:	In-Depth Changes
% ------------------------------------------------------------------------------

\section{Änderungen im System}
\begin{frame}[fragile]
	\frametitle{Änderungen im System}

	\begin{center}\huge{Kapitel 3:}\end{center}
	\begin{center}\huge{\color{typo3darkgrey}\textbf{Änderungen im System}}\end{center}

\end{frame}

% ------------------------------------------------------------------------------
% LTXE-SLIDE-START
% LTXE-SLIDE-UID:		547708aa-92282fd1-6c27d618-4f13f86b
% LTXE-SLIDE-TITLE:		Add SVG support
% LTXE-SLIDE-REFERENCE:	Feature-50136-AddSVGSupport.rst
% ------------------------------------------------------------------------------
\begin{frame}[fragile]
	\frametitle{Änderungen im System}
	\framesubtitle{SVG Support im Core}

	\begin{itemize}
		\item Der Core unterstützt nun SVG-Bilder ("Scalable Vector Graphics")

		\item Wenn ein SVG-Bild skaliert wird, wird kein prozessiertes Bild abgelegt,
			dafür aber die neuen Größenangaben in einem Datensatz \texttt{sys\_file\_processedfile}
			gespeichert\newline
			\small(außer, das Bild wird - z.B. durch Cropping - weiterverarbeitet)\normalsize.

		\item Zudem wurde ein zusätzlicher Fallback eingebaut, falls ImageMagick/GraphicsMagick nicht
			in der Lage sein sollte, die Dimensionen zu berechnen: in diesem Fall wird das XML ausgelesen.

		\item SVG wurde außerdem zur Liste der zulässigen Bildtypen hinzugefügt:\newline
			\texttt{\$GLOBALS['TYPO3\_CONF\_VARS']['GFX']['imagefile\_ext']}

	\end{itemize}

\end{frame}

% ------------------------------------------------------------------------------
% LTXE-SLIDE-START
% LTXE-SLIDE-UID:		bd03e141-3842b75e-8a44920f-a8c139e5
% LTXE-SLIDE-TITLE:		Add count methods and sort functionality to FAL drivers
% LTXE-SLIDE-REFERENCE:	Breaking-56746-AddCountMethodsAndSortFunctionalityToFalDrivers.rst
% ------------------------------------------------------------------------------
\begin{frame}[fragile]
	\frametitle{Änderungen im System}
	\framesubtitle{Erweiterung der FAL-Treiber}

	% decrease font size for code listing
	\lstset{basicstyle=\tiny\ttfamily}

	\begin{itemize}
		\item Um die Performance der Dateiliste bei (Remote-)Storages innerhalb von FAL zu erhöhen,
			ist es notwendig, die Sortierung und das Ermitteln der Anzahl direkt im Treiber zu
			erledigen. Dafür wurden zwei neue Parameter \texttt{sort} und \texttt{sortRev} eingebracht:

			\begin{lstlisting}
				public function getFilesInFolder($folderIdentifier, $start = 0, $numberOfItems = 0,
				  $recursive = FALSE, array $filenameFilterCallbacks = array(), $sort = '', $sortRev = FALSE);

				public function getFoldersInFolder($folderIdentifier, $start = 0, $numberOfItems = 0,
				  $recursive = FALSE, array $folderNameFilterCallbacks = array(), $sort = '', $sortRev = FALSE);
			\end{lstlisting}

		\item Außerdem wurden zwei neue Methoden eingeführt:

			\begin{lstlisting}
				public function getFilesInFolderCount($folderIdentifier, $recursive = FALSE,
				  array $filenameFilterCallbacks = array());

				public function getFoldersInFolderCount($folderIdentifier, $recursive = FALSE,
				  array $folderNameFilterCallbacks = array());
			\end{lstlisting}

	\end{itemize}

\end{frame}

% ------------------------------------------------------------------------------
% LTXE-SLIDE-START
% LTXE-SLIDE-UID:		d0df6060-4b8d33a9-69adad59-422e9ada
% LTXE-SLIDE-TITLE:		Introduce Backend Routing (1)
% LTXE-SLIDE-REFERENCE:	commit a08ce7238e583d1962edfe998e04c7d9c3d7c2ae
% ------------------------------------------------------------------------------
\begin{frame}[fragile]
	\frametitle{Änderungen im System}
	\framesubtitle{Backend Routing API (1)}

	\begin{itemize}
		\item Der Core enthält nun eine Backend Routing API, welche die Entry Points ins Backend verwaltet
		\item Die Routing API wurde vom Symfony Routing Framework inspiriert und ist weitgehend kompatibel mit dieser\newline
			\small(auch wenn für TYPO3 zur Zeit nur ca. 20\% genutzt werden)\normalsize

		\item Grundsätzlich existieren hierfür drei Klassen:
			\begin{itemize}
				\item \texttt{class Route}:\tabto{3.6cm}enthält Angaben zum Pfad und Optionen
				\item \texttt{class Router}:\tabto{3.6cm}API, um die Route zu matchen
				\item \texttt{class UrlGenerator}:\tabto{3.6cm}generiert die URL
			\end{itemize}
	\end{itemize}

\end{frame}

% ------------------------------------------------------------------------------
% LTXE-SLIDE-START
% LTXE-SLIDE-UID:		fe1eec67-bbf31ed9-57b8e83a-3a922f0e
% LTXE-SLIDE-TITLE:		Introduce Backend Routing (2)
% LTXE-SLIDE-REFERENCE:	commit a08ce7238e583d1962edfe998e04c7d9c3d7c2ae
% ------------------------------------------------------------------------------
\begin{frame}[fragile]
	\frametitle{Änderungen im System}
	\framesubtitle{Backend Routing API (2)}

	\begin{itemize}

		\item Routen werden dabei in folgender Datei in der entsprechenden Extension definiert:
			\texttt{Configuration/Backend/Routes.php}\newline
			\small(siehe Systemextension \texttt{backend} als Beispiel)\normalsize

		\item Weitere Informationen zur Backend Routing API:\newline
			\small\url{http://wiki.typo3.org/Blueprints/BackendRouting}\normalsize

	\end{itemize}

\end{frame}

% ------------------------------------------------------------------------------
% LTXE-SLIDE-START
% LTXE-SLIDE-UID:		9d56ca19-04a98896-ba0a973c-8a16e877
% LTXE-SLIDE-TITLE:		New system extension: mediace
% LTXE-SLIDE-REFERENCE:	Breaking-64719-MediaContentMovedToSystemExtension.rst
% LTXE-SLIDE-REFERENCE:	Breaking-65778-MediaWizardProviderMovedToSystemExtension.rst
% ------------------------------------------------------------------------------
\begin{frame}[fragile]
	\frametitle{Änderungen im System}
	\framesubtitle{Neue Systemextension für Media Inhaltselemente}

	\begin{itemize}

		\item Neue Systemextension "\texttt{mediace}" enthält folgende cObjects:

			\begin{itemize}
				\item \texttt{MULTIMEDIA}
				\item \texttt{MEDIA}
				\item \texttt{SWFOBJECT}
				\item \texttt{FLOWPLAYER}
				\item \texttt{QTOBJECT}
			\end{itemize}

		\item Die Inhaltselemente \texttt{media} und \texttt{multimedia} wurden ebenfalls in
			die Systemextension verschoben, ebenso der "Media Wizard Provider"

		\item Die Extension ist standardmäßig \underline{\textbf{nicht}} installiert!

	\end{itemize}

\end{frame}

% ------------------------------------------------------------------------------
% LTXE-SLIDE-START
% LTXE-SLIDE-UID:		cc63cd98-524ee6df-38430963-f7c7067c
% LTXE-SLIDE-TITLE:		Composer vendor directory
% LTXE-SLIDE-REFERENCE:	Breaking-66001-ComposerVendorDirectoryChanged.rst
% ------------------------------------------------------------------------------
\begin{frame}[fragile]
	\frametitle{Änderungen im System}
	\framesubtitle{Fremd-Bibliotheken an neuer Stelle}

	\begin{itemize}
		\item Sämtliche Fremd-Bibliotheken werden nicht mehr in \texttt{Packages/Libraries},
			sondern in \texttt{typo3/contrib/vendor} abgelegt

		\item Grundsätzlich ist dafür die Installation der Bibliotheken mittels \texttt{composer install} notwendig

		\item Probleme gibt es beim Upgrade einer Installation, wenn dort \texttt{phpunit} verwendet wurde!
			Dies kann wie folgt behoben werden:
			\begin{lstlisting}
				# cd htdocs/
				# rm -rf typo3/contrib/vendor/ bin/ Packages/Libraries/ composer.lock
				# composer install
			\end{lstlisting}
	\end{itemize}

\end{frame}

% ------------------------------------------------------------------------------
% LTXE-SLIDE-START
% LTXE-SLIDE-UID:		c9a5288c-f45aac1e-34606612-fdf6ec18
% LTXE-SLIDE-TITLE:		Introduce JavaScript notification API
% LTXE-SLIDE-REFERENCE:	Feature-66047-IntroduceJavascriptNotificationApi.rst
% ------------------------------------------------------------------------------
\begin{frame}[fragile]
	\frametitle{Änderungen im System}
	\framesubtitle{API für JavaScript Notifikationen}

%	% decrease font size for code listing
%	\lstset{basicstyle=\smaller\ttfamily}

	\begin{itemize}

		\item Neue API, um JavaScript Notifikationen zu erzeugen:
			\begin{lstlisting}
				// Bisheriger (veralteter) Weg:
				top.TYPO3.Flashmessages.display(TYPO3.Severity.notice)

				// Neuer Weg:
				top.TYPO3.Notification.notice(title, message)
    		\end{lstlisting}

		\item Es existieren folgende API-Funktionen:\newline
			\small(Parameter \texttt{duration} ist optional und standardmäßig auf 5s eingestellt)\normalsize
			\begin{itemize}
				\item \normalsize\smaller\texttt{top.TYPO3.Notification.notice(title, message, duration)}\normalsize
				\item \smaller\texttt{top.TYPO3.Notification.info(title, message, duration)}\normalsize
				\item \smaller\texttt{top.TYPO3.Notification.success(title, message, duration)}\normalsize
				\item \smaller\texttt{top.TYPO3.Notification.warning(title, message, duration)}\normalsize
				\item \smaller\texttt{top.TYPO3.Notification.error(title, message, duration)}\normalsize
			\end{itemize}

	\end{itemize}

\end{frame}

% ------------------------------------------------------------------------------
% LTXE-SLIDE-START
% LTXE-SLIDE-UID:		b2acb5df-60b9a7dd-35cf697d-2a50d5db
% LTXE-SLIDE-TITLE:		System Information Dropdown (1)
% LTXE-SLIDE-REFERENCE:	Feature-65767-SystemInformationDropdown.rst
% ------------------------------------------------------------------------------
\begin{frame}[fragile]
	\frametitle{Änderungen im System}
	\framesubtitle{Systeminformationen (1)}

	% decrease font size for code listing
	\lstset{basicstyle=\tiny\ttfamily}

	\begin{itemize}
		\item Das Dropdown mit Systeminformationen kann über folgenden Slot erweitert werden:

			\begin{lstlisting}
				$signalSlotDispatcher = \TYPO3\CMS\Core\Utility\GeneralUtility::makeInstance(
				  \TYPO3\CMS\Extbase\SignalSlot\Dispatcher::class);

				$signalSlotDispatcher->connect(
				  \TYPO3\CMS\Backend\Backend\ToolbarItems\SystemInformationToolbarItem::class,
				  'getSystemInformation',
				  \Vendor\Extension\SystemInformation\Item::class,
				  'getItem'
				);
			\end{lstlisting}

	\end{itemize}

\end{frame}

% ------------------------------------------------------------------------------
% LTXE-SLIDE-START
% LTXE-SLIDE-UID:		e0f13209-22e0ff8c-7ff23bf9-31b1bb52
% LTXE-SLIDE-TITLE:		System Information Dropdown (2)
% LTXE-SLIDE-REFERENCE:	Feature-65767-SystemInformationDropdown.rst
% ------------------------------------------------------------------------------
\begin{frame}[fragile]
	\frametitle{Änderungen im System}
	\framesubtitle{Systeminformationen (2)}

	% decrease font size for code listing
	\lstset{basicstyle=\tiny\ttfamily}

	\begin{itemize}
		\item Zur Ansprache benötigt man die Klasse \texttt{Item} und dazugehörig die Methode
			\texttt{getItem()} innerhalb einer Extension
			\small
				\texttt{EXT:extension\textbackslash
					Classes\textbackslash
					SystemInformation\textbackslash
					Item.php}:
			\normalsize

		\begin{lstlisting}
			class Item {
			  public function getItem() {
			    return array(array(
			      'title' => 'The title shown on hover',
			      'value' => 'Description shown in the list',
			      'status' => SystemInformationHookInterface::STATUS_OK,
			      'count' => 4,
			      'icon' => \TYPO3\CMS\Backend\Utility\IconUtility::getSpriteIcon(
				    'extensions-example-information-icon')
			    ));
			  }
			}
		\end{lstlisting}

	\end{itemize}

\end{frame}

% ------------------------------------------------------------------------------
% LTXE-SLIDE-START
% LTXE-SLIDE-UID:		7d27df49-260500ef-0fc78a89-6e80069b
% LTXE-SLIDE-TITLE:		System Information Dropdown (3)
% LTXE-SLIDE-REFERENCE:	Feature-65767-SystemInformationDropdown.rst
% ------------------------------------------------------------------------------
\begin{frame}[fragile]
	\frametitle{Änderungen im System}
	\framesubtitle{Systeminformationen (3)}

	% decrease font size for code listing
	\lstset{basicstyle=\tiny\ttfamily}

	\begin{itemize}
		\item Das Icon \texttt{extensions-example-information-icon} wird in der Datei \texttt{ext\_localconf.php} registriert:
		\begin{lstlisting}
			\TYPO3\CMS\Backend\Sprite\SpriteManager::addSingleIcons(
			  array(
			    'information-icon' => \TYPO3\CMS\Core\Utility\ExtensionManagementUtility::extRelPath(
			      $_EXTKEY) . 'Resources/Public/Images/Icons/information-icon.png'
			    ),
			   $_EXTKEY
			);
		\end{lstlisting}

	\end{itemize}

\end{frame}


% ------------------------------------------------------------------------------
% LTXE-SLIDE-START
% LTXE-SLIDE-UID:		d7449c27-fd9676cf-d12be80b-c24a4536
% LTXE-SLIDE-TITLE:		System Information Dropdown (4)
% LTXE-SLIDE-REFERENCE:	Feature-65767-SystemInformationDropdown.rst
% ------------------------------------------------------------------------------
\begin{frame}[fragile]
	\frametitle{Änderungen im System}
	\framesubtitle{Systeminformationen (4)}

	% decrease font size for code listing
	\lstset{basicstyle=\tiny\ttfamily}

	\begin{itemize}
		\item Nachrichten werden am unteren Ende des Dropdowns angezeigt
		\item Über den folgenden Slot können eigene Nachrichten eingebracht werden:

		\begin{lstlisting}
			$signalSlotDispatcher = \TYPO3\CMS\Core\Utility\GeneralUtility::makeInstance(
			  \TYPO3\CMS\Extbase\SignalSlot\Dispatcher::class);

			$signalSlotDispatcher->connect(
			  \TYPO3\CMS\Backend\Backend\ToolbarItems\SystemInformationToolbarItem::class,
			  'loadMessages',
			  \Vendor\Extension\SystemInformation\Message::class,
			  'getMessage'
			);
		\end{lstlisting}

	\end{itemize}

\end{frame}

% ------------------------------------------------------------------------------
% LTXE-SLIDE-START
% LTXE-SLIDE-UID:		ab3b46ce-c7d7a228-7f7888af-4312fcdd
% LTXE-SLIDE-TITLE:		System Information Dropdown (5)
% LTXE-SLIDE-REFERENCE:	Feature-65767-SystemInformationDropdown.rst
% ------------------------------------------------------------------------------
\begin{frame}[fragile]
	\frametitle{Änderungen im System}
	\framesubtitle{Systeminformationen (5)}

	% decrease font size for code listing
	\lstset{basicstyle=\tiny\ttfamily}

	\begin{itemize}
		\item Zur Ansprache benötigt man die Klasse \texttt{Message} und dazugehörig die Methode
			\texttt{getMessage()} innerhalb einer Extension
			\small
				\texttt{EXT:extension\textbackslash
					Classes\textbackslash
					SystemInformation\textbackslash
					Message.php}:
			\normalsize

		\begin{lstlisting}
			class Message {
			  public function getMessage() {
			    return array(array(
			      'status' => SystemInformationHookInterface::STATUS_OK,
			      'text' => 'Something went wrong. Take a look at the reports module.'
			    ));
			  }
			}
		\end{lstlisting}

	\end{itemize}

\end{frame}


% ------------------------------------------------------------------------------
% LTXE-SLIDE-START
% LTXE-SLIDE-UID:		98e2056a-d165b638-b1a1ee08-206bc15a
% LTXE-SLIDE-TITLE:		Add image cropping (1)
% LTXE-SLIDE-REFERENCE:	Feature-65584-AddImageCropping.rst
% ------------------------------------------------------------------------------
\begin{frame}[fragile]
	\frametitle{Änderungen im System}
	\framesubtitle{Einstellungen für Bild-Manipulation (1)}

	\begin{itemize}
		\item Folgende Einstellungen können über TypoScript getätigt werden:
			\begin{lstlisting}
				# Cropping fuer alle Bilder deaktivieren
				tt_content.image.20.1.file.crop =

				# Ueberschreiben/Setzen der Cropping Eigenschaften
				# offsetX,offsetY,width,height
				tt_content.image.20.1.file.crop = 50,50,100,100
			\end{lstlisting}

	\end{itemize}

\end{frame}

% ------------------------------------------------------------------------------
% LTXE-SLIDE-START
% LTXE-SLIDE-UID:		55e12732-2568b1f8-3d460a5d-68a68445
% LTXE-SLIDE-TITLE:		Add image cropping (2)
% LTXE-SLIDE-REFERENCE:	Feature-65584-AddImageCropping.rst
% ------------------------------------------------------------------------------
\begin{frame}[fragile]
	\frametitle{Änderungen im System}
	\framesubtitle{Einstellungen für Bild-Manipulation (2)}

	\begin{itemize}
		\item Das Cropping kann auch in Fluid verwendet werden:
			\begin{lstlisting}
				# Cropping fuer alle Bilder deaktivieren
				<f:image image="{imageObject}" crop="" ></f:image>

				# Ueberschreiben/Setzen der Cropping Eigenschaften
				# offsetX,offsetY,width,height
				<f:image image="{imageObject}" crop="50,50,100,100" ></f:image>
			\end{lstlisting}

	\end{itemize}

\end{frame}

% ------------------------------------------------------------------------------
% LTXE-SLIDE-START
% LTXE-SLIDE-UID:		e9cce085-6c300f1f-010de4ee-2c744704
% LTXE-SLIDE-TITLE:		Add image cropping (3)
% LTXE-SLIDE-REFERENCE:	Feature-65584-AddImageCropping.rst
% ------------------------------------------------------------------------------
\begin{frame}[fragile]
	\frametitle{Änderungen im System}
	\framesubtitle{Einstellungen für Bild-Manipulation (3)}

	% decrease font size for code listing
	\lstset{basicstyle=\smaller\ttfamily}

	\begin{itemize}
		\item Im TCA wird das Image-Cropping wie folgt zur Verfügung gestellt:

			\begin{itemize}
				\item Column Type: \texttt{image\_manipulation}
				\item Config \texttt{file\_field}: string	\tabto{5.6cm}(default: \texttt{uid\_local})
				\item Config \texttt{enableZoom}: boolean	\tabto{5.6cm}(default: \texttt{FALSE})
				\item Config \texttt{allowedExtensions}: string\newline
					(default: \smaller\texttt{\$GLOBALS['TYPO3\_CONF\_VARS']['GFX']['imagefile\_ext']}\small)
				\item Config \texttt{ratios}: array, default:

					\begin{lstlisting}
						array(
						  '1.7777777777777777' => '16:9',
						  '1.3333333333333333' => '4:3',
						  '1' => '1:1',
						  'NaN' => 'Free'
						)
					\end{lstlisting}
			\end{itemize}

	\end{itemize}

\end{frame}

% ------------------------------------------------------------------------------
% LTXE-SLIDE-START
% LTXE-SLIDE-UID:		ed863269-574e4061-d69f855e-6494c062
% LTXE-SLIDE-TITLE:		Additional params for HTMLparser userFunc
% LTXE-SLIDE-REFERENCE:	Feature-59712-HtmlParserAdditionalUserFuncParams.rst
% ------------------------------------------------------------------------------
\begin{frame}[fragile]
	\frametitle{Änderungen im System}
	\framesubtitle{Zusätzliche Parameter für HTMLparser userFunc}

	% decrease font size for code listing
	\lstset{basicstyle=\tiny\ttfamily}

	\begin{itemize}

		\item Die userFunc im HTMLparser kann nun zusätzliche Parameter aufnehmen:

			\begin{lstlisting}
				myobj = TEXT
				myobj.value = <a href="/" class="myclass">MyText</a>
				myobj.HTMLparser.tags.a.fixAttrib.class {
				  userFunc = Tx\MyExt\Myclass->htmlUserFunc
				  userFunc.myparam = test
				}
			\end{lstlisting}

		\item Diese können in einer Extension wie folgt abgerufen werden:

			\begin{lstlisting}
				function htmlUserFunc(array $params, HtmlParser $htmlParser) {
				  // $params['attributeValue'] enthaelt den Wert der
				  // verarbeiteten Eigenschaft - hier also "myclass"
				  // $params['myparam'] enthaelt den Wert "test"
				  ...
				}
			\end{lstlisting}

	\end{itemize}

\end{frame}

% ------------------------------------------------------------------------------
% LTXE-SLIDE-START
% LTXE-SLIDE-UID:		b04f8d44-fd695c73-b768f3cb-fd6d8643
% LTXE-SLIDE-TITLE:		New Locking API (1)
% LTXE-SLIDE-REFERENCE:	Feature-47712-NewLockingAPI.rst
% ------------------------------------------------------------------------------
\begin{frame}[fragile]
	\frametitle{Änderungen im System}
	\framesubtitle{Locking API (1)}

	\begin{itemize}

		\item Es wurde eine neue Locking-API eingeführt, welche verschiedene Locking-Methoden (SimpleFile, Semaphore, ...) zur Verfügung stellt
		\item Eine Locking-Methode muss dabei das \texttt{LockingStrategyInterface} implementieren
		\begin{lstlisting}
			$lockFactory = GeneralUtility::makeInstance(LockFactory::class);
			$locker = $lockFactory->createLocker('someId');
			$locker->acquire() || die('Could not acquire lock.');
			...
			$locker->release();
		\end{lstlisting}

	\end{itemize}

\end{frame}

% ------------------------------------------------------------------------------
% LTXE-SLIDE-START
% LTXE-SLIDE-UID:		79a00412-c72401e6-4d9667c9-0e15d3f4
% LTXE-SLIDE-TITLE:		New Locking API (2)
% LTXE-SLIDE-REFERENCE:	Feature-47712-NewLockingAPI.rst
% ------------------------------------------------------------------------------
\begin{frame}[fragile]
	\frametitle{Änderungen im System}
	\framesubtitle{Locking API (2)}

	% decrease font size for code listing
	\lstset{basicstyle=\tiny\ttfamily}

	\begin{itemize}

		\item Man kann außerdem "non-blocking" Locks realisieren:

		\begin{lstlisting}
			$lockFactory = GeneralUtility::makeInstance(LockFactory::class);
			$locker = $lockFactory->createLocker(
			  'someId',
			  LockingStrategyInterface::LOCK_CAPABILITY_SHARED |
			    LockingStrategyInterface::LOCK_CAPABILITY_NOBLOCK
			);
			try {
			  $result = $locker->acquire(LockingStrategyInterface::LOCK_CAPABILITY_SHARED | LockingStrategyInterface::LOCK_CAPABILITY_NOBLOCK);
			  catch (\RuntimeException $e) {
			  if ($e->getCode() === 1428700748) {
			    // einige Prozesse haben noch ein Lock
			    // daher sollte etwas in der Zwischenzeit gemacht werden
			    ...
			  }
			}
			if ($result) {
			  $locker->release();
			}
		\end{lstlisting}

	\end{itemize}

\end{frame}

% ------------------------------------------------------------------------------
% LTXE-SLIDE-START
% LTXE-SLIDE-UID:		108cb4a7-6e006507-c25a91e7-3a867f09
% LTXE-SLIDE-TITLE:		Add signal after extension installation
% LTXE-SLIDE-REFERENCE:	commit 39e0fd0a2c919da270cb49223433bcf95febbb10
% ------------------------------------------------------------------------------
\begin{frame}[fragile]
	\frametitle{Änderungen im System}
	\framesubtitle{Signal nach Installation von Extensions}

	% decrease font size for code listing
	\lstset{basicstyle=\tiny\ttfamily}

	\begin{itemize}

		\item In der Methode
			\smaller
				\texttt{\textbackslash
					TYPO3\textbackslash
					CMS\textbackslash
					Extensionmanager\textbackslash
					Utility\textbackslash
					InstallUtility::install()}
			\normalsize
			wurde ein Signal eingebaut, welches emmitiert wird, sobald eine Extension
			fertig installiert ist und alle Imports/Updates durchgelaufen sind

		\begin{lstlisting}
			// Aufruf
			$this->emitAfterExtensionInstallSignal($extensionKey);

			// Methode
			protected function emitAfterExtensionInstallSignal($extensionKey) {
			  $this->signalSlotDispatcher->dispatch(
			    __CLASS__,
			    'afterExtensionInstall',
			    array($extensionKey, $this)
			  );
			}
		\end{lstlisting}

	\end{itemize}

\end{frame}

% ------------------------------------------------------------------------------
% LTXE-SLIDE-START
% LTXE-SLIDE-UID:		acb1ba95-1a401bef-549072a1-f32263a9
% LTXE-SLIDE-TITLE:		Registry for adding text extractor classes
% LTXE-SLIDE-REFERENCE:	Feature-36743-FAL-TextExtractorRegistry.rst
% ------------------------------------------------------------------------------
\begin{frame}[fragile]
	\frametitle{Änderungen im System}
	\framesubtitle{Registry für Text-Extraktoren}

	% decrease font size for code listing
	\lstset{basicstyle=\tiny\ttfamily}

	\begin{itemize}

		\item Der Core enthält nun eine Registry um Text-Extraktoren  anzumelden

		\item Dabei prüft \texttt{canExtractText()} ob eine Extrahierung möglich ist
			und \texttt{extractText()} führt diese durch

		\item Die Registrierung erfolgt in \texttt{ext\_localconf.php}:

			\begin{lstlisting}
				$textExtractorRegistry = \TYPO3\CMS\Core\Resource\TextExtraction\TextExtractorRegistry::getInstance();
				$textExtractorRegistry->registerTextExtractor(
				  \TYPO3\CMS\Core\Resource\TextExtraction\PlainTextExtractor::class
				);
			\end{lstlisting}

		\item Die Verwendung erfolgt folgendermaßen:

			\begin{lstlisting}
				$textExtractorRegistry = \TYPO3\CMS\Core\Resource\TextExtraction\TextExtractorRegistry::getInstance();
				$extractor = $textExtractorRegistry->getTextExtractor($file);
				if($extractor !== NULL) {
				  $content = $extractor->extractText($file);
				}
			\end{lstlisting}
	\end{itemize}

\end{frame}

% ------------------------------------------------------------------------------
% LTXE-SLIDE-START
% LTXE-SLIDE-UID:		d9657b5d-5cd56c65-d7266a27-401dd233
% LTXE-SLIDE-TITLE:		Miscellaneous
% LTXE-SLIDE-REFERENCE:	Feature-66042-WebLibrariesLoadedViaBower.rst
% LTXE-SLIDE-REFERENCE:	Feature-63703-AddOptionToStopTask.rst
% LTXE-SLIDE-REFERENCE:	Feature-61463-AllowProcessedFoldersInDifferentStorage.rst
% LTXE-SLIDE-REFERENCE:	Feature-52693-TSFE-RequestedId.rst
% ------------------------------------------------------------------------------
% LTXE-SLIDE-TOPIC:		Web Libraries are included via bower
% LTXE-SLIDE-TOPIC:		Add option to stop a running task in the scheduler
% LTXE-SLIDE-TOPIC:		Allow processed folders in different storage
% LTXE-SLIDE-TOPIC:		Add TSFE property $requestedId

\begin{frame}[fragile]
	\frametitle{Änderungen im System}
	\framesubtitle{Diverses}

	\begin{itemize}

		\item Alle Web-Bibliotheken (wie z.B. Twitter Bootstrap, jQuery, Font Awesome usw.) verwenden
			nun "Bower" (\url{http://bower.io}) zur Installation, und sind nicht mehr im TYPO3 Core Git
			enthalten\newline
			\small
				\texttt{bower install}	\tabto{3cm}führt eine Installation durch\newline
				\texttt{bower update}	\tabto{3cm}führt ein Update durch\newline
			\normalsize
			(die zugehörige Datei \texttt{bower.json} befindet sich im Verzeichnis \texttt{Build})

		\item Ein laufender Scheduler Task kann nun in der Kommandozeile mit der Option \texttt{-s}
			wieder gestoppt werden

		\item Der "Processing" Ordner eines Storages kann nun auch außerhalb von diesem liegen
			(z.B. bei read-only Storages)

		\item Man kann nun auf die ID der ursprünglich angefragten Seite über das TSFE zugreifen:
			\texttt{\$TSFE->getRequestedId()}

	\end{itemize}

\end{frame}

% ------------------------------------------------------------------------------
